\begin{comment}
\begin{table}[]
    \centering
    \begin{tabular}{c|c|c}
        Model & Decompression event & Closed box  \\
        Width ($\mathrm{km}$) & 1 & 300 \\
        Height ($\mathrm{km}$) & 120 & 120 \\
         & 
    \end{tabular}
    \caption{Caption}
    \label{tab:decompression_params}
\end{table}
\end{comment}

\subsection{Closed box}

\begin{figure}
    \centering
    \begin{subfigure}{0.49\textwidth}
        \centering
        \includegraphics[width=\textwidth]{figures/box_batch0.png}
        \caption{Caption}
        \label{fig:my_label}
    \end{subfigure}
    \hfill
    \begin{subfigure}{0.49\textwidth}
        \centering
        \includegraphics[width=\textwidth]{figures/box_2pp0.png}
        \caption{Caption}
        \label{fig:my_label}
    \end{subfigure}
    %
    \begin{subfigure}{0.49\textwidth}
        \centering
        \includegraphics[width=\textwidth]{figures/box_batch38.png}
        \caption{Caption}
        \label{fig:my_label}
    \end{subfigure}
    \hfill
    \begin{subfigure}{0.49\textwidth}
        \centering
        \includegraphics[width=\textwidth]{figures/box_2pp38.png}
        \caption{Caption}
        \label{fig:my_label}
    \end{subfigure}
    %
    \begin{subfigure}{0.49\textwidth}
        \centering
        \includegraphics[width=\textwidth]{figures/box_batch79.png}
        \caption{Caption}
        \label{fig:my_label}
    \end{subfigure}
    \hfill
    \begin{subfigure}{0.49\textwidth}
        \centering
        \includegraphics[width=\textwidth]{figures/box_2pp77.png}
        \caption{Caption}
        \label{fig:my_label}
    \end{subfigure}
\end{figure}
\subsection{Decompression event}

\begin{figure}
    \centering
    %% Creator: Matplotlib, PGF backend
%%
%% To include the figure in your LaTeX document, write
%%   \input{<filename>.pgf}
%%
%% Make sure the required packages are loaded in your preamble
%%   \usepackage{pgf}
%%
%% Figures using additional raster images can only be included by \input if
%% they are in the same directory as the main LaTeX file. For loading figures
%% from other directories you can use the `import` package
%%   \usepackage{import}
%% and then include the figures with
%%   \import{<path to file>}{<filename>.pgf}
%%
%% Matplotlib used the following preamble
%%   \usepackage{fontspec}
%%   \setmainfont{DejaVuSerif.ttf}[Path=/home/connor/.local/lib/python3.8/site-packages/matplotlib/mpl-data/fonts/ttf/]
%%   \setsansfont{DejaVuSans.ttf}[Path=/home/connor/.local/lib/python3.8/site-packages/matplotlib/mpl-data/fonts/ttf/]
%%   \setmonofont{DejaVuSansMono.ttf}[Path=/home/connor/.local/lib/python3.8/site-packages/matplotlib/mpl-data/fonts/ttf/]
%%
\begingroup%
\makeatletter%
\begin{pgfpicture}%
\pgfpathrectangle{\pgfpointorigin}{\pgfqpoint{4.004230in}{3.132284in}}%
\pgfusepath{use as bounding box, clip}%
\begin{pgfscope}%
\pgfsetbuttcap%
\pgfsetmiterjoin%
\definecolor{currentfill}{rgb}{1.000000,1.000000,1.000000}%
\pgfsetfillcolor{currentfill}%
\pgfsetlinewidth{0.000000pt}%
\definecolor{currentstroke}{rgb}{1.000000,1.000000,1.000000}%
\pgfsetstrokecolor{currentstroke}%
\pgfsetdash{}{0pt}%
\pgfpathmoveto{\pgfqpoint{0.000000in}{0.000000in}}%
\pgfpathlineto{\pgfqpoint{4.004230in}{0.000000in}}%
\pgfpathlineto{\pgfqpoint{4.004230in}{3.132284in}}%
\pgfpathlineto{\pgfqpoint{0.000000in}{3.132284in}}%
\pgfpathclose%
\pgfusepath{fill}%
\end{pgfscope}%
\begin{pgfscope}%
\pgfsetbuttcap%
\pgfsetmiterjoin%
\definecolor{currentfill}{rgb}{1.000000,1.000000,1.000000}%
\pgfsetfillcolor{currentfill}%
\pgfsetlinewidth{0.000000pt}%
\definecolor{currentstroke}{rgb}{0.000000,0.000000,0.000000}%
\pgfsetstrokecolor{currentstroke}%
\pgfsetstrokeopacity{0.000000}%
\pgfsetdash{}{0pt}%
\pgfpathmoveto{\pgfqpoint{0.511159in}{1.741813in}}%
\pgfpathlineto{\pgfqpoint{3.904230in}{1.741813in}}%
\pgfpathlineto{\pgfqpoint{3.904230in}{3.014215in}}%
\pgfpathlineto{\pgfqpoint{0.511159in}{3.014215in}}%
\pgfpathclose%
\pgfusepath{fill}%
\end{pgfscope}%
\begin{pgfscope}%
\pgfsetbuttcap%
\pgfsetroundjoin%
\definecolor{currentfill}{rgb}{0.000000,0.000000,0.000000}%
\pgfsetfillcolor{currentfill}%
\pgfsetlinewidth{0.803000pt}%
\definecolor{currentstroke}{rgb}{0.000000,0.000000,0.000000}%
\pgfsetstrokecolor{currentstroke}%
\pgfsetdash{}{0pt}%
\pgfsys@defobject{currentmarker}{\pgfqpoint{0.000000in}{-0.048611in}}{\pgfqpoint{0.000000in}{0.000000in}}{%
\pgfpathmoveto{\pgfqpoint{0.000000in}{0.000000in}}%
\pgfpathlineto{\pgfqpoint{0.000000in}{-0.048611in}}%
\pgfusepath{stroke,fill}%
}%
\begin{pgfscope}%
\pgfsys@transformshift{0.665390in}{1.741813in}%
\pgfsys@useobject{currentmarker}{}%
\end{pgfscope}%
\end{pgfscope}%
\begin{pgfscope}%
\definecolor{textcolor}{rgb}{0.000000,0.000000,0.000000}%
\pgfsetstrokecolor{textcolor}%
\pgfsetfillcolor{textcolor}%
\pgftext[x=0.665390in,y=1.644591in,,top]{\color{textcolor}\rmfamily\fontsize{8.000000}{9.600000}\selectfont \(\displaystyle 0\)}%
\end{pgfscope}%
\begin{pgfscope}%
\pgfsetbuttcap%
\pgfsetroundjoin%
\definecolor{currentfill}{rgb}{0.000000,0.000000,0.000000}%
\pgfsetfillcolor{currentfill}%
\pgfsetlinewidth{0.803000pt}%
\definecolor{currentstroke}{rgb}{0.000000,0.000000,0.000000}%
\pgfsetstrokecolor{currentstroke}%
\pgfsetdash{}{0pt}%
\pgfsys@defobject{currentmarker}{\pgfqpoint{0.000000in}{-0.048611in}}{\pgfqpoint{0.000000in}{0.000000in}}{%
\pgfpathmoveto{\pgfqpoint{0.000000in}{0.000000in}}%
\pgfpathlineto{\pgfqpoint{0.000000in}{-0.048611in}}%
\pgfusepath{stroke,fill}%
}%
\begin{pgfscope}%
\pgfsys@transformshift{1.078738in}{1.741813in}%
\pgfsys@useobject{currentmarker}{}%
\end{pgfscope}%
\end{pgfscope}%
\begin{pgfscope}%
\definecolor{textcolor}{rgb}{0.000000,0.000000,0.000000}%
\pgfsetstrokecolor{textcolor}%
\pgfsetfillcolor{textcolor}%
\pgftext[x=1.078738in,y=1.644591in,,top]{\color{textcolor}\rmfamily\fontsize{8.000000}{9.600000}\selectfont \(\displaystyle 20000\)}%
\end{pgfscope}%
\begin{pgfscope}%
\pgfsetbuttcap%
\pgfsetroundjoin%
\definecolor{currentfill}{rgb}{0.000000,0.000000,0.000000}%
\pgfsetfillcolor{currentfill}%
\pgfsetlinewidth{0.803000pt}%
\definecolor{currentstroke}{rgb}{0.000000,0.000000,0.000000}%
\pgfsetstrokecolor{currentstroke}%
\pgfsetdash{}{0pt}%
\pgfsys@defobject{currentmarker}{\pgfqpoint{0.000000in}{-0.048611in}}{\pgfqpoint{0.000000in}{0.000000in}}{%
\pgfpathmoveto{\pgfqpoint{0.000000in}{0.000000in}}%
\pgfpathlineto{\pgfqpoint{0.000000in}{-0.048611in}}%
\pgfusepath{stroke,fill}%
}%
\begin{pgfscope}%
\pgfsys@transformshift{1.492086in}{1.741813in}%
\pgfsys@useobject{currentmarker}{}%
\end{pgfscope}%
\end{pgfscope}%
\begin{pgfscope}%
\definecolor{textcolor}{rgb}{0.000000,0.000000,0.000000}%
\pgfsetstrokecolor{textcolor}%
\pgfsetfillcolor{textcolor}%
\pgftext[x=1.492086in,y=1.644591in,,top]{\color{textcolor}\rmfamily\fontsize{8.000000}{9.600000}\selectfont \(\displaystyle 40000\)}%
\end{pgfscope}%
\begin{pgfscope}%
\pgfsetbuttcap%
\pgfsetroundjoin%
\definecolor{currentfill}{rgb}{0.000000,0.000000,0.000000}%
\pgfsetfillcolor{currentfill}%
\pgfsetlinewidth{0.803000pt}%
\definecolor{currentstroke}{rgb}{0.000000,0.000000,0.000000}%
\pgfsetstrokecolor{currentstroke}%
\pgfsetdash{}{0pt}%
\pgfsys@defobject{currentmarker}{\pgfqpoint{0.000000in}{-0.048611in}}{\pgfqpoint{0.000000in}{0.000000in}}{%
\pgfpathmoveto{\pgfqpoint{0.000000in}{0.000000in}}%
\pgfpathlineto{\pgfqpoint{0.000000in}{-0.048611in}}%
\pgfusepath{stroke,fill}%
}%
\begin{pgfscope}%
\pgfsys@transformshift{1.905434in}{1.741813in}%
\pgfsys@useobject{currentmarker}{}%
\end{pgfscope}%
\end{pgfscope}%
\begin{pgfscope}%
\definecolor{textcolor}{rgb}{0.000000,0.000000,0.000000}%
\pgfsetstrokecolor{textcolor}%
\pgfsetfillcolor{textcolor}%
\pgftext[x=1.905434in,y=1.644591in,,top]{\color{textcolor}\rmfamily\fontsize{8.000000}{9.600000}\selectfont \(\displaystyle 60000\)}%
\end{pgfscope}%
\begin{pgfscope}%
\pgfsetbuttcap%
\pgfsetroundjoin%
\definecolor{currentfill}{rgb}{0.000000,0.000000,0.000000}%
\pgfsetfillcolor{currentfill}%
\pgfsetlinewidth{0.803000pt}%
\definecolor{currentstroke}{rgb}{0.000000,0.000000,0.000000}%
\pgfsetstrokecolor{currentstroke}%
\pgfsetdash{}{0pt}%
\pgfsys@defobject{currentmarker}{\pgfqpoint{0.000000in}{-0.048611in}}{\pgfqpoint{0.000000in}{0.000000in}}{%
\pgfpathmoveto{\pgfqpoint{0.000000in}{0.000000in}}%
\pgfpathlineto{\pgfqpoint{0.000000in}{-0.048611in}}%
\pgfusepath{stroke,fill}%
}%
\begin{pgfscope}%
\pgfsys@transformshift{2.318782in}{1.741813in}%
\pgfsys@useobject{currentmarker}{}%
\end{pgfscope}%
\end{pgfscope}%
\begin{pgfscope}%
\definecolor{textcolor}{rgb}{0.000000,0.000000,0.000000}%
\pgfsetstrokecolor{textcolor}%
\pgfsetfillcolor{textcolor}%
\pgftext[x=2.318782in,y=1.644591in,,top]{\color{textcolor}\rmfamily\fontsize{8.000000}{9.600000}\selectfont \(\displaystyle 80000\)}%
\end{pgfscope}%
\begin{pgfscope}%
\pgfsetbuttcap%
\pgfsetroundjoin%
\definecolor{currentfill}{rgb}{0.000000,0.000000,0.000000}%
\pgfsetfillcolor{currentfill}%
\pgfsetlinewidth{0.803000pt}%
\definecolor{currentstroke}{rgb}{0.000000,0.000000,0.000000}%
\pgfsetstrokecolor{currentstroke}%
\pgfsetdash{}{0pt}%
\pgfsys@defobject{currentmarker}{\pgfqpoint{0.000000in}{-0.048611in}}{\pgfqpoint{0.000000in}{0.000000in}}{%
\pgfpathmoveto{\pgfqpoint{0.000000in}{0.000000in}}%
\pgfpathlineto{\pgfqpoint{0.000000in}{-0.048611in}}%
\pgfusepath{stroke,fill}%
}%
\begin{pgfscope}%
\pgfsys@transformshift{2.732130in}{1.741813in}%
\pgfsys@useobject{currentmarker}{}%
\end{pgfscope}%
\end{pgfscope}%
\begin{pgfscope}%
\definecolor{textcolor}{rgb}{0.000000,0.000000,0.000000}%
\pgfsetstrokecolor{textcolor}%
\pgfsetfillcolor{textcolor}%
\pgftext[x=2.732130in,y=1.644591in,,top]{\color{textcolor}\rmfamily\fontsize{8.000000}{9.600000}\selectfont \(\displaystyle 100000\)}%
\end{pgfscope}%
\begin{pgfscope}%
\pgfsetbuttcap%
\pgfsetroundjoin%
\definecolor{currentfill}{rgb}{0.000000,0.000000,0.000000}%
\pgfsetfillcolor{currentfill}%
\pgfsetlinewidth{0.803000pt}%
\definecolor{currentstroke}{rgb}{0.000000,0.000000,0.000000}%
\pgfsetstrokecolor{currentstroke}%
\pgfsetdash{}{0pt}%
\pgfsys@defobject{currentmarker}{\pgfqpoint{0.000000in}{-0.048611in}}{\pgfqpoint{0.000000in}{0.000000in}}{%
\pgfpathmoveto{\pgfqpoint{0.000000in}{0.000000in}}%
\pgfpathlineto{\pgfqpoint{0.000000in}{-0.048611in}}%
\pgfusepath{stroke,fill}%
}%
\begin{pgfscope}%
\pgfsys@transformshift{3.145478in}{1.741813in}%
\pgfsys@useobject{currentmarker}{}%
\end{pgfscope}%
\end{pgfscope}%
\begin{pgfscope}%
\definecolor{textcolor}{rgb}{0.000000,0.000000,0.000000}%
\pgfsetstrokecolor{textcolor}%
\pgfsetfillcolor{textcolor}%
\pgftext[x=3.145478in,y=1.644591in,,top]{\color{textcolor}\rmfamily\fontsize{8.000000}{9.600000}\selectfont \(\displaystyle 120000\)}%
\end{pgfscope}%
\begin{pgfscope}%
\pgfsetbuttcap%
\pgfsetroundjoin%
\definecolor{currentfill}{rgb}{0.000000,0.000000,0.000000}%
\pgfsetfillcolor{currentfill}%
\pgfsetlinewidth{0.803000pt}%
\definecolor{currentstroke}{rgb}{0.000000,0.000000,0.000000}%
\pgfsetstrokecolor{currentstroke}%
\pgfsetdash{}{0pt}%
\pgfsys@defobject{currentmarker}{\pgfqpoint{0.000000in}{-0.048611in}}{\pgfqpoint{0.000000in}{0.000000in}}{%
\pgfpathmoveto{\pgfqpoint{0.000000in}{0.000000in}}%
\pgfpathlineto{\pgfqpoint{0.000000in}{-0.048611in}}%
\pgfusepath{stroke,fill}%
}%
\begin{pgfscope}%
\pgfsys@transformshift{3.558826in}{1.741813in}%
\pgfsys@useobject{currentmarker}{}%
\end{pgfscope}%
\end{pgfscope}%
\begin{pgfscope}%
\definecolor{textcolor}{rgb}{0.000000,0.000000,0.000000}%
\pgfsetstrokecolor{textcolor}%
\pgfsetfillcolor{textcolor}%
\pgftext[x=3.558826in,y=1.644591in,,top]{\color{textcolor}\rmfamily\fontsize{8.000000}{9.600000}\selectfont \(\displaystyle 140000\)}%
\end{pgfscope}%
\begin{pgfscope}%
\pgfsetbuttcap%
\pgfsetroundjoin%
\definecolor{currentfill}{rgb}{0.000000,0.000000,0.000000}%
\pgfsetfillcolor{currentfill}%
\pgfsetlinewidth{0.803000pt}%
\definecolor{currentstroke}{rgb}{0.000000,0.000000,0.000000}%
\pgfsetstrokecolor{currentstroke}%
\pgfsetdash{}{0pt}%
\pgfsys@defobject{currentmarker}{\pgfqpoint{-0.048611in}{0.000000in}}{\pgfqpoint{0.000000in}{0.000000in}}{%
\pgfpathmoveto{\pgfqpoint{0.000000in}{0.000000in}}%
\pgfpathlineto{\pgfqpoint{-0.048611in}{0.000000in}}%
\pgfusepath{stroke,fill}%
}%
\begin{pgfscope}%
\pgfsys@transformshift{0.511159in}{1.799650in}%
\pgfsys@useobject{currentmarker}{}%
\end{pgfscope}%
\end{pgfscope}%
\begin{pgfscope}%
\definecolor{textcolor}{rgb}{0.000000,0.000000,0.000000}%
\pgfsetstrokecolor{textcolor}%
\pgfsetfillcolor{textcolor}%
\pgftext[x=0.263086in,y=1.757440in,left,base]{\color{textcolor}\rmfamily\fontsize{8.000000}{9.600000}\selectfont \(\displaystyle 0.0\)}%
\end{pgfscope}%
\begin{pgfscope}%
\pgfsetbuttcap%
\pgfsetroundjoin%
\definecolor{currentfill}{rgb}{0.000000,0.000000,0.000000}%
\pgfsetfillcolor{currentfill}%
\pgfsetlinewidth{0.803000pt}%
\definecolor{currentstroke}{rgb}{0.000000,0.000000,0.000000}%
\pgfsetstrokecolor{currentstroke}%
\pgfsetdash{}{0pt}%
\pgfsys@defobject{currentmarker}{\pgfqpoint{-0.048611in}{0.000000in}}{\pgfqpoint{0.000000in}{0.000000in}}{%
\pgfpathmoveto{\pgfqpoint{0.000000in}{0.000000in}}%
\pgfpathlineto{\pgfqpoint{-0.048611in}{0.000000in}}%
\pgfusepath{stroke,fill}%
}%
\begin{pgfscope}%
\pgfsys@transformshift{0.511159in}{2.070813in}%
\pgfsys@useobject{currentmarker}{}%
\end{pgfscope}%
\end{pgfscope}%
\begin{pgfscope}%
\definecolor{textcolor}{rgb}{0.000000,0.000000,0.000000}%
\pgfsetstrokecolor{textcolor}%
\pgfsetfillcolor{textcolor}%
\pgftext[x=0.263086in,y=2.028604in,left,base]{\color{textcolor}\rmfamily\fontsize{8.000000}{9.600000}\selectfont \(\displaystyle 0.5\)}%
\end{pgfscope}%
\begin{pgfscope}%
\pgfsetbuttcap%
\pgfsetroundjoin%
\definecolor{currentfill}{rgb}{0.000000,0.000000,0.000000}%
\pgfsetfillcolor{currentfill}%
\pgfsetlinewidth{0.803000pt}%
\definecolor{currentstroke}{rgb}{0.000000,0.000000,0.000000}%
\pgfsetstrokecolor{currentstroke}%
\pgfsetdash{}{0pt}%
\pgfsys@defobject{currentmarker}{\pgfqpoint{-0.048611in}{0.000000in}}{\pgfqpoint{0.000000in}{0.000000in}}{%
\pgfpathmoveto{\pgfqpoint{0.000000in}{0.000000in}}%
\pgfpathlineto{\pgfqpoint{-0.048611in}{0.000000in}}%
\pgfusepath{stroke,fill}%
}%
\begin{pgfscope}%
\pgfsys@transformshift{0.511159in}{2.341976in}%
\pgfsys@useobject{currentmarker}{}%
\end{pgfscope}%
\end{pgfscope}%
\begin{pgfscope}%
\definecolor{textcolor}{rgb}{0.000000,0.000000,0.000000}%
\pgfsetstrokecolor{textcolor}%
\pgfsetfillcolor{textcolor}%
\pgftext[x=0.263086in,y=2.299767in,left,base]{\color{textcolor}\rmfamily\fontsize{8.000000}{9.600000}\selectfont \(\displaystyle 1.0\)}%
\end{pgfscope}%
\begin{pgfscope}%
\pgfsetbuttcap%
\pgfsetroundjoin%
\definecolor{currentfill}{rgb}{0.000000,0.000000,0.000000}%
\pgfsetfillcolor{currentfill}%
\pgfsetlinewidth{0.803000pt}%
\definecolor{currentstroke}{rgb}{0.000000,0.000000,0.000000}%
\pgfsetstrokecolor{currentstroke}%
\pgfsetdash{}{0pt}%
\pgfsys@defobject{currentmarker}{\pgfqpoint{-0.048611in}{0.000000in}}{\pgfqpoint{0.000000in}{0.000000in}}{%
\pgfpathmoveto{\pgfqpoint{0.000000in}{0.000000in}}%
\pgfpathlineto{\pgfqpoint{-0.048611in}{0.000000in}}%
\pgfusepath{stroke,fill}%
}%
\begin{pgfscope}%
\pgfsys@transformshift{0.511159in}{2.613140in}%
\pgfsys@useobject{currentmarker}{}%
\end{pgfscope}%
\end{pgfscope}%
\begin{pgfscope}%
\definecolor{textcolor}{rgb}{0.000000,0.000000,0.000000}%
\pgfsetstrokecolor{textcolor}%
\pgfsetfillcolor{textcolor}%
\pgftext[x=0.263086in,y=2.570931in,left,base]{\color{textcolor}\rmfamily\fontsize{8.000000}{9.600000}\selectfont \(\displaystyle 1.5\)}%
\end{pgfscope}%
\begin{pgfscope}%
\pgfsetbuttcap%
\pgfsetroundjoin%
\definecolor{currentfill}{rgb}{0.000000,0.000000,0.000000}%
\pgfsetfillcolor{currentfill}%
\pgfsetlinewidth{0.803000pt}%
\definecolor{currentstroke}{rgb}{0.000000,0.000000,0.000000}%
\pgfsetstrokecolor{currentstroke}%
\pgfsetdash{}{0pt}%
\pgfsys@defobject{currentmarker}{\pgfqpoint{-0.048611in}{0.000000in}}{\pgfqpoint{0.000000in}{0.000000in}}{%
\pgfpathmoveto{\pgfqpoint{0.000000in}{0.000000in}}%
\pgfpathlineto{\pgfqpoint{-0.048611in}{0.000000in}}%
\pgfusepath{stroke,fill}%
}%
\begin{pgfscope}%
\pgfsys@transformshift{0.511159in}{2.884303in}%
\pgfsys@useobject{currentmarker}{}%
\end{pgfscope}%
\end{pgfscope}%
\begin{pgfscope}%
\definecolor{textcolor}{rgb}{0.000000,0.000000,0.000000}%
\pgfsetstrokecolor{textcolor}%
\pgfsetfillcolor{textcolor}%
\pgftext[x=0.263086in,y=2.842094in,left,base]{\color{textcolor}\rmfamily\fontsize{8.000000}{9.600000}\selectfont \(\displaystyle 2.0\)}%
\end{pgfscope}%
\begin{pgfscope}%
\definecolor{textcolor}{rgb}{0.000000,0.000000,0.000000}%
\pgfsetstrokecolor{textcolor}%
\pgfsetfillcolor{textcolor}%
\pgftext[x=0.207530in,y=2.378014in,,bottom,rotate=90.000000]{\color{textcolor}\rmfamily\fontsize{8.000000}{9.600000}\selectfont Melt composition (mol)}%
\end{pgfscope}%
\begin{pgfscope}%
\pgfpathrectangle{\pgfqpoint{0.511159in}{1.741813in}}{\pgfqpoint{3.393071in}{1.272402in}}%
\pgfusepath{clip}%
\pgfsetrectcap%
\pgfsetroundjoin%
\pgfsetlinewidth{1.505625pt}%
\definecolor{currentstroke}{rgb}{0.121569,0.466667,0.705882}%
\pgfsetstrokecolor{currentstroke}%
\pgfsetdash{}{0pt}%
\pgfpathmoveto{\pgfqpoint{0.665390in}{1.799650in}}%
\pgfpathlineto{\pgfqpoint{1.393916in}{1.799650in}}%
\pgfpathlineto{\pgfqpoint{1.414583in}{1.825296in}}%
\pgfpathlineto{\pgfqpoint{1.435250in}{2.081757in}}%
\pgfpathlineto{\pgfqpoint{1.497253in}{2.081757in}}%
\pgfpathlineto{\pgfqpoint{1.517920in}{2.093059in}}%
\pgfpathlineto{\pgfqpoint{1.538587in}{2.091644in}}%
\pgfpathlineto{\pgfqpoint{1.559255in}{2.090939in}}%
\pgfpathlineto{\pgfqpoint{1.579922in}{2.081757in}}%
\pgfpathlineto{\pgfqpoint{1.600590in}{2.089529in}}%
\pgfpathlineto{\pgfqpoint{1.641924in}{2.086351in}}%
\pgfpathlineto{\pgfqpoint{1.662592in}{2.070461in}}%
\pgfpathlineto{\pgfqpoint{1.765929in}{2.070461in}}%
\pgfpathlineto{\pgfqpoint{1.786596in}{2.064810in}}%
\pgfpathlineto{\pgfqpoint{1.807264in}{2.070461in}}%
\pgfpathlineto{\pgfqpoint{1.848598in}{2.070461in}}%
\pgfpathlineto{\pgfqpoint{1.869266in}{2.068308in}}%
\pgfpathlineto{\pgfqpoint{1.889933in}{2.068340in}}%
\pgfpathlineto{\pgfqpoint{1.910601in}{2.064115in}}%
\pgfpathlineto{\pgfqpoint{1.931268in}{2.062808in}}%
\pgfpathlineto{\pgfqpoint{1.951935in}{2.059869in}}%
\pgfpathlineto{\pgfqpoint{1.972603in}{2.059869in}}%
\pgfpathlineto{\pgfqpoint{1.993270in}{2.057396in}}%
\pgfpathlineto{\pgfqpoint{2.075940in}{2.050687in}}%
\pgfpathlineto{\pgfqpoint{2.096607in}{2.047862in}}%
\pgfpathlineto{\pgfqpoint{2.117275in}{2.047412in}}%
\pgfpathlineto{\pgfqpoint{2.158609in}{2.042759in}}%
\pgfpathlineto{\pgfqpoint{2.179277in}{2.041506in}}%
\pgfpathlineto{\pgfqpoint{2.261946in}{2.030209in}}%
\pgfpathlineto{\pgfqpoint{2.282614in}{2.028799in}}%
\pgfpathlineto{\pgfqpoint{2.303281in}{2.026679in}}%
\pgfpathlineto{\pgfqpoint{2.323949in}{2.025572in}}%
\pgfpathlineto{\pgfqpoint{2.344616in}{2.023853in}}%
\pgfpathlineto{\pgfqpoint{2.406618in}{2.016087in}}%
\pgfpathlineto{\pgfqpoint{2.427286in}{2.014319in}}%
\pgfpathlineto{\pgfqpoint{2.489288in}{2.004215in}}%
\pgfpathlineto{\pgfqpoint{2.509955in}{2.003375in}}%
\pgfpathlineto{\pgfqpoint{2.551290in}{2.003375in}}%
\pgfpathlineto{\pgfqpoint{2.571958in}{1.997724in}}%
\pgfpathlineto{\pgfqpoint{2.592625in}{1.995961in}}%
\pgfpathlineto{\pgfqpoint{2.654627in}{1.986731in}}%
\pgfpathlineto{\pgfqpoint{2.675295in}{1.985245in}}%
\pgfpathlineto{\pgfqpoint{2.695962in}{1.982191in}}%
\pgfpathlineto{\pgfqpoint{2.716629in}{1.982099in}}%
\pgfpathlineto{\pgfqpoint{2.737297in}{1.983075in}}%
\pgfpathlineto{\pgfqpoint{2.757964in}{1.982186in}}%
\pgfpathlineto{\pgfqpoint{3.005973in}{1.982962in}}%
\pgfpathlineto{\pgfqpoint{3.150645in}{1.982897in}}%
\pgfpathlineto{\pgfqpoint{3.274649in}{1.983200in}}%
\pgfpathlineto{\pgfqpoint{3.315984in}{1.982897in}}%
\pgfpathlineto{\pgfqpoint{3.481323in}{1.982436in}}%
\pgfpathlineto{\pgfqpoint{3.605328in}{1.981774in}}%
\pgfpathlineto{\pgfqpoint{3.646662in}{1.982392in}}%
\pgfpathlineto{\pgfqpoint{3.667330in}{1.981503in}}%
\pgfpathlineto{\pgfqpoint{3.687997in}{1.981644in}}%
\pgfpathlineto{\pgfqpoint{3.708665in}{1.982436in}}%
\pgfpathlineto{\pgfqpoint{3.729332in}{1.981091in}}%
\pgfpathlineto{\pgfqpoint{3.750000in}{1.981053in}}%
\pgfpathlineto{\pgfqpoint{3.750000in}{1.981053in}}%
\pgfusepath{stroke}%
\end{pgfscope}%
\begin{pgfscope}%
\pgfpathrectangle{\pgfqpoint{0.511159in}{1.741813in}}{\pgfqpoint{3.393071in}{1.272402in}}%
\pgfusepath{clip}%
\pgfsetrectcap%
\pgfsetroundjoin%
\pgfsetlinewidth{1.505625pt}%
\definecolor{currentstroke}{rgb}{1.000000,0.498039,0.054902}%
\pgfsetstrokecolor{currentstroke}%
\pgfsetdash{}{0pt}%
\pgfpathmoveto{\pgfqpoint{0.665390in}{1.799650in}}%
\pgfpathlineto{\pgfqpoint{1.393916in}{1.799650in}}%
\pgfpathlineto{\pgfqpoint{1.414583in}{1.821605in}}%
\pgfpathlineto{\pgfqpoint{1.435250in}{2.039743in}}%
\pgfpathlineto{\pgfqpoint{1.455918in}{2.036918in}}%
\pgfpathlineto{\pgfqpoint{1.476585in}{2.032682in}}%
\pgfpathlineto{\pgfqpoint{1.497253in}{2.031267in}}%
\pgfpathlineto{\pgfqpoint{1.517920in}{2.024206in}}%
\pgfpathlineto{\pgfqpoint{1.538587in}{2.022795in}}%
\pgfpathlineto{\pgfqpoint{1.559255in}{2.022795in}}%
\pgfpathlineto{\pgfqpoint{1.579922in}{2.024206in}}%
\pgfpathlineto{\pgfqpoint{1.600590in}{2.021380in}}%
\pgfpathlineto{\pgfqpoint{1.641924in}{2.018560in}}%
\pgfpathlineto{\pgfqpoint{1.662592in}{2.021380in}}%
\pgfpathlineto{\pgfqpoint{1.724594in}{2.017144in}}%
\pgfpathlineto{\pgfqpoint{1.745261in}{2.014319in}}%
\pgfpathlineto{\pgfqpoint{1.786596in}{2.014319in}}%
\pgfpathlineto{\pgfqpoint{1.807264in}{2.011493in}}%
\pgfpathlineto{\pgfqpoint{1.848598in}{2.011493in}}%
\pgfpathlineto{\pgfqpoint{1.869266in}{2.008798in}}%
\pgfpathlineto{\pgfqpoint{1.889933in}{2.003022in}}%
\pgfpathlineto{\pgfqpoint{1.910601in}{2.011266in}}%
\pgfpathlineto{\pgfqpoint{1.931268in}{2.010083in}}%
\pgfpathlineto{\pgfqpoint{1.951935in}{2.010083in}}%
\pgfpathlineto{\pgfqpoint{1.972603in}{2.004432in}}%
\pgfpathlineto{\pgfqpoint{1.993270in}{2.004432in}}%
\pgfpathlineto{\pgfqpoint{2.013938in}{2.003022in}}%
\pgfpathlineto{\pgfqpoint{2.034605in}{2.003022in}}%
\pgfpathlineto{\pgfqpoint{2.055272in}{2.001612in}}%
\pgfpathlineto{\pgfqpoint{2.096607in}{2.001612in}}%
\pgfpathlineto{\pgfqpoint{2.137942in}{1.998787in}}%
\pgfpathlineto{\pgfqpoint{2.158609in}{1.998787in}}%
\pgfpathlineto{\pgfqpoint{2.179277in}{1.997707in}}%
\pgfpathlineto{\pgfqpoint{2.199944in}{1.997371in}}%
\pgfpathlineto{\pgfqpoint{2.261946in}{1.993136in}}%
\pgfpathlineto{\pgfqpoint{2.323949in}{1.992935in}}%
\pgfpathlineto{\pgfqpoint{2.365283in}{1.991026in}}%
\pgfpathlineto{\pgfqpoint{2.385951in}{1.990310in}}%
\pgfpathlineto{\pgfqpoint{2.406618in}{1.990310in}}%
\pgfpathlineto{\pgfqpoint{2.447953in}{1.988900in}}%
\pgfpathlineto{\pgfqpoint{2.489288in}{1.986617in}}%
\pgfpathlineto{\pgfqpoint{2.530623in}{1.984795in}}%
\pgfpathlineto{\pgfqpoint{2.571958in}{1.984220in}}%
\pgfpathlineto{\pgfqpoint{2.654627in}{1.980424in}}%
\pgfpathlineto{\pgfqpoint{2.675295in}{1.980277in}}%
\pgfpathlineto{\pgfqpoint{2.695962in}{1.979013in}}%
\pgfpathlineto{\pgfqpoint{2.737297in}{1.979333in}}%
\pgfpathlineto{\pgfqpoint{2.778632in}{1.981123in}}%
\pgfpathlineto{\pgfqpoint{2.819966in}{1.981150in}}%
\pgfpathlineto{\pgfqpoint{2.881969in}{1.983038in}}%
\pgfpathlineto{\pgfqpoint{2.943971in}{1.983249in}}%
\pgfpathlineto{\pgfqpoint{2.964638in}{1.984664in}}%
\pgfpathlineto{\pgfqpoint{3.005973in}{1.984968in}}%
\pgfpathlineto{\pgfqpoint{3.047308in}{1.986834in}}%
\pgfpathlineto{\pgfqpoint{3.067975in}{1.986839in}}%
\pgfpathlineto{\pgfqpoint{3.088643in}{1.987485in}}%
\pgfpathlineto{\pgfqpoint{3.109310in}{1.987485in}}%
\pgfpathlineto{\pgfqpoint{3.129977in}{1.988705in}}%
\pgfpathlineto{\pgfqpoint{3.171312in}{1.988900in}}%
\pgfpathlineto{\pgfqpoint{3.191980in}{1.990261in}}%
\pgfpathlineto{\pgfqpoint{3.233314in}{1.990310in}}%
\pgfpathlineto{\pgfqpoint{3.253982in}{1.991726in}}%
\pgfpathlineto{\pgfqpoint{3.274649in}{1.992105in}}%
\pgfpathlineto{\pgfqpoint{3.295317in}{1.993136in}}%
\pgfpathlineto{\pgfqpoint{3.315984in}{1.993136in}}%
\pgfpathlineto{\pgfqpoint{3.357319in}{1.994204in}}%
\pgfpathlineto{\pgfqpoint{3.439988in}{1.995961in}}%
\pgfpathlineto{\pgfqpoint{3.460656in}{1.997371in}}%
\pgfpathlineto{\pgfqpoint{3.501991in}{1.997458in}}%
\pgfpathlineto{\pgfqpoint{3.522658in}{1.999221in}}%
\pgfpathlineto{\pgfqpoint{3.563993in}{2.000728in}}%
\pgfpathlineto{\pgfqpoint{3.584660in}{2.000929in}}%
\pgfpathlineto{\pgfqpoint{3.605328in}{2.000197in}}%
\pgfpathlineto{\pgfqpoint{3.646662in}{1.997496in}}%
\pgfpathlineto{\pgfqpoint{3.667330in}{2.000929in}}%
\pgfpathlineto{\pgfqpoint{3.687997in}{2.002811in}}%
\pgfpathlineto{\pgfqpoint{3.708665in}{2.001612in}}%
\pgfpathlineto{\pgfqpoint{3.729332in}{2.002632in}}%
\pgfpathlineto{\pgfqpoint{3.750000in}{2.002616in}}%
\pgfpathlineto{\pgfqpoint{3.750000in}{2.002616in}}%
\pgfusepath{stroke}%
\end{pgfscope}%
\begin{pgfscope}%
\pgfpathrectangle{\pgfqpoint{0.511159in}{1.741813in}}{\pgfqpoint{3.393071in}{1.272402in}}%
\pgfusepath{clip}%
\pgfsetrectcap%
\pgfsetroundjoin%
\pgfsetlinewidth{1.505625pt}%
\definecolor{currentstroke}{rgb}{0.172549,0.627451,0.172549}%
\pgfsetstrokecolor{currentstroke}%
\pgfsetdash{}{0pt}%
\pgfpathmoveto{\pgfqpoint{0.665390in}{1.799650in}}%
\pgfpathlineto{\pgfqpoint{1.393916in}{1.799650in}}%
\pgfpathlineto{\pgfqpoint{1.414583in}{1.860379in}}%
\pgfpathlineto{\pgfqpoint{1.435250in}{2.467688in}}%
\pgfpathlineto{\pgfqpoint{1.476585in}{2.467688in}}%
\pgfpathlineto{\pgfqpoint{1.497253in}{2.466278in}}%
\pgfpathlineto{\pgfqpoint{1.517920in}{2.440843in}}%
\pgfpathlineto{\pgfqpoint{1.538587in}{2.443663in}}%
\pgfpathlineto{\pgfqpoint{1.559255in}{2.443663in}}%
\pgfpathlineto{\pgfqpoint{1.579922in}{2.464868in}}%
\pgfpathlineto{\pgfqpoint{1.600590in}{2.446483in}}%
\pgfpathlineto{\pgfqpoint{1.641924in}{2.452123in}}%
\pgfpathlineto{\pgfqpoint{1.662592in}{2.486018in}}%
\pgfpathlineto{\pgfqpoint{1.683259in}{2.486018in}}%
\pgfpathlineto{\pgfqpoint{1.703927in}{2.484608in}}%
\pgfpathlineto{\pgfqpoint{1.724594in}{2.481788in}}%
\pgfpathlineto{\pgfqpoint{1.745261in}{2.481788in}}%
\pgfpathlineto{\pgfqpoint{1.765929in}{2.478968in}}%
\pgfpathlineto{\pgfqpoint{1.786596in}{2.490249in}}%
\pgfpathlineto{\pgfqpoint{1.807264in}{2.476148in}}%
\pgfpathlineto{\pgfqpoint{1.827931in}{2.476148in}}%
\pgfpathlineto{\pgfqpoint{1.848598in}{2.473328in}}%
\pgfpathlineto{\pgfqpoint{1.869266in}{2.476148in}}%
\pgfpathlineto{\pgfqpoint{1.889933in}{2.477558in}}%
\pgfpathlineto{\pgfqpoint{1.910601in}{2.480378in}}%
\pgfpathlineto{\pgfqpoint{1.931268in}{2.481788in}}%
\pgfpathlineto{\pgfqpoint{1.951935in}{2.484608in}}%
\pgfpathlineto{\pgfqpoint{1.972603in}{2.486018in}}%
\pgfpathlineto{\pgfqpoint{2.034605in}{2.494479in}}%
\pgfpathlineto{\pgfqpoint{2.075940in}{2.497353in}}%
\pgfpathlineto{\pgfqpoint{2.096607in}{2.501583in}}%
\pgfpathlineto{\pgfqpoint{2.117275in}{2.500173in}}%
\pgfpathlineto{\pgfqpoint{2.137942in}{2.502993in}}%
\pgfpathlineto{\pgfqpoint{2.199944in}{2.508633in}}%
\pgfpathlineto{\pgfqpoint{2.220612in}{2.512864in}}%
\pgfpathlineto{\pgfqpoint{2.261946in}{2.518504in}}%
\pgfpathlineto{\pgfqpoint{2.282614in}{2.518504in}}%
\pgfpathlineto{\pgfqpoint{2.303281in}{2.521324in}}%
\pgfpathlineto{\pgfqpoint{2.323949in}{2.521541in}}%
\pgfpathlineto{\pgfqpoint{2.344616in}{2.522734in}}%
\pgfpathlineto{\pgfqpoint{2.385951in}{2.528428in}}%
\pgfpathlineto{\pgfqpoint{2.427286in}{2.531249in}}%
\pgfpathlineto{\pgfqpoint{2.489288in}{2.539166in}}%
\pgfpathlineto{\pgfqpoint{2.509955in}{2.537431in}}%
\pgfpathlineto{\pgfqpoint{2.530623in}{2.534448in}}%
\pgfpathlineto{\pgfqpoint{2.551290in}{2.532659in}}%
\pgfpathlineto{\pgfqpoint{2.571958in}{2.542963in}}%
\pgfpathlineto{\pgfqpoint{2.592625in}{2.545349in}}%
\pgfpathlineto{\pgfqpoint{2.633960in}{2.548169in}}%
\pgfpathlineto{\pgfqpoint{2.654627in}{2.550447in}}%
\pgfpathlineto{\pgfqpoint{2.675295in}{2.550989in}}%
\pgfpathlineto{\pgfqpoint{2.695962in}{2.552399in}}%
\pgfpathlineto{\pgfqpoint{2.716629in}{2.551369in}}%
\pgfpathlineto{\pgfqpoint{2.737297in}{2.544536in}}%
\pgfpathlineto{\pgfqpoint{2.757964in}{2.549579in}}%
\pgfpathlineto{\pgfqpoint{2.799299in}{2.546759in}}%
\pgfpathlineto{\pgfqpoint{2.819966in}{2.546759in}}%
\pgfpathlineto{\pgfqpoint{2.840634in}{2.545349in}}%
\pgfpathlineto{\pgfqpoint{2.861301in}{2.545349in}}%
\pgfpathlineto{\pgfqpoint{2.881969in}{2.543939in}}%
\pgfpathlineto{\pgfqpoint{2.902636in}{2.543939in}}%
\pgfpathlineto{\pgfqpoint{2.923303in}{2.542529in}}%
\pgfpathlineto{\pgfqpoint{2.943971in}{2.542529in}}%
\pgfpathlineto{\pgfqpoint{2.964638in}{2.541119in}}%
\pgfpathlineto{\pgfqpoint{2.985306in}{2.541119in}}%
\pgfpathlineto{\pgfqpoint{3.005973in}{2.539709in}}%
\pgfpathlineto{\pgfqpoint{3.047308in}{2.538299in}}%
\pgfpathlineto{\pgfqpoint{3.067975in}{2.538245in}}%
\pgfpathlineto{\pgfqpoint{3.088643in}{2.536889in}}%
\pgfpathlineto{\pgfqpoint{3.150645in}{2.536401in}}%
\pgfpathlineto{\pgfqpoint{3.171312in}{2.535479in}}%
\pgfpathlineto{\pgfqpoint{3.212647in}{2.535316in}}%
\pgfpathlineto{\pgfqpoint{3.233314in}{2.534069in}}%
\pgfpathlineto{\pgfqpoint{3.274649in}{2.534069in}}%
\pgfpathlineto{\pgfqpoint{3.295317in}{2.535099in}}%
\pgfpathlineto{\pgfqpoint{3.315984in}{2.535099in}}%
\pgfpathlineto{\pgfqpoint{3.357319in}{2.534069in}}%
\pgfpathlineto{\pgfqpoint{3.481323in}{2.534069in}}%
\pgfpathlineto{\pgfqpoint{3.501991in}{2.535208in}}%
\pgfpathlineto{\pgfqpoint{3.543325in}{2.535479in}}%
\pgfpathlineto{\pgfqpoint{3.563993in}{2.536889in}}%
\pgfpathlineto{\pgfqpoint{3.584660in}{2.536889in}}%
\pgfpathlineto{\pgfqpoint{3.625995in}{2.534286in}}%
\pgfpathlineto{\pgfqpoint{3.646662in}{2.534231in}}%
\pgfpathlineto{\pgfqpoint{3.667330in}{2.536889in}}%
\pgfpathlineto{\pgfqpoint{3.687997in}{2.534069in}}%
\pgfpathlineto{\pgfqpoint{3.708665in}{2.529838in}}%
\pgfpathlineto{\pgfqpoint{3.729332in}{2.538028in}}%
\pgfpathlineto{\pgfqpoint{3.750000in}{2.538299in}}%
\pgfpathlineto{\pgfqpoint{3.750000in}{2.538299in}}%
\pgfusepath{stroke}%
\end{pgfscope}%
\begin{pgfscope}%
\pgfpathrectangle{\pgfqpoint{0.511159in}{1.741813in}}{\pgfqpoint{3.393071in}{1.272402in}}%
\pgfusepath{clip}%
\pgfsetrectcap%
\pgfsetroundjoin%
\pgfsetlinewidth{1.505625pt}%
\definecolor{currentstroke}{rgb}{0.839216,0.152941,0.156863}%
\pgfsetstrokecolor{currentstroke}%
\pgfsetdash{}{0pt}%
\pgfpathmoveto{\pgfqpoint{0.665390in}{1.799650in}}%
\pgfpathlineto{\pgfqpoint{1.393916in}{1.799650in}}%
\pgfpathlineto{\pgfqpoint{1.414583in}{1.878580in}}%
\pgfpathlineto{\pgfqpoint{1.435250in}{2.668566in}}%
\pgfpathlineto{\pgfqpoint{1.455918in}{2.669976in}}%
\pgfpathlineto{\pgfqpoint{1.476585in}{2.672091in}}%
\pgfpathlineto{\pgfqpoint{1.497253in}{2.673501in}}%
\pgfpathlineto{\pgfqpoint{1.517920in}{2.655875in}}%
\pgfpathlineto{\pgfqpoint{1.538587in}{2.659400in}}%
\pgfpathlineto{\pgfqpoint{1.559255in}{2.661515in}}%
\pgfpathlineto{\pgfqpoint{1.579922in}{2.677731in}}%
\pgfpathlineto{\pgfqpoint{1.600590in}{2.665041in}}%
\pgfpathlineto{\pgfqpoint{1.621257in}{2.669650in}}%
\pgfpathlineto{\pgfqpoint{1.641924in}{2.673175in}}%
\pgfpathlineto{\pgfqpoint{1.662592in}{2.702461in}}%
\pgfpathlineto{\pgfqpoint{1.683259in}{2.703166in}}%
\pgfpathlineto{\pgfqpoint{1.703927in}{2.704576in}}%
\pgfpathlineto{\pgfqpoint{1.724594in}{2.706691in}}%
\pgfpathlineto{\pgfqpoint{1.765929in}{2.709511in}}%
\pgfpathlineto{\pgfqpoint{1.786596in}{2.720846in}}%
\pgfpathlineto{\pgfqpoint{1.807264in}{2.712331in}}%
\pgfpathlineto{\pgfqpoint{1.827931in}{2.712331in}}%
\pgfpathlineto{\pgfqpoint{1.848598in}{2.713741in}}%
\pgfpathlineto{\pgfqpoint{1.869266in}{2.720141in}}%
\pgfpathlineto{\pgfqpoint{1.889933in}{2.722256in}}%
\pgfpathlineto{\pgfqpoint{1.910601in}{2.729415in}}%
\pgfpathlineto{\pgfqpoint{1.931268in}{2.733211in}}%
\pgfpathlineto{\pgfqpoint{1.951935in}{2.740587in}}%
\pgfpathlineto{\pgfqpoint{1.972603in}{2.742702in}}%
\pgfpathlineto{\pgfqpoint{1.993270in}{2.748722in}}%
\pgfpathlineto{\pgfqpoint{2.055272in}{2.762497in}}%
\pgfpathlineto{\pgfqpoint{2.075940in}{2.766022in}}%
\pgfpathlineto{\pgfqpoint{2.096607in}{2.772367in}}%
\pgfpathlineto{\pgfqpoint{2.117275in}{2.775133in}}%
\pgfpathlineto{\pgfqpoint{2.158609in}{2.787389in}}%
\pgfpathlineto{\pgfqpoint{2.179277in}{2.790752in}}%
\pgfpathlineto{\pgfqpoint{2.199944in}{2.798507in}}%
\pgfpathlineto{\pgfqpoint{2.220612in}{2.807022in}}%
\pgfpathlineto{\pgfqpoint{2.241279in}{2.813313in}}%
\pgfpathlineto{\pgfqpoint{2.261946in}{2.821122in}}%
\pgfpathlineto{\pgfqpoint{2.282614in}{2.825352in}}%
\pgfpathlineto{\pgfqpoint{2.303281in}{2.830287in}}%
\pgfpathlineto{\pgfqpoint{2.323949in}{2.833596in}}%
\pgfpathlineto{\pgfqpoint{2.344616in}{2.838748in}}%
\pgfpathlineto{\pgfqpoint{2.365283in}{2.845418in}}%
\pgfpathlineto{\pgfqpoint{2.385951in}{2.853282in}}%
\pgfpathlineto{\pgfqpoint{2.406618in}{2.859248in}}%
\pgfpathlineto{\pgfqpoint{2.427286in}{2.864183in}}%
\pgfpathlineto{\pgfqpoint{2.468620in}{2.883436in}}%
\pgfpathlineto{\pgfqpoint{2.489288in}{2.892004in}}%
\pgfpathlineto{\pgfqpoint{2.509955in}{2.895963in}}%
\pgfpathlineto{\pgfqpoint{2.530623in}{2.897807in}}%
\pgfpathlineto{\pgfqpoint{2.551290in}{2.898783in}}%
\pgfpathlineto{\pgfqpoint{2.571958in}{2.910769in}}%
\pgfpathlineto{\pgfqpoint{2.592625in}{2.915379in}}%
\pgfpathlineto{\pgfqpoint{2.613292in}{2.924924in}}%
\pgfpathlineto{\pgfqpoint{2.654627in}{2.941898in}}%
\pgfpathlineto{\pgfqpoint{2.675295in}{2.946183in}}%
\pgfpathlineto{\pgfqpoint{2.695962in}{2.955294in}}%
\pgfpathlineto{\pgfqpoint{2.737297in}{2.956378in}}%
\pgfpathlineto{\pgfqpoint{2.778632in}{2.956053in}}%
\pgfpathlineto{\pgfqpoint{3.109310in}{2.956378in}}%
\pgfpathlineto{\pgfqpoint{3.171312in}{2.956378in}}%
\pgfpathlineto{\pgfqpoint{3.191980in}{2.954968in}}%
\pgfpathlineto{\pgfqpoint{3.605328in}{2.954263in}}%
\pgfpathlineto{\pgfqpoint{3.646662in}{2.954535in}}%
\pgfpathlineto{\pgfqpoint{3.687997in}{2.954209in}}%
\pgfpathlineto{\pgfqpoint{3.708665in}{2.954535in}}%
\pgfpathlineto{\pgfqpoint{3.729332in}{2.953938in}}%
\pgfpathlineto{\pgfqpoint{3.750000in}{2.953938in}}%
\pgfpathlineto{\pgfqpoint{3.750000in}{2.953938in}}%
\pgfusepath{stroke}%
\end{pgfscope}%
\begin{pgfscope}%
\pgfsetrectcap%
\pgfsetmiterjoin%
\pgfsetlinewidth{0.803000pt}%
\definecolor{currentstroke}{rgb}{0.000000,0.000000,0.000000}%
\pgfsetstrokecolor{currentstroke}%
\pgfsetdash{}{0pt}%
\pgfpathmoveto{\pgfqpoint{0.511159in}{1.741813in}}%
\pgfpathlineto{\pgfqpoint{0.511159in}{3.014215in}}%
\pgfusepath{stroke}%
\end{pgfscope}%
\begin{pgfscope}%
\pgfsetrectcap%
\pgfsetmiterjoin%
\pgfsetlinewidth{0.803000pt}%
\definecolor{currentstroke}{rgb}{0.000000,0.000000,0.000000}%
\pgfsetstrokecolor{currentstroke}%
\pgfsetdash{}{0pt}%
\pgfpathmoveto{\pgfqpoint{3.904230in}{1.741813in}}%
\pgfpathlineto{\pgfqpoint{3.904230in}{3.014215in}}%
\pgfusepath{stroke}%
\end{pgfscope}%
\begin{pgfscope}%
\pgfsetrectcap%
\pgfsetmiterjoin%
\pgfsetlinewidth{0.803000pt}%
\definecolor{currentstroke}{rgb}{0.000000,0.000000,0.000000}%
\pgfsetstrokecolor{currentstroke}%
\pgfsetdash{}{0pt}%
\pgfpathmoveto{\pgfqpoint{0.511159in}{1.741813in}}%
\pgfpathlineto{\pgfqpoint{3.904230in}{1.741813in}}%
\pgfusepath{stroke}%
\end{pgfscope}%
\begin{pgfscope}%
\pgfsetrectcap%
\pgfsetmiterjoin%
\pgfsetlinewidth{0.803000pt}%
\definecolor{currentstroke}{rgb}{0.000000,0.000000,0.000000}%
\pgfsetstrokecolor{currentstroke}%
\pgfsetdash{}{0pt}%
\pgfpathmoveto{\pgfqpoint{0.511159in}{3.014215in}}%
\pgfpathlineto{\pgfqpoint{3.904230in}{3.014215in}}%
\pgfusepath{stroke}%
\end{pgfscope}%
\begin{pgfscope}%
\pgfsetrectcap%
\pgfsetroundjoin%
\pgfsetlinewidth{1.505625pt}%
\definecolor{currentstroke}{rgb}{0.121569,0.466667,0.705882}%
\pgfsetstrokecolor{currentstroke}%
\pgfsetdash{}{0pt}%
\pgfpathmoveto{\pgfqpoint{0.611159in}{2.868685in}}%
\pgfpathlineto{\pgfqpoint{0.833381in}{2.868685in}}%
\pgfusepath{stroke}%
\end{pgfscope}%
\begin{pgfscope}%
\definecolor{textcolor}{rgb}{0.000000,0.000000,0.000000}%
\pgfsetstrokecolor{textcolor}%
\pgfsetfillcolor{textcolor}%
\pgftext[x=0.922270in,y=2.829796in,left,base]{\color{textcolor}\rmfamily\fontsize{8.000000}{9.600000}\selectfont CaO}%
\end{pgfscope}%
\begin{pgfscope}%
\pgfsetrectcap%
\pgfsetroundjoin%
\pgfsetlinewidth{1.505625pt}%
\definecolor{currentstroke}{rgb}{1.000000,0.498039,0.054902}%
\pgfsetstrokecolor{currentstroke}%
\pgfsetdash{}{0pt}%
\pgfpathmoveto{\pgfqpoint{0.611159in}{2.705599in}}%
\pgfpathlineto{\pgfqpoint{0.833381in}{2.705599in}}%
\pgfusepath{stroke}%
\end{pgfscope}%
\begin{pgfscope}%
\definecolor{textcolor}{rgb}{0.000000,0.000000,0.000000}%
\pgfsetstrokecolor{textcolor}%
\pgfsetfillcolor{textcolor}%
\pgftext[x=0.922270in,y=2.666711in,left,base]{\color{textcolor}\rmfamily\fontsize{8.000000}{9.600000}\selectfont FeO}%
\end{pgfscope}%
\begin{pgfscope}%
\pgfsetrectcap%
\pgfsetroundjoin%
\pgfsetlinewidth{1.505625pt}%
\definecolor{currentstroke}{rgb}{0.172549,0.627451,0.172549}%
\pgfsetstrokecolor{currentstroke}%
\pgfsetdash{}{0pt}%
\pgfpathmoveto{\pgfqpoint{0.611159in}{2.542514in}}%
\pgfpathlineto{\pgfqpoint{0.833381in}{2.542514in}}%
\pgfusepath{stroke}%
\end{pgfscope}%
\begin{pgfscope}%
\definecolor{textcolor}{rgb}{0.000000,0.000000,0.000000}%
\pgfsetstrokecolor{textcolor}%
\pgfsetfillcolor{textcolor}%
\pgftext[x=0.922270in,y=2.503625in,left,base]{\color{textcolor}\rmfamily\fontsize{8.000000}{9.600000}\selectfont MgO}%
\end{pgfscope}%
\begin{pgfscope}%
\pgfsetrectcap%
\pgfsetroundjoin%
\pgfsetlinewidth{1.505625pt}%
\definecolor{currentstroke}{rgb}{0.839216,0.152941,0.156863}%
\pgfsetstrokecolor{currentstroke}%
\pgfsetdash{}{0pt}%
\pgfpathmoveto{\pgfqpoint{0.611159in}{2.377854in}}%
\pgfpathlineto{\pgfqpoint{0.833381in}{2.377854in}}%
\pgfusepath{stroke}%
\end{pgfscope}%
\begin{pgfscope}%
\definecolor{textcolor}{rgb}{0.000000,0.000000,0.000000}%
\pgfsetstrokecolor{textcolor}%
\pgfsetfillcolor{textcolor}%
\pgftext[x=0.922270in,y=2.338966in,left,base]{\color{textcolor}\rmfamily\fontsize{8.000000}{9.600000}\selectfont SiO2}%
\end{pgfscope}%
\begin{pgfscope}%
\pgfsetbuttcap%
\pgfsetmiterjoin%
\definecolor{currentfill}{rgb}{1.000000,1.000000,1.000000}%
\pgfsetfillcolor{currentfill}%
\pgfsetlinewidth{0.000000pt}%
\definecolor{currentstroke}{rgb}{0.000000,0.000000,0.000000}%
\pgfsetstrokecolor{currentstroke}%
\pgfsetstrokeopacity{0.000000}%
\pgfsetdash{}{0pt}%
\pgfpathmoveto{\pgfqpoint{0.511159in}{0.469412in}}%
\pgfpathlineto{\pgfqpoint{3.904230in}{0.469412in}}%
\pgfpathlineto{\pgfqpoint{3.904230in}{1.423713in}}%
\pgfpathlineto{\pgfqpoint{0.511159in}{1.423713in}}%
\pgfpathclose%
\pgfusepath{fill}%
\end{pgfscope}%
\begin{pgfscope}%
\pgfsetbuttcap%
\pgfsetroundjoin%
\definecolor{currentfill}{rgb}{0.000000,0.000000,0.000000}%
\pgfsetfillcolor{currentfill}%
\pgfsetlinewidth{0.803000pt}%
\definecolor{currentstroke}{rgb}{0.000000,0.000000,0.000000}%
\pgfsetstrokecolor{currentstroke}%
\pgfsetdash{}{0pt}%
\pgfsys@defobject{currentmarker}{\pgfqpoint{0.000000in}{-0.048611in}}{\pgfqpoint{0.000000in}{0.000000in}}{%
\pgfpathmoveto{\pgfqpoint{0.000000in}{0.000000in}}%
\pgfpathlineto{\pgfqpoint{0.000000in}{-0.048611in}}%
\pgfusepath{stroke,fill}%
}%
\begin{pgfscope}%
\pgfsys@transformshift{0.665390in}{0.469412in}%
\pgfsys@useobject{currentmarker}{}%
\end{pgfscope}%
\end{pgfscope}%
\begin{pgfscope}%
\definecolor{textcolor}{rgb}{0.000000,0.000000,0.000000}%
\pgfsetstrokecolor{textcolor}%
\pgfsetfillcolor{textcolor}%
\pgftext[x=0.665390in,y=0.372189in,,top]{\color{textcolor}\rmfamily\fontsize{8.000000}{9.600000}\selectfont \(\displaystyle 0\)}%
\end{pgfscope}%
\begin{pgfscope}%
\pgfsetbuttcap%
\pgfsetroundjoin%
\definecolor{currentfill}{rgb}{0.000000,0.000000,0.000000}%
\pgfsetfillcolor{currentfill}%
\pgfsetlinewidth{0.803000pt}%
\definecolor{currentstroke}{rgb}{0.000000,0.000000,0.000000}%
\pgfsetstrokecolor{currentstroke}%
\pgfsetdash{}{0pt}%
\pgfsys@defobject{currentmarker}{\pgfqpoint{0.000000in}{-0.048611in}}{\pgfqpoint{0.000000in}{0.000000in}}{%
\pgfpathmoveto{\pgfqpoint{0.000000in}{0.000000in}}%
\pgfpathlineto{\pgfqpoint{0.000000in}{-0.048611in}}%
\pgfusepath{stroke,fill}%
}%
\begin{pgfscope}%
\pgfsys@transformshift{1.078738in}{0.469412in}%
\pgfsys@useobject{currentmarker}{}%
\end{pgfscope}%
\end{pgfscope}%
\begin{pgfscope}%
\definecolor{textcolor}{rgb}{0.000000,0.000000,0.000000}%
\pgfsetstrokecolor{textcolor}%
\pgfsetfillcolor{textcolor}%
\pgftext[x=1.078738in,y=0.372189in,,top]{\color{textcolor}\rmfamily\fontsize{8.000000}{9.600000}\selectfont \(\displaystyle 20000\)}%
\end{pgfscope}%
\begin{pgfscope}%
\pgfsetbuttcap%
\pgfsetroundjoin%
\definecolor{currentfill}{rgb}{0.000000,0.000000,0.000000}%
\pgfsetfillcolor{currentfill}%
\pgfsetlinewidth{0.803000pt}%
\definecolor{currentstroke}{rgb}{0.000000,0.000000,0.000000}%
\pgfsetstrokecolor{currentstroke}%
\pgfsetdash{}{0pt}%
\pgfsys@defobject{currentmarker}{\pgfqpoint{0.000000in}{-0.048611in}}{\pgfqpoint{0.000000in}{0.000000in}}{%
\pgfpathmoveto{\pgfqpoint{0.000000in}{0.000000in}}%
\pgfpathlineto{\pgfqpoint{0.000000in}{-0.048611in}}%
\pgfusepath{stroke,fill}%
}%
\begin{pgfscope}%
\pgfsys@transformshift{1.492086in}{0.469412in}%
\pgfsys@useobject{currentmarker}{}%
\end{pgfscope}%
\end{pgfscope}%
\begin{pgfscope}%
\definecolor{textcolor}{rgb}{0.000000,0.000000,0.000000}%
\pgfsetstrokecolor{textcolor}%
\pgfsetfillcolor{textcolor}%
\pgftext[x=1.492086in,y=0.372189in,,top]{\color{textcolor}\rmfamily\fontsize{8.000000}{9.600000}\selectfont \(\displaystyle 40000\)}%
\end{pgfscope}%
\begin{pgfscope}%
\pgfsetbuttcap%
\pgfsetroundjoin%
\definecolor{currentfill}{rgb}{0.000000,0.000000,0.000000}%
\pgfsetfillcolor{currentfill}%
\pgfsetlinewidth{0.803000pt}%
\definecolor{currentstroke}{rgb}{0.000000,0.000000,0.000000}%
\pgfsetstrokecolor{currentstroke}%
\pgfsetdash{}{0pt}%
\pgfsys@defobject{currentmarker}{\pgfqpoint{0.000000in}{-0.048611in}}{\pgfqpoint{0.000000in}{0.000000in}}{%
\pgfpathmoveto{\pgfqpoint{0.000000in}{0.000000in}}%
\pgfpathlineto{\pgfqpoint{0.000000in}{-0.048611in}}%
\pgfusepath{stroke,fill}%
}%
\begin{pgfscope}%
\pgfsys@transformshift{1.905434in}{0.469412in}%
\pgfsys@useobject{currentmarker}{}%
\end{pgfscope}%
\end{pgfscope}%
\begin{pgfscope}%
\definecolor{textcolor}{rgb}{0.000000,0.000000,0.000000}%
\pgfsetstrokecolor{textcolor}%
\pgfsetfillcolor{textcolor}%
\pgftext[x=1.905434in,y=0.372189in,,top]{\color{textcolor}\rmfamily\fontsize{8.000000}{9.600000}\selectfont \(\displaystyle 60000\)}%
\end{pgfscope}%
\begin{pgfscope}%
\pgfsetbuttcap%
\pgfsetroundjoin%
\definecolor{currentfill}{rgb}{0.000000,0.000000,0.000000}%
\pgfsetfillcolor{currentfill}%
\pgfsetlinewidth{0.803000pt}%
\definecolor{currentstroke}{rgb}{0.000000,0.000000,0.000000}%
\pgfsetstrokecolor{currentstroke}%
\pgfsetdash{}{0pt}%
\pgfsys@defobject{currentmarker}{\pgfqpoint{0.000000in}{-0.048611in}}{\pgfqpoint{0.000000in}{0.000000in}}{%
\pgfpathmoveto{\pgfqpoint{0.000000in}{0.000000in}}%
\pgfpathlineto{\pgfqpoint{0.000000in}{-0.048611in}}%
\pgfusepath{stroke,fill}%
}%
\begin{pgfscope}%
\pgfsys@transformshift{2.318782in}{0.469412in}%
\pgfsys@useobject{currentmarker}{}%
\end{pgfscope}%
\end{pgfscope}%
\begin{pgfscope}%
\definecolor{textcolor}{rgb}{0.000000,0.000000,0.000000}%
\pgfsetstrokecolor{textcolor}%
\pgfsetfillcolor{textcolor}%
\pgftext[x=2.318782in,y=0.372189in,,top]{\color{textcolor}\rmfamily\fontsize{8.000000}{9.600000}\selectfont \(\displaystyle 80000\)}%
\end{pgfscope}%
\begin{pgfscope}%
\pgfsetbuttcap%
\pgfsetroundjoin%
\definecolor{currentfill}{rgb}{0.000000,0.000000,0.000000}%
\pgfsetfillcolor{currentfill}%
\pgfsetlinewidth{0.803000pt}%
\definecolor{currentstroke}{rgb}{0.000000,0.000000,0.000000}%
\pgfsetstrokecolor{currentstroke}%
\pgfsetdash{}{0pt}%
\pgfsys@defobject{currentmarker}{\pgfqpoint{0.000000in}{-0.048611in}}{\pgfqpoint{0.000000in}{0.000000in}}{%
\pgfpathmoveto{\pgfqpoint{0.000000in}{0.000000in}}%
\pgfpathlineto{\pgfqpoint{0.000000in}{-0.048611in}}%
\pgfusepath{stroke,fill}%
}%
\begin{pgfscope}%
\pgfsys@transformshift{2.732130in}{0.469412in}%
\pgfsys@useobject{currentmarker}{}%
\end{pgfscope}%
\end{pgfscope}%
\begin{pgfscope}%
\definecolor{textcolor}{rgb}{0.000000,0.000000,0.000000}%
\pgfsetstrokecolor{textcolor}%
\pgfsetfillcolor{textcolor}%
\pgftext[x=2.732130in,y=0.372189in,,top]{\color{textcolor}\rmfamily\fontsize{8.000000}{9.600000}\selectfont \(\displaystyle 100000\)}%
\end{pgfscope}%
\begin{pgfscope}%
\pgfsetbuttcap%
\pgfsetroundjoin%
\definecolor{currentfill}{rgb}{0.000000,0.000000,0.000000}%
\pgfsetfillcolor{currentfill}%
\pgfsetlinewidth{0.803000pt}%
\definecolor{currentstroke}{rgb}{0.000000,0.000000,0.000000}%
\pgfsetstrokecolor{currentstroke}%
\pgfsetdash{}{0pt}%
\pgfsys@defobject{currentmarker}{\pgfqpoint{0.000000in}{-0.048611in}}{\pgfqpoint{0.000000in}{0.000000in}}{%
\pgfpathmoveto{\pgfqpoint{0.000000in}{0.000000in}}%
\pgfpathlineto{\pgfqpoint{0.000000in}{-0.048611in}}%
\pgfusepath{stroke,fill}%
}%
\begin{pgfscope}%
\pgfsys@transformshift{3.145478in}{0.469412in}%
\pgfsys@useobject{currentmarker}{}%
\end{pgfscope}%
\end{pgfscope}%
\begin{pgfscope}%
\definecolor{textcolor}{rgb}{0.000000,0.000000,0.000000}%
\pgfsetstrokecolor{textcolor}%
\pgfsetfillcolor{textcolor}%
\pgftext[x=3.145478in,y=0.372189in,,top]{\color{textcolor}\rmfamily\fontsize{8.000000}{9.600000}\selectfont \(\displaystyle 120000\)}%
\end{pgfscope}%
\begin{pgfscope}%
\pgfsetbuttcap%
\pgfsetroundjoin%
\definecolor{currentfill}{rgb}{0.000000,0.000000,0.000000}%
\pgfsetfillcolor{currentfill}%
\pgfsetlinewidth{0.803000pt}%
\definecolor{currentstroke}{rgb}{0.000000,0.000000,0.000000}%
\pgfsetstrokecolor{currentstroke}%
\pgfsetdash{}{0pt}%
\pgfsys@defobject{currentmarker}{\pgfqpoint{0.000000in}{-0.048611in}}{\pgfqpoint{0.000000in}{0.000000in}}{%
\pgfpathmoveto{\pgfqpoint{0.000000in}{0.000000in}}%
\pgfpathlineto{\pgfqpoint{0.000000in}{-0.048611in}}%
\pgfusepath{stroke,fill}%
}%
\begin{pgfscope}%
\pgfsys@transformshift{3.558826in}{0.469412in}%
\pgfsys@useobject{currentmarker}{}%
\end{pgfscope}%
\end{pgfscope}%
\begin{pgfscope}%
\definecolor{textcolor}{rgb}{0.000000,0.000000,0.000000}%
\pgfsetstrokecolor{textcolor}%
\pgfsetfillcolor{textcolor}%
\pgftext[x=3.558826in,y=0.372189in,,top]{\color{textcolor}\rmfamily\fontsize{8.000000}{9.600000}\selectfont \(\displaystyle 140000\)}%
\end{pgfscope}%
\begin{pgfscope}%
\definecolor{textcolor}{rgb}{0.000000,0.000000,0.000000}%
\pgfsetstrokecolor{textcolor}%
\pgfsetfillcolor{textcolor}%
\pgftext[x=2.207695in,y=0.209104in,,top]{\color{textcolor}\rmfamily\fontsize{8.000000}{9.600000}\selectfont Time (yr)}%
\end{pgfscope}%
\begin{pgfscope}%
\pgfsetbuttcap%
\pgfsetroundjoin%
\definecolor{currentfill}{rgb}{0.000000,0.000000,0.000000}%
\pgfsetfillcolor{currentfill}%
\pgfsetlinewidth{0.803000pt}%
\definecolor{currentstroke}{rgb}{0.000000,0.000000,0.000000}%
\pgfsetstrokecolor{currentstroke}%
\pgfsetdash{}{0pt}%
\pgfsys@defobject{currentmarker}{\pgfqpoint{-0.048611in}{0.000000in}}{\pgfqpoint{0.000000in}{0.000000in}}{%
\pgfpathmoveto{\pgfqpoint{0.000000in}{0.000000in}}%
\pgfpathlineto{\pgfqpoint{-0.048611in}{0.000000in}}%
\pgfusepath{stroke,fill}%
}%
\begin{pgfscope}%
\pgfsys@transformshift{0.511159in}{0.512789in}%
\pgfsys@useobject{currentmarker}{}%
\end{pgfscope}%
\end{pgfscope}%
\begin{pgfscope}%
\definecolor{textcolor}{rgb}{0.000000,0.000000,0.000000}%
\pgfsetstrokecolor{textcolor}%
\pgfsetfillcolor{textcolor}%
\pgftext[x=0.263086in,y=0.470580in,left,base]{\color{textcolor}\rmfamily\fontsize{8.000000}{9.600000}\selectfont \(\displaystyle 0.0\)}%
\end{pgfscope}%
\begin{pgfscope}%
\pgfsetbuttcap%
\pgfsetroundjoin%
\definecolor{currentfill}{rgb}{0.000000,0.000000,0.000000}%
\pgfsetfillcolor{currentfill}%
\pgfsetlinewidth{0.803000pt}%
\definecolor{currentstroke}{rgb}{0.000000,0.000000,0.000000}%
\pgfsetstrokecolor{currentstroke}%
\pgfsetdash{}{0pt}%
\pgfsys@defobject{currentmarker}{\pgfqpoint{-0.048611in}{0.000000in}}{\pgfqpoint{0.000000in}{0.000000in}}{%
\pgfpathmoveto{\pgfqpoint{0.000000in}{0.000000in}}%
\pgfpathlineto{\pgfqpoint{-0.048611in}{0.000000in}}%
\pgfusepath{stroke,fill}%
}%
\begin{pgfscope}%
\pgfsys@transformshift{0.511159in}{0.779159in}%
\pgfsys@useobject{currentmarker}{}%
\end{pgfscope}%
\end{pgfscope}%
\begin{pgfscope}%
\definecolor{textcolor}{rgb}{0.000000,0.000000,0.000000}%
\pgfsetstrokecolor{textcolor}%
\pgfsetfillcolor{textcolor}%
\pgftext[x=0.263086in,y=0.736950in,left,base]{\color{textcolor}\rmfamily\fontsize{8.000000}{9.600000}\selectfont \(\displaystyle 2.5\)}%
\end{pgfscope}%
\begin{pgfscope}%
\pgfsetbuttcap%
\pgfsetroundjoin%
\definecolor{currentfill}{rgb}{0.000000,0.000000,0.000000}%
\pgfsetfillcolor{currentfill}%
\pgfsetlinewidth{0.803000pt}%
\definecolor{currentstroke}{rgb}{0.000000,0.000000,0.000000}%
\pgfsetstrokecolor{currentstroke}%
\pgfsetdash{}{0pt}%
\pgfsys@defobject{currentmarker}{\pgfqpoint{-0.048611in}{0.000000in}}{\pgfqpoint{0.000000in}{0.000000in}}{%
\pgfpathmoveto{\pgfqpoint{0.000000in}{0.000000in}}%
\pgfpathlineto{\pgfqpoint{-0.048611in}{0.000000in}}%
\pgfusepath{stroke,fill}%
}%
\begin{pgfscope}%
\pgfsys@transformshift{0.511159in}{1.045529in}%
\pgfsys@useobject{currentmarker}{}%
\end{pgfscope}%
\end{pgfscope}%
\begin{pgfscope}%
\definecolor{textcolor}{rgb}{0.000000,0.000000,0.000000}%
\pgfsetstrokecolor{textcolor}%
\pgfsetfillcolor{textcolor}%
\pgftext[x=0.263086in,y=1.003320in,left,base]{\color{textcolor}\rmfamily\fontsize{8.000000}{9.600000}\selectfont \(\displaystyle 5.0\)}%
\end{pgfscope}%
\begin{pgfscope}%
\pgfsetbuttcap%
\pgfsetroundjoin%
\definecolor{currentfill}{rgb}{0.000000,0.000000,0.000000}%
\pgfsetfillcolor{currentfill}%
\pgfsetlinewidth{0.803000pt}%
\definecolor{currentstroke}{rgb}{0.000000,0.000000,0.000000}%
\pgfsetstrokecolor{currentstroke}%
\pgfsetdash{}{0pt}%
\pgfsys@defobject{currentmarker}{\pgfqpoint{-0.048611in}{0.000000in}}{\pgfqpoint{0.000000in}{0.000000in}}{%
\pgfpathmoveto{\pgfqpoint{0.000000in}{0.000000in}}%
\pgfpathlineto{\pgfqpoint{-0.048611in}{0.000000in}}%
\pgfusepath{stroke,fill}%
}%
\begin{pgfscope}%
\pgfsys@transformshift{0.511159in}{1.311900in}%
\pgfsys@useobject{currentmarker}{}%
\end{pgfscope}%
\end{pgfscope}%
\begin{pgfscope}%
\definecolor{textcolor}{rgb}{0.000000,0.000000,0.000000}%
\pgfsetstrokecolor{textcolor}%
\pgfsetfillcolor{textcolor}%
\pgftext[x=0.263086in,y=1.269690in,left,base]{\color{textcolor}\rmfamily\fontsize{8.000000}{9.600000}\selectfont \(\displaystyle 7.5\)}%
\end{pgfscope}%
\begin{pgfscope}%
\definecolor{textcolor}{rgb}{0.000000,0.000000,0.000000}%
\pgfsetstrokecolor{textcolor}%
\pgfsetfillcolor{textcolor}%
\pgftext[x=0.207530in,y=0.946562in,,bottom,rotate=90.000000]{\color{textcolor}\rmfamily\fontsize{8.000000}{9.600000}\selectfont Melt amount (mol)}%
\end{pgfscope}%
\begin{pgfscope}%
\pgfpathrectangle{\pgfqpoint{0.511159in}{0.469412in}}{\pgfqpoint{3.393071in}{0.954301in}}%
\pgfusepath{clip}%
\pgfsetrectcap%
\pgfsetroundjoin%
\pgfsetlinewidth{1.505625pt}%
\definecolor{currentstroke}{rgb}{0.121569,0.466667,0.705882}%
\pgfsetstrokecolor{currentstroke}%
\pgfsetdash{}{0pt}%
\pgfpathmoveto{\pgfqpoint{0.665390in}{0.512789in}}%
\pgfpathlineto{\pgfqpoint{1.414583in}{0.512914in}}%
\pgfpathlineto{\pgfqpoint{1.435250in}{0.525855in}}%
\pgfpathlineto{\pgfqpoint{1.455918in}{0.555963in}}%
\pgfpathlineto{\pgfqpoint{1.476585in}{0.584902in}}%
\pgfpathlineto{\pgfqpoint{1.497253in}{0.607320in}}%
\pgfpathlineto{\pgfqpoint{1.517920in}{0.626998in}}%
\pgfpathlineto{\pgfqpoint{1.538587in}{0.636630in}}%
\pgfpathlineto{\pgfqpoint{1.559255in}{0.651344in}}%
\pgfpathlineto{\pgfqpoint{1.600590in}{0.672089in}}%
\pgfpathlineto{\pgfqpoint{1.621257in}{0.679601in}}%
\pgfpathlineto{\pgfqpoint{1.641924in}{0.703031in}}%
\pgfpathlineto{\pgfqpoint{1.662592in}{0.729955in}}%
\pgfpathlineto{\pgfqpoint{1.683259in}{0.739023in}}%
\pgfpathlineto{\pgfqpoint{1.703927in}{0.759895in}}%
\pgfpathlineto{\pgfqpoint{1.724594in}{0.771712in}}%
\pgfpathlineto{\pgfqpoint{1.745261in}{0.793117in}}%
\pgfpathlineto{\pgfqpoint{1.765929in}{0.804603in}}%
\pgfpathlineto{\pgfqpoint{1.786596in}{0.819893in}}%
\pgfpathlineto{\pgfqpoint{1.807264in}{0.816046in}}%
\pgfpathlineto{\pgfqpoint{1.827931in}{0.816046in}}%
\pgfpathlineto{\pgfqpoint{1.848598in}{0.820106in}}%
\pgfpathlineto{\pgfqpoint{1.869266in}{0.863439in}}%
\pgfpathlineto{\pgfqpoint{1.889933in}{0.962614in}}%
\pgfpathlineto{\pgfqpoint{1.910601in}{0.833350in}}%
\pgfpathlineto{\pgfqpoint{1.931268in}{0.836503in}}%
\pgfpathlineto{\pgfqpoint{1.951935in}{0.843259in}}%
\pgfpathlineto{\pgfqpoint{1.972603in}{0.914720in}}%
\pgfpathlineto{\pgfqpoint{1.993270in}{0.931097in}}%
\pgfpathlineto{\pgfqpoint{2.034605in}{0.953664in}}%
\pgfpathlineto{\pgfqpoint{2.055272in}{0.961942in}}%
\pgfpathlineto{\pgfqpoint{2.075940in}{0.969103in}}%
\pgfpathlineto{\pgfqpoint{2.096607in}{0.981100in}}%
\pgfpathlineto{\pgfqpoint{2.117275in}{0.988388in}}%
\pgfpathlineto{\pgfqpoint{2.137942in}{1.004871in}}%
\pgfpathlineto{\pgfqpoint{2.158609in}{1.017017in}}%
\pgfpathlineto{\pgfqpoint{2.179277in}{1.023474in}}%
\pgfpathlineto{\pgfqpoint{2.199944in}{1.038753in}}%
\pgfpathlineto{\pgfqpoint{2.220612in}{1.056312in}}%
\pgfpathlineto{\pgfqpoint{2.241279in}{1.115436in}}%
\pgfpathlineto{\pgfqpoint{2.261946in}{1.126399in}}%
\pgfpathlineto{\pgfqpoint{2.282614in}{1.132792in}}%
\pgfpathlineto{\pgfqpoint{2.303281in}{1.141369in}}%
\pgfpathlineto{\pgfqpoint{2.323949in}{1.146047in}}%
\pgfpathlineto{\pgfqpoint{2.344616in}{1.152962in}}%
\pgfpathlineto{\pgfqpoint{2.365283in}{1.167325in}}%
\pgfpathlineto{\pgfqpoint{2.385951in}{1.180387in}}%
\pgfpathlineto{\pgfqpoint{2.406618in}{1.189668in}}%
\pgfpathlineto{\pgfqpoint{2.427286in}{1.197659in}}%
\pgfpathlineto{\pgfqpoint{2.447953in}{1.212959in}}%
\pgfpathlineto{\pgfqpoint{2.468620in}{1.216315in}}%
\pgfpathlineto{\pgfqpoint{2.489288in}{1.233139in}}%
\pgfpathlineto{\pgfqpoint{2.509955in}{1.239340in}}%
\pgfpathlineto{\pgfqpoint{2.530623in}{1.241471in}}%
\pgfpathlineto{\pgfqpoint{2.551290in}{1.242750in}}%
\pgfpathlineto{\pgfqpoint{2.571958in}{1.273926in}}%
\pgfpathlineto{\pgfqpoint{2.592625in}{1.283739in}}%
\pgfpathlineto{\pgfqpoint{2.633960in}{1.320093in}}%
\pgfpathlineto{\pgfqpoint{2.654627in}{1.337429in}}%
\pgfpathlineto{\pgfqpoint{2.675295in}{1.351365in}}%
\pgfpathlineto{\pgfqpoint{2.695962in}{1.370000in}}%
\pgfpathlineto{\pgfqpoint{2.716629in}{1.370725in}}%
\pgfpathlineto{\pgfqpoint{2.737297in}{1.366026in}}%
\pgfpathlineto{\pgfqpoint{2.757964in}{1.370309in}}%
\pgfpathlineto{\pgfqpoint{2.840634in}{1.368328in}}%
\pgfpathlineto{\pgfqpoint{2.861301in}{1.368573in}}%
\pgfpathlineto{\pgfqpoint{2.943971in}{1.367347in}}%
\pgfpathlineto{\pgfqpoint{2.985306in}{1.367347in}}%
\pgfpathlineto{\pgfqpoint{3.005973in}{1.366570in}}%
\pgfpathlineto{\pgfqpoint{3.047308in}{1.366889in}}%
\pgfpathlineto{\pgfqpoint{3.171312in}{1.366356in}}%
\pgfpathlineto{\pgfqpoint{3.191980in}{1.369510in}}%
\pgfpathlineto{\pgfqpoint{3.212647in}{1.369468in}}%
\pgfpathlineto{\pgfqpoint{3.233314in}{1.368647in}}%
\pgfpathlineto{\pgfqpoint{3.253982in}{1.369627in}}%
\pgfpathlineto{\pgfqpoint{3.274649in}{1.369894in}}%
\pgfpathlineto{\pgfqpoint{3.295317in}{1.371364in}}%
\pgfpathlineto{\pgfqpoint{3.398654in}{1.371641in}}%
\pgfpathlineto{\pgfqpoint{3.419321in}{1.372430in}}%
\pgfpathlineto{\pgfqpoint{3.439988in}{1.372600in}}%
\pgfpathlineto{\pgfqpoint{3.460656in}{1.373591in}}%
\pgfpathlineto{\pgfqpoint{3.481323in}{1.373591in}}%
\pgfpathlineto{\pgfqpoint{3.543325in}{1.376500in}}%
\pgfpathlineto{\pgfqpoint{3.563993in}{1.377981in}}%
\pgfpathlineto{\pgfqpoint{3.584660in}{1.378119in}}%
\pgfpathlineto{\pgfqpoint{3.605328in}{1.376819in}}%
\pgfpathlineto{\pgfqpoint{3.625995in}{1.374774in}}%
\pgfpathlineto{\pgfqpoint{3.646662in}{1.373804in}}%
\pgfpathlineto{\pgfqpoint{3.667330in}{1.378119in}}%
\pgfpathlineto{\pgfqpoint{3.687997in}{1.377448in}}%
\pgfpathlineto{\pgfqpoint{3.708665in}{1.373591in}}%
\pgfpathlineto{\pgfqpoint{3.729332in}{1.380144in}}%
\pgfpathlineto{\pgfqpoint{3.750000in}{1.380336in}}%
\pgfpathlineto{\pgfqpoint{3.750000in}{1.380336in}}%
\pgfusepath{stroke}%
\end{pgfscope}%
\begin{pgfscope}%
\pgfsetrectcap%
\pgfsetmiterjoin%
\pgfsetlinewidth{0.803000pt}%
\definecolor{currentstroke}{rgb}{0.000000,0.000000,0.000000}%
\pgfsetstrokecolor{currentstroke}%
\pgfsetdash{}{0pt}%
\pgfpathmoveto{\pgfqpoint{0.511159in}{0.469412in}}%
\pgfpathlineto{\pgfqpoint{0.511159in}{1.423713in}}%
\pgfusepath{stroke}%
\end{pgfscope}%
\begin{pgfscope}%
\pgfsetrectcap%
\pgfsetmiterjoin%
\pgfsetlinewidth{0.803000pt}%
\definecolor{currentstroke}{rgb}{0.000000,0.000000,0.000000}%
\pgfsetstrokecolor{currentstroke}%
\pgfsetdash{}{0pt}%
\pgfpathmoveto{\pgfqpoint{3.904230in}{0.469412in}}%
\pgfpathlineto{\pgfqpoint{3.904230in}{1.423713in}}%
\pgfusepath{stroke}%
\end{pgfscope}%
\begin{pgfscope}%
\pgfsetrectcap%
\pgfsetmiterjoin%
\pgfsetlinewidth{0.803000pt}%
\definecolor{currentstroke}{rgb}{0.000000,0.000000,0.000000}%
\pgfsetstrokecolor{currentstroke}%
\pgfsetdash{}{0pt}%
\pgfpathmoveto{\pgfqpoint{0.511159in}{0.469412in}}%
\pgfpathlineto{\pgfqpoint{3.904230in}{0.469412in}}%
\pgfusepath{stroke}%
\end{pgfscope}%
\begin{pgfscope}%
\pgfsetrectcap%
\pgfsetmiterjoin%
\pgfsetlinewidth{0.803000pt}%
\definecolor{currentstroke}{rgb}{0.000000,0.000000,0.000000}%
\pgfsetstrokecolor{currentstroke}%
\pgfsetdash{}{0pt}%
\pgfpathmoveto{\pgfqpoint{0.511159in}{1.423713in}}%
\pgfpathlineto{\pgfqpoint{3.904230in}{1.423713in}}%
\pgfusepath{stroke}%
\end{pgfscope}%
\end{pgfpicture}%
\makeatother%
\endgroup%

    \caption{}
    \label{fig:decompression_batch}
\end{figure}

\begin{figure}
    \centering
    %% Creator: Matplotlib, PGF backend
%%
%% To include the figure in your LaTeX document, write
%%   \input{<filename>.pgf}
%%
%% Make sure the required packages are loaded in your preamble
%%   \usepackage{pgf}
%%
%% Figures using additional raster images can only be included by \input if
%% they are in the same directory as the main LaTeX file. For loading figures
%% from other directories you can use the `import` package
%%   \usepackage{import}
%% and then include the figures with
%%   \import{<path to file>}{<filename>.pgf}
%%
%% Matplotlib used the following preamble
%%   \usepackage{fontspec}
%%   \setmainfont{DejaVuSerif.ttf}[Path=/home/connor/.local/lib/python3.8/site-packages/matplotlib/mpl-data/fonts/ttf/]
%%   \setsansfont{DejaVuSans.ttf}[Path=/home/connor/.local/lib/python3.8/site-packages/matplotlib/mpl-data/fonts/ttf/]
%%   \setmonofont{DejaVuSansMono.ttf}[Path=/home/connor/.local/lib/python3.8/site-packages/matplotlib/mpl-data/fonts/ttf/]
%%
\begingroup%
\makeatletter%
\begin{pgfpicture}%
\pgfpathrectangle{\pgfpointorigin}{\pgfqpoint{4.004230in}{3.132284in}}%
\pgfusepath{use as bounding box, clip}%
\begin{pgfscope}%
\pgfsetbuttcap%
\pgfsetmiterjoin%
\definecolor{currentfill}{rgb}{1.000000,1.000000,1.000000}%
\pgfsetfillcolor{currentfill}%
\pgfsetlinewidth{0.000000pt}%
\definecolor{currentstroke}{rgb}{1.000000,1.000000,1.000000}%
\pgfsetstrokecolor{currentstroke}%
\pgfsetdash{}{0pt}%
\pgfpathmoveto{\pgfqpoint{0.000000in}{0.000000in}}%
\pgfpathlineto{\pgfqpoint{4.004230in}{0.000000in}}%
\pgfpathlineto{\pgfqpoint{4.004230in}{3.132284in}}%
\pgfpathlineto{\pgfqpoint{0.000000in}{3.132284in}}%
\pgfpathclose%
\pgfusepath{fill}%
\end{pgfscope}%
\begin{pgfscope}%
\pgfsetbuttcap%
\pgfsetmiterjoin%
\definecolor{currentfill}{rgb}{1.000000,1.000000,1.000000}%
\pgfsetfillcolor{currentfill}%
\pgfsetlinewidth{0.000000pt}%
\definecolor{currentstroke}{rgb}{0.000000,0.000000,0.000000}%
\pgfsetstrokecolor{currentstroke}%
\pgfsetstrokeopacity{0.000000}%
\pgfsetdash{}{0pt}%
\pgfpathmoveto{\pgfqpoint{0.511159in}{1.741813in}}%
\pgfpathlineto{\pgfqpoint{3.904230in}{1.741813in}}%
\pgfpathlineto{\pgfqpoint{3.904230in}{3.014215in}}%
\pgfpathlineto{\pgfqpoint{0.511159in}{3.014215in}}%
\pgfpathclose%
\pgfusepath{fill}%
\end{pgfscope}%
\begin{pgfscope}%
\pgfsetbuttcap%
\pgfsetroundjoin%
\definecolor{currentfill}{rgb}{0.000000,0.000000,0.000000}%
\pgfsetfillcolor{currentfill}%
\pgfsetlinewidth{0.803000pt}%
\definecolor{currentstroke}{rgb}{0.000000,0.000000,0.000000}%
\pgfsetstrokecolor{currentstroke}%
\pgfsetdash{}{0pt}%
\pgfsys@defobject{currentmarker}{\pgfqpoint{0.000000in}{-0.048611in}}{\pgfqpoint{0.000000in}{0.000000in}}{%
\pgfpathmoveto{\pgfqpoint{0.000000in}{0.000000in}}%
\pgfpathlineto{\pgfqpoint{0.000000in}{-0.048611in}}%
\pgfusepath{stroke,fill}%
}%
\begin{pgfscope}%
\pgfsys@transformshift{0.665390in}{1.741813in}%
\pgfsys@useobject{currentmarker}{}%
\end{pgfscope}%
\end{pgfscope}%
\begin{pgfscope}%
\definecolor{textcolor}{rgb}{0.000000,0.000000,0.000000}%
\pgfsetstrokecolor{textcolor}%
\pgfsetfillcolor{textcolor}%
\pgftext[x=0.665390in,y=1.644591in,,top]{\color{textcolor}\rmfamily\fontsize{8.000000}{9.600000}\selectfont \(\displaystyle 0\)}%
\end{pgfscope}%
\begin{pgfscope}%
\pgfsetbuttcap%
\pgfsetroundjoin%
\definecolor{currentfill}{rgb}{0.000000,0.000000,0.000000}%
\pgfsetfillcolor{currentfill}%
\pgfsetlinewidth{0.803000pt}%
\definecolor{currentstroke}{rgb}{0.000000,0.000000,0.000000}%
\pgfsetstrokecolor{currentstroke}%
\pgfsetdash{}{0pt}%
\pgfsys@defobject{currentmarker}{\pgfqpoint{0.000000in}{-0.048611in}}{\pgfqpoint{0.000000in}{0.000000in}}{%
\pgfpathmoveto{\pgfqpoint{0.000000in}{0.000000in}}%
\pgfpathlineto{\pgfqpoint{0.000000in}{-0.048611in}}%
\pgfusepath{stroke,fill}%
}%
\begin{pgfscope}%
\pgfsys@transformshift{1.078738in}{1.741813in}%
\pgfsys@useobject{currentmarker}{}%
\end{pgfscope}%
\end{pgfscope}%
\begin{pgfscope}%
\definecolor{textcolor}{rgb}{0.000000,0.000000,0.000000}%
\pgfsetstrokecolor{textcolor}%
\pgfsetfillcolor{textcolor}%
\pgftext[x=1.078738in,y=1.644591in,,top]{\color{textcolor}\rmfamily\fontsize{8.000000}{9.600000}\selectfont \(\displaystyle 20000\)}%
\end{pgfscope}%
\begin{pgfscope}%
\pgfsetbuttcap%
\pgfsetroundjoin%
\definecolor{currentfill}{rgb}{0.000000,0.000000,0.000000}%
\pgfsetfillcolor{currentfill}%
\pgfsetlinewidth{0.803000pt}%
\definecolor{currentstroke}{rgb}{0.000000,0.000000,0.000000}%
\pgfsetstrokecolor{currentstroke}%
\pgfsetdash{}{0pt}%
\pgfsys@defobject{currentmarker}{\pgfqpoint{0.000000in}{-0.048611in}}{\pgfqpoint{0.000000in}{0.000000in}}{%
\pgfpathmoveto{\pgfqpoint{0.000000in}{0.000000in}}%
\pgfpathlineto{\pgfqpoint{0.000000in}{-0.048611in}}%
\pgfusepath{stroke,fill}%
}%
\begin{pgfscope}%
\pgfsys@transformshift{1.492086in}{1.741813in}%
\pgfsys@useobject{currentmarker}{}%
\end{pgfscope}%
\end{pgfscope}%
\begin{pgfscope}%
\definecolor{textcolor}{rgb}{0.000000,0.000000,0.000000}%
\pgfsetstrokecolor{textcolor}%
\pgfsetfillcolor{textcolor}%
\pgftext[x=1.492086in,y=1.644591in,,top]{\color{textcolor}\rmfamily\fontsize{8.000000}{9.600000}\selectfont \(\displaystyle 40000\)}%
\end{pgfscope}%
\begin{pgfscope}%
\pgfsetbuttcap%
\pgfsetroundjoin%
\definecolor{currentfill}{rgb}{0.000000,0.000000,0.000000}%
\pgfsetfillcolor{currentfill}%
\pgfsetlinewidth{0.803000pt}%
\definecolor{currentstroke}{rgb}{0.000000,0.000000,0.000000}%
\pgfsetstrokecolor{currentstroke}%
\pgfsetdash{}{0pt}%
\pgfsys@defobject{currentmarker}{\pgfqpoint{0.000000in}{-0.048611in}}{\pgfqpoint{0.000000in}{0.000000in}}{%
\pgfpathmoveto{\pgfqpoint{0.000000in}{0.000000in}}%
\pgfpathlineto{\pgfqpoint{0.000000in}{-0.048611in}}%
\pgfusepath{stroke,fill}%
}%
\begin{pgfscope}%
\pgfsys@transformshift{1.905434in}{1.741813in}%
\pgfsys@useobject{currentmarker}{}%
\end{pgfscope}%
\end{pgfscope}%
\begin{pgfscope}%
\definecolor{textcolor}{rgb}{0.000000,0.000000,0.000000}%
\pgfsetstrokecolor{textcolor}%
\pgfsetfillcolor{textcolor}%
\pgftext[x=1.905434in,y=1.644591in,,top]{\color{textcolor}\rmfamily\fontsize{8.000000}{9.600000}\selectfont \(\displaystyle 60000\)}%
\end{pgfscope}%
\begin{pgfscope}%
\pgfsetbuttcap%
\pgfsetroundjoin%
\definecolor{currentfill}{rgb}{0.000000,0.000000,0.000000}%
\pgfsetfillcolor{currentfill}%
\pgfsetlinewidth{0.803000pt}%
\definecolor{currentstroke}{rgb}{0.000000,0.000000,0.000000}%
\pgfsetstrokecolor{currentstroke}%
\pgfsetdash{}{0pt}%
\pgfsys@defobject{currentmarker}{\pgfqpoint{0.000000in}{-0.048611in}}{\pgfqpoint{0.000000in}{0.000000in}}{%
\pgfpathmoveto{\pgfqpoint{0.000000in}{0.000000in}}%
\pgfpathlineto{\pgfqpoint{0.000000in}{-0.048611in}}%
\pgfusepath{stroke,fill}%
}%
\begin{pgfscope}%
\pgfsys@transformshift{2.318782in}{1.741813in}%
\pgfsys@useobject{currentmarker}{}%
\end{pgfscope}%
\end{pgfscope}%
\begin{pgfscope}%
\definecolor{textcolor}{rgb}{0.000000,0.000000,0.000000}%
\pgfsetstrokecolor{textcolor}%
\pgfsetfillcolor{textcolor}%
\pgftext[x=2.318782in,y=1.644591in,,top]{\color{textcolor}\rmfamily\fontsize{8.000000}{9.600000}\selectfont \(\displaystyle 80000\)}%
\end{pgfscope}%
\begin{pgfscope}%
\pgfsetbuttcap%
\pgfsetroundjoin%
\definecolor{currentfill}{rgb}{0.000000,0.000000,0.000000}%
\pgfsetfillcolor{currentfill}%
\pgfsetlinewidth{0.803000pt}%
\definecolor{currentstroke}{rgb}{0.000000,0.000000,0.000000}%
\pgfsetstrokecolor{currentstroke}%
\pgfsetdash{}{0pt}%
\pgfsys@defobject{currentmarker}{\pgfqpoint{0.000000in}{-0.048611in}}{\pgfqpoint{0.000000in}{0.000000in}}{%
\pgfpathmoveto{\pgfqpoint{0.000000in}{0.000000in}}%
\pgfpathlineto{\pgfqpoint{0.000000in}{-0.048611in}}%
\pgfusepath{stroke,fill}%
}%
\begin{pgfscope}%
\pgfsys@transformshift{2.732130in}{1.741813in}%
\pgfsys@useobject{currentmarker}{}%
\end{pgfscope}%
\end{pgfscope}%
\begin{pgfscope}%
\definecolor{textcolor}{rgb}{0.000000,0.000000,0.000000}%
\pgfsetstrokecolor{textcolor}%
\pgfsetfillcolor{textcolor}%
\pgftext[x=2.732130in,y=1.644591in,,top]{\color{textcolor}\rmfamily\fontsize{8.000000}{9.600000}\selectfont \(\displaystyle 100000\)}%
\end{pgfscope}%
\begin{pgfscope}%
\pgfsetbuttcap%
\pgfsetroundjoin%
\definecolor{currentfill}{rgb}{0.000000,0.000000,0.000000}%
\pgfsetfillcolor{currentfill}%
\pgfsetlinewidth{0.803000pt}%
\definecolor{currentstroke}{rgb}{0.000000,0.000000,0.000000}%
\pgfsetstrokecolor{currentstroke}%
\pgfsetdash{}{0pt}%
\pgfsys@defobject{currentmarker}{\pgfqpoint{0.000000in}{-0.048611in}}{\pgfqpoint{0.000000in}{0.000000in}}{%
\pgfpathmoveto{\pgfqpoint{0.000000in}{0.000000in}}%
\pgfpathlineto{\pgfqpoint{0.000000in}{-0.048611in}}%
\pgfusepath{stroke,fill}%
}%
\begin{pgfscope}%
\pgfsys@transformshift{3.145478in}{1.741813in}%
\pgfsys@useobject{currentmarker}{}%
\end{pgfscope}%
\end{pgfscope}%
\begin{pgfscope}%
\definecolor{textcolor}{rgb}{0.000000,0.000000,0.000000}%
\pgfsetstrokecolor{textcolor}%
\pgfsetfillcolor{textcolor}%
\pgftext[x=3.145478in,y=1.644591in,,top]{\color{textcolor}\rmfamily\fontsize{8.000000}{9.600000}\selectfont \(\displaystyle 120000\)}%
\end{pgfscope}%
\begin{pgfscope}%
\pgfsetbuttcap%
\pgfsetroundjoin%
\definecolor{currentfill}{rgb}{0.000000,0.000000,0.000000}%
\pgfsetfillcolor{currentfill}%
\pgfsetlinewidth{0.803000pt}%
\definecolor{currentstroke}{rgb}{0.000000,0.000000,0.000000}%
\pgfsetstrokecolor{currentstroke}%
\pgfsetdash{}{0pt}%
\pgfsys@defobject{currentmarker}{\pgfqpoint{0.000000in}{-0.048611in}}{\pgfqpoint{0.000000in}{0.000000in}}{%
\pgfpathmoveto{\pgfqpoint{0.000000in}{0.000000in}}%
\pgfpathlineto{\pgfqpoint{0.000000in}{-0.048611in}}%
\pgfusepath{stroke,fill}%
}%
\begin{pgfscope}%
\pgfsys@transformshift{3.558826in}{1.741813in}%
\pgfsys@useobject{currentmarker}{}%
\end{pgfscope}%
\end{pgfscope}%
\begin{pgfscope}%
\definecolor{textcolor}{rgb}{0.000000,0.000000,0.000000}%
\pgfsetstrokecolor{textcolor}%
\pgfsetfillcolor{textcolor}%
\pgftext[x=3.558826in,y=1.644591in,,top]{\color{textcolor}\rmfamily\fontsize{8.000000}{9.600000}\selectfont \(\displaystyle 140000\)}%
\end{pgfscope}%
\begin{pgfscope}%
\pgfsetbuttcap%
\pgfsetroundjoin%
\definecolor{currentfill}{rgb}{0.000000,0.000000,0.000000}%
\pgfsetfillcolor{currentfill}%
\pgfsetlinewidth{0.803000pt}%
\definecolor{currentstroke}{rgb}{0.000000,0.000000,0.000000}%
\pgfsetstrokecolor{currentstroke}%
\pgfsetdash{}{0pt}%
\pgfsys@defobject{currentmarker}{\pgfqpoint{-0.048611in}{0.000000in}}{\pgfqpoint{0.000000in}{0.000000in}}{%
\pgfpathmoveto{\pgfqpoint{0.000000in}{0.000000in}}%
\pgfpathlineto{\pgfqpoint{-0.048611in}{0.000000in}}%
\pgfusepath{stroke,fill}%
}%
\begin{pgfscope}%
\pgfsys@transformshift{0.511159in}{1.799650in}%
\pgfsys@useobject{currentmarker}{}%
\end{pgfscope}%
\end{pgfscope}%
\begin{pgfscope}%
\definecolor{textcolor}{rgb}{0.000000,0.000000,0.000000}%
\pgfsetstrokecolor{textcolor}%
\pgfsetfillcolor{textcolor}%
\pgftext[x=0.263086in,y=1.757440in,left,base]{\color{textcolor}\rmfamily\fontsize{8.000000}{9.600000}\selectfont \(\displaystyle 0.0\)}%
\end{pgfscope}%
\begin{pgfscope}%
\pgfsetbuttcap%
\pgfsetroundjoin%
\definecolor{currentfill}{rgb}{0.000000,0.000000,0.000000}%
\pgfsetfillcolor{currentfill}%
\pgfsetlinewidth{0.803000pt}%
\definecolor{currentstroke}{rgb}{0.000000,0.000000,0.000000}%
\pgfsetstrokecolor{currentstroke}%
\pgfsetdash{}{0pt}%
\pgfsys@defobject{currentmarker}{\pgfqpoint{-0.048611in}{0.000000in}}{\pgfqpoint{0.000000in}{0.000000in}}{%
\pgfpathmoveto{\pgfqpoint{0.000000in}{0.000000in}}%
\pgfpathlineto{\pgfqpoint{-0.048611in}{0.000000in}}%
\pgfusepath{stroke,fill}%
}%
\begin{pgfscope}%
\pgfsys@transformshift{0.511159in}{2.112838in}%
\pgfsys@useobject{currentmarker}{}%
\end{pgfscope}%
\end{pgfscope}%
\begin{pgfscope}%
\definecolor{textcolor}{rgb}{0.000000,0.000000,0.000000}%
\pgfsetstrokecolor{textcolor}%
\pgfsetfillcolor{textcolor}%
\pgftext[x=0.263086in,y=2.070629in,left,base]{\color{textcolor}\rmfamily\fontsize{8.000000}{9.600000}\selectfont \(\displaystyle 0.5\)}%
\end{pgfscope}%
\begin{pgfscope}%
\pgfsetbuttcap%
\pgfsetroundjoin%
\definecolor{currentfill}{rgb}{0.000000,0.000000,0.000000}%
\pgfsetfillcolor{currentfill}%
\pgfsetlinewidth{0.803000pt}%
\definecolor{currentstroke}{rgb}{0.000000,0.000000,0.000000}%
\pgfsetstrokecolor{currentstroke}%
\pgfsetdash{}{0pt}%
\pgfsys@defobject{currentmarker}{\pgfqpoint{-0.048611in}{0.000000in}}{\pgfqpoint{0.000000in}{0.000000in}}{%
\pgfpathmoveto{\pgfqpoint{0.000000in}{0.000000in}}%
\pgfpathlineto{\pgfqpoint{-0.048611in}{0.000000in}}%
\pgfusepath{stroke,fill}%
}%
\begin{pgfscope}%
\pgfsys@transformshift{0.511159in}{2.426026in}%
\pgfsys@useobject{currentmarker}{}%
\end{pgfscope}%
\end{pgfscope}%
\begin{pgfscope}%
\definecolor{textcolor}{rgb}{0.000000,0.000000,0.000000}%
\pgfsetstrokecolor{textcolor}%
\pgfsetfillcolor{textcolor}%
\pgftext[x=0.263086in,y=2.383817in,left,base]{\color{textcolor}\rmfamily\fontsize{8.000000}{9.600000}\selectfont \(\displaystyle 1.0\)}%
\end{pgfscope}%
\begin{pgfscope}%
\pgfsetbuttcap%
\pgfsetroundjoin%
\definecolor{currentfill}{rgb}{0.000000,0.000000,0.000000}%
\pgfsetfillcolor{currentfill}%
\pgfsetlinewidth{0.803000pt}%
\definecolor{currentstroke}{rgb}{0.000000,0.000000,0.000000}%
\pgfsetstrokecolor{currentstroke}%
\pgfsetdash{}{0pt}%
\pgfsys@defobject{currentmarker}{\pgfqpoint{-0.048611in}{0.000000in}}{\pgfqpoint{0.000000in}{0.000000in}}{%
\pgfpathmoveto{\pgfqpoint{0.000000in}{0.000000in}}%
\pgfpathlineto{\pgfqpoint{-0.048611in}{0.000000in}}%
\pgfusepath{stroke,fill}%
}%
\begin{pgfscope}%
\pgfsys@transformshift{0.511159in}{2.739214in}%
\pgfsys@useobject{currentmarker}{}%
\end{pgfscope}%
\end{pgfscope}%
\begin{pgfscope}%
\definecolor{textcolor}{rgb}{0.000000,0.000000,0.000000}%
\pgfsetstrokecolor{textcolor}%
\pgfsetfillcolor{textcolor}%
\pgftext[x=0.263086in,y=2.697005in,left,base]{\color{textcolor}\rmfamily\fontsize{8.000000}{9.600000}\selectfont \(\displaystyle 1.5\)}%
\end{pgfscope}%
\begin{pgfscope}%
\definecolor{textcolor}{rgb}{0.000000,0.000000,0.000000}%
\pgfsetstrokecolor{textcolor}%
\pgfsetfillcolor{textcolor}%
\pgftext[x=0.207530in,y=2.378014in,,bottom,rotate=90.000000]{\color{textcolor}\rmfamily\fontsize{8.000000}{9.600000}\selectfont Melt composition (mol)}%
\end{pgfscope}%
\begin{pgfscope}%
\pgfpathrectangle{\pgfqpoint{0.511159in}{1.741813in}}{\pgfqpoint{3.393071in}{1.272402in}}%
\pgfusepath{clip}%
\pgfsetrectcap%
\pgfsetroundjoin%
\pgfsetlinewidth{1.505625pt}%
\definecolor{currentstroke}{rgb}{0.121569,0.466667,0.705882}%
\pgfsetstrokecolor{currentstroke}%
\pgfsetdash{}{0pt}%
\pgfpathmoveto{\pgfqpoint{0.665390in}{1.799650in}}%
\pgfpathlineto{\pgfqpoint{2.365283in}{1.799650in}}%
\pgfpathlineto{\pgfqpoint{2.385951in}{2.125478in}}%
\pgfpathlineto{\pgfqpoint{2.571958in}{2.125478in}}%
\pgfpathlineto{\pgfqpoint{2.592625in}{2.122716in}}%
\pgfpathlineto{\pgfqpoint{2.861301in}{2.122716in}}%
\pgfpathlineto{\pgfqpoint{2.881969in}{2.112180in}}%
\pgfpathlineto{\pgfqpoint{3.212647in}{2.112180in}}%
\pgfpathlineto{\pgfqpoint{3.233314in}{2.096727in}}%
\pgfpathlineto{\pgfqpoint{3.563993in}{2.096727in}}%
\pgfpathlineto{\pgfqpoint{3.584660in}{2.077579in}}%
\pgfpathlineto{\pgfqpoint{3.750000in}{2.077579in}}%
\pgfpathlineto{\pgfqpoint{3.750000in}{2.077579in}}%
\pgfusepath{stroke}%
\end{pgfscope}%
\begin{pgfscope}%
\pgfpathrectangle{\pgfqpoint{0.511159in}{1.741813in}}{\pgfqpoint{3.393071in}{1.272402in}}%
\pgfusepath{clip}%
\pgfsetrectcap%
\pgfsetroundjoin%
\pgfsetlinewidth{1.505625pt}%
\definecolor{currentstroke}{rgb}{1.000000,0.498039,0.054902}%
\pgfsetstrokecolor{currentstroke}%
\pgfsetdash{}{0pt}%
\pgfpathmoveto{\pgfqpoint{0.665390in}{1.799650in}}%
\pgfpathlineto{\pgfqpoint{2.365283in}{1.799650in}}%
\pgfpathlineto{\pgfqpoint{2.385951in}{2.057378in}}%
\pgfpathlineto{\pgfqpoint{2.571958in}{2.057378in}}%
\pgfpathlineto{\pgfqpoint{2.592625in}{2.049480in}}%
\pgfpathlineto{\pgfqpoint{2.861301in}{2.049480in}}%
\pgfpathlineto{\pgfqpoint{2.881969in}{2.043961in}}%
\pgfpathlineto{\pgfqpoint{3.212647in}{2.043961in}}%
\pgfpathlineto{\pgfqpoint{3.233314in}{2.039508in}}%
\pgfpathlineto{\pgfqpoint{3.563993in}{2.039508in}}%
\pgfpathlineto{\pgfqpoint{3.584660in}{2.035875in}}%
\pgfpathlineto{\pgfqpoint{3.750000in}{2.035875in}}%
\pgfpathlineto{\pgfqpoint{3.750000in}{2.035875in}}%
\pgfusepath{stroke}%
\end{pgfscope}%
\begin{pgfscope}%
\pgfpathrectangle{\pgfqpoint{0.511159in}{1.741813in}}{\pgfqpoint{3.393071in}{1.272402in}}%
\pgfusepath{clip}%
\pgfsetrectcap%
\pgfsetroundjoin%
\pgfsetlinewidth{1.505625pt}%
\definecolor{currentstroke}{rgb}{0.172549,0.627451,0.172549}%
\pgfsetstrokecolor{currentstroke}%
\pgfsetdash{}{0pt}%
\pgfpathmoveto{\pgfqpoint{0.665390in}{1.799650in}}%
\pgfpathlineto{\pgfqpoint{2.365283in}{1.799650in}}%
\pgfpathlineto{\pgfqpoint{2.385951in}{2.558129in}}%
\pgfpathlineto{\pgfqpoint{2.571958in}{2.558129in}}%
\pgfpathlineto{\pgfqpoint{2.592625in}{2.556563in}}%
\pgfpathlineto{\pgfqpoint{2.861301in}{2.556563in}}%
\pgfpathlineto{\pgfqpoint{2.881969in}{2.566960in}}%
\pgfpathlineto{\pgfqpoint{3.212647in}{2.566960in}}%
\pgfpathlineto{\pgfqpoint{3.233314in}{2.580428in}}%
\pgfpathlineto{\pgfqpoint{3.563993in}{2.580428in}}%
\pgfpathlineto{\pgfqpoint{3.584660in}{2.593456in}}%
\pgfpathlineto{\pgfqpoint{3.750000in}{2.593456in}}%
\pgfpathlineto{\pgfqpoint{3.750000in}{2.593456in}}%
\pgfusepath{stroke}%
\end{pgfscope}%
\begin{pgfscope}%
\pgfpathrectangle{\pgfqpoint{0.511159in}{1.741813in}}{\pgfqpoint{3.393071in}{1.272402in}}%
\pgfusepath{clip}%
\pgfsetrectcap%
\pgfsetroundjoin%
\pgfsetlinewidth{1.505625pt}%
\definecolor{currentstroke}{rgb}{0.839216,0.152941,0.156863}%
\pgfsetstrokecolor{currentstroke}%
\pgfsetdash{}{0pt}%
\pgfpathmoveto{\pgfqpoint{0.665390in}{1.799650in}}%
\pgfpathlineto{\pgfqpoint{2.365283in}{1.799650in}}%
\pgfpathlineto{\pgfqpoint{2.385951in}{2.819578in}}%
\pgfpathlineto{\pgfqpoint{2.571958in}{2.819578in}}%
\pgfpathlineto{\pgfqpoint{2.592625in}{2.832607in}}%
\pgfpathlineto{\pgfqpoint{2.861301in}{2.832607in}}%
\pgfpathlineto{\pgfqpoint{2.881969in}{2.861796in}}%
\pgfpathlineto{\pgfqpoint{3.212647in}{2.861796in}}%
\pgfpathlineto{\pgfqpoint{3.233314in}{2.903638in}}%
\pgfpathlineto{\pgfqpoint{3.563993in}{2.903638in}}%
\pgfpathlineto{\pgfqpoint{3.584660in}{2.956378in}}%
\pgfpathlineto{\pgfqpoint{3.750000in}{2.956378in}}%
\pgfpathlineto{\pgfqpoint{3.750000in}{2.956378in}}%
\pgfusepath{stroke}%
\end{pgfscope}%
\begin{pgfscope}%
\pgfsetrectcap%
\pgfsetmiterjoin%
\pgfsetlinewidth{0.803000pt}%
\definecolor{currentstroke}{rgb}{0.000000,0.000000,0.000000}%
\pgfsetstrokecolor{currentstroke}%
\pgfsetdash{}{0pt}%
\pgfpathmoveto{\pgfqpoint{0.511159in}{1.741813in}}%
\pgfpathlineto{\pgfqpoint{0.511159in}{3.014215in}}%
\pgfusepath{stroke}%
\end{pgfscope}%
\begin{pgfscope}%
\pgfsetrectcap%
\pgfsetmiterjoin%
\pgfsetlinewidth{0.803000pt}%
\definecolor{currentstroke}{rgb}{0.000000,0.000000,0.000000}%
\pgfsetstrokecolor{currentstroke}%
\pgfsetdash{}{0pt}%
\pgfpathmoveto{\pgfqpoint{3.904230in}{1.741813in}}%
\pgfpathlineto{\pgfqpoint{3.904230in}{3.014215in}}%
\pgfusepath{stroke}%
\end{pgfscope}%
\begin{pgfscope}%
\pgfsetrectcap%
\pgfsetmiterjoin%
\pgfsetlinewidth{0.803000pt}%
\definecolor{currentstroke}{rgb}{0.000000,0.000000,0.000000}%
\pgfsetstrokecolor{currentstroke}%
\pgfsetdash{}{0pt}%
\pgfpathmoveto{\pgfqpoint{0.511159in}{1.741813in}}%
\pgfpathlineto{\pgfqpoint{3.904230in}{1.741813in}}%
\pgfusepath{stroke}%
\end{pgfscope}%
\begin{pgfscope}%
\pgfsetrectcap%
\pgfsetmiterjoin%
\pgfsetlinewidth{0.803000pt}%
\definecolor{currentstroke}{rgb}{0.000000,0.000000,0.000000}%
\pgfsetstrokecolor{currentstroke}%
\pgfsetdash{}{0pt}%
\pgfpathmoveto{\pgfqpoint{0.511159in}{3.014215in}}%
\pgfpathlineto{\pgfqpoint{3.904230in}{3.014215in}}%
\pgfusepath{stroke}%
\end{pgfscope}%
\begin{pgfscope}%
\pgfsetrectcap%
\pgfsetroundjoin%
\pgfsetlinewidth{1.505625pt}%
\definecolor{currentstroke}{rgb}{0.121569,0.466667,0.705882}%
\pgfsetstrokecolor{currentstroke}%
\pgfsetdash{}{0pt}%
\pgfpathmoveto{\pgfqpoint{0.611159in}{2.868685in}}%
\pgfpathlineto{\pgfqpoint{0.833381in}{2.868685in}}%
\pgfusepath{stroke}%
\end{pgfscope}%
\begin{pgfscope}%
\definecolor{textcolor}{rgb}{0.000000,0.000000,0.000000}%
\pgfsetstrokecolor{textcolor}%
\pgfsetfillcolor{textcolor}%
\pgftext[x=0.922270in,y=2.829796in,left,base]{\color{textcolor}\rmfamily\fontsize{8.000000}{9.600000}\selectfont CaO}%
\end{pgfscope}%
\begin{pgfscope}%
\pgfsetrectcap%
\pgfsetroundjoin%
\pgfsetlinewidth{1.505625pt}%
\definecolor{currentstroke}{rgb}{1.000000,0.498039,0.054902}%
\pgfsetstrokecolor{currentstroke}%
\pgfsetdash{}{0pt}%
\pgfpathmoveto{\pgfqpoint{0.611159in}{2.705599in}}%
\pgfpathlineto{\pgfqpoint{0.833381in}{2.705599in}}%
\pgfusepath{stroke}%
\end{pgfscope}%
\begin{pgfscope}%
\definecolor{textcolor}{rgb}{0.000000,0.000000,0.000000}%
\pgfsetstrokecolor{textcolor}%
\pgfsetfillcolor{textcolor}%
\pgftext[x=0.922270in,y=2.666711in,left,base]{\color{textcolor}\rmfamily\fontsize{8.000000}{9.600000}\selectfont FeO}%
\end{pgfscope}%
\begin{pgfscope}%
\pgfsetrectcap%
\pgfsetroundjoin%
\pgfsetlinewidth{1.505625pt}%
\definecolor{currentstroke}{rgb}{0.172549,0.627451,0.172549}%
\pgfsetstrokecolor{currentstroke}%
\pgfsetdash{}{0pt}%
\pgfpathmoveto{\pgfqpoint{0.611159in}{2.542514in}}%
\pgfpathlineto{\pgfqpoint{0.833381in}{2.542514in}}%
\pgfusepath{stroke}%
\end{pgfscope}%
\begin{pgfscope}%
\definecolor{textcolor}{rgb}{0.000000,0.000000,0.000000}%
\pgfsetstrokecolor{textcolor}%
\pgfsetfillcolor{textcolor}%
\pgftext[x=0.922270in,y=2.503625in,left,base]{\color{textcolor}\rmfamily\fontsize{8.000000}{9.600000}\selectfont MgO}%
\end{pgfscope}%
\begin{pgfscope}%
\pgfsetrectcap%
\pgfsetroundjoin%
\pgfsetlinewidth{1.505625pt}%
\definecolor{currentstroke}{rgb}{0.839216,0.152941,0.156863}%
\pgfsetstrokecolor{currentstroke}%
\pgfsetdash{}{0pt}%
\pgfpathmoveto{\pgfqpoint{0.611159in}{2.377854in}}%
\pgfpathlineto{\pgfqpoint{0.833381in}{2.377854in}}%
\pgfusepath{stroke}%
\end{pgfscope}%
\begin{pgfscope}%
\definecolor{textcolor}{rgb}{0.000000,0.000000,0.000000}%
\pgfsetstrokecolor{textcolor}%
\pgfsetfillcolor{textcolor}%
\pgftext[x=0.922270in,y=2.338966in,left,base]{\color{textcolor}\rmfamily\fontsize{8.000000}{9.600000}\selectfont SiO2}%
\end{pgfscope}%
\begin{pgfscope}%
\pgfsetbuttcap%
\pgfsetmiterjoin%
\definecolor{currentfill}{rgb}{1.000000,1.000000,1.000000}%
\pgfsetfillcolor{currentfill}%
\pgfsetlinewidth{0.000000pt}%
\definecolor{currentstroke}{rgb}{0.000000,0.000000,0.000000}%
\pgfsetstrokecolor{currentstroke}%
\pgfsetstrokeopacity{0.000000}%
\pgfsetdash{}{0pt}%
\pgfpathmoveto{\pgfqpoint{0.511159in}{0.469412in}}%
\pgfpathlineto{\pgfqpoint{3.904230in}{0.469412in}}%
\pgfpathlineto{\pgfqpoint{3.904230in}{1.423713in}}%
\pgfpathlineto{\pgfqpoint{0.511159in}{1.423713in}}%
\pgfpathclose%
\pgfusepath{fill}%
\end{pgfscope}%
\begin{pgfscope}%
\pgfsetbuttcap%
\pgfsetroundjoin%
\definecolor{currentfill}{rgb}{0.000000,0.000000,0.000000}%
\pgfsetfillcolor{currentfill}%
\pgfsetlinewidth{0.803000pt}%
\definecolor{currentstroke}{rgb}{0.000000,0.000000,0.000000}%
\pgfsetstrokecolor{currentstroke}%
\pgfsetdash{}{0pt}%
\pgfsys@defobject{currentmarker}{\pgfqpoint{0.000000in}{-0.048611in}}{\pgfqpoint{0.000000in}{0.000000in}}{%
\pgfpathmoveto{\pgfqpoint{0.000000in}{0.000000in}}%
\pgfpathlineto{\pgfqpoint{0.000000in}{-0.048611in}}%
\pgfusepath{stroke,fill}%
}%
\begin{pgfscope}%
\pgfsys@transformshift{0.665390in}{0.469412in}%
\pgfsys@useobject{currentmarker}{}%
\end{pgfscope}%
\end{pgfscope}%
\begin{pgfscope}%
\definecolor{textcolor}{rgb}{0.000000,0.000000,0.000000}%
\pgfsetstrokecolor{textcolor}%
\pgfsetfillcolor{textcolor}%
\pgftext[x=0.665390in,y=0.372189in,,top]{\color{textcolor}\rmfamily\fontsize{8.000000}{9.600000}\selectfont \(\displaystyle 0\)}%
\end{pgfscope}%
\begin{pgfscope}%
\pgfsetbuttcap%
\pgfsetroundjoin%
\definecolor{currentfill}{rgb}{0.000000,0.000000,0.000000}%
\pgfsetfillcolor{currentfill}%
\pgfsetlinewidth{0.803000pt}%
\definecolor{currentstroke}{rgb}{0.000000,0.000000,0.000000}%
\pgfsetstrokecolor{currentstroke}%
\pgfsetdash{}{0pt}%
\pgfsys@defobject{currentmarker}{\pgfqpoint{0.000000in}{-0.048611in}}{\pgfqpoint{0.000000in}{0.000000in}}{%
\pgfpathmoveto{\pgfqpoint{0.000000in}{0.000000in}}%
\pgfpathlineto{\pgfqpoint{0.000000in}{-0.048611in}}%
\pgfusepath{stroke,fill}%
}%
\begin{pgfscope}%
\pgfsys@transformshift{1.078738in}{0.469412in}%
\pgfsys@useobject{currentmarker}{}%
\end{pgfscope}%
\end{pgfscope}%
\begin{pgfscope}%
\definecolor{textcolor}{rgb}{0.000000,0.000000,0.000000}%
\pgfsetstrokecolor{textcolor}%
\pgfsetfillcolor{textcolor}%
\pgftext[x=1.078738in,y=0.372189in,,top]{\color{textcolor}\rmfamily\fontsize{8.000000}{9.600000}\selectfont \(\displaystyle 20000\)}%
\end{pgfscope}%
\begin{pgfscope}%
\pgfsetbuttcap%
\pgfsetroundjoin%
\definecolor{currentfill}{rgb}{0.000000,0.000000,0.000000}%
\pgfsetfillcolor{currentfill}%
\pgfsetlinewidth{0.803000pt}%
\definecolor{currentstroke}{rgb}{0.000000,0.000000,0.000000}%
\pgfsetstrokecolor{currentstroke}%
\pgfsetdash{}{0pt}%
\pgfsys@defobject{currentmarker}{\pgfqpoint{0.000000in}{-0.048611in}}{\pgfqpoint{0.000000in}{0.000000in}}{%
\pgfpathmoveto{\pgfqpoint{0.000000in}{0.000000in}}%
\pgfpathlineto{\pgfqpoint{0.000000in}{-0.048611in}}%
\pgfusepath{stroke,fill}%
}%
\begin{pgfscope}%
\pgfsys@transformshift{1.492086in}{0.469412in}%
\pgfsys@useobject{currentmarker}{}%
\end{pgfscope}%
\end{pgfscope}%
\begin{pgfscope}%
\definecolor{textcolor}{rgb}{0.000000,0.000000,0.000000}%
\pgfsetstrokecolor{textcolor}%
\pgfsetfillcolor{textcolor}%
\pgftext[x=1.492086in,y=0.372189in,,top]{\color{textcolor}\rmfamily\fontsize{8.000000}{9.600000}\selectfont \(\displaystyle 40000\)}%
\end{pgfscope}%
\begin{pgfscope}%
\pgfsetbuttcap%
\pgfsetroundjoin%
\definecolor{currentfill}{rgb}{0.000000,0.000000,0.000000}%
\pgfsetfillcolor{currentfill}%
\pgfsetlinewidth{0.803000pt}%
\definecolor{currentstroke}{rgb}{0.000000,0.000000,0.000000}%
\pgfsetstrokecolor{currentstroke}%
\pgfsetdash{}{0pt}%
\pgfsys@defobject{currentmarker}{\pgfqpoint{0.000000in}{-0.048611in}}{\pgfqpoint{0.000000in}{0.000000in}}{%
\pgfpathmoveto{\pgfqpoint{0.000000in}{0.000000in}}%
\pgfpathlineto{\pgfqpoint{0.000000in}{-0.048611in}}%
\pgfusepath{stroke,fill}%
}%
\begin{pgfscope}%
\pgfsys@transformshift{1.905434in}{0.469412in}%
\pgfsys@useobject{currentmarker}{}%
\end{pgfscope}%
\end{pgfscope}%
\begin{pgfscope}%
\definecolor{textcolor}{rgb}{0.000000,0.000000,0.000000}%
\pgfsetstrokecolor{textcolor}%
\pgfsetfillcolor{textcolor}%
\pgftext[x=1.905434in,y=0.372189in,,top]{\color{textcolor}\rmfamily\fontsize{8.000000}{9.600000}\selectfont \(\displaystyle 60000\)}%
\end{pgfscope}%
\begin{pgfscope}%
\pgfsetbuttcap%
\pgfsetroundjoin%
\definecolor{currentfill}{rgb}{0.000000,0.000000,0.000000}%
\pgfsetfillcolor{currentfill}%
\pgfsetlinewidth{0.803000pt}%
\definecolor{currentstroke}{rgb}{0.000000,0.000000,0.000000}%
\pgfsetstrokecolor{currentstroke}%
\pgfsetdash{}{0pt}%
\pgfsys@defobject{currentmarker}{\pgfqpoint{0.000000in}{-0.048611in}}{\pgfqpoint{0.000000in}{0.000000in}}{%
\pgfpathmoveto{\pgfqpoint{0.000000in}{0.000000in}}%
\pgfpathlineto{\pgfqpoint{0.000000in}{-0.048611in}}%
\pgfusepath{stroke,fill}%
}%
\begin{pgfscope}%
\pgfsys@transformshift{2.318782in}{0.469412in}%
\pgfsys@useobject{currentmarker}{}%
\end{pgfscope}%
\end{pgfscope}%
\begin{pgfscope}%
\definecolor{textcolor}{rgb}{0.000000,0.000000,0.000000}%
\pgfsetstrokecolor{textcolor}%
\pgfsetfillcolor{textcolor}%
\pgftext[x=2.318782in,y=0.372189in,,top]{\color{textcolor}\rmfamily\fontsize{8.000000}{9.600000}\selectfont \(\displaystyle 80000\)}%
\end{pgfscope}%
\begin{pgfscope}%
\pgfsetbuttcap%
\pgfsetroundjoin%
\definecolor{currentfill}{rgb}{0.000000,0.000000,0.000000}%
\pgfsetfillcolor{currentfill}%
\pgfsetlinewidth{0.803000pt}%
\definecolor{currentstroke}{rgb}{0.000000,0.000000,0.000000}%
\pgfsetstrokecolor{currentstroke}%
\pgfsetdash{}{0pt}%
\pgfsys@defobject{currentmarker}{\pgfqpoint{0.000000in}{-0.048611in}}{\pgfqpoint{0.000000in}{0.000000in}}{%
\pgfpathmoveto{\pgfqpoint{0.000000in}{0.000000in}}%
\pgfpathlineto{\pgfqpoint{0.000000in}{-0.048611in}}%
\pgfusepath{stroke,fill}%
}%
\begin{pgfscope}%
\pgfsys@transformshift{2.732130in}{0.469412in}%
\pgfsys@useobject{currentmarker}{}%
\end{pgfscope}%
\end{pgfscope}%
\begin{pgfscope}%
\definecolor{textcolor}{rgb}{0.000000,0.000000,0.000000}%
\pgfsetstrokecolor{textcolor}%
\pgfsetfillcolor{textcolor}%
\pgftext[x=2.732130in,y=0.372189in,,top]{\color{textcolor}\rmfamily\fontsize{8.000000}{9.600000}\selectfont \(\displaystyle 100000\)}%
\end{pgfscope}%
\begin{pgfscope}%
\pgfsetbuttcap%
\pgfsetroundjoin%
\definecolor{currentfill}{rgb}{0.000000,0.000000,0.000000}%
\pgfsetfillcolor{currentfill}%
\pgfsetlinewidth{0.803000pt}%
\definecolor{currentstroke}{rgb}{0.000000,0.000000,0.000000}%
\pgfsetstrokecolor{currentstroke}%
\pgfsetdash{}{0pt}%
\pgfsys@defobject{currentmarker}{\pgfqpoint{0.000000in}{-0.048611in}}{\pgfqpoint{0.000000in}{0.000000in}}{%
\pgfpathmoveto{\pgfqpoint{0.000000in}{0.000000in}}%
\pgfpathlineto{\pgfqpoint{0.000000in}{-0.048611in}}%
\pgfusepath{stroke,fill}%
}%
\begin{pgfscope}%
\pgfsys@transformshift{3.145478in}{0.469412in}%
\pgfsys@useobject{currentmarker}{}%
\end{pgfscope}%
\end{pgfscope}%
\begin{pgfscope}%
\definecolor{textcolor}{rgb}{0.000000,0.000000,0.000000}%
\pgfsetstrokecolor{textcolor}%
\pgfsetfillcolor{textcolor}%
\pgftext[x=3.145478in,y=0.372189in,,top]{\color{textcolor}\rmfamily\fontsize{8.000000}{9.600000}\selectfont \(\displaystyle 120000\)}%
\end{pgfscope}%
\begin{pgfscope}%
\pgfsetbuttcap%
\pgfsetroundjoin%
\definecolor{currentfill}{rgb}{0.000000,0.000000,0.000000}%
\pgfsetfillcolor{currentfill}%
\pgfsetlinewidth{0.803000pt}%
\definecolor{currentstroke}{rgb}{0.000000,0.000000,0.000000}%
\pgfsetstrokecolor{currentstroke}%
\pgfsetdash{}{0pt}%
\pgfsys@defobject{currentmarker}{\pgfqpoint{0.000000in}{-0.048611in}}{\pgfqpoint{0.000000in}{0.000000in}}{%
\pgfpathmoveto{\pgfqpoint{0.000000in}{0.000000in}}%
\pgfpathlineto{\pgfqpoint{0.000000in}{-0.048611in}}%
\pgfusepath{stroke,fill}%
}%
\begin{pgfscope}%
\pgfsys@transformshift{3.558826in}{0.469412in}%
\pgfsys@useobject{currentmarker}{}%
\end{pgfscope}%
\end{pgfscope}%
\begin{pgfscope}%
\definecolor{textcolor}{rgb}{0.000000,0.000000,0.000000}%
\pgfsetstrokecolor{textcolor}%
\pgfsetfillcolor{textcolor}%
\pgftext[x=3.558826in,y=0.372189in,,top]{\color{textcolor}\rmfamily\fontsize{8.000000}{9.600000}\selectfont \(\displaystyle 140000\)}%
\end{pgfscope}%
\begin{pgfscope}%
\definecolor{textcolor}{rgb}{0.000000,0.000000,0.000000}%
\pgfsetstrokecolor{textcolor}%
\pgfsetfillcolor{textcolor}%
\pgftext[x=2.207695in,y=0.209104in,,top]{\color{textcolor}\rmfamily\fontsize{8.000000}{9.600000}\selectfont Time (yr)}%
\end{pgfscope}%
\begin{pgfscope}%
\pgfsetbuttcap%
\pgfsetroundjoin%
\definecolor{currentfill}{rgb}{0.000000,0.000000,0.000000}%
\pgfsetfillcolor{currentfill}%
\pgfsetlinewidth{0.803000pt}%
\definecolor{currentstroke}{rgb}{0.000000,0.000000,0.000000}%
\pgfsetstrokecolor{currentstroke}%
\pgfsetdash{}{0pt}%
\pgfsys@defobject{currentmarker}{\pgfqpoint{-0.048611in}{0.000000in}}{\pgfqpoint{0.000000in}{0.000000in}}{%
\pgfpathmoveto{\pgfqpoint{0.000000in}{0.000000in}}%
\pgfpathlineto{\pgfqpoint{-0.048611in}{0.000000in}}%
\pgfusepath{stroke,fill}%
}%
\begin{pgfscope}%
\pgfsys@transformshift{0.511159in}{0.512789in}%
\pgfsys@useobject{currentmarker}{}%
\end{pgfscope}%
\end{pgfscope}%
\begin{pgfscope}%
\definecolor{textcolor}{rgb}{0.000000,0.000000,0.000000}%
\pgfsetstrokecolor{textcolor}%
\pgfsetfillcolor{textcolor}%
\pgftext[x=0.354908in,y=0.470580in,left,base]{\color{textcolor}\rmfamily\fontsize{8.000000}{9.600000}\selectfont \(\displaystyle 0\)}%
\end{pgfscope}%
\begin{pgfscope}%
\pgfsetbuttcap%
\pgfsetroundjoin%
\definecolor{currentfill}{rgb}{0.000000,0.000000,0.000000}%
\pgfsetfillcolor{currentfill}%
\pgfsetlinewidth{0.803000pt}%
\definecolor{currentstroke}{rgb}{0.000000,0.000000,0.000000}%
\pgfsetstrokecolor{currentstroke}%
\pgfsetdash{}{0pt}%
\pgfsys@defobject{currentmarker}{\pgfqpoint{-0.048611in}{0.000000in}}{\pgfqpoint{0.000000in}{0.000000in}}{%
\pgfpathmoveto{\pgfqpoint{0.000000in}{0.000000in}}%
\pgfpathlineto{\pgfqpoint{-0.048611in}{0.000000in}}%
\pgfusepath{stroke,fill}%
}%
\begin{pgfscope}%
\pgfsys@transformshift{0.511159in}{0.805035in}%
\pgfsys@useobject{currentmarker}{}%
\end{pgfscope}%
\end{pgfscope}%
\begin{pgfscope}%
\definecolor{textcolor}{rgb}{0.000000,0.000000,0.000000}%
\pgfsetstrokecolor{textcolor}%
\pgfsetfillcolor{textcolor}%
\pgftext[x=0.354908in,y=0.762826in,left,base]{\color{textcolor}\rmfamily\fontsize{8.000000}{9.600000}\selectfont \(\displaystyle 2\)}%
\end{pgfscope}%
\begin{pgfscope}%
\pgfsetbuttcap%
\pgfsetroundjoin%
\definecolor{currentfill}{rgb}{0.000000,0.000000,0.000000}%
\pgfsetfillcolor{currentfill}%
\pgfsetlinewidth{0.803000pt}%
\definecolor{currentstroke}{rgb}{0.000000,0.000000,0.000000}%
\pgfsetstrokecolor{currentstroke}%
\pgfsetdash{}{0pt}%
\pgfsys@defobject{currentmarker}{\pgfqpoint{-0.048611in}{0.000000in}}{\pgfqpoint{0.000000in}{0.000000in}}{%
\pgfpathmoveto{\pgfqpoint{0.000000in}{0.000000in}}%
\pgfpathlineto{\pgfqpoint{-0.048611in}{0.000000in}}%
\pgfusepath{stroke,fill}%
}%
\begin{pgfscope}%
\pgfsys@transformshift{0.511159in}{1.097281in}%
\pgfsys@useobject{currentmarker}{}%
\end{pgfscope}%
\end{pgfscope}%
\begin{pgfscope}%
\definecolor{textcolor}{rgb}{0.000000,0.000000,0.000000}%
\pgfsetstrokecolor{textcolor}%
\pgfsetfillcolor{textcolor}%
\pgftext[x=0.354908in,y=1.055072in,left,base]{\color{textcolor}\rmfamily\fontsize{8.000000}{9.600000}\selectfont \(\displaystyle 4\)}%
\end{pgfscope}%
\begin{pgfscope}%
\pgfsetbuttcap%
\pgfsetroundjoin%
\definecolor{currentfill}{rgb}{0.000000,0.000000,0.000000}%
\pgfsetfillcolor{currentfill}%
\pgfsetlinewidth{0.803000pt}%
\definecolor{currentstroke}{rgb}{0.000000,0.000000,0.000000}%
\pgfsetstrokecolor{currentstroke}%
\pgfsetdash{}{0pt}%
\pgfsys@defobject{currentmarker}{\pgfqpoint{-0.048611in}{0.000000in}}{\pgfqpoint{0.000000in}{0.000000in}}{%
\pgfpathmoveto{\pgfqpoint{0.000000in}{0.000000in}}%
\pgfpathlineto{\pgfqpoint{-0.048611in}{0.000000in}}%
\pgfusepath{stroke,fill}%
}%
\begin{pgfscope}%
\pgfsys@transformshift{0.511159in}{1.389527in}%
\pgfsys@useobject{currentmarker}{}%
\end{pgfscope}%
\end{pgfscope}%
\begin{pgfscope}%
\definecolor{textcolor}{rgb}{0.000000,0.000000,0.000000}%
\pgfsetstrokecolor{textcolor}%
\pgfsetfillcolor{textcolor}%
\pgftext[x=0.354908in,y=1.347317in,left,base]{\color{textcolor}\rmfamily\fontsize{8.000000}{9.600000}\selectfont \(\displaystyle 6\)}%
\end{pgfscope}%
\begin{pgfscope}%
\definecolor{textcolor}{rgb}{0.000000,0.000000,0.000000}%
\pgfsetstrokecolor{textcolor}%
\pgfsetfillcolor{textcolor}%
\pgftext[x=0.299353in,y=0.946562in,,bottom,rotate=90.000000]{\color{textcolor}\rmfamily\fontsize{8.000000}{9.600000}\selectfont Melt amount (mol)}%
\end{pgfscope}%
\begin{pgfscope}%
\pgfpathrectangle{\pgfqpoint{0.511159in}{0.469412in}}{\pgfqpoint{3.393071in}{0.954301in}}%
\pgfusepath{clip}%
\pgfsetrectcap%
\pgfsetroundjoin%
\pgfsetlinewidth{1.505625pt}%
\definecolor{currentstroke}{rgb}{0.121569,0.466667,0.705882}%
\pgfsetstrokecolor{currentstroke}%
\pgfsetdash{}{0pt}%
\pgfpathmoveto{\pgfqpoint{0.665390in}{0.512789in}}%
\pgfpathlineto{\pgfqpoint{2.365283in}{0.512789in}}%
\pgfpathlineto{\pgfqpoint{2.385951in}{0.719042in}}%
\pgfpathlineto{\pgfqpoint{2.571958in}{0.719042in}}%
\pgfpathlineto{\pgfqpoint{2.592625in}{0.912728in}}%
\pgfpathlineto{\pgfqpoint{2.861301in}{0.912728in}}%
\pgfpathlineto{\pgfqpoint{2.881969in}{1.086716in}}%
\pgfpathlineto{\pgfqpoint{3.212647in}{1.086716in}}%
\pgfpathlineto{\pgfqpoint{3.233314in}{1.242629in}}%
\pgfpathlineto{\pgfqpoint{3.563993in}{1.242629in}}%
\pgfpathlineto{\pgfqpoint{3.584660in}{1.380336in}}%
\pgfpathlineto{\pgfqpoint{3.750000in}{1.380336in}}%
\pgfpathlineto{\pgfqpoint{3.750000in}{1.380336in}}%
\pgfusepath{stroke}%
\end{pgfscope}%
\begin{pgfscope}%
\pgfsetrectcap%
\pgfsetmiterjoin%
\pgfsetlinewidth{0.803000pt}%
\definecolor{currentstroke}{rgb}{0.000000,0.000000,0.000000}%
\pgfsetstrokecolor{currentstroke}%
\pgfsetdash{}{0pt}%
\pgfpathmoveto{\pgfqpoint{0.511159in}{0.469412in}}%
\pgfpathlineto{\pgfqpoint{0.511159in}{1.423713in}}%
\pgfusepath{stroke}%
\end{pgfscope}%
\begin{pgfscope}%
\pgfsetrectcap%
\pgfsetmiterjoin%
\pgfsetlinewidth{0.803000pt}%
\definecolor{currentstroke}{rgb}{0.000000,0.000000,0.000000}%
\pgfsetstrokecolor{currentstroke}%
\pgfsetdash{}{0pt}%
\pgfpathmoveto{\pgfqpoint{3.904230in}{0.469412in}}%
\pgfpathlineto{\pgfqpoint{3.904230in}{1.423713in}}%
\pgfusepath{stroke}%
\end{pgfscope}%
\begin{pgfscope}%
\pgfsetrectcap%
\pgfsetmiterjoin%
\pgfsetlinewidth{0.803000pt}%
\definecolor{currentstroke}{rgb}{0.000000,0.000000,0.000000}%
\pgfsetstrokecolor{currentstroke}%
\pgfsetdash{}{0pt}%
\pgfpathmoveto{\pgfqpoint{0.511159in}{0.469412in}}%
\pgfpathlineto{\pgfqpoint{3.904230in}{0.469412in}}%
\pgfusepath{stroke}%
\end{pgfscope}%
\begin{pgfscope}%
\pgfsetrectcap%
\pgfsetmiterjoin%
\pgfsetlinewidth{0.803000pt}%
\definecolor{currentstroke}{rgb}{0.000000,0.000000,0.000000}%
\pgfsetstrokecolor{currentstroke}%
\pgfsetdash{}{0pt}%
\pgfpathmoveto{\pgfqpoint{0.511159in}{1.423713in}}%
\pgfpathlineto{\pgfqpoint{3.904230in}{1.423713in}}%
\pgfusepath{stroke}%
\end{pgfscope}%
\end{pgfpicture}%
\makeatother%
\endgroup%

    \caption{Caption}
    \label{fig:decompression_frac}
\end{figure}

\begin{figure}
    \centering
    %% Creator: Matplotlib, PGF backend
%%
%% To include the figure in your LaTeX document, write
%%   \input{<filename>.pgf}
%%
%% Make sure the required packages are loaded in your preamble
%%   \usepackage{pgf}
%%
%% Figures using additional raster images can only be included by \input if
%% they are in the same directory as the main LaTeX file. For loading figures
%% from other directories you can use the `import` package
%%   \usepackage{import}
%% and then include the figures with
%%   \import{<path to file>}{<filename>.pgf}
%%
%% Matplotlib used the following preamble
%%   \usepackage{fontspec}
%%   \setmainfont{DejaVuSerif.ttf}[Path=/home/connor/.local/lib/python3.8/site-packages/matplotlib/mpl-data/fonts/ttf/]
%%   \setsansfont{DejaVuSans.ttf}[Path=/home/connor/.local/lib/python3.8/site-packages/matplotlib/mpl-data/fonts/ttf/]
%%   \setmonofont{DejaVuSansMono.ttf}[Path=/home/connor/.local/lib/python3.8/site-packages/matplotlib/mpl-data/fonts/ttf/]
%%
\begingroup%
\makeatletter%
\begin{pgfpicture}%
\pgfpathrectangle{\pgfpointorigin}{\pgfqpoint{4.063259in}{3.132284in}}%
\pgfusepath{use as bounding box, clip}%
\begin{pgfscope}%
\pgfsetbuttcap%
\pgfsetmiterjoin%
\definecolor{currentfill}{rgb}{1.000000,1.000000,1.000000}%
\pgfsetfillcolor{currentfill}%
\pgfsetlinewidth{0.000000pt}%
\definecolor{currentstroke}{rgb}{1.000000,1.000000,1.000000}%
\pgfsetstrokecolor{currentstroke}%
\pgfsetdash{}{0pt}%
\pgfpathmoveto{\pgfqpoint{0.000000in}{0.000000in}}%
\pgfpathlineto{\pgfqpoint{4.063259in}{0.000000in}}%
\pgfpathlineto{\pgfqpoint{4.063259in}{3.132284in}}%
\pgfpathlineto{\pgfqpoint{0.000000in}{3.132284in}}%
\pgfpathclose%
\pgfusepath{fill}%
\end{pgfscope}%
\begin{pgfscope}%
\pgfsetbuttcap%
\pgfsetmiterjoin%
\definecolor{currentfill}{rgb}{1.000000,1.000000,1.000000}%
\pgfsetfillcolor{currentfill}%
\pgfsetlinewidth{0.000000pt}%
\definecolor{currentstroke}{rgb}{0.000000,0.000000,0.000000}%
\pgfsetstrokecolor{currentstroke}%
\pgfsetstrokeopacity{0.000000}%
\pgfsetdash{}{0pt}%
\pgfpathmoveto{\pgfqpoint{0.570188in}{1.741813in}}%
\pgfpathlineto{\pgfqpoint{3.963259in}{1.741813in}}%
\pgfpathlineto{\pgfqpoint{3.963259in}{3.014215in}}%
\pgfpathlineto{\pgfqpoint{0.570188in}{3.014215in}}%
\pgfpathclose%
\pgfusepath{fill}%
\end{pgfscope}%
\begin{pgfscope}%
\pgfsetbuttcap%
\pgfsetroundjoin%
\definecolor{currentfill}{rgb}{0.000000,0.000000,0.000000}%
\pgfsetfillcolor{currentfill}%
\pgfsetlinewidth{0.803000pt}%
\definecolor{currentstroke}{rgb}{0.000000,0.000000,0.000000}%
\pgfsetstrokecolor{currentstroke}%
\pgfsetdash{}{0pt}%
\pgfsys@defobject{currentmarker}{\pgfqpoint{0.000000in}{-0.048611in}}{\pgfqpoint{0.000000in}{0.000000in}}{%
\pgfpathmoveto{\pgfqpoint{0.000000in}{0.000000in}}%
\pgfpathlineto{\pgfqpoint{0.000000in}{-0.048611in}}%
\pgfusepath{stroke,fill}%
}%
\begin{pgfscope}%
\pgfsys@transformshift{0.724418in}{1.741813in}%
\pgfsys@useobject{currentmarker}{}%
\end{pgfscope}%
\end{pgfscope}%
\begin{pgfscope}%
\definecolor{textcolor}{rgb}{0.000000,0.000000,0.000000}%
\pgfsetstrokecolor{textcolor}%
\pgfsetfillcolor{textcolor}%
\pgftext[x=0.724418in,y=1.644591in,,top]{\color{textcolor}\rmfamily\fontsize{8.000000}{9.600000}\selectfont \(\displaystyle 0\)}%
\end{pgfscope}%
\begin{pgfscope}%
\pgfsetbuttcap%
\pgfsetroundjoin%
\definecolor{currentfill}{rgb}{0.000000,0.000000,0.000000}%
\pgfsetfillcolor{currentfill}%
\pgfsetlinewidth{0.803000pt}%
\definecolor{currentstroke}{rgb}{0.000000,0.000000,0.000000}%
\pgfsetstrokecolor{currentstroke}%
\pgfsetdash{}{0pt}%
\pgfsys@defobject{currentmarker}{\pgfqpoint{0.000000in}{-0.048611in}}{\pgfqpoint{0.000000in}{0.000000in}}{%
\pgfpathmoveto{\pgfqpoint{0.000000in}{0.000000in}}%
\pgfpathlineto{\pgfqpoint{0.000000in}{-0.048611in}}%
\pgfusepath{stroke,fill}%
}%
\begin{pgfscope}%
\pgfsys@transformshift{1.138071in}{1.741813in}%
\pgfsys@useobject{currentmarker}{}%
\end{pgfscope}%
\end{pgfscope}%
\begin{pgfscope}%
\definecolor{textcolor}{rgb}{0.000000,0.000000,0.000000}%
\pgfsetstrokecolor{textcolor}%
\pgfsetfillcolor{textcolor}%
\pgftext[x=1.138071in,y=1.644591in,,top]{\color{textcolor}\rmfamily\fontsize{8.000000}{9.600000}\selectfont \(\displaystyle 20000\)}%
\end{pgfscope}%
\begin{pgfscope}%
\pgfsetbuttcap%
\pgfsetroundjoin%
\definecolor{currentfill}{rgb}{0.000000,0.000000,0.000000}%
\pgfsetfillcolor{currentfill}%
\pgfsetlinewidth{0.803000pt}%
\definecolor{currentstroke}{rgb}{0.000000,0.000000,0.000000}%
\pgfsetstrokecolor{currentstroke}%
\pgfsetdash{}{0pt}%
\pgfsys@defobject{currentmarker}{\pgfqpoint{0.000000in}{-0.048611in}}{\pgfqpoint{0.000000in}{0.000000in}}{%
\pgfpathmoveto{\pgfqpoint{0.000000in}{0.000000in}}%
\pgfpathlineto{\pgfqpoint{0.000000in}{-0.048611in}}%
\pgfusepath{stroke,fill}%
}%
\begin{pgfscope}%
\pgfsys@transformshift{1.551724in}{1.741813in}%
\pgfsys@useobject{currentmarker}{}%
\end{pgfscope}%
\end{pgfscope}%
\begin{pgfscope}%
\definecolor{textcolor}{rgb}{0.000000,0.000000,0.000000}%
\pgfsetstrokecolor{textcolor}%
\pgfsetfillcolor{textcolor}%
\pgftext[x=1.551724in,y=1.644591in,,top]{\color{textcolor}\rmfamily\fontsize{8.000000}{9.600000}\selectfont \(\displaystyle 40000\)}%
\end{pgfscope}%
\begin{pgfscope}%
\pgfsetbuttcap%
\pgfsetroundjoin%
\definecolor{currentfill}{rgb}{0.000000,0.000000,0.000000}%
\pgfsetfillcolor{currentfill}%
\pgfsetlinewidth{0.803000pt}%
\definecolor{currentstroke}{rgb}{0.000000,0.000000,0.000000}%
\pgfsetstrokecolor{currentstroke}%
\pgfsetdash{}{0pt}%
\pgfsys@defobject{currentmarker}{\pgfqpoint{0.000000in}{-0.048611in}}{\pgfqpoint{0.000000in}{0.000000in}}{%
\pgfpathmoveto{\pgfqpoint{0.000000in}{0.000000in}}%
\pgfpathlineto{\pgfqpoint{0.000000in}{-0.048611in}}%
\pgfusepath{stroke,fill}%
}%
\begin{pgfscope}%
\pgfsys@transformshift{1.965377in}{1.741813in}%
\pgfsys@useobject{currentmarker}{}%
\end{pgfscope}%
\end{pgfscope}%
\begin{pgfscope}%
\definecolor{textcolor}{rgb}{0.000000,0.000000,0.000000}%
\pgfsetstrokecolor{textcolor}%
\pgfsetfillcolor{textcolor}%
\pgftext[x=1.965377in,y=1.644591in,,top]{\color{textcolor}\rmfamily\fontsize{8.000000}{9.600000}\selectfont \(\displaystyle 60000\)}%
\end{pgfscope}%
\begin{pgfscope}%
\pgfsetbuttcap%
\pgfsetroundjoin%
\definecolor{currentfill}{rgb}{0.000000,0.000000,0.000000}%
\pgfsetfillcolor{currentfill}%
\pgfsetlinewidth{0.803000pt}%
\definecolor{currentstroke}{rgb}{0.000000,0.000000,0.000000}%
\pgfsetstrokecolor{currentstroke}%
\pgfsetdash{}{0pt}%
\pgfsys@defobject{currentmarker}{\pgfqpoint{0.000000in}{-0.048611in}}{\pgfqpoint{0.000000in}{0.000000in}}{%
\pgfpathmoveto{\pgfqpoint{0.000000in}{0.000000in}}%
\pgfpathlineto{\pgfqpoint{0.000000in}{-0.048611in}}%
\pgfusepath{stroke,fill}%
}%
\begin{pgfscope}%
\pgfsys@transformshift{2.379030in}{1.741813in}%
\pgfsys@useobject{currentmarker}{}%
\end{pgfscope}%
\end{pgfscope}%
\begin{pgfscope}%
\definecolor{textcolor}{rgb}{0.000000,0.000000,0.000000}%
\pgfsetstrokecolor{textcolor}%
\pgfsetfillcolor{textcolor}%
\pgftext[x=2.379030in,y=1.644591in,,top]{\color{textcolor}\rmfamily\fontsize{8.000000}{9.600000}\selectfont \(\displaystyle 80000\)}%
\end{pgfscope}%
\begin{pgfscope}%
\pgfsetbuttcap%
\pgfsetroundjoin%
\definecolor{currentfill}{rgb}{0.000000,0.000000,0.000000}%
\pgfsetfillcolor{currentfill}%
\pgfsetlinewidth{0.803000pt}%
\definecolor{currentstroke}{rgb}{0.000000,0.000000,0.000000}%
\pgfsetstrokecolor{currentstroke}%
\pgfsetdash{}{0pt}%
\pgfsys@defobject{currentmarker}{\pgfqpoint{0.000000in}{-0.048611in}}{\pgfqpoint{0.000000in}{0.000000in}}{%
\pgfpathmoveto{\pgfqpoint{0.000000in}{0.000000in}}%
\pgfpathlineto{\pgfqpoint{0.000000in}{-0.048611in}}%
\pgfusepath{stroke,fill}%
}%
\begin{pgfscope}%
\pgfsys@transformshift{2.792683in}{1.741813in}%
\pgfsys@useobject{currentmarker}{}%
\end{pgfscope}%
\end{pgfscope}%
\begin{pgfscope}%
\definecolor{textcolor}{rgb}{0.000000,0.000000,0.000000}%
\pgfsetstrokecolor{textcolor}%
\pgfsetfillcolor{textcolor}%
\pgftext[x=2.792683in,y=1.644591in,,top]{\color{textcolor}\rmfamily\fontsize{8.000000}{9.600000}\selectfont \(\displaystyle 100000\)}%
\end{pgfscope}%
\begin{pgfscope}%
\pgfsetbuttcap%
\pgfsetroundjoin%
\definecolor{currentfill}{rgb}{0.000000,0.000000,0.000000}%
\pgfsetfillcolor{currentfill}%
\pgfsetlinewidth{0.803000pt}%
\definecolor{currentstroke}{rgb}{0.000000,0.000000,0.000000}%
\pgfsetstrokecolor{currentstroke}%
\pgfsetdash{}{0pt}%
\pgfsys@defobject{currentmarker}{\pgfqpoint{0.000000in}{-0.048611in}}{\pgfqpoint{0.000000in}{0.000000in}}{%
\pgfpathmoveto{\pgfqpoint{0.000000in}{0.000000in}}%
\pgfpathlineto{\pgfqpoint{0.000000in}{-0.048611in}}%
\pgfusepath{stroke,fill}%
}%
\begin{pgfscope}%
\pgfsys@transformshift{3.206336in}{1.741813in}%
\pgfsys@useobject{currentmarker}{}%
\end{pgfscope}%
\end{pgfscope}%
\begin{pgfscope}%
\definecolor{textcolor}{rgb}{0.000000,0.000000,0.000000}%
\pgfsetstrokecolor{textcolor}%
\pgfsetfillcolor{textcolor}%
\pgftext[x=3.206336in,y=1.644591in,,top]{\color{textcolor}\rmfamily\fontsize{8.000000}{9.600000}\selectfont \(\displaystyle 120000\)}%
\end{pgfscope}%
\begin{pgfscope}%
\pgfsetbuttcap%
\pgfsetroundjoin%
\definecolor{currentfill}{rgb}{0.000000,0.000000,0.000000}%
\pgfsetfillcolor{currentfill}%
\pgfsetlinewidth{0.803000pt}%
\definecolor{currentstroke}{rgb}{0.000000,0.000000,0.000000}%
\pgfsetstrokecolor{currentstroke}%
\pgfsetdash{}{0pt}%
\pgfsys@defobject{currentmarker}{\pgfqpoint{0.000000in}{-0.048611in}}{\pgfqpoint{0.000000in}{0.000000in}}{%
\pgfpathmoveto{\pgfqpoint{0.000000in}{0.000000in}}%
\pgfpathlineto{\pgfqpoint{0.000000in}{-0.048611in}}%
\pgfusepath{stroke,fill}%
}%
\begin{pgfscope}%
\pgfsys@transformshift{3.619989in}{1.741813in}%
\pgfsys@useobject{currentmarker}{}%
\end{pgfscope}%
\end{pgfscope}%
\begin{pgfscope}%
\definecolor{textcolor}{rgb}{0.000000,0.000000,0.000000}%
\pgfsetstrokecolor{textcolor}%
\pgfsetfillcolor{textcolor}%
\pgftext[x=3.619989in,y=1.644591in,,top]{\color{textcolor}\rmfamily\fontsize{8.000000}{9.600000}\selectfont \(\displaystyle 140000\)}%
\end{pgfscope}%
\begin{pgfscope}%
\pgfsetbuttcap%
\pgfsetroundjoin%
\definecolor{currentfill}{rgb}{0.000000,0.000000,0.000000}%
\pgfsetfillcolor{currentfill}%
\pgfsetlinewidth{0.803000pt}%
\definecolor{currentstroke}{rgb}{0.000000,0.000000,0.000000}%
\pgfsetstrokecolor{currentstroke}%
\pgfsetdash{}{0pt}%
\pgfsys@defobject{currentmarker}{\pgfqpoint{-0.048611in}{0.000000in}}{\pgfqpoint{0.000000in}{0.000000in}}{%
\pgfpathmoveto{\pgfqpoint{0.000000in}{0.000000in}}%
\pgfpathlineto{\pgfqpoint{-0.048611in}{0.000000in}}%
\pgfusepath{stroke,fill}%
}%
\begin{pgfscope}%
\pgfsys@transformshift{0.570188in}{1.799650in}%
\pgfsys@useobject{currentmarker}{}%
\end{pgfscope}%
\end{pgfscope}%
\begin{pgfscope}%
\definecolor{textcolor}{rgb}{0.000000,0.000000,0.000000}%
\pgfsetstrokecolor{textcolor}%
\pgfsetfillcolor{textcolor}%
\pgftext[x=0.413937in,y=1.757440in,left,base]{\color{textcolor}\rmfamily\fontsize{8.000000}{9.600000}\selectfont \(\displaystyle 0\)}%
\end{pgfscope}%
\begin{pgfscope}%
\pgfsetbuttcap%
\pgfsetroundjoin%
\definecolor{currentfill}{rgb}{0.000000,0.000000,0.000000}%
\pgfsetfillcolor{currentfill}%
\pgfsetlinewidth{0.803000pt}%
\definecolor{currentstroke}{rgb}{0.000000,0.000000,0.000000}%
\pgfsetstrokecolor{currentstroke}%
\pgfsetdash{}{0pt}%
\pgfsys@defobject{currentmarker}{\pgfqpoint{-0.048611in}{0.000000in}}{\pgfqpoint{0.000000in}{0.000000in}}{%
\pgfpathmoveto{\pgfqpoint{0.000000in}{0.000000in}}%
\pgfpathlineto{\pgfqpoint{-0.048611in}{0.000000in}}%
\pgfusepath{stroke,fill}%
}%
\begin{pgfscope}%
\pgfsys@transformshift{0.570188in}{2.191974in}%
\pgfsys@useobject{currentmarker}{}%
\end{pgfscope}%
\end{pgfscope}%
\begin{pgfscope}%
\definecolor{textcolor}{rgb}{0.000000,0.000000,0.000000}%
\pgfsetstrokecolor{textcolor}%
\pgfsetfillcolor{textcolor}%
\pgftext[x=0.413937in,y=2.149765in,left,base]{\color{textcolor}\rmfamily\fontsize{8.000000}{9.600000}\selectfont \(\displaystyle 5\)}%
\end{pgfscope}%
\begin{pgfscope}%
\pgfsetbuttcap%
\pgfsetroundjoin%
\definecolor{currentfill}{rgb}{0.000000,0.000000,0.000000}%
\pgfsetfillcolor{currentfill}%
\pgfsetlinewidth{0.803000pt}%
\definecolor{currentstroke}{rgb}{0.000000,0.000000,0.000000}%
\pgfsetstrokecolor{currentstroke}%
\pgfsetdash{}{0pt}%
\pgfsys@defobject{currentmarker}{\pgfqpoint{-0.048611in}{0.000000in}}{\pgfqpoint{0.000000in}{0.000000in}}{%
\pgfpathmoveto{\pgfqpoint{0.000000in}{0.000000in}}%
\pgfpathlineto{\pgfqpoint{-0.048611in}{0.000000in}}%
\pgfusepath{stroke,fill}%
}%
\begin{pgfscope}%
\pgfsys@transformshift{0.570188in}{2.584298in}%
\pgfsys@useobject{currentmarker}{}%
\end{pgfscope}%
\end{pgfscope}%
\begin{pgfscope}%
\definecolor{textcolor}{rgb}{0.000000,0.000000,0.000000}%
\pgfsetstrokecolor{textcolor}%
\pgfsetfillcolor{textcolor}%
\pgftext[x=0.354908in,y=2.542089in,left,base]{\color{textcolor}\rmfamily\fontsize{8.000000}{9.600000}\selectfont \(\displaystyle 10\)}%
\end{pgfscope}%
\begin{pgfscope}%
\pgfsetbuttcap%
\pgfsetroundjoin%
\definecolor{currentfill}{rgb}{0.000000,0.000000,0.000000}%
\pgfsetfillcolor{currentfill}%
\pgfsetlinewidth{0.803000pt}%
\definecolor{currentstroke}{rgb}{0.000000,0.000000,0.000000}%
\pgfsetstrokecolor{currentstroke}%
\pgfsetdash{}{0pt}%
\pgfsys@defobject{currentmarker}{\pgfqpoint{-0.048611in}{0.000000in}}{\pgfqpoint{0.000000in}{0.000000in}}{%
\pgfpathmoveto{\pgfqpoint{0.000000in}{0.000000in}}%
\pgfpathlineto{\pgfqpoint{-0.048611in}{0.000000in}}%
\pgfusepath{stroke,fill}%
}%
\begin{pgfscope}%
\pgfsys@transformshift{0.570188in}{2.976622in}%
\pgfsys@useobject{currentmarker}{}%
\end{pgfscope}%
\end{pgfscope}%
\begin{pgfscope}%
\definecolor{textcolor}{rgb}{0.000000,0.000000,0.000000}%
\pgfsetstrokecolor{textcolor}%
\pgfsetfillcolor{textcolor}%
\pgftext[x=0.354908in,y=2.934413in,left,base]{\color{textcolor}\rmfamily\fontsize{8.000000}{9.600000}\selectfont \(\displaystyle 15\)}%
\end{pgfscope}%
\begin{pgfscope}%
\definecolor{textcolor}{rgb}{0.000000,0.000000,0.000000}%
\pgfsetstrokecolor{textcolor}%
\pgfsetfillcolor{textcolor}%
\pgftext[x=0.299353in,y=2.378014in,,bottom,rotate=90.000000]{\color{textcolor}\rmfamily\fontsize{8.000000}{9.600000}\selectfont Melt composition (mol)}%
\end{pgfscope}%
\begin{pgfscope}%
\pgfpathrectangle{\pgfqpoint{0.570188in}{1.741813in}}{\pgfqpoint{3.393071in}{1.272402in}}%
\pgfusepath{clip}%
\pgfsetrectcap%
\pgfsetroundjoin%
\pgfsetlinewidth{1.505625pt}%
\definecolor{currentstroke}{rgb}{0.121569,0.466667,0.705882}%
\pgfsetstrokecolor{currentstroke}%
\pgfsetdash{}{0pt}%
\pgfpathmoveto{\pgfqpoint{0.724418in}{1.799650in}}%
\pgfpathlineto{\pgfqpoint{0.746290in}{1.854707in}}%
\pgfpathlineto{\pgfqpoint{0.769568in}{1.857611in}}%
\pgfpathlineto{\pgfqpoint{1.182746in}{1.861144in}}%
\pgfpathlineto{\pgfqpoint{1.244049in}{1.862511in}}%
\pgfpathlineto{\pgfqpoint{1.305290in}{1.863664in}}%
\pgfpathlineto{\pgfqpoint{1.389407in}{1.865594in}}%
\pgfpathlineto{\pgfqpoint{1.431413in}{1.866871in}}%
\pgfpathlineto{\pgfqpoint{1.469614in}{1.868067in}}%
\pgfpathlineto{\pgfqpoint{1.595571in}{1.871852in}}%
\pgfpathlineto{\pgfqpoint{1.698385in}{1.875550in}}%
\pgfpathlineto{\pgfqpoint{1.740122in}{1.877252in}}%
\pgfpathlineto{\pgfqpoint{1.864942in}{1.882116in}}%
\pgfpathlineto{\pgfqpoint{2.174499in}{1.895620in}}%
\pgfpathlineto{\pgfqpoint{2.193382in}{1.896781in}}%
\pgfpathlineto{\pgfqpoint{2.257519in}{1.899622in}}%
\pgfpathlineto{\pgfqpoint{2.399899in}{1.905907in}}%
\pgfpathlineto{\pgfqpoint{2.444573in}{1.907453in}}%
\pgfpathlineto{\pgfqpoint{2.485442in}{1.909391in}}%
\pgfpathlineto{\pgfqpoint{2.526208in}{1.910317in}}%
\pgfpathlineto{\pgfqpoint{2.566911in}{1.912176in}}%
\pgfpathlineto{\pgfqpoint{2.629745in}{1.914201in}}%
\pgfpathlineto{\pgfqpoint{2.648215in}{1.914365in}}%
\pgfpathlineto{\pgfqpoint{2.711007in}{1.916876in}}%
\pgfpathlineto{\pgfqpoint{2.751628in}{1.917818in}}%
\pgfpathlineto{\pgfqpoint{2.773800in}{1.917629in}}%
\pgfpathlineto{\pgfqpoint{2.855144in}{1.920062in}}%
\pgfpathlineto{\pgfqpoint{2.940564in}{1.920509in}}%
\pgfpathlineto{\pgfqpoint{2.981515in}{1.920643in}}%
\pgfpathlineto{\pgfqpoint{3.085963in}{1.925727in}}%
\pgfpathlineto{\pgfqpoint{3.127121in}{1.926739in}}%
\pgfpathlineto{\pgfqpoint{3.187308in}{1.930764in}}%
\pgfpathlineto{\pgfqpoint{3.209852in}{1.930929in}}%
\pgfpathlineto{\pgfqpoint{3.228673in}{1.932475in}}%
\pgfpathlineto{\pgfqpoint{3.270038in}{1.934311in}}%
\pgfpathlineto{\pgfqpoint{3.292582in}{1.799650in}}%
\pgfpathlineto{\pgfqpoint{3.809028in}{1.799650in}}%
\pgfpathlineto{\pgfqpoint{3.809028in}{1.799650in}}%
\pgfusepath{stroke}%
\end{pgfscope}%
\begin{pgfscope}%
\pgfpathrectangle{\pgfqpoint{0.570188in}{1.741813in}}{\pgfqpoint{3.393071in}{1.272402in}}%
\pgfusepath{clip}%
\pgfsetrectcap%
\pgfsetroundjoin%
\pgfsetlinewidth{1.505625pt}%
\definecolor{currentstroke}{rgb}{1.000000,0.498039,0.054902}%
\pgfsetstrokecolor{currentstroke}%
\pgfsetdash{}{0pt}%
\pgfpathmoveto{\pgfqpoint{0.724418in}{1.799650in}}%
\pgfpathlineto{\pgfqpoint{0.746290in}{1.890951in}}%
\pgfpathlineto{\pgfqpoint{0.769568in}{1.895447in}}%
\pgfpathlineto{\pgfqpoint{0.808394in}{1.895699in}}%
\pgfpathlineto{\pgfqpoint{1.036623in}{1.896640in}}%
\pgfpathlineto{\pgfqpoint{1.286159in}{1.901277in}}%
\pgfpathlineto{\pgfqpoint{1.347359in}{1.902964in}}%
\pgfpathlineto{\pgfqpoint{1.389407in}{1.904267in}}%
\pgfpathlineto{\pgfqpoint{1.469614in}{1.907123in}}%
\pgfpathlineto{\pgfqpoint{1.572676in}{1.911023in}}%
\pgfpathlineto{\pgfqpoint{1.637516in}{1.913895in}}%
\pgfpathlineto{\pgfqpoint{1.717371in}{1.917967in}}%
\pgfpathlineto{\pgfqpoint{1.781777in}{1.921168in}}%
\pgfpathlineto{\pgfqpoint{1.906452in}{1.928261in}}%
\pgfpathlineto{\pgfqpoint{2.008335in}{1.934366in}}%
\pgfpathlineto{\pgfqpoint{2.298864in}{1.954021in}}%
\pgfpathlineto{\pgfqpoint{2.648215in}{1.979020in}}%
\pgfpathlineto{\pgfqpoint{2.670386in}{1.980354in}}%
\pgfpathlineto{\pgfqpoint{2.711007in}{1.983532in}}%
\pgfpathlineto{\pgfqpoint{2.855144in}{1.992006in}}%
\pgfpathlineto{\pgfqpoint{2.981515in}{1.998448in}}%
\pgfpathlineto{\pgfqpoint{3.041081in}{2.002183in}}%
\pgfpathlineto{\pgfqpoint{3.145942in}{2.008382in}}%
\pgfpathlineto{\pgfqpoint{3.187308in}{2.011316in}}%
\pgfpathlineto{\pgfqpoint{3.228673in}{2.013553in}}%
\pgfpathlineto{\pgfqpoint{3.270038in}{2.015946in}}%
\pgfpathlineto{\pgfqpoint{3.292582in}{1.799650in}}%
\pgfpathlineto{\pgfqpoint{3.809028in}{1.799650in}}%
\pgfpathlineto{\pgfqpoint{3.809028in}{1.799650in}}%
\pgfusepath{stroke}%
\end{pgfscope}%
\begin{pgfscope}%
\pgfpathrectangle{\pgfqpoint{0.570188in}{1.741813in}}{\pgfqpoint{3.393071in}{1.272402in}}%
\pgfusepath{clip}%
\pgfsetrectcap%
\pgfsetroundjoin%
\pgfsetlinewidth{1.505625pt}%
\definecolor{currentstroke}{rgb}{0.172549,0.627451,0.172549}%
\pgfsetstrokecolor{currentstroke}%
\pgfsetdash{}{0pt}%
\pgfpathmoveto{\pgfqpoint{0.724418in}{1.799650in}}%
\pgfpathlineto{\pgfqpoint{0.746290in}{2.416564in}}%
\pgfpathlineto{\pgfqpoint{0.769568in}{2.447612in}}%
\pgfpathlineto{\pgfqpoint{0.788985in}{2.449119in}}%
\pgfpathlineto{\pgfqpoint{0.808394in}{2.448185in}}%
\pgfpathlineto{\pgfqpoint{0.932361in}{2.439476in}}%
\pgfpathlineto{\pgfqpoint{0.955567in}{2.438683in}}%
\pgfpathlineto{\pgfqpoint{0.974885in}{2.436729in}}%
\pgfpathlineto{\pgfqpoint{1.017346in}{2.436635in}}%
\pgfpathlineto{\pgfqpoint{1.036623in}{2.435435in}}%
\pgfpathlineto{\pgfqpoint{1.055899in}{2.435960in}}%
\pgfpathlineto{\pgfqpoint{1.079001in}{2.434203in}}%
\pgfpathlineto{\pgfqpoint{1.117471in}{2.435749in}}%
\pgfpathlineto{\pgfqpoint{1.140532in}{2.434564in}}%
\pgfpathlineto{\pgfqpoint{1.182746in}{2.437067in}}%
\pgfpathlineto{\pgfqpoint{1.201918in}{2.436596in}}%
\pgfpathlineto{\pgfqpoint{1.221071in}{2.436800in}}%
\pgfpathlineto{\pgfqpoint{1.244049in}{2.437993in}}%
\pgfpathlineto{\pgfqpoint{1.263201in}{2.439829in}}%
\pgfpathlineto{\pgfqpoint{1.305290in}{2.441578in}}%
\pgfpathlineto{\pgfqpoint{1.324422in}{2.443391in}}%
\pgfpathlineto{\pgfqpoint{1.347359in}{2.443524in}}%
\pgfpathlineto{\pgfqpoint{1.366470in}{2.446020in}}%
\pgfpathlineto{\pgfqpoint{1.389407in}{2.448122in}}%
\pgfpathlineto{\pgfqpoint{1.408497in}{2.448374in}}%
\pgfpathlineto{\pgfqpoint{1.450524in}{2.453890in}}%
\pgfpathlineto{\pgfqpoint{1.469614in}{2.454470in}}%
\pgfpathlineto{\pgfqpoint{1.492510in}{2.457334in}}%
\pgfpathlineto{\pgfqpoint{1.511600in}{2.460379in}}%
\pgfpathlineto{\pgfqpoint{1.534516in}{2.462944in}}%
\pgfpathlineto{\pgfqpoint{1.553585in}{2.464475in}}%
\pgfpathlineto{\pgfqpoint{1.572676in}{2.467652in}}%
\pgfpathlineto{\pgfqpoint{1.679398in}{2.482671in}}%
\pgfpathlineto{\pgfqpoint{1.698385in}{2.485809in}}%
\pgfpathlineto{\pgfqpoint{1.717371in}{2.487661in}}%
\pgfpathlineto{\pgfqpoint{1.740122in}{2.492377in}}%
\pgfpathlineto{\pgfqpoint{1.800702in}{2.502208in}}%
\pgfpathlineto{\pgfqpoint{1.883805in}{2.520098in}}%
\pgfpathlineto{\pgfqpoint{1.947962in}{2.532315in}}%
\pgfpathlineto{\pgfqpoint{1.966825in}{2.537345in}}%
\pgfpathlineto{\pgfqpoint{1.989472in}{2.540836in}}%
\pgfpathlineto{\pgfqpoint{2.008335in}{2.546290in}}%
\pgfpathlineto{\pgfqpoint{2.049866in}{2.556255in}}%
\pgfpathlineto{\pgfqpoint{2.072534in}{2.560453in}}%
\pgfpathlineto{\pgfqpoint{2.110300in}{2.570104in}}%
\pgfpathlineto{\pgfqpoint{2.151852in}{2.580053in}}%
\pgfpathlineto{\pgfqpoint{2.174499in}{2.586417in}}%
\pgfpathlineto{\pgfqpoint{2.193382in}{2.589398in}}%
\pgfpathlineto{\pgfqpoint{2.216030in}{2.595676in}}%
\pgfpathlineto{\pgfqpoint{2.234913in}{2.599913in}}%
\pgfpathlineto{\pgfqpoint{2.276320in}{2.610819in}}%
\pgfpathlineto{\pgfqpoint{2.298864in}{2.617489in}}%
\pgfpathlineto{\pgfqpoint{2.317644in}{2.621569in}}%
\pgfpathlineto{\pgfqpoint{2.399899in}{2.643931in}}%
\pgfpathlineto{\pgfqpoint{2.422257in}{2.652406in}}%
\pgfpathlineto{\pgfqpoint{2.485442in}{2.670060in}}%
\pgfpathlineto{\pgfqpoint{2.526208in}{2.685675in}}%
\pgfpathlineto{\pgfqpoint{2.544719in}{2.690226in}}%
\pgfpathlineto{\pgfqpoint{2.566911in}{2.696974in}}%
\pgfpathlineto{\pgfqpoint{2.589104in}{2.702545in}}%
\pgfpathlineto{\pgfqpoint{2.607594in}{2.707880in}}%
\pgfpathlineto{\pgfqpoint{2.629745in}{2.716276in}}%
\pgfpathlineto{\pgfqpoint{2.648215in}{2.723966in}}%
\pgfpathlineto{\pgfqpoint{2.670386in}{2.729929in}}%
\pgfpathlineto{\pgfqpoint{2.692537in}{2.735108in}}%
\pgfpathlineto{\pgfqpoint{2.711007in}{2.740600in}}%
\pgfpathlineto{\pgfqpoint{2.751628in}{2.754881in}}%
\pgfpathlineto{\pgfqpoint{2.773800in}{2.765866in}}%
\pgfpathlineto{\pgfqpoint{2.814400in}{2.778263in}}%
\pgfpathlineto{\pgfqpoint{2.836737in}{2.786424in}}%
\pgfpathlineto{\pgfqpoint{2.855144in}{2.793799in}}%
\pgfpathlineto{\pgfqpoint{2.877482in}{2.803607in}}%
\pgfpathlineto{\pgfqpoint{2.918227in}{2.818830in}}%
\pgfpathlineto{\pgfqpoint{2.981515in}{2.843311in}}%
\pgfpathlineto{\pgfqpoint{3.000130in}{2.846528in}}%
\pgfpathlineto{\pgfqpoint{3.022467in}{2.852805in}}%
\pgfpathlineto{\pgfqpoint{3.041081in}{2.857434in}}%
\pgfpathlineto{\pgfqpoint{3.063419in}{2.861044in}}%
\pgfpathlineto{\pgfqpoint{3.085963in}{2.866065in}}%
\pgfpathlineto{\pgfqpoint{3.127121in}{2.879797in}}%
\pgfpathlineto{\pgfqpoint{3.168487in}{2.883249in}}%
\pgfpathlineto{\pgfqpoint{3.187308in}{2.886309in}}%
\pgfpathlineto{\pgfqpoint{3.209852in}{2.895019in}}%
\pgfpathlineto{\pgfqpoint{3.228673in}{2.895804in}}%
\pgfpathlineto{\pgfqpoint{3.251217in}{2.899413in}}%
\pgfpathlineto{\pgfqpoint{3.270038in}{2.904278in}}%
\pgfpathlineto{\pgfqpoint{3.292582in}{1.799650in}}%
\pgfpathlineto{\pgfqpoint{3.809028in}{1.799650in}}%
\pgfpathlineto{\pgfqpoint{3.809028in}{1.799650in}}%
\pgfusepath{stroke}%
\end{pgfscope}%
\begin{pgfscope}%
\pgfpathrectangle{\pgfqpoint{0.570188in}{1.741813in}}{\pgfqpoint{3.393071in}{1.272402in}}%
\pgfusepath{clip}%
\pgfsetrectcap%
\pgfsetroundjoin%
\pgfsetlinewidth{1.505625pt}%
\definecolor{currentstroke}{rgb}{0.839216,0.152941,0.156863}%
\pgfsetstrokecolor{currentstroke}%
\pgfsetdash{}{0pt}%
\pgfpathmoveto{\pgfqpoint{0.724418in}{1.799650in}}%
\pgfpathlineto{\pgfqpoint{0.746290in}{2.357864in}}%
\pgfpathlineto{\pgfqpoint{0.769568in}{2.385570in}}%
\pgfpathlineto{\pgfqpoint{0.788985in}{2.387069in}}%
\pgfpathlineto{\pgfqpoint{0.831670in}{2.386614in}}%
\pgfpathlineto{\pgfqpoint{0.870436in}{2.385868in}}%
\pgfpathlineto{\pgfqpoint{0.913027in}{2.385272in}}%
\pgfpathlineto{\pgfqpoint{0.994203in}{2.385641in}}%
\pgfpathlineto{\pgfqpoint{1.017346in}{2.385963in}}%
\pgfpathlineto{\pgfqpoint{1.036623in}{2.386928in}}%
\pgfpathlineto{\pgfqpoint{1.055899in}{2.387304in}}%
\pgfpathlineto{\pgfqpoint{1.079001in}{2.388960in}}%
\pgfpathlineto{\pgfqpoint{1.117471in}{2.390576in}}%
\pgfpathlineto{\pgfqpoint{1.140532in}{2.392656in}}%
\pgfpathlineto{\pgfqpoint{1.182746in}{2.395386in}}%
\pgfpathlineto{\pgfqpoint{1.244049in}{2.401679in}}%
\pgfpathlineto{\pgfqpoint{1.263201in}{2.403452in}}%
\pgfpathlineto{\pgfqpoint{1.324422in}{2.410907in}}%
\pgfpathlineto{\pgfqpoint{1.347359in}{2.414641in}}%
\pgfpathlineto{\pgfqpoint{1.389407in}{2.420291in}}%
\pgfpathlineto{\pgfqpoint{1.408497in}{2.423806in}}%
\pgfpathlineto{\pgfqpoint{1.450524in}{2.430264in}}%
\pgfpathlineto{\pgfqpoint{1.469614in}{2.434132in}}%
\pgfpathlineto{\pgfqpoint{1.534516in}{2.445925in}}%
\pgfpathlineto{\pgfqpoint{1.553585in}{2.450108in}}%
\pgfpathlineto{\pgfqpoint{1.595571in}{2.458464in}}%
\pgfpathlineto{\pgfqpoint{1.698385in}{2.482247in}}%
\pgfpathlineto{\pgfqpoint{1.717371in}{2.487331in}}%
\pgfpathlineto{\pgfqpoint{1.759068in}{2.497226in}}%
\pgfpathlineto{\pgfqpoint{1.823391in}{2.514386in}}%
\pgfpathlineto{\pgfqpoint{1.883805in}{2.531162in}}%
\pgfpathlineto{\pgfqpoint{1.966825in}{2.555227in}}%
\pgfpathlineto{\pgfqpoint{1.989472in}{2.562885in}}%
\pgfpathlineto{\pgfqpoint{2.030982in}{2.575447in}}%
\pgfpathlineto{\pgfqpoint{2.049866in}{2.581301in}}%
\pgfpathlineto{\pgfqpoint{2.072534in}{2.588928in}}%
\pgfpathlineto{\pgfqpoint{2.091417in}{2.594342in}}%
\pgfpathlineto{\pgfqpoint{2.257519in}{2.648012in}}%
\pgfpathlineto{\pgfqpoint{2.399899in}{2.695561in}}%
\pgfpathlineto{\pgfqpoint{2.422257in}{2.702309in}}%
\pgfpathlineto{\pgfqpoint{2.503974in}{2.730478in}}%
\pgfpathlineto{\pgfqpoint{2.526208in}{2.737226in}}%
\pgfpathlineto{\pgfqpoint{2.544719in}{2.743817in}}%
\pgfpathlineto{\pgfqpoint{2.607594in}{2.764061in}}%
\pgfpathlineto{\pgfqpoint{2.670386in}{2.782657in}}%
\pgfpathlineto{\pgfqpoint{2.711007in}{2.795369in}}%
\pgfpathlineto{\pgfqpoint{2.751628in}{2.807217in}}%
\pgfpathlineto{\pgfqpoint{2.773800in}{2.812552in}}%
\pgfpathlineto{\pgfqpoint{2.814400in}{2.825578in}}%
\pgfpathlineto{\pgfqpoint{2.855144in}{2.836641in}}%
\pgfpathlineto{\pgfqpoint{2.981515in}{2.866536in}}%
\pgfpathlineto{\pgfqpoint{3.085963in}{2.900041in}}%
\pgfpathlineto{\pgfqpoint{3.127121in}{2.911575in}}%
\pgfpathlineto{\pgfqpoint{3.187308in}{2.931976in}}%
\pgfpathlineto{\pgfqpoint{3.209852in}{2.937468in}}%
\pgfpathlineto{\pgfqpoint{3.228673in}{2.944138in}}%
\pgfpathlineto{\pgfqpoint{3.251217in}{2.951200in}}%
\pgfpathlineto{\pgfqpoint{3.270038in}{2.956378in}}%
\pgfpathlineto{\pgfqpoint{3.292582in}{1.799650in}}%
\pgfpathlineto{\pgfqpoint{3.809028in}{1.799650in}}%
\pgfpathlineto{\pgfqpoint{3.809028in}{1.799650in}}%
\pgfusepath{stroke}%
\end{pgfscope}%
\begin{pgfscope}%
\pgfsetrectcap%
\pgfsetmiterjoin%
\pgfsetlinewidth{0.803000pt}%
\definecolor{currentstroke}{rgb}{0.000000,0.000000,0.000000}%
\pgfsetstrokecolor{currentstroke}%
\pgfsetdash{}{0pt}%
\pgfpathmoveto{\pgfqpoint{0.570188in}{1.741813in}}%
\pgfpathlineto{\pgfqpoint{0.570188in}{3.014215in}}%
\pgfusepath{stroke}%
\end{pgfscope}%
\begin{pgfscope}%
\pgfsetrectcap%
\pgfsetmiterjoin%
\pgfsetlinewidth{0.803000pt}%
\definecolor{currentstroke}{rgb}{0.000000,0.000000,0.000000}%
\pgfsetstrokecolor{currentstroke}%
\pgfsetdash{}{0pt}%
\pgfpathmoveto{\pgfqpoint{3.963259in}{1.741813in}}%
\pgfpathlineto{\pgfqpoint{3.963259in}{3.014215in}}%
\pgfusepath{stroke}%
\end{pgfscope}%
\begin{pgfscope}%
\pgfsetrectcap%
\pgfsetmiterjoin%
\pgfsetlinewidth{0.803000pt}%
\definecolor{currentstroke}{rgb}{0.000000,0.000000,0.000000}%
\pgfsetstrokecolor{currentstroke}%
\pgfsetdash{}{0pt}%
\pgfpathmoveto{\pgfqpoint{0.570188in}{1.741813in}}%
\pgfpathlineto{\pgfqpoint{3.963259in}{1.741813in}}%
\pgfusepath{stroke}%
\end{pgfscope}%
\begin{pgfscope}%
\pgfsetrectcap%
\pgfsetmiterjoin%
\pgfsetlinewidth{0.803000pt}%
\definecolor{currentstroke}{rgb}{0.000000,0.000000,0.000000}%
\pgfsetstrokecolor{currentstroke}%
\pgfsetdash{}{0pt}%
\pgfpathmoveto{\pgfqpoint{0.570188in}{3.014215in}}%
\pgfpathlineto{\pgfqpoint{3.963259in}{3.014215in}}%
\pgfusepath{stroke}%
\end{pgfscope}%
\begin{pgfscope}%
\pgfsetrectcap%
\pgfsetroundjoin%
\pgfsetlinewidth{1.505625pt}%
\definecolor{currentstroke}{rgb}{0.121569,0.466667,0.705882}%
\pgfsetstrokecolor{currentstroke}%
\pgfsetdash{}{0pt}%
\pgfpathmoveto{\pgfqpoint{3.276160in}{2.631665in}}%
\pgfpathlineto{\pgfqpoint{3.498382in}{2.631665in}}%
\pgfusepath{stroke}%
\end{pgfscope}%
\begin{pgfscope}%
\definecolor{textcolor}{rgb}{0.000000,0.000000,0.000000}%
\pgfsetstrokecolor{textcolor}%
\pgfsetfillcolor{textcolor}%
\pgftext[x=3.587271in,y=2.592776in,left,base]{\color{textcolor}\rmfamily\fontsize{8.000000}{9.600000}\selectfont CaO}%
\end{pgfscope}%
\begin{pgfscope}%
\pgfsetrectcap%
\pgfsetroundjoin%
\pgfsetlinewidth{1.505625pt}%
\definecolor{currentstroke}{rgb}{1.000000,0.498039,0.054902}%
\pgfsetstrokecolor{currentstroke}%
\pgfsetdash{}{0pt}%
\pgfpathmoveto{\pgfqpoint{3.276160in}{2.468579in}}%
\pgfpathlineto{\pgfqpoint{3.498382in}{2.468579in}}%
\pgfusepath{stroke}%
\end{pgfscope}%
\begin{pgfscope}%
\definecolor{textcolor}{rgb}{0.000000,0.000000,0.000000}%
\pgfsetstrokecolor{textcolor}%
\pgfsetfillcolor{textcolor}%
\pgftext[x=3.587271in,y=2.429690in,left,base]{\color{textcolor}\rmfamily\fontsize{8.000000}{9.600000}\selectfont FeO}%
\end{pgfscope}%
\begin{pgfscope}%
\pgfsetrectcap%
\pgfsetroundjoin%
\pgfsetlinewidth{1.505625pt}%
\definecolor{currentstroke}{rgb}{0.172549,0.627451,0.172549}%
\pgfsetstrokecolor{currentstroke}%
\pgfsetdash{}{0pt}%
\pgfpathmoveto{\pgfqpoint{3.276160in}{2.305493in}}%
\pgfpathlineto{\pgfqpoint{3.498382in}{2.305493in}}%
\pgfusepath{stroke}%
\end{pgfscope}%
\begin{pgfscope}%
\definecolor{textcolor}{rgb}{0.000000,0.000000,0.000000}%
\pgfsetstrokecolor{textcolor}%
\pgfsetfillcolor{textcolor}%
\pgftext[x=3.587271in,y=2.266605in,left,base]{\color{textcolor}\rmfamily\fontsize{8.000000}{9.600000}\selectfont MgO}%
\end{pgfscope}%
\begin{pgfscope}%
\pgfsetrectcap%
\pgfsetroundjoin%
\pgfsetlinewidth{1.505625pt}%
\definecolor{currentstroke}{rgb}{0.839216,0.152941,0.156863}%
\pgfsetstrokecolor{currentstroke}%
\pgfsetdash{}{0pt}%
\pgfpathmoveto{\pgfqpoint{3.276160in}{2.140834in}}%
\pgfpathlineto{\pgfqpoint{3.498382in}{2.140834in}}%
\pgfusepath{stroke}%
\end{pgfscope}%
\begin{pgfscope}%
\definecolor{textcolor}{rgb}{0.000000,0.000000,0.000000}%
\pgfsetstrokecolor{textcolor}%
\pgfsetfillcolor{textcolor}%
\pgftext[x=3.587271in,y=2.101945in,left,base]{\color{textcolor}\rmfamily\fontsize{8.000000}{9.600000}\selectfont SiO2}%
\end{pgfscope}%
\begin{pgfscope}%
\pgfsetbuttcap%
\pgfsetmiterjoin%
\definecolor{currentfill}{rgb}{1.000000,1.000000,1.000000}%
\pgfsetfillcolor{currentfill}%
\pgfsetlinewidth{0.000000pt}%
\definecolor{currentstroke}{rgb}{0.000000,0.000000,0.000000}%
\pgfsetstrokecolor{currentstroke}%
\pgfsetstrokeopacity{0.000000}%
\pgfsetdash{}{0pt}%
\pgfpathmoveto{\pgfqpoint{0.570188in}{0.469412in}}%
\pgfpathlineto{\pgfqpoint{3.963259in}{0.469412in}}%
\pgfpathlineto{\pgfqpoint{3.963259in}{1.423713in}}%
\pgfpathlineto{\pgfqpoint{0.570188in}{1.423713in}}%
\pgfpathclose%
\pgfusepath{fill}%
\end{pgfscope}%
\begin{pgfscope}%
\pgfsetbuttcap%
\pgfsetroundjoin%
\definecolor{currentfill}{rgb}{0.000000,0.000000,0.000000}%
\pgfsetfillcolor{currentfill}%
\pgfsetlinewidth{0.803000pt}%
\definecolor{currentstroke}{rgb}{0.000000,0.000000,0.000000}%
\pgfsetstrokecolor{currentstroke}%
\pgfsetdash{}{0pt}%
\pgfsys@defobject{currentmarker}{\pgfqpoint{0.000000in}{-0.048611in}}{\pgfqpoint{0.000000in}{0.000000in}}{%
\pgfpathmoveto{\pgfqpoint{0.000000in}{0.000000in}}%
\pgfpathlineto{\pgfqpoint{0.000000in}{-0.048611in}}%
\pgfusepath{stroke,fill}%
}%
\begin{pgfscope}%
\pgfsys@transformshift{0.724418in}{0.469412in}%
\pgfsys@useobject{currentmarker}{}%
\end{pgfscope}%
\end{pgfscope}%
\begin{pgfscope}%
\definecolor{textcolor}{rgb}{0.000000,0.000000,0.000000}%
\pgfsetstrokecolor{textcolor}%
\pgfsetfillcolor{textcolor}%
\pgftext[x=0.724418in,y=0.372189in,,top]{\color{textcolor}\rmfamily\fontsize{8.000000}{9.600000}\selectfont \(\displaystyle 0\)}%
\end{pgfscope}%
\begin{pgfscope}%
\pgfsetbuttcap%
\pgfsetroundjoin%
\definecolor{currentfill}{rgb}{0.000000,0.000000,0.000000}%
\pgfsetfillcolor{currentfill}%
\pgfsetlinewidth{0.803000pt}%
\definecolor{currentstroke}{rgb}{0.000000,0.000000,0.000000}%
\pgfsetstrokecolor{currentstroke}%
\pgfsetdash{}{0pt}%
\pgfsys@defobject{currentmarker}{\pgfqpoint{0.000000in}{-0.048611in}}{\pgfqpoint{0.000000in}{0.000000in}}{%
\pgfpathmoveto{\pgfqpoint{0.000000in}{0.000000in}}%
\pgfpathlineto{\pgfqpoint{0.000000in}{-0.048611in}}%
\pgfusepath{stroke,fill}%
}%
\begin{pgfscope}%
\pgfsys@transformshift{1.138071in}{0.469412in}%
\pgfsys@useobject{currentmarker}{}%
\end{pgfscope}%
\end{pgfscope}%
\begin{pgfscope}%
\definecolor{textcolor}{rgb}{0.000000,0.000000,0.000000}%
\pgfsetstrokecolor{textcolor}%
\pgfsetfillcolor{textcolor}%
\pgftext[x=1.138071in,y=0.372189in,,top]{\color{textcolor}\rmfamily\fontsize{8.000000}{9.600000}\selectfont \(\displaystyle 20000\)}%
\end{pgfscope}%
\begin{pgfscope}%
\pgfsetbuttcap%
\pgfsetroundjoin%
\definecolor{currentfill}{rgb}{0.000000,0.000000,0.000000}%
\pgfsetfillcolor{currentfill}%
\pgfsetlinewidth{0.803000pt}%
\definecolor{currentstroke}{rgb}{0.000000,0.000000,0.000000}%
\pgfsetstrokecolor{currentstroke}%
\pgfsetdash{}{0pt}%
\pgfsys@defobject{currentmarker}{\pgfqpoint{0.000000in}{-0.048611in}}{\pgfqpoint{0.000000in}{0.000000in}}{%
\pgfpathmoveto{\pgfqpoint{0.000000in}{0.000000in}}%
\pgfpathlineto{\pgfqpoint{0.000000in}{-0.048611in}}%
\pgfusepath{stroke,fill}%
}%
\begin{pgfscope}%
\pgfsys@transformshift{1.551724in}{0.469412in}%
\pgfsys@useobject{currentmarker}{}%
\end{pgfscope}%
\end{pgfscope}%
\begin{pgfscope}%
\definecolor{textcolor}{rgb}{0.000000,0.000000,0.000000}%
\pgfsetstrokecolor{textcolor}%
\pgfsetfillcolor{textcolor}%
\pgftext[x=1.551724in,y=0.372189in,,top]{\color{textcolor}\rmfamily\fontsize{8.000000}{9.600000}\selectfont \(\displaystyle 40000\)}%
\end{pgfscope}%
\begin{pgfscope}%
\pgfsetbuttcap%
\pgfsetroundjoin%
\definecolor{currentfill}{rgb}{0.000000,0.000000,0.000000}%
\pgfsetfillcolor{currentfill}%
\pgfsetlinewidth{0.803000pt}%
\definecolor{currentstroke}{rgb}{0.000000,0.000000,0.000000}%
\pgfsetstrokecolor{currentstroke}%
\pgfsetdash{}{0pt}%
\pgfsys@defobject{currentmarker}{\pgfqpoint{0.000000in}{-0.048611in}}{\pgfqpoint{0.000000in}{0.000000in}}{%
\pgfpathmoveto{\pgfqpoint{0.000000in}{0.000000in}}%
\pgfpathlineto{\pgfqpoint{0.000000in}{-0.048611in}}%
\pgfusepath{stroke,fill}%
}%
\begin{pgfscope}%
\pgfsys@transformshift{1.965377in}{0.469412in}%
\pgfsys@useobject{currentmarker}{}%
\end{pgfscope}%
\end{pgfscope}%
\begin{pgfscope}%
\definecolor{textcolor}{rgb}{0.000000,0.000000,0.000000}%
\pgfsetstrokecolor{textcolor}%
\pgfsetfillcolor{textcolor}%
\pgftext[x=1.965377in,y=0.372189in,,top]{\color{textcolor}\rmfamily\fontsize{8.000000}{9.600000}\selectfont \(\displaystyle 60000\)}%
\end{pgfscope}%
\begin{pgfscope}%
\pgfsetbuttcap%
\pgfsetroundjoin%
\definecolor{currentfill}{rgb}{0.000000,0.000000,0.000000}%
\pgfsetfillcolor{currentfill}%
\pgfsetlinewidth{0.803000pt}%
\definecolor{currentstroke}{rgb}{0.000000,0.000000,0.000000}%
\pgfsetstrokecolor{currentstroke}%
\pgfsetdash{}{0pt}%
\pgfsys@defobject{currentmarker}{\pgfqpoint{0.000000in}{-0.048611in}}{\pgfqpoint{0.000000in}{0.000000in}}{%
\pgfpathmoveto{\pgfqpoint{0.000000in}{0.000000in}}%
\pgfpathlineto{\pgfqpoint{0.000000in}{-0.048611in}}%
\pgfusepath{stroke,fill}%
}%
\begin{pgfscope}%
\pgfsys@transformshift{2.379030in}{0.469412in}%
\pgfsys@useobject{currentmarker}{}%
\end{pgfscope}%
\end{pgfscope}%
\begin{pgfscope}%
\definecolor{textcolor}{rgb}{0.000000,0.000000,0.000000}%
\pgfsetstrokecolor{textcolor}%
\pgfsetfillcolor{textcolor}%
\pgftext[x=2.379030in,y=0.372189in,,top]{\color{textcolor}\rmfamily\fontsize{8.000000}{9.600000}\selectfont \(\displaystyle 80000\)}%
\end{pgfscope}%
\begin{pgfscope}%
\pgfsetbuttcap%
\pgfsetroundjoin%
\definecolor{currentfill}{rgb}{0.000000,0.000000,0.000000}%
\pgfsetfillcolor{currentfill}%
\pgfsetlinewidth{0.803000pt}%
\definecolor{currentstroke}{rgb}{0.000000,0.000000,0.000000}%
\pgfsetstrokecolor{currentstroke}%
\pgfsetdash{}{0pt}%
\pgfsys@defobject{currentmarker}{\pgfqpoint{0.000000in}{-0.048611in}}{\pgfqpoint{0.000000in}{0.000000in}}{%
\pgfpathmoveto{\pgfqpoint{0.000000in}{0.000000in}}%
\pgfpathlineto{\pgfqpoint{0.000000in}{-0.048611in}}%
\pgfusepath{stroke,fill}%
}%
\begin{pgfscope}%
\pgfsys@transformshift{2.792683in}{0.469412in}%
\pgfsys@useobject{currentmarker}{}%
\end{pgfscope}%
\end{pgfscope}%
\begin{pgfscope}%
\definecolor{textcolor}{rgb}{0.000000,0.000000,0.000000}%
\pgfsetstrokecolor{textcolor}%
\pgfsetfillcolor{textcolor}%
\pgftext[x=2.792683in,y=0.372189in,,top]{\color{textcolor}\rmfamily\fontsize{8.000000}{9.600000}\selectfont \(\displaystyle 100000\)}%
\end{pgfscope}%
\begin{pgfscope}%
\pgfsetbuttcap%
\pgfsetroundjoin%
\definecolor{currentfill}{rgb}{0.000000,0.000000,0.000000}%
\pgfsetfillcolor{currentfill}%
\pgfsetlinewidth{0.803000pt}%
\definecolor{currentstroke}{rgb}{0.000000,0.000000,0.000000}%
\pgfsetstrokecolor{currentstroke}%
\pgfsetdash{}{0pt}%
\pgfsys@defobject{currentmarker}{\pgfqpoint{0.000000in}{-0.048611in}}{\pgfqpoint{0.000000in}{0.000000in}}{%
\pgfpathmoveto{\pgfqpoint{0.000000in}{0.000000in}}%
\pgfpathlineto{\pgfqpoint{0.000000in}{-0.048611in}}%
\pgfusepath{stroke,fill}%
}%
\begin{pgfscope}%
\pgfsys@transformshift{3.206336in}{0.469412in}%
\pgfsys@useobject{currentmarker}{}%
\end{pgfscope}%
\end{pgfscope}%
\begin{pgfscope}%
\definecolor{textcolor}{rgb}{0.000000,0.000000,0.000000}%
\pgfsetstrokecolor{textcolor}%
\pgfsetfillcolor{textcolor}%
\pgftext[x=3.206336in,y=0.372189in,,top]{\color{textcolor}\rmfamily\fontsize{8.000000}{9.600000}\selectfont \(\displaystyle 120000\)}%
\end{pgfscope}%
\begin{pgfscope}%
\pgfsetbuttcap%
\pgfsetroundjoin%
\definecolor{currentfill}{rgb}{0.000000,0.000000,0.000000}%
\pgfsetfillcolor{currentfill}%
\pgfsetlinewidth{0.803000pt}%
\definecolor{currentstroke}{rgb}{0.000000,0.000000,0.000000}%
\pgfsetstrokecolor{currentstroke}%
\pgfsetdash{}{0pt}%
\pgfsys@defobject{currentmarker}{\pgfqpoint{0.000000in}{-0.048611in}}{\pgfqpoint{0.000000in}{0.000000in}}{%
\pgfpathmoveto{\pgfqpoint{0.000000in}{0.000000in}}%
\pgfpathlineto{\pgfqpoint{0.000000in}{-0.048611in}}%
\pgfusepath{stroke,fill}%
}%
\begin{pgfscope}%
\pgfsys@transformshift{3.619989in}{0.469412in}%
\pgfsys@useobject{currentmarker}{}%
\end{pgfscope}%
\end{pgfscope}%
\begin{pgfscope}%
\definecolor{textcolor}{rgb}{0.000000,0.000000,0.000000}%
\pgfsetstrokecolor{textcolor}%
\pgfsetfillcolor{textcolor}%
\pgftext[x=3.619989in,y=0.372189in,,top]{\color{textcolor}\rmfamily\fontsize{8.000000}{9.600000}\selectfont \(\displaystyle 140000\)}%
\end{pgfscope}%
\begin{pgfscope}%
\definecolor{textcolor}{rgb}{0.000000,0.000000,0.000000}%
\pgfsetstrokecolor{textcolor}%
\pgfsetfillcolor{textcolor}%
\pgftext[x=2.266723in,y=0.209104in,,top]{\color{textcolor}\rmfamily\fontsize{8.000000}{9.600000}\selectfont Time (yr)}%
\end{pgfscope}%
\begin{pgfscope}%
\pgfsetbuttcap%
\pgfsetroundjoin%
\definecolor{currentfill}{rgb}{0.000000,0.000000,0.000000}%
\pgfsetfillcolor{currentfill}%
\pgfsetlinewidth{0.803000pt}%
\definecolor{currentstroke}{rgb}{0.000000,0.000000,0.000000}%
\pgfsetstrokecolor{currentstroke}%
\pgfsetdash{}{0pt}%
\pgfsys@defobject{currentmarker}{\pgfqpoint{-0.048611in}{0.000000in}}{\pgfqpoint{0.000000in}{0.000000in}}{%
\pgfpathmoveto{\pgfqpoint{0.000000in}{0.000000in}}%
\pgfpathlineto{\pgfqpoint{-0.048611in}{0.000000in}}%
\pgfusepath{stroke,fill}%
}%
\begin{pgfscope}%
\pgfsys@transformshift{0.570188in}{0.512789in}%
\pgfsys@useobject{currentmarker}{}%
\end{pgfscope}%
\end{pgfscope}%
\begin{pgfscope}%
\definecolor{textcolor}{rgb}{0.000000,0.000000,0.000000}%
\pgfsetstrokecolor{textcolor}%
\pgfsetfillcolor{textcolor}%
\pgftext[x=0.263086in,y=0.470580in,left,base]{\color{textcolor}\rmfamily\fontsize{8.000000}{9.600000}\selectfont \(\displaystyle 0.00\)}%
\end{pgfscope}%
\begin{pgfscope}%
\pgfsetbuttcap%
\pgfsetroundjoin%
\definecolor{currentfill}{rgb}{0.000000,0.000000,0.000000}%
\pgfsetfillcolor{currentfill}%
\pgfsetlinewidth{0.803000pt}%
\definecolor{currentstroke}{rgb}{0.000000,0.000000,0.000000}%
\pgfsetstrokecolor{currentstroke}%
\pgfsetdash{}{0pt}%
\pgfsys@defobject{currentmarker}{\pgfqpoint{-0.048611in}{0.000000in}}{\pgfqpoint{0.000000in}{0.000000in}}{%
\pgfpathmoveto{\pgfqpoint{0.000000in}{0.000000in}}%
\pgfpathlineto{\pgfqpoint{-0.048611in}{0.000000in}}%
\pgfusepath{stroke,fill}%
}%
\begin{pgfscope}%
\pgfsys@transformshift{0.570188in}{0.805259in}%
\pgfsys@useobject{currentmarker}{}%
\end{pgfscope}%
\end{pgfscope}%
\begin{pgfscope}%
\definecolor{textcolor}{rgb}{0.000000,0.000000,0.000000}%
\pgfsetstrokecolor{textcolor}%
\pgfsetfillcolor{textcolor}%
\pgftext[x=0.263086in,y=0.763049in,left,base]{\color{textcolor}\rmfamily\fontsize{8.000000}{9.600000}\selectfont \(\displaystyle 0.25\)}%
\end{pgfscope}%
\begin{pgfscope}%
\pgfsetbuttcap%
\pgfsetroundjoin%
\definecolor{currentfill}{rgb}{0.000000,0.000000,0.000000}%
\pgfsetfillcolor{currentfill}%
\pgfsetlinewidth{0.803000pt}%
\definecolor{currentstroke}{rgb}{0.000000,0.000000,0.000000}%
\pgfsetstrokecolor{currentstroke}%
\pgfsetdash{}{0pt}%
\pgfsys@defobject{currentmarker}{\pgfqpoint{-0.048611in}{0.000000in}}{\pgfqpoint{0.000000in}{0.000000in}}{%
\pgfpathmoveto{\pgfqpoint{0.000000in}{0.000000in}}%
\pgfpathlineto{\pgfqpoint{-0.048611in}{0.000000in}}%
\pgfusepath{stroke,fill}%
}%
\begin{pgfscope}%
\pgfsys@transformshift{0.570188in}{1.097728in}%
\pgfsys@useobject{currentmarker}{}%
\end{pgfscope}%
\end{pgfscope}%
\begin{pgfscope}%
\definecolor{textcolor}{rgb}{0.000000,0.000000,0.000000}%
\pgfsetstrokecolor{textcolor}%
\pgfsetfillcolor{textcolor}%
\pgftext[x=0.263086in,y=1.055519in,left,base]{\color{textcolor}\rmfamily\fontsize{8.000000}{9.600000}\selectfont \(\displaystyle 0.50\)}%
\end{pgfscope}%
\begin{pgfscope}%
\pgfsetbuttcap%
\pgfsetroundjoin%
\definecolor{currentfill}{rgb}{0.000000,0.000000,0.000000}%
\pgfsetfillcolor{currentfill}%
\pgfsetlinewidth{0.803000pt}%
\definecolor{currentstroke}{rgb}{0.000000,0.000000,0.000000}%
\pgfsetstrokecolor{currentstroke}%
\pgfsetdash{}{0pt}%
\pgfsys@defobject{currentmarker}{\pgfqpoint{-0.048611in}{0.000000in}}{\pgfqpoint{0.000000in}{0.000000in}}{%
\pgfpathmoveto{\pgfqpoint{0.000000in}{0.000000in}}%
\pgfpathlineto{\pgfqpoint{-0.048611in}{0.000000in}}%
\pgfusepath{stroke,fill}%
}%
\begin{pgfscope}%
\pgfsys@transformshift{0.570188in}{1.390198in}%
\pgfsys@useobject{currentmarker}{}%
\end{pgfscope}%
\end{pgfscope}%
\begin{pgfscope}%
\definecolor{textcolor}{rgb}{0.000000,0.000000,0.000000}%
\pgfsetstrokecolor{textcolor}%
\pgfsetfillcolor{textcolor}%
\pgftext[x=0.263086in,y=1.347988in,left,base]{\color{textcolor}\rmfamily\fontsize{8.000000}{9.600000}\selectfont \(\displaystyle 0.75\)}%
\end{pgfscope}%
\begin{pgfscope}%
\definecolor{textcolor}{rgb}{0.000000,0.000000,0.000000}%
\pgfsetstrokecolor{textcolor}%
\pgfsetfillcolor{textcolor}%
\pgftext[x=0.207530in,y=0.946562in,,bottom,rotate=90.000000]{\color{textcolor}\rmfamily\fontsize{8.000000}{9.600000}\selectfont Melt amount (mol)}%
\end{pgfscope}%
\begin{pgfscope}%
\pgfpathrectangle{\pgfqpoint{0.570188in}{0.469412in}}{\pgfqpoint{3.393071in}{0.954301in}}%
\pgfusepath{clip}%
\pgfsetrectcap%
\pgfsetroundjoin%
\pgfsetlinewidth{1.505625pt}%
\definecolor{currentstroke}{rgb}{0.121569,0.466667,0.705882}%
\pgfsetstrokecolor{currentstroke}%
\pgfsetdash{}{0pt}%
\pgfpathmoveto{\pgfqpoint{0.724418in}{0.758440in}}%
\pgfpathlineto{\pgfqpoint{0.746290in}{0.956477in}}%
\pgfpathlineto{\pgfqpoint{0.769568in}{0.951376in}}%
\pgfpathlineto{\pgfqpoint{0.808394in}{0.949072in}}%
\pgfpathlineto{\pgfqpoint{0.870436in}{0.946767in}}%
\pgfpathlineto{\pgfqpoint{0.932361in}{0.945749in}}%
\pgfpathlineto{\pgfqpoint{0.994203in}{0.945913in}}%
\pgfpathlineto{\pgfqpoint{1.055899in}{0.947106in}}%
\pgfpathlineto{\pgfqpoint{1.117471in}{0.949282in}}%
\pgfpathlineto{\pgfqpoint{1.182746in}{0.952628in}}%
\pgfpathlineto{\pgfqpoint{1.244049in}{0.956734in}}%
\pgfpathlineto{\pgfqpoint{1.305290in}{0.961765in}}%
\pgfpathlineto{\pgfqpoint{1.366470in}{0.967661in}}%
\pgfpathlineto{\pgfqpoint{1.431413in}{0.974914in}}%
\pgfpathlineto{\pgfqpoint{1.511600in}{0.985186in}}%
\pgfpathlineto{\pgfqpoint{1.595571in}{0.997446in}}%
\pgfpathlineto{\pgfqpoint{1.679398in}{1.011192in}}%
\pgfpathlineto{\pgfqpoint{1.759068in}{1.025547in}}%
\pgfpathlineto{\pgfqpoint{1.842274in}{1.041808in}}%
\pgfpathlineto{\pgfqpoint{1.925315in}{1.059192in}}%
\pgfpathlineto{\pgfqpoint{2.008335in}{1.077594in}}%
\pgfpathlineto{\pgfqpoint{2.110300in}{1.101366in}}%
\pgfpathlineto{\pgfqpoint{2.234913in}{1.131760in}}%
\pgfpathlineto{\pgfqpoint{2.399899in}{1.173384in}}%
\pgfpathlineto{\pgfqpoint{2.733158in}{1.257803in}}%
\pgfpathlineto{\pgfqpoint{2.877482in}{1.292946in}}%
\pgfpathlineto{\pgfqpoint{2.981515in}{1.317384in}}%
\pgfpathlineto{\pgfqpoint{3.085963in}{1.341063in}}%
\pgfpathlineto{\pgfqpoint{3.209852in}{1.367841in}}%
\pgfpathlineto{\pgfqpoint{3.270038in}{1.380336in}}%
\pgfpathlineto{\pgfqpoint{3.292582in}{0.512789in}}%
\pgfpathlineto{\pgfqpoint{3.809028in}{0.512789in}}%
\pgfpathlineto{\pgfqpoint{3.809028in}{0.512789in}}%
\pgfusepath{stroke}%
\end{pgfscope}%
\begin{pgfscope}%
\pgfsetrectcap%
\pgfsetmiterjoin%
\pgfsetlinewidth{0.803000pt}%
\definecolor{currentstroke}{rgb}{0.000000,0.000000,0.000000}%
\pgfsetstrokecolor{currentstroke}%
\pgfsetdash{}{0pt}%
\pgfpathmoveto{\pgfqpoint{0.570188in}{0.469412in}}%
\pgfpathlineto{\pgfqpoint{0.570188in}{1.423713in}}%
\pgfusepath{stroke}%
\end{pgfscope}%
\begin{pgfscope}%
\pgfsetrectcap%
\pgfsetmiterjoin%
\pgfsetlinewidth{0.803000pt}%
\definecolor{currentstroke}{rgb}{0.000000,0.000000,0.000000}%
\pgfsetstrokecolor{currentstroke}%
\pgfsetdash{}{0pt}%
\pgfpathmoveto{\pgfqpoint{3.963259in}{0.469412in}}%
\pgfpathlineto{\pgfqpoint{3.963259in}{1.423713in}}%
\pgfusepath{stroke}%
\end{pgfscope}%
\begin{pgfscope}%
\pgfsetrectcap%
\pgfsetmiterjoin%
\pgfsetlinewidth{0.803000pt}%
\definecolor{currentstroke}{rgb}{0.000000,0.000000,0.000000}%
\pgfsetstrokecolor{currentstroke}%
\pgfsetdash{}{0pt}%
\pgfpathmoveto{\pgfqpoint{0.570188in}{0.469412in}}%
\pgfpathlineto{\pgfqpoint{3.963259in}{0.469412in}}%
\pgfusepath{stroke}%
\end{pgfscope}%
\begin{pgfscope}%
\pgfsetrectcap%
\pgfsetmiterjoin%
\pgfsetlinewidth{0.803000pt}%
\definecolor{currentstroke}{rgb}{0.000000,0.000000,0.000000}%
\pgfsetstrokecolor{currentstroke}%
\pgfsetdash{}{0pt}%
\pgfpathmoveto{\pgfqpoint{0.570188in}{1.423713in}}%
\pgfpathlineto{\pgfqpoint{3.963259in}{1.423713in}}%
\pgfusepath{stroke}%
\end{pgfscope}%
\end{pgfpicture}%
\makeatother%
\endgroup%

    \caption{Caption}
    \label{fig:decompression_2pp}
\end{figure}

\subsection{Performance analysis}

\begin{figure}
    \centering
    \begin{subfigure}{0.49\columnwidth}
        \centering
        \begin{tikzpicture}
            \pie[hide number, radius=1.3, rotate=45, text=legend] {
                83/ \footnotesize{Update particle properties},
                5.2/ \footnotesize{Assemble temperature system},
                2.8/ \footnotesize{Assemble Stokes system},
                2.3/ \footnotesize{Postprocessing},
                6.7/ \footnotesize{Other}
            }
        \end{tikzpicture}
        \caption{A breakdown of the runtime for the particle property plugin.}
        \label{fig:perf_batch_pie}
    \end{subfigure}
    \hfill
    \begin{subfigure}{0.49\columnwidth}
        \centering
        \begin{tikzpicture}
            \pie[hide number, radius=1.3, rotate=45, text=legend] {
                65/ \footnotesize{Assemble composition system},
                6.6/ \footnotesize{Assemble temperature system},
                4.8/ \footnotesize{Solve Stokes system},
                3.9/ \footnotesize{Solve composition reactions},
                19.7/ \footnotesize{Other}
            }
        \end{tikzpicture}
        \caption{A breakdown of the runtime for the material model plugin.}
        \label{fig:perf_2pp_pie}
    \end{subfigure}
    %
    \caption{
        A comparison of the breakdown of runtime spent in the different elements of ASPECT between the particle property and material model approaches to tracking the composition.
        In both cases the \texttt{simple.dat} data set for Perple\_X has been used and the analysis has been drawn from the closed box experiments.
        The actual runtime comparison between the two approaches is discussed in Section~\ref{sec:???}.
    }
    \label{fig:perf_pie}
\end{figure}

\subsubsection{Load balancing}
\begin{figure}
    \centering
    %% Creator: Matplotlib, PGF backend
%%
%% To include the figure in your LaTeX document, write
%%   \input{<filename>.pgf}
%%
%% Make sure the required packages are loaded in your preamble
%%   \usepackage{pgf}
%%
%% Figures using additional raster images can only be included by \input if
%% they are in the same directory as the main LaTeX file. For loading figures
%% from other directories you can use the `import` package
%%   \usepackage{import}
%% and then include the figures with
%%   \import{<path to file>}{<filename>.pgf}
%%
%% Matplotlib used the following preamble
%%   \usepackage{fontspec}
%%   \setmainfont{DejaVuSerif.ttf}[Path=/home/connor/.local/lib/python3.8/site-packages/matplotlib/mpl-data/fonts/ttf/]
%%   \setsansfont{DejaVuSans.ttf}[Path=/home/connor/.local/lib/python3.8/site-packages/matplotlib/mpl-data/fonts/ttf/]
%%   \setmonofont{DejaVuSansMono.ttf}[Path=/home/connor/.local/lib/python3.8/site-packages/matplotlib/mpl-data/fonts/ttf/]
%%
\begingroup%
\makeatletter%
\begin{pgfpicture}%
\pgfpathrectangle{\pgfpointorigin}{\pgfqpoint{3.924431in}{3.046073in}}%
\pgfusepath{use as bounding box, clip}%
\begin{pgfscope}%
\pgfsetbuttcap%
\pgfsetmiterjoin%
\definecolor{currentfill}{rgb}{1.000000,1.000000,1.000000}%
\pgfsetfillcolor{currentfill}%
\pgfsetlinewidth{0.000000pt}%
\definecolor{currentstroke}{rgb}{1.000000,1.000000,1.000000}%
\pgfsetstrokecolor{currentstroke}%
\pgfsetdash{}{0pt}%
\pgfpathmoveto{\pgfqpoint{0.000000in}{0.000000in}}%
\pgfpathlineto{\pgfqpoint{3.924431in}{0.000000in}}%
\pgfpathlineto{\pgfqpoint{3.924431in}{3.046073in}}%
\pgfpathlineto{\pgfqpoint{0.000000in}{3.046073in}}%
\pgfpathclose%
\pgfusepath{fill}%
\end{pgfscope}%
\begin{pgfscope}%
\pgfsetbuttcap%
\pgfsetmiterjoin%
\definecolor{currentfill}{rgb}{1.000000,1.000000,1.000000}%
\pgfsetfillcolor{currentfill}%
\pgfsetlinewidth{0.000000pt}%
\definecolor{currentstroke}{rgb}{0.000000,0.000000,0.000000}%
\pgfsetstrokecolor{currentstroke}%
\pgfsetstrokeopacity{0.000000}%
\pgfsetdash{}{0pt}%
\pgfpathmoveto{\pgfqpoint{0.537394in}{0.484854in}}%
\pgfpathlineto{\pgfqpoint{3.824431in}{0.484854in}}%
\pgfpathlineto{\pgfqpoint{3.824431in}{2.934227in}}%
\pgfpathlineto{\pgfqpoint{0.537394in}{2.934227in}}%
\pgfpathclose%
\pgfusepath{fill}%
\end{pgfscope}%
\begin{pgfscope}%
\pgfpathrectangle{\pgfqpoint{0.537394in}{0.484854in}}{\pgfqpoint{3.287038in}{2.449373in}}%
\pgfusepath{clip}%
\pgfsetbuttcap%
\pgfsetroundjoin%
\definecolor{currentfill}{rgb}{0.121569,0.466667,0.705882}%
\pgfsetfillcolor{currentfill}%
\pgfsetfillopacity{0.300000}%
\pgfsetlinewidth{0.000000pt}%
\definecolor{currentstroke}{rgb}{0.000000,0.000000,0.000000}%
\pgfsetstrokecolor{currentstroke}%
\pgfsetdash{}{0pt}%
\pgfpathmoveto{\pgfqpoint{0.537394in}{1.567842in}}%
\pgfpathlineto{\pgfqpoint{0.537394in}{1.527356in}}%
\pgfpathlineto{\pgfqpoint{0.756530in}{1.527356in}}%
\pgfpathlineto{\pgfqpoint{0.975666in}{1.527356in}}%
\pgfpathlineto{\pgfqpoint{1.194801in}{1.527356in}}%
\pgfpathlineto{\pgfqpoint{1.413937in}{1.527356in}}%
\pgfpathlineto{\pgfqpoint{1.633073in}{1.486871in}}%
\pgfpathlineto{\pgfqpoint{1.852209in}{1.446385in}}%
\pgfpathlineto{\pgfqpoint{2.071345in}{1.324929in}}%
\pgfpathlineto{\pgfqpoint{2.217215in}{1.162987in}}%
\pgfpathlineto{\pgfqpoint{2.303424in}{1.162987in}}%
\pgfpathlineto{\pgfqpoint{2.365092in}{1.284443in}}%
\pgfpathlineto{\pgfqpoint{2.413246in}{1.243958in}}%
\pgfpathlineto{\pgfqpoint{2.452847in}{1.243958in}}%
\pgfpathlineto{\pgfqpoint{2.486549in}{1.122501in}}%
\pgfpathlineto{\pgfqpoint{2.515976in}{1.001045in}}%
\pgfpathlineto{\pgfqpoint{2.542171in}{0.960559in}}%
\pgfpathlineto{\pgfqpoint{2.565846in}{0.960559in}}%
\pgfpathlineto{\pgfqpoint{2.587505in}{0.960559in}}%
\pgfpathlineto{\pgfqpoint{2.607523in}{0.960559in}}%
\pgfpathlineto{\pgfqpoint{2.626183in}{1.162987in}}%
\pgfpathlineto{\pgfqpoint{2.643572in}{1.122501in}}%
\pgfpathlineto{\pgfqpoint{2.659840in}{1.162987in}}%
\pgfpathlineto{\pgfqpoint{2.675166in}{1.122501in}}%
\pgfpathlineto{\pgfqpoint{2.689692in}{1.122501in}}%
\pgfpathlineto{\pgfqpoint{2.703534in}{1.122501in}}%
\pgfpathlineto{\pgfqpoint{2.716788in}{1.122501in}}%
\pgfpathlineto{\pgfqpoint{2.729536in}{0.960559in}}%
\pgfpathlineto{\pgfqpoint{2.741843in}{0.879588in}}%
\pgfpathlineto{\pgfqpoint{2.753768in}{0.879588in}}%
\pgfpathlineto{\pgfqpoint{2.765360in}{0.960559in}}%
\pgfpathlineto{\pgfqpoint{2.776662in}{0.758131in}}%
\pgfpathlineto{\pgfqpoint{2.787712in}{1.122501in}}%
\pgfpathlineto{\pgfqpoint{2.798544in}{1.041530in}}%
\pgfpathlineto{\pgfqpoint{2.809184in}{0.960559in}}%
\pgfpathlineto{\pgfqpoint{2.819657in}{0.920073in}}%
\pgfpathlineto{\pgfqpoint{2.829990in}{0.839102in}}%
\pgfpathlineto{\pgfqpoint{2.840204in}{0.798617in}}%
\pgfpathlineto{\pgfqpoint{2.850323in}{0.798617in}}%
\pgfpathlineto{\pgfqpoint{2.860365in}{0.879588in}}%
\pgfpathlineto{\pgfqpoint{2.870349in}{1.041530in}}%
\pgfpathlineto{\pgfqpoint{2.880294in}{1.122501in}}%
\pgfpathlineto{\pgfqpoint{2.890216in}{1.203472in}}%
\pgfpathlineto{\pgfqpoint{2.900131in}{1.203472in}}%
\pgfpathlineto{\pgfqpoint{2.910056in}{1.324929in}}%
\pgfpathlineto{\pgfqpoint{2.920006in}{1.243958in}}%
\pgfpathlineto{\pgfqpoint{2.929996in}{1.041530in}}%
\pgfpathlineto{\pgfqpoint{2.940040in}{0.960559in}}%
\pgfpathlineto{\pgfqpoint{2.950155in}{1.162987in}}%
\pgfpathlineto{\pgfqpoint{2.960354in}{1.122501in}}%
\pgfpathlineto{\pgfqpoint{2.970653in}{1.082016in}}%
\pgfpathlineto{\pgfqpoint{2.981067in}{1.041530in}}%
\pgfpathlineto{\pgfqpoint{2.991611in}{1.082016in}}%
\pgfpathlineto{\pgfqpoint{3.002300in}{0.839102in}}%
\pgfpathlineto{\pgfqpoint{3.013152in}{0.839102in}}%
\pgfpathlineto{\pgfqpoint{3.024182in}{0.677160in}}%
\pgfpathlineto{\pgfqpoint{3.035406in}{0.717646in}}%
\pgfpathlineto{\pgfqpoint{3.046826in}{0.636675in}}%
\pgfpathlineto{\pgfqpoint{3.058134in}{0.636675in}}%
\pgfpathlineto{\pgfqpoint{3.069290in}{0.717646in}}%
\pgfpathlineto{\pgfqpoint{3.080344in}{0.798617in}}%
\pgfpathlineto{\pgfqpoint{3.091337in}{1.082016in}}%
\pgfpathlineto{\pgfqpoint{3.102311in}{1.041530in}}%
\pgfpathlineto{\pgfqpoint{3.113303in}{1.082016in}}%
\pgfpathlineto{\pgfqpoint{3.124347in}{1.082016in}}%
\pgfpathlineto{\pgfqpoint{3.135474in}{0.960559in}}%
\pgfpathlineto{\pgfqpoint{3.146717in}{0.920073in}}%
\pgfpathlineto{\pgfqpoint{3.158082in}{0.879588in}}%
\pgfpathlineto{\pgfqpoint{3.169596in}{0.758131in}}%
\pgfpathlineto{\pgfqpoint{3.181287in}{0.798617in}}%
\pgfpathlineto{\pgfqpoint{3.193182in}{0.798617in}}%
\pgfpathlineto{\pgfqpoint{3.205304in}{0.879588in}}%
\pgfpathlineto{\pgfqpoint{3.217678in}{0.879588in}}%
\pgfpathlineto{\pgfqpoint{3.230327in}{0.920073in}}%
\pgfpathlineto{\pgfqpoint{3.243273in}{0.879588in}}%
\pgfpathlineto{\pgfqpoint{3.256537in}{0.960559in}}%
\pgfpathlineto{\pgfqpoint{3.270140in}{0.798617in}}%
\pgfpathlineto{\pgfqpoint{3.284100in}{0.798617in}}%
\pgfpathlineto{\pgfqpoint{3.298435in}{0.879588in}}%
\pgfpathlineto{\pgfqpoint{3.313164in}{0.758131in}}%
\pgfpathlineto{\pgfqpoint{3.328301in}{0.596189in}}%
\pgfpathlineto{\pgfqpoint{3.343863in}{0.677160in}}%
\pgfpathlineto{\pgfqpoint{3.359865in}{0.879588in}}%
\pgfpathlineto{\pgfqpoint{3.376321in}{0.879588in}}%
\pgfpathlineto{\pgfqpoint{3.393243in}{1.001045in}}%
\pgfpathlineto{\pgfqpoint{3.410646in}{1.122501in}}%
\pgfpathlineto{\pgfqpoint{3.428543in}{1.001045in}}%
\pgfpathlineto{\pgfqpoint{3.446946in}{1.001045in}}%
\pgfpathlineto{\pgfqpoint{3.465870in}{1.082016in}}%
\pgfpathlineto{\pgfqpoint{3.485327in}{0.960559in}}%
\pgfpathlineto{\pgfqpoint{3.505319in}{0.879588in}}%
\pgfpathlineto{\pgfqpoint{3.525848in}{0.717646in}}%
\pgfpathlineto{\pgfqpoint{3.546928in}{0.717646in}}%
\pgfpathlineto{\pgfqpoint{3.568573in}{0.677160in}}%
\pgfpathlineto{\pgfqpoint{3.590798in}{0.717646in}}%
\pgfpathlineto{\pgfqpoint{3.613621in}{0.717646in}}%
\pgfpathlineto{\pgfqpoint{3.637059in}{0.879588in}}%
\pgfpathlineto{\pgfqpoint{3.661130in}{0.960559in}}%
\pgfpathlineto{\pgfqpoint{3.685855in}{0.920073in}}%
\pgfpathlineto{\pgfqpoint{3.711237in}{0.879588in}}%
\pgfpathlineto{\pgfqpoint{3.737282in}{0.920073in}}%
\pgfpathlineto{\pgfqpoint{3.764011in}{0.839102in}}%
\pgfpathlineto{\pgfqpoint{3.791445in}{0.717646in}}%
\pgfpathlineto{\pgfqpoint{3.819606in}{0.596189in}}%
\pgfpathlineto{\pgfqpoint{3.824431in}{0.596189in}}%
\pgfpathlineto{\pgfqpoint{3.824431in}{2.620465in}}%
\pgfpathlineto{\pgfqpoint{3.824431in}{2.620465in}}%
\pgfpathlineto{\pgfqpoint{3.819606in}{2.579979in}}%
\pgfpathlineto{\pgfqpoint{3.791445in}{2.458523in}}%
\pgfpathlineto{\pgfqpoint{3.764011in}{2.418037in}}%
\pgfpathlineto{\pgfqpoint{3.737282in}{2.377552in}}%
\pgfpathlineto{\pgfqpoint{3.711237in}{2.499008in}}%
\pgfpathlineto{\pgfqpoint{3.685855in}{2.579979in}}%
\pgfpathlineto{\pgfqpoint{3.661130in}{2.822892in}}%
\pgfpathlineto{\pgfqpoint{3.637059in}{2.822892in}}%
\pgfpathlineto{\pgfqpoint{3.613621in}{2.458523in}}%
\pgfpathlineto{\pgfqpoint{3.590798in}{2.337066in}}%
\pgfpathlineto{\pgfqpoint{3.568573in}{2.256095in}}%
\pgfpathlineto{\pgfqpoint{3.546928in}{2.013182in}}%
\pgfpathlineto{\pgfqpoint{3.525848in}{2.053668in}}%
\pgfpathlineto{\pgfqpoint{3.505319in}{2.094153in}}%
\pgfpathlineto{\pgfqpoint{3.485327in}{1.972697in}}%
\pgfpathlineto{\pgfqpoint{3.465870in}{2.013182in}}%
\pgfpathlineto{\pgfqpoint{3.446946in}{2.013182in}}%
\pgfpathlineto{\pgfqpoint{3.428543in}{2.215610in}}%
\pgfpathlineto{\pgfqpoint{3.410646in}{2.215610in}}%
\pgfpathlineto{\pgfqpoint{3.393243in}{2.134639in}}%
\pgfpathlineto{\pgfqpoint{3.376321in}{2.094153in}}%
\pgfpathlineto{\pgfqpoint{3.359865in}{2.256095in}}%
\pgfpathlineto{\pgfqpoint{3.343863in}{2.134639in}}%
\pgfpathlineto{\pgfqpoint{3.328301in}{2.215610in}}%
\pgfpathlineto{\pgfqpoint{3.313164in}{2.134639in}}%
\pgfpathlineto{\pgfqpoint{3.298435in}{2.013182in}}%
\pgfpathlineto{\pgfqpoint{3.284100in}{2.134639in}}%
\pgfpathlineto{\pgfqpoint{3.270140in}{2.053668in}}%
\pgfpathlineto{\pgfqpoint{3.256537in}{2.175124in}}%
\pgfpathlineto{\pgfqpoint{3.243273in}{2.215610in}}%
\pgfpathlineto{\pgfqpoint{3.230327in}{2.175124in}}%
\pgfpathlineto{\pgfqpoint{3.217678in}{2.296581in}}%
\pgfpathlineto{\pgfqpoint{3.205304in}{2.296581in}}%
\pgfpathlineto{\pgfqpoint{3.193182in}{2.134639in}}%
\pgfpathlineto{\pgfqpoint{3.181287in}{2.094153in}}%
\pgfpathlineto{\pgfqpoint{3.169596in}{2.094153in}}%
\pgfpathlineto{\pgfqpoint{3.158082in}{1.972697in}}%
\pgfpathlineto{\pgfqpoint{3.146717in}{1.851240in}}%
\pgfpathlineto{\pgfqpoint{3.135474in}{1.891726in}}%
\pgfpathlineto{\pgfqpoint{3.124347in}{1.851240in}}%
\pgfpathlineto{\pgfqpoint{3.113303in}{2.053668in}}%
\pgfpathlineto{\pgfqpoint{3.102311in}{2.094153in}}%
\pgfpathlineto{\pgfqpoint{3.091337in}{2.094153in}}%
\pgfpathlineto{\pgfqpoint{3.080344in}{2.094153in}}%
\pgfpathlineto{\pgfqpoint{3.069290in}{2.094153in}}%
\pgfpathlineto{\pgfqpoint{3.058134in}{2.215610in}}%
\pgfpathlineto{\pgfqpoint{3.046826in}{2.296581in}}%
\pgfpathlineto{\pgfqpoint{3.035406in}{2.134639in}}%
\pgfpathlineto{\pgfqpoint{3.024182in}{2.215610in}}%
\pgfpathlineto{\pgfqpoint{3.013152in}{1.972697in}}%
\pgfpathlineto{\pgfqpoint{3.002300in}{1.972697in}}%
\pgfpathlineto{\pgfqpoint{2.991611in}{2.134639in}}%
\pgfpathlineto{\pgfqpoint{2.981067in}{2.215610in}}%
\pgfpathlineto{\pgfqpoint{2.970653in}{2.013182in}}%
\pgfpathlineto{\pgfqpoint{2.960354in}{2.013182in}}%
\pgfpathlineto{\pgfqpoint{2.950155in}{2.013182in}}%
\pgfpathlineto{\pgfqpoint{2.940040in}{1.932211in}}%
\pgfpathlineto{\pgfqpoint{2.929996in}{1.932211in}}%
\pgfpathlineto{\pgfqpoint{2.920006in}{1.851240in}}%
\pgfpathlineto{\pgfqpoint{2.910056in}{2.013182in}}%
\pgfpathlineto{\pgfqpoint{2.900131in}{2.053668in}}%
\pgfpathlineto{\pgfqpoint{2.890216in}{2.053668in}}%
\pgfpathlineto{\pgfqpoint{2.880294in}{2.094153in}}%
\pgfpathlineto{\pgfqpoint{2.870349in}{2.175124in}}%
\pgfpathlineto{\pgfqpoint{2.860365in}{2.296581in}}%
\pgfpathlineto{\pgfqpoint{2.850323in}{2.377552in}}%
\pgfpathlineto{\pgfqpoint{2.840204in}{2.256095in}}%
\pgfpathlineto{\pgfqpoint{2.829990in}{2.256095in}}%
\pgfpathlineto{\pgfqpoint{2.819657in}{2.175124in}}%
\pgfpathlineto{\pgfqpoint{2.809184in}{2.418037in}}%
\pgfpathlineto{\pgfqpoint{2.798544in}{2.296581in}}%
\pgfpathlineto{\pgfqpoint{2.787712in}{2.175124in}}%
\pgfpathlineto{\pgfqpoint{2.776662in}{2.134639in}}%
\pgfpathlineto{\pgfqpoint{2.765360in}{2.175124in}}%
\pgfpathlineto{\pgfqpoint{2.753768in}{2.296581in}}%
\pgfpathlineto{\pgfqpoint{2.741843in}{2.215610in}}%
\pgfpathlineto{\pgfqpoint{2.729536in}{2.134639in}}%
\pgfpathlineto{\pgfqpoint{2.716788in}{2.175124in}}%
\pgfpathlineto{\pgfqpoint{2.703534in}{2.175124in}}%
\pgfpathlineto{\pgfqpoint{2.689692in}{2.215610in}}%
\pgfpathlineto{\pgfqpoint{2.675166in}{2.215610in}}%
\pgfpathlineto{\pgfqpoint{2.659840in}{2.215610in}}%
\pgfpathlineto{\pgfqpoint{2.643572in}{2.134639in}}%
\pgfpathlineto{\pgfqpoint{2.626183in}{2.134639in}}%
\pgfpathlineto{\pgfqpoint{2.607523in}{2.215610in}}%
\pgfpathlineto{\pgfqpoint{2.587505in}{2.134639in}}%
\pgfpathlineto{\pgfqpoint{2.565846in}{2.053668in}}%
\pgfpathlineto{\pgfqpoint{2.542171in}{2.013182in}}%
\pgfpathlineto{\pgfqpoint{2.515976in}{2.094153in}}%
\pgfpathlineto{\pgfqpoint{2.486549in}{2.053668in}}%
\pgfpathlineto{\pgfqpoint{2.452847in}{1.972697in}}%
\pgfpathlineto{\pgfqpoint{2.413246in}{1.770269in}}%
\pgfpathlineto{\pgfqpoint{2.365092in}{1.729784in}}%
\pgfpathlineto{\pgfqpoint{2.303424in}{1.810755in}}%
\pgfpathlineto{\pgfqpoint{2.217215in}{1.770269in}}%
\pgfpathlineto{\pgfqpoint{2.071345in}{1.729784in}}%
\pgfpathlineto{\pgfqpoint{1.852209in}{1.648813in}}%
\pgfpathlineto{\pgfqpoint{1.633073in}{1.689298in}}%
\pgfpathlineto{\pgfqpoint{1.413937in}{1.608327in}}%
\pgfpathlineto{\pgfqpoint{1.194801in}{1.608327in}}%
\pgfpathlineto{\pgfqpoint{0.975666in}{1.608327in}}%
\pgfpathlineto{\pgfqpoint{0.756530in}{1.567842in}}%
\pgfpathlineto{\pgfqpoint{0.537394in}{1.567842in}}%
\pgfpathclose%
\pgfusepath{fill}%
\end{pgfscope}%
\begin{pgfscope}%
\pgfsetbuttcap%
\pgfsetroundjoin%
\definecolor{currentfill}{rgb}{0.000000,0.000000,0.000000}%
\pgfsetfillcolor{currentfill}%
\pgfsetlinewidth{0.803000pt}%
\definecolor{currentstroke}{rgb}{0.000000,0.000000,0.000000}%
\pgfsetstrokecolor{currentstroke}%
\pgfsetdash{}{0pt}%
\pgfsys@defobject{currentmarker}{\pgfqpoint{0.000000in}{-0.048611in}}{\pgfqpoint{0.000000in}{0.000000in}}{%
\pgfpathmoveto{\pgfqpoint{0.000000in}{0.000000in}}%
\pgfpathlineto{\pgfqpoint{0.000000in}{-0.048611in}}%
\pgfusepath{stroke,fill}%
}%
\begin{pgfscope}%
\pgfsys@transformshift{0.537394in}{0.484854in}%
\pgfsys@useobject{currentmarker}{}%
\end{pgfscope}%
\end{pgfscope}%
\begin{pgfscope}%
\definecolor{textcolor}{rgb}{0.000000,0.000000,0.000000}%
\pgfsetstrokecolor{textcolor}%
\pgfsetfillcolor{textcolor}%
\pgftext[x=0.537394in,y=0.387632in,,top]{\color{textcolor}\rmfamily\fontsize{8.000000}{9.600000}\selectfont \(\displaystyle 0.0\)}%
\end{pgfscope}%
\begin{pgfscope}%
\pgfsetbuttcap%
\pgfsetroundjoin%
\definecolor{currentfill}{rgb}{0.000000,0.000000,0.000000}%
\pgfsetfillcolor{currentfill}%
\pgfsetlinewidth{0.803000pt}%
\definecolor{currentstroke}{rgb}{0.000000,0.000000,0.000000}%
\pgfsetstrokecolor{currentstroke}%
\pgfsetdash{}{0pt}%
\pgfsys@defobject{currentmarker}{\pgfqpoint{0.000000in}{-0.048611in}}{\pgfqpoint{0.000000in}{0.000000in}}{%
\pgfpathmoveto{\pgfqpoint{0.000000in}{0.000000in}}%
\pgfpathlineto{\pgfqpoint{0.000000in}{-0.048611in}}%
\pgfusepath{stroke,fill}%
}%
\begin{pgfscope}%
\pgfsys@transformshift{0.975666in}{0.484854in}%
\pgfsys@useobject{currentmarker}{}%
\end{pgfscope}%
\end{pgfscope}%
\begin{pgfscope}%
\definecolor{textcolor}{rgb}{0.000000,0.000000,0.000000}%
\pgfsetstrokecolor{textcolor}%
\pgfsetfillcolor{textcolor}%
\pgftext[x=0.975666in,y=0.387632in,,top]{\color{textcolor}\rmfamily\fontsize{8.000000}{9.600000}\selectfont \(\displaystyle 0.2\)}%
\end{pgfscope}%
\begin{pgfscope}%
\pgfsetbuttcap%
\pgfsetroundjoin%
\definecolor{currentfill}{rgb}{0.000000,0.000000,0.000000}%
\pgfsetfillcolor{currentfill}%
\pgfsetlinewidth{0.803000pt}%
\definecolor{currentstroke}{rgb}{0.000000,0.000000,0.000000}%
\pgfsetstrokecolor{currentstroke}%
\pgfsetdash{}{0pt}%
\pgfsys@defobject{currentmarker}{\pgfqpoint{0.000000in}{-0.048611in}}{\pgfqpoint{0.000000in}{0.000000in}}{%
\pgfpathmoveto{\pgfqpoint{0.000000in}{0.000000in}}%
\pgfpathlineto{\pgfqpoint{0.000000in}{-0.048611in}}%
\pgfusepath{stroke,fill}%
}%
\begin{pgfscope}%
\pgfsys@transformshift{1.413937in}{0.484854in}%
\pgfsys@useobject{currentmarker}{}%
\end{pgfscope}%
\end{pgfscope}%
\begin{pgfscope}%
\definecolor{textcolor}{rgb}{0.000000,0.000000,0.000000}%
\pgfsetstrokecolor{textcolor}%
\pgfsetfillcolor{textcolor}%
\pgftext[x=1.413937in,y=0.387632in,,top]{\color{textcolor}\rmfamily\fontsize{8.000000}{9.600000}\selectfont \(\displaystyle 0.4\)}%
\end{pgfscope}%
\begin{pgfscope}%
\pgfsetbuttcap%
\pgfsetroundjoin%
\definecolor{currentfill}{rgb}{0.000000,0.000000,0.000000}%
\pgfsetfillcolor{currentfill}%
\pgfsetlinewidth{0.803000pt}%
\definecolor{currentstroke}{rgb}{0.000000,0.000000,0.000000}%
\pgfsetstrokecolor{currentstroke}%
\pgfsetdash{}{0pt}%
\pgfsys@defobject{currentmarker}{\pgfqpoint{0.000000in}{-0.048611in}}{\pgfqpoint{0.000000in}{0.000000in}}{%
\pgfpathmoveto{\pgfqpoint{0.000000in}{0.000000in}}%
\pgfpathlineto{\pgfqpoint{0.000000in}{-0.048611in}}%
\pgfusepath{stroke,fill}%
}%
\begin{pgfscope}%
\pgfsys@transformshift{1.852209in}{0.484854in}%
\pgfsys@useobject{currentmarker}{}%
\end{pgfscope}%
\end{pgfscope}%
\begin{pgfscope}%
\definecolor{textcolor}{rgb}{0.000000,0.000000,0.000000}%
\pgfsetstrokecolor{textcolor}%
\pgfsetfillcolor{textcolor}%
\pgftext[x=1.852209in,y=0.387632in,,top]{\color{textcolor}\rmfamily\fontsize{8.000000}{9.600000}\selectfont \(\displaystyle 0.6\)}%
\end{pgfscope}%
\begin{pgfscope}%
\pgfsetbuttcap%
\pgfsetroundjoin%
\definecolor{currentfill}{rgb}{0.000000,0.000000,0.000000}%
\pgfsetfillcolor{currentfill}%
\pgfsetlinewidth{0.803000pt}%
\definecolor{currentstroke}{rgb}{0.000000,0.000000,0.000000}%
\pgfsetstrokecolor{currentstroke}%
\pgfsetdash{}{0pt}%
\pgfsys@defobject{currentmarker}{\pgfqpoint{0.000000in}{-0.048611in}}{\pgfqpoint{0.000000in}{0.000000in}}{%
\pgfpathmoveto{\pgfqpoint{0.000000in}{0.000000in}}%
\pgfpathlineto{\pgfqpoint{0.000000in}{-0.048611in}}%
\pgfusepath{stroke,fill}%
}%
\begin{pgfscope}%
\pgfsys@transformshift{2.290481in}{0.484854in}%
\pgfsys@useobject{currentmarker}{}%
\end{pgfscope}%
\end{pgfscope}%
\begin{pgfscope}%
\definecolor{textcolor}{rgb}{0.000000,0.000000,0.000000}%
\pgfsetstrokecolor{textcolor}%
\pgfsetfillcolor{textcolor}%
\pgftext[x=2.290481in,y=0.387632in,,top]{\color{textcolor}\rmfamily\fontsize{8.000000}{9.600000}\selectfont \(\displaystyle 0.8\)}%
\end{pgfscope}%
\begin{pgfscope}%
\pgfsetbuttcap%
\pgfsetroundjoin%
\definecolor{currentfill}{rgb}{0.000000,0.000000,0.000000}%
\pgfsetfillcolor{currentfill}%
\pgfsetlinewidth{0.803000pt}%
\definecolor{currentstroke}{rgb}{0.000000,0.000000,0.000000}%
\pgfsetstrokecolor{currentstroke}%
\pgfsetdash{}{0pt}%
\pgfsys@defobject{currentmarker}{\pgfqpoint{0.000000in}{-0.048611in}}{\pgfqpoint{0.000000in}{0.000000in}}{%
\pgfpathmoveto{\pgfqpoint{0.000000in}{0.000000in}}%
\pgfpathlineto{\pgfqpoint{0.000000in}{-0.048611in}}%
\pgfusepath{stroke,fill}%
}%
\begin{pgfscope}%
\pgfsys@transformshift{2.728752in}{0.484854in}%
\pgfsys@useobject{currentmarker}{}%
\end{pgfscope}%
\end{pgfscope}%
\begin{pgfscope}%
\definecolor{textcolor}{rgb}{0.000000,0.000000,0.000000}%
\pgfsetstrokecolor{textcolor}%
\pgfsetfillcolor{textcolor}%
\pgftext[x=2.728752in,y=0.387632in,,top]{\color{textcolor}\rmfamily\fontsize{8.000000}{9.600000}\selectfont \(\displaystyle 1.0\)}%
\end{pgfscope}%
\begin{pgfscope}%
\pgfsetbuttcap%
\pgfsetroundjoin%
\definecolor{currentfill}{rgb}{0.000000,0.000000,0.000000}%
\pgfsetfillcolor{currentfill}%
\pgfsetlinewidth{0.803000pt}%
\definecolor{currentstroke}{rgb}{0.000000,0.000000,0.000000}%
\pgfsetstrokecolor{currentstroke}%
\pgfsetdash{}{0pt}%
\pgfsys@defobject{currentmarker}{\pgfqpoint{0.000000in}{-0.048611in}}{\pgfqpoint{0.000000in}{0.000000in}}{%
\pgfpathmoveto{\pgfqpoint{0.000000in}{0.000000in}}%
\pgfpathlineto{\pgfqpoint{0.000000in}{-0.048611in}}%
\pgfusepath{stroke,fill}%
}%
\begin{pgfscope}%
\pgfsys@transformshift{3.167024in}{0.484854in}%
\pgfsys@useobject{currentmarker}{}%
\end{pgfscope}%
\end{pgfscope}%
\begin{pgfscope}%
\definecolor{textcolor}{rgb}{0.000000,0.000000,0.000000}%
\pgfsetstrokecolor{textcolor}%
\pgfsetfillcolor{textcolor}%
\pgftext[x=3.167024in,y=0.387632in,,top]{\color{textcolor}\rmfamily\fontsize{8.000000}{9.600000}\selectfont \(\displaystyle 1.2\)}%
\end{pgfscope}%
\begin{pgfscope}%
\pgfsetbuttcap%
\pgfsetroundjoin%
\definecolor{currentfill}{rgb}{0.000000,0.000000,0.000000}%
\pgfsetfillcolor{currentfill}%
\pgfsetlinewidth{0.803000pt}%
\definecolor{currentstroke}{rgb}{0.000000,0.000000,0.000000}%
\pgfsetstrokecolor{currentstroke}%
\pgfsetdash{}{0pt}%
\pgfsys@defobject{currentmarker}{\pgfqpoint{0.000000in}{-0.048611in}}{\pgfqpoint{0.000000in}{0.000000in}}{%
\pgfpathmoveto{\pgfqpoint{0.000000in}{0.000000in}}%
\pgfpathlineto{\pgfqpoint{0.000000in}{-0.048611in}}%
\pgfusepath{stroke,fill}%
}%
\begin{pgfscope}%
\pgfsys@transformshift{3.605296in}{0.484854in}%
\pgfsys@useobject{currentmarker}{}%
\end{pgfscope}%
\end{pgfscope}%
\begin{pgfscope}%
\definecolor{textcolor}{rgb}{0.000000,0.000000,0.000000}%
\pgfsetstrokecolor{textcolor}%
\pgfsetfillcolor{textcolor}%
\pgftext[x=3.605296in,y=0.387632in,,top]{\color{textcolor}\rmfamily\fontsize{8.000000}{9.600000}\selectfont \(\displaystyle 1.4\)}%
\end{pgfscope}%
\begin{pgfscope}%
\definecolor{textcolor}{rgb}{0.000000,0.000000,0.000000}%
\pgfsetstrokecolor{textcolor}%
\pgfsetfillcolor{textcolor}%
\pgftext[x=2.180913in,y=0.224546in,,top]{\color{textcolor}\rmfamily\fontsize{8.000000}{9.600000}\selectfont Time (\(\displaystyle \times 10^6 \, \mathrm{yr}\))}%
\end{pgfscope}%
\begin{pgfscope}%
\pgfsetbuttcap%
\pgfsetroundjoin%
\definecolor{currentfill}{rgb}{0.000000,0.000000,0.000000}%
\pgfsetfillcolor{currentfill}%
\pgfsetlinewidth{0.803000pt}%
\definecolor{currentstroke}{rgb}{0.000000,0.000000,0.000000}%
\pgfsetstrokecolor{currentstroke}%
\pgfsetdash{}{0pt}%
\pgfsys@defobject{currentmarker}{\pgfqpoint{-0.048611in}{0.000000in}}{\pgfqpoint{0.000000in}{0.000000in}}{%
\pgfpathmoveto{\pgfqpoint{0.000000in}{0.000000in}}%
\pgfpathlineto{\pgfqpoint{-0.048611in}{0.000000in}}%
\pgfusepath{stroke,fill}%
}%
\begin{pgfscope}%
\pgfsys@transformshift{0.537394in}{0.879588in}%
\pgfsys@useobject{currentmarker}{}%
\end{pgfscope}%
\end{pgfscope}%
\begin{pgfscope}%
\definecolor{textcolor}{rgb}{0.000000,0.000000,0.000000}%
\pgfsetstrokecolor{textcolor}%
\pgfsetfillcolor{textcolor}%
\pgftext[x=0.263086in,y=0.837379in,left,base]{\color{textcolor}\rmfamily\fontsize{8.000000}{9.600000}\selectfont \(\displaystyle 150\)}%
\end{pgfscope}%
\begin{pgfscope}%
\pgfsetbuttcap%
\pgfsetroundjoin%
\definecolor{currentfill}{rgb}{0.000000,0.000000,0.000000}%
\pgfsetfillcolor{currentfill}%
\pgfsetlinewidth{0.803000pt}%
\definecolor{currentstroke}{rgb}{0.000000,0.000000,0.000000}%
\pgfsetstrokecolor{currentstroke}%
\pgfsetdash{}{0pt}%
\pgfsys@defobject{currentmarker}{\pgfqpoint{-0.048611in}{0.000000in}}{\pgfqpoint{0.000000in}{0.000000in}}{%
\pgfpathmoveto{\pgfqpoint{0.000000in}{0.000000in}}%
\pgfpathlineto{\pgfqpoint{-0.048611in}{0.000000in}}%
\pgfusepath{stroke,fill}%
}%
\begin{pgfscope}%
\pgfsys@transformshift{0.537394in}{1.284443in}%
\pgfsys@useobject{currentmarker}{}%
\end{pgfscope}%
\end{pgfscope}%
\begin{pgfscope}%
\definecolor{textcolor}{rgb}{0.000000,0.000000,0.000000}%
\pgfsetstrokecolor{textcolor}%
\pgfsetfillcolor{textcolor}%
\pgftext[x=0.263086in,y=1.242234in,left,base]{\color{textcolor}\rmfamily\fontsize{8.000000}{9.600000}\selectfont \(\displaystyle 160\)}%
\end{pgfscope}%
\begin{pgfscope}%
\pgfsetbuttcap%
\pgfsetroundjoin%
\definecolor{currentfill}{rgb}{0.000000,0.000000,0.000000}%
\pgfsetfillcolor{currentfill}%
\pgfsetlinewidth{0.803000pt}%
\definecolor{currentstroke}{rgb}{0.000000,0.000000,0.000000}%
\pgfsetstrokecolor{currentstroke}%
\pgfsetdash{}{0pt}%
\pgfsys@defobject{currentmarker}{\pgfqpoint{-0.048611in}{0.000000in}}{\pgfqpoint{0.000000in}{0.000000in}}{%
\pgfpathmoveto{\pgfqpoint{0.000000in}{0.000000in}}%
\pgfpathlineto{\pgfqpoint{-0.048611in}{0.000000in}}%
\pgfusepath{stroke,fill}%
}%
\begin{pgfscope}%
\pgfsys@transformshift{0.537394in}{1.689298in}%
\pgfsys@useobject{currentmarker}{}%
\end{pgfscope}%
\end{pgfscope}%
\begin{pgfscope}%
\definecolor{textcolor}{rgb}{0.000000,0.000000,0.000000}%
\pgfsetstrokecolor{textcolor}%
\pgfsetfillcolor{textcolor}%
\pgftext[x=0.263086in,y=1.647089in,left,base]{\color{textcolor}\rmfamily\fontsize{8.000000}{9.600000}\selectfont \(\displaystyle 170\)}%
\end{pgfscope}%
\begin{pgfscope}%
\pgfsetbuttcap%
\pgfsetroundjoin%
\definecolor{currentfill}{rgb}{0.000000,0.000000,0.000000}%
\pgfsetfillcolor{currentfill}%
\pgfsetlinewidth{0.803000pt}%
\definecolor{currentstroke}{rgb}{0.000000,0.000000,0.000000}%
\pgfsetstrokecolor{currentstroke}%
\pgfsetdash{}{0pt}%
\pgfsys@defobject{currentmarker}{\pgfqpoint{-0.048611in}{0.000000in}}{\pgfqpoint{0.000000in}{0.000000in}}{%
\pgfpathmoveto{\pgfqpoint{0.000000in}{0.000000in}}%
\pgfpathlineto{\pgfqpoint{-0.048611in}{0.000000in}}%
\pgfusepath{stroke,fill}%
}%
\begin{pgfscope}%
\pgfsys@transformshift{0.537394in}{2.094153in}%
\pgfsys@useobject{currentmarker}{}%
\end{pgfscope}%
\end{pgfscope}%
\begin{pgfscope}%
\definecolor{textcolor}{rgb}{0.000000,0.000000,0.000000}%
\pgfsetstrokecolor{textcolor}%
\pgfsetfillcolor{textcolor}%
\pgftext[x=0.263086in,y=2.051944in,left,base]{\color{textcolor}\rmfamily\fontsize{8.000000}{9.600000}\selectfont \(\displaystyle 180\)}%
\end{pgfscope}%
\begin{pgfscope}%
\pgfsetbuttcap%
\pgfsetroundjoin%
\definecolor{currentfill}{rgb}{0.000000,0.000000,0.000000}%
\pgfsetfillcolor{currentfill}%
\pgfsetlinewidth{0.803000pt}%
\definecolor{currentstroke}{rgb}{0.000000,0.000000,0.000000}%
\pgfsetstrokecolor{currentstroke}%
\pgfsetdash{}{0pt}%
\pgfsys@defobject{currentmarker}{\pgfqpoint{-0.048611in}{0.000000in}}{\pgfqpoint{0.000000in}{0.000000in}}{%
\pgfpathmoveto{\pgfqpoint{0.000000in}{0.000000in}}%
\pgfpathlineto{\pgfqpoint{-0.048611in}{0.000000in}}%
\pgfusepath{stroke,fill}%
}%
\begin{pgfscope}%
\pgfsys@transformshift{0.537394in}{2.499008in}%
\pgfsys@useobject{currentmarker}{}%
\end{pgfscope}%
\end{pgfscope}%
\begin{pgfscope}%
\definecolor{textcolor}{rgb}{0.000000,0.000000,0.000000}%
\pgfsetstrokecolor{textcolor}%
\pgfsetfillcolor{textcolor}%
\pgftext[x=0.263086in,y=2.456799in,left,base]{\color{textcolor}\rmfamily\fontsize{8.000000}{9.600000}\selectfont \(\displaystyle 190\)}%
\end{pgfscope}%
\begin{pgfscope}%
\pgfsetbuttcap%
\pgfsetroundjoin%
\definecolor{currentfill}{rgb}{0.000000,0.000000,0.000000}%
\pgfsetfillcolor{currentfill}%
\pgfsetlinewidth{0.803000pt}%
\definecolor{currentstroke}{rgb}{0.000000,0.000000,0.000000}%
\pgfsetstrokecolor{currentstroke}%
\pgfsetdash{}{0pt}%
\pgfsys@defobject{currentmarker}{\pgfqpoint{-0.048611in}{0.000000in}}{\pgfqpoint{0.000000in}{0.000000in}}{%
\pgfpathmoveto{\pgfqpoint{0.000000in}{0.000000in}}%
\pgfpathlineto{\pgfqpoint{-0.048611in}{0.000000in}}%
\pgfusepath{stroke,fill}%
}%
\begin{pgfscope}%
\pgfsys@transformshift{0.537394in}{2.903863in}%
\pgfsys@useobject{currentmarker}{}%
\end{pgfscope}%
\end{pgfscope}%
\begin{pgfscope}%
\definecolor{textcolor}{rgb}{0.000000,0.000000,0.000000}%
\pgfsetstrokecolor{textcolor}%
\pgfsetfillcolor{textcolor}%
\pgftext[x=0.263086in,y=2.861654in,left,base]{\color{textcolor}\rmfamily\fontsize{8.000000}{9.600000}\selectfont \(\displaystyle 200\)}%
\end{pgfscope}%
\begin{pgfscope}%
\definecolor{textcolor}{rgb}{0.000000,0.000000,0.000000}%
\pgfsetstrokecolor{textcolor}%
\pgfsetfillcolor{textcolor}%
\pgftext[x=0.207530in,y=1.709541in,,bottom,rotate=90.000000]{\color{textcolor}\rmfamily\fontsize{8.000000}{9.600000}\selectfont Number of particles}%
\end{pgfscope}%
\begin{pgfscope}%
\pgfpathrectangle{\pgfqpoint{0.537394in}{0.484854in}}{\pgfqpoint{3.287038in}{2.449373in}}%
\pgfusepath{clip}%
\pgfsetrectcap%
\pgfsetroundjoin%
\pgfsetlinewidth{1.505625pt}%
\definecolor{currentstroke}{rgb}{0.121569,0.466667,0.705882}%
\pgfsetstrokecolor{currentstroke}%
\pgfsetdash{}{0pt}%
\pgfpathmoveto{\pgfqpoint{0.537394in}{1.554348in}}%
\pgfpathlineto{\pgfqpoint{0.756530in}{1.554348in}}%
\pgfpathlineto{\pgfqpoint{0.975666in}{1.554348in}}%
\pgfpathlineto{\pgfqpoint{1.194801in}{1.554348in}}%
\pgfpathlineto{\pgfqpoint{1.413937in}{1.554348in}}%
\pgfpathlineto{\pgfqpoint{1.633073in}{1.554348in}}%
\pgfpathlineto{\pgfqpoint{1.852209in}{1.554348in}}%
\pgfpathlineto{\pgfqpoint{2.071345in}{1.554348in}}%
\pgfpathlineto{\pgfqpoint{2.217215in}{1.554348in}}%
\pgfpathlineto{\pgfqpoint{2.303424in}{1.554348in}}%
\pgfpathlineto{\pgfqpoint{2.365092in}{1.554348in}}%
\pgfpathlineto{\pgfqpoint{2.413246in}{1.554348in}}%
\pgfpathlineto{\pgfqpoint{2.452847in}{1.554348in}}%
\pgfpathlineto{\pgfqpoint{2.486549in}{1.554348in}}%
\pgfpathlineto{\pgfqpoint{2.515976in}{1.554348in}}%
\pgfpathlineto{\pgfqpoint{2.542171in}{1.554348in}}%
\pgfpathlineto{\pgfqpoint{2.565846in}{1.554348in}}%
\pgfpathlineto{\pgfqpoint{2.587505in}{1.554348in}}%
\pgfpathlineto{\pgfqpoint{2.607523in}{1.554348in}}%
\pgfpathlineto{\pgfqpoint{2.626183in}{1.554348in}}%
\pgfpathlineto{\pgfqpoint{2.643572in}{1.554348in}}%
\pgfpathlineto{\pgfqpoint{2.659840in}{1.554348in}}%
\pgfpathlineto{\pgfqpoint{2.675166in}{1.554348in}}%
\pgfpathlineto{\pgfqpoint{2.689692in}{1.554348in}}%
\pgfpathlineto{\pgfqpoint{2.703534in}{1.554348in}}%
\pgfpathlineto{\pgfqpoint{2.716788in}{1.554348in}}%
\pgfpathlineto{\pgfqpoint{2.729536in}{1.554348in}}%
\pgfpathlineto{\pgfqpoint{2.741843in}{1.554348in}}%
\pgfpathlineto{\pgfqpoint{2.753768in}{1.554348in}}%
\pgfpathlineto{\pgfqpoint{2.765360in}{1.554348in}}%
\pgfpathlineto{\pgfqpoint{2.776662in}{1.554348in}}%
\pgfpathlineto{\pgfqpoint{2.787712in}{1.554348in}}%
\pgfpathlineto{\pgfqpoint{2.798544in}{1.554348in}}%
\pgfpathlineto{\pgfqpoint{2.809184in}{1.554348in}}%
\pgfpathlineto{\pgfqpoint{2.819657in}{1.554348in}}%
\pgfpathlineto{\pgfqpoint{2.829990in}{1.554348in}}%
\pgfpathlineto{\pgfqpoint{2.840204in}{1.554348in}}%
\pgfpathlineto{\pgfqpoint{2.850323in}{1.554348in}}%
\pgfpathlineto{\pgfqpoint{2.860365in}{1.554348in}}%
\pgfpathlineto{\pgfqpoint{2.870349in}{1.554348in}}%
\pgfpathlineto{\pgfqpoint{2.880294in}{1.554348in}}%
\pgfpathlineto{\pgfqpoint{2.890216in}{1.554348in}}%
\pgfpathlineto{\pgfqpoint{2.900131in}{1.554348in}}%
\pgfpathlineto{\pgfqpoint{2.910056in}{1.554348in}}%
\pgfpathlineto{\pgfqpoint{2.920006in}{1.554348in}}%
\pgfpathlineto{\pgfqpoint{2.929996in}{1.554348in}}%
\pgfpathlineto{\pgfqpoint{2.940040in}{1.554348in}}%
\pgfpathlineto{\pgfqpoint{2.950155in}{1.554348in}}%
\pgfpathlineto{\pgfqpoint{2.960354in}{1.554348in}}%
\pgfpathlineto{\pgfqpoint{2.970653in}{1.554348in}}%
\pgfpathlineto{\pgfqpoint{2.981067in}{1.554348in}}%
\pgfpathlineto{\pgfqpoint{2.991611in}{1.554348in}}%
\pgfpathlineto{\pgfqpoint{3.002300in}{1.554348in}}%
\pgfpathlineto{\pgfqpoint{3.013152in}{1.554348in}}%
\pgfpathlineto{\pgfqpoint{3.024182in}{1.554348in}}%
\pgfpathlineto{\pgfqpoint{3.035406in}{1.554348in}}%
\pgfpathlineto{\pgfqpoint{3.046826in}{1.554348in}}%
\pgfpathlineto{\pgfqpoint{3.058134in}{1.554348in}}%
\pgfpathlineto{\pgfqpoint{3.069290in}{1.554348in}}%
\pgfpathlineto{\pgfqpoint{3.080344in}{1.554348in}}%
\pgfpathlineto{\pgfqpoint{3.091337in}{1.554348in}}%
\pgfpathlineto{\pgfqpoint{3.102311in}{1.554348in}}%
\pgfpathlineto{\pgfqpoint{3.113303in}{1.554348in}}%
\pgfpathlineto{\pgfqpoint{3.124347in}{1.554348in}}%
\pgfpathlineto{\pgfqpoint{3.135474in}{1.554348in}}%
\pgfpathlineto{\pgfqpoint{3.146717in}{1.554348in}}%
\pgfpathlineto{\pgfqpoint{3.158082in}{1.554348in}}%
\pgfpathlineto{\pgfqpoint{3.169596in}{1.554348in}}%
\pgfpathlineto{\pgfqpoint{3.181287in}{1.554348in}}%
\pgfpathlineto{\pgfqpoint{3.193182in}{1.554348in}}%
\pgfpathlineto{\pgfqpoint{3.205304in}{1.554348in}}%
\pgfpathlineto{\pgfqpoint{3.217678in}{1.554348in}}%
\pgfpathlineto{\pgfqpoint{3.230327in}{1.554348in}}%
\pgfpathlineto{\pgfqpoint{3.243273in}{1.554348in}}%
\pgfpathlineto{\pgfqpoint{3.256537in}{1.554348in}}%
\pgfpathlineto{\pgfqpoint{3.270140in}{1.554348in}}%
\pgfpathlineto{\pgfqpoint{3.284100in}{1.554348in}}%
\pgfpathlineto{\pgfqpoint{3.298435in}{1.554348in}}%
\pgfpathlineto{\pgfqpoint{3.313164in}{1.554348in}}%
\pgfpathlineto{\pgfqpoint{3.328301in}{1.554348in}}%
\pgfpathlineto{\pgfqpoint{3.343863in}{1.554348in}}%
\pgfpathlineto{\pgfqpoint{3.359865in}{1.554348in}}%
\pgfpathlineto{\pgfqpoint{3.376321in}{1.554348in}}%
\pgfpathlineto{\pgfqpoint{3.393243in}{1.554348in}}%
\pgfpathlineto{\pgfqpoint{3.410646in}{1.554348in}}%
\pgfpathlineto{\pgfqpoint{3.428543in}{1.554348in}}%
\pgfpathlineto{\pgfqpoint{3.446946in}{1.554348in}}%
\pgfpathlineto{\pgfqpoint{3.465870in}{1.554348in}}%
\pgfpathlineto{\pgfqpoint{3.485327in}{1.554348in}}%
\pgfpathlineto{\pgfqpoint{3.505319in}{1.554348in}}%
\pgfpathlineto{\pgfqpoint{3.525848in}{1.554348in}}%
\pgfpathlineto{\pgfqpoint{3.546928in}{1.554348in}}%
\pgfpathlineto{\pgfqpoint{3.568573in}{1.554348in}}%
\pgfpathlineto{\pgfqpoint{3.590798in}{1.554348in}}%
\pgfpathlineto{\pgfqpoint{3.613621in}{1.554348in}}%
\pgfpathlineto{\pgfqpoint{3.637059in}{1.554348in}}%
\pgfpathlineto{\pgfqpoint{3.661130in}{1.554348in}}%
\pgfpathlineto{\pgfqpoint{3.685855in}{1.554348in}}%
\pgfpathlineto{\pgfqpoint{3.711237in}{1.554348in}}%
\pgfpathlineto{\pgfqpoint{3.737282in}{1.554348in}}%
\pgfpathlineto{\pgfqpoint{3.764011in}{1.554348in}}%
\pgfpathlineto{\pgfqpoint{3.791445in}{1.554348in}}%
\pgfpathlineto{\pgfqpoint{3.819606in}{1.554348in}}%
\pgfpathlineto{\pgfqpoint{3.824431in}{1.554348in}}%
\pgfusepath{stroke}%
\end{pgfscope}%
\begin{pgfscope}%
\pgfsetrectcap%
\pgfsetmiterjoin%
\pgfsetlinewidth{0.803000pt}%
\definecolor{currentstroke}{rgb}{0.000000,0.000000,0.000000}%
\pgfsetstrokecolor{currentstroke}%
\pgfsetdash{}{0pt}%
\pgfpathmoveto{\pgfqpoint{0.537394in}{0.484854in}}%
\pgfpathlineto{\pgfqpoint{0.537394in}{2.934227in}}%
\pgfusepath{stroke}%
\end{pgfscope}%
\begin{pgfscope}%
\pgfsetrectcap%
\pgfsetmiterjoin%
\pgfsetlinewidth{0.803000pt}%
\definecolor{currentstroke}{rgb}{0.000000,0.000000,0.000000}%
\pgfsetstrokecolor{currentstroke}%
\pgfsetdash{}{0pt}%
\pgfpathmoveto{\pgfqpoint{3.824431in}{0.484854in}}%
\pgfpathlineto{\pgfqpoint{3.824431in}{2.934227in}}%
\pgfusepath{stroke}%
\end{pgfscope}%
\begin{pgfscope}%
\pgfsetrectcap%
\pgfsetmiterjoin%
\pgfsetlinewidth{0.803000pt}%
\definecolor{currentstroke}{rgb}{0.000000,0.000000,0.000000}%
\pgfsetstrokecolor{currentstroke}%
\pgfsetdash{}{0pt}%
\pgfpathmoveto{\pgfqpoint{0.537394in}{0.484854in}}%
\pgfpathlineto{\pgfqpoint{3.824431in}{0.484854in}}%
\pgfusepath{stroke}%
\end{pgfscope}%
\begin{pgfscope}%
\pgfsetrectcap%
\pgfsetmiterjoin%
\pgfsetlinewidth{0.803000pt}%
\definecolor{currentstroke}{rgb}{0.000000,0.000000,0.000000}%
\pgfsetstrokecolor{currentstroke}%
\pgfsetdash{}{0pt}%
\pgfpathmoveto{\pgfqpoint{0.537394in}{2.934227in}}%
\pgfpathlineto{\pgfqpoint{3.824431in}{2.934227in}}%
\pgfusepath{stroke}%
\end{pgfscope}%
\begin{pgfscope}%
\pgfsetrectcap%
\pgfsetroundjoin%
\pgfsetlinewidth{1.505625pt}%
\definecolor{currentstroke}{rgb}{0.121569,0.466667,0.705882}%
\pgfsetstrokecolor{currentstroke}%
\pgfsetdash{}{0pt}%
\pgfpathmoveto{\pgfqpoint{0.637394in}{2.788698in}}%
\pgfpathlineto{\pgfqpoint{0.859616in}{2.788698in}}%
\pgfusepath{stroke}%
\end{pgfscope}%
\begin{pgfscope}%
\definecolor{textcolor}{rgb}{0.000000,0.000000,0.000000}%
\pgfsetstrokecolor{textcolor}%
\pgfsetfillcolor{textcolor}%
\pgftext[x=0.948505in,y=2.749809in,left,base]{\color{textcolor}\rmfamily\fontsize{8.000000}{9.600000}\selectfont Average}%
\end{pgfscope}%
\begin{pgfscope}%
\pgfsetbuttcap%
\pgfsetmiterjoin%
\definecolor{currentfill}{rgb}{0.121569,0.466667,0.705882}%
\pgfsetfillcolor{currentfill}%
\pgfsetfillopacity{0.300000}%
\pgfsetlinewidth{0.000000pt}%
\definecolor{currentstroke}{rgb}{0.000000,0.000000,0.000000}%
\pgfsetstrokecolor{currentstroke}%
\pgfsetstrokeopacity{0.300000}%
\pgfsetdash{}{0pt}%
\pgfpathmoveto{\pgfqpoint{0.637394in}{2.585150in}}%
\pgfpathlineto{\pgfqpoint{0.859616in}{2.585150in}}%
\pgfpathlineto{\pgfqpoint{0.859616in}{2.662928in}}%
\pgfpathlineto{\pgfqpoint{0.637394in}{2.662928in}}%
\pgfpathclose%
\pgfusepath{fill}%
\end{pgfscope}%
\begin{pgfscope}%
\definecolor{textcolor}{rgb}{0.000000,0.000000,0.000000}%
\pgfsetstrokecolor{textcolor}%
\pgfsetfillcolor{textcolor}%
\pgftext[x=0.948505in,y=2.585150in,left,base]{\color{textcolor}\rmfamily\fontsize{8.000000}{9.600000}\selectfont Range}%
\end{pgfscope}%
\end{pgfpicture}%
\makeatother%
\endgroup%

    \caption{Caption}
    \label{fig:load_balancing}
\end{figure}

\begin{figure}
    \centering
    %% Creator: Matplotlib, PGF backend
%%
%% To include the figure in your LaTeX document, write
%%   \input{<filename>.pgf}
%%
%% Make sure the required packages are loaded in your preamble
%%   \usepackage{pgf}
%%
%% Figures using additional raster images can only be included by \input if
%% they are in the same directory as the main LaTeX file. For loading figures
%% from other directories you can use the `import` package
%%   \usepackage{import}
%% and then include the figures with
%%   \import{<path to file>}{<filename>.pgf}
%%
%% Matplotlib used the following preamble
%%   \usepackage{fontspec}
%%   \setmainfont{DejaVuSerif.ttf}[Path=/home/connor/.local/lib/python3.8/site-packages/matplotlib/mpl-data/fonts/ttf/]
%%   \setsansfont{DejaVuSans.ttf}[Path=/home/connor/.local/lib/python3.8/site-packages/matplotlib/mpl-data/fonts/ttf/]
%%   \setmonofont{DejaVuSansMono.ttf}[Path=/home/connor/.local/lib/python3.8/site-packages/matplotlib/mpl-data/fonts/ttf/]
%%
\begingroup%
\makeatletter%
\begin{pgfpicture}%
\pgfpathrectangle{\pgfpointorigin}{\pgfqpoint{4.049047in}{3.018785in}}%
\pgfusepath{use as bounding box, clip}%
\begin{pgfscope}%
\pgfsetbuttcap%
\pgfsetmiterjoin%
\definecolor{currentfill}{rgb}{1.000000,1.000000,1.000000}%
\pgfsetfillcolor{currentfill}%
\pgfsetlinewidth{0.000000pt}%
\definecolor{currentstroke}{rgb}{1.000000,1.000000,1.000000}%
\pgfsetstrokecolor{currentstroke}%
\pgfsetdash{}{0pt}%
\pgfpathmoveto{\pgfqpoint{0.000000in}{0.000000in}}%
\pgfpathlineto{\pgfqpoint{4.049047in}{0.000000in}}%
\pgfpathlineto{\pgfqpoint{4.049047in}{3.018785in}}%
\pgfpathlineto{\pgfqpoint{0.000000in}{3.018785in}}%
\pgfpathclose%
\pgfusepath{fill}%
\end{pgfscope}%
\begin{pgfscope}%
\pgfsetbuttcap%
\pgfsetmiterjoin%
\definecolor{currentfill}{rgb}{1.000000,1.000000,1.000000}%
\pgfsetfillcolor{currentfill}%
\pgfsetlinewidth{0.000000pt}%
\definecolor{currentstroke}{rgb}{0.000000,0.000000,0.000000}%
\pgfsetstrokecolor{currentstroke}%
\pgfsetstrokeopacity{0.000000}%
\pgfsetdash{}{0pt}%
\pgfpathmoveto{\pgfqpoint{0.662010in}{0.469412in}}%
\pgfpathlineto{\pgfqpoint{3.949047in}{0.469412in}}%
\pgfpathlineto{\pgfqpoint{3.949047in}{2.918785in}}%
\pgfpathlineto{\pgfqpoint{0.662010in}{2.918785in}}%
\pgfpathclose%
\pgfusepath{fill}%
\end{pgfscope}%
\begin{pgfscope}%
\pgfsetbuttcap%
\pgfsetroundjoin%
\definecolor{currentfill}{rgb}{0.000000,0.000000,0.000000}%
\pgfsetfillcolor{currentfill}%
\pgfsetlinewidth{0.803000pt}%
\definecolor{currentstroke}{rgb}{0.000000,0.000000,0.000000}%
\pgfsetstrokecolor{currentstroke}%
\pgfsetdash{}{0pt}%
\pgfsys@defobject{currentmarker}{\pgfqpoint{0.000000in}{-0.048611in}}{\pgfqpoint{0.000000in}{0.000000in}}{%
\pgfpathmoveto{\pgfqpoint{0.000000in}{0.000000in}}%
\pgfpathlineto{\pgfqpoint{0.000000in}{-0.048611in}}%
\pgfusepath{stroke,fill}%
}%
\begin{pgfscope}%
\pgfsys@transformshift{0.811421in}{0.469412in}%
\pgfsys@useobject{currentmarker}{}%
\end{pgfscope}%
\end{pgfscope}%
\begin{pgfscope}%
\definecolor{textcolor}{rgb}{0.000000,0.000000,0.000000}%
\pgfsetstrokecolor{textcolor}%
\pgfsetfillcolor{textcolor}%
\pgftext[x=0.811421in,y=0.372189in,,top]{\color{textcolor}\rmfamily\fontsize{8.000000}{9.600000}\selectfont \(\displaystyle 0\)}%
\end{pgfscope}%
\begin{pgfscope}%
\pgfsetbuttcap%
\pgfsetroundjoin%
\definecolor{currentfill}{rgb}{0.000000,0.000000,0.000000}%
\pgfsetfillcolor{currentfill}%
\pgfsetlinewidth{0.803000pt}%
\definecolor{currentstroke}{rgb}{0.000000,0.000000,0.000000}%
\pgfsetstrokecolor{currentstroke}%
\pgfsetdash{}{0pt}%
\pgfsys@defobject{currentmarker}{\pgfqpoint{0.000000in}{-0.048611in}}{\pgfqpoint{0.000000in}{0.000000in}}{%
\pgfpathmoveto{\pgfqpoint{0.000000in}{0.000000in}}%
\pgfpathlineto{\pgfqpoint{0.000000in}{-0.048611in}}%
\pgfusepath{stroke,fill}%
}%
\begin{pgfscope}%
\pgfsys@transformshift{1.209849in}{0.469412in}%
\pgfsys@useobject{currentmarker}{}%
\end{pgfscope}%
\end{pgfscope}%
\begin{pgfscope}%
\definecolor{textcolor}{rgb}{0.000000,0.000000,0.000000}%
\pgfsetstrokecolor{textcolor}%
\pgfsetfillcolor{textcolor}%
\pgftext[x=1.209849in,y=0.372189in,,top]{\color{textcolor}\rmfamily\fontsize{8.000000}{9.600000}\selectfont \(\displaystyle 200000\)}%
\end{pgfscope}%
\begin{pgfscope}%
\pgfsetbuttcap%
\pgfsetroundjoin%
\definecolor{currentfill}{rgb}{0.000000,0.000000,0.000000}%
\pgfsetfillcolor{currentfill}%
\pgfsetlinewidth{0.803000pt}%
\definecolor{currentstroke}{rgb}{0.000000,0.000000,0.000000}%
\pgfsetstrokecolor{currentstroke}%
\pgfsetdash{}{0pt}%
\pgfsys@defobject{currentmarker}{\pgfqpoint{0.000000in}{-0.048611in}}{\pgfqpoint{0.000000in}{0.000000in}}{%
\pgfpathmoveto{\pgfqpoint{0.000000in}{0.000000in}}%
\pgfpathlineto{\pgfqpoint{0.000000in}{-0.048611in}}%
\pgfusepath{stroke,fill}%
}%
\begin{pgfscope}%
\pgfsys@transformshift{1.608278in}{0.469412in}%
\pgfsys@useobject{currentmarker}{}%
\end{pgfscope}%
\end{pgfscope}%
\begin{pgfscope}%
\definecolor{textcolor}{rgb}{0.000000,0.000000,0.000000}%
\pgfsetstrokecolor{textcolor}%
\pgfsetfillcolor{textcolor}%
\pgftext[x=1.608278in,y=0.372189in,,top]{\color{textcolor}\rmfamily\fontsize{8.000000}{9.600000}\selectfont \(\displaystyle 400000\)}%
\end{pgfscope}%
\begin{pgfscope}%
\pgfsetbuttcap%
\pgfsetroundjoin%
\definecolor{currentfill}{rgb}{0.000000,0.000000,0.000000}%
\pgfsetfillcolor{currentfill}%
\pgfsetlinewidth{0.803000pt}%
\definecolor{currentstroke}{rgb}{0.000000,0.000000,0.000000}%
\pgfsetstrokecolor{currentstroke}%
\pgfsetdash{}{0pt}%
\pgfsys@defobject{currentmarker}{\pgfqpoint{0.000000in}{-0.048611in}}{\pgfqpoint{0.000000in}{0.000000in}}{%
\pgfpathmoveto{\pgfqpoint{0.000000in}{0.000000in}}%
\pgfpathlineto{\pgfqpoint{0.000000in}{-0.048611in}}%
\pgfusepath{stroke,fill}%
}%
\begin{pgfscope}%
\pgfsys@transformshift{2.006707in}{0.469412in}%
\pgfsys@useobject{currentmarker}{}%
\end{pgfscope}%
\end{pgfscope}%
\begin{pgfscope}%
\definecolor{textcolor}{rgb}{0.000000,0.000000,0.000000}%
\pgfsetstrokecolor{textcolor}%
\pgfsetfillcolor{textcolor}%
\pgftext[x=2.006707in,y=0.372189in,,top]{\color{textcolor}\rmfamily\fontsize{8.000000}{9.600000}\selectfont \(\displaystyle 600000\)}%
\end{pgfscope}%
\begin{pgfscope}%
\pgfsetbuttcap%
\pgfsetroundjoin%
\definecolor{currentfill}{rgb}{0.000000,0.000000,0.000000}%
\pgfsetfillcolor{currentfill}%
\pgfsetlinewidth{0.803000pt}%
\definecolor{currentstroke}{rgb}{0.000000,0.000000,0.000000}%
\pgfsetstrokecolor{currentstroke}%
\pgfsetdash{}{0pt}%
\pgfsys@defobject{currentmarker}{\pgfqpoint{0.000000in}{-0.048611in}}{\pgfqpoint{0.000000in}{0.000000in}}{%
\pgfpathmoveto{\pgfqpoint{0.000000in}{0.000000in}}%
\pgfpathlineto{\pgfqpoint{0.000000in}{-0.048611in}}%
\pgfusepath{stroke,fill}%
}%
\begin{pgfscope}%
\pgfsys@transformshift{2.405136in}{0.469412in}%
\pgfsys@useobject{currentmarker}{}%
\end{pgfscope}%
\end{pgfscope}%
\begin{pgfscope}%
\definecolor{textcolor}{rgb}{0.000000,0.000000,0.000000}%
\pgfsetstrokecolor{textcolor}%
\pgfsetfillcolor{textcolor}%
\pgftext[x=2.405136in,y=0.372189in,,top]{\color{textcolor}\rmfamily\fontsize{8.000000}{9.600000}\selectfont \(\displaystyle 800000\)}%
\end{pgfscope}%
\begin{pgfscope}%
\pgfsetbuttcap%
\pgfsetroundjoin%
\definecolor{currentfill}{rgb}{0.000000,0.000000,0.000000}%
\pgfsetfillcolor{currentfill}%
\pgfsetlinewidth{0.803000pt}%
\definecolor{currentstroke}{rgb}{0.000000,0.000000,0.000000}%
\pgfsetstrokecolor{currentstroke}%
\pgfsetdash{}{0pt}%
\pgfsys@defobject{currentmarker}{\pgfqpoint{0.000000in}{-0.048611in}}{\pgfqpoint{0.000000in}{0.000000in}}{%
\pgfpathmoveto{\pgfqpoint{0.000000in}{0.000000in}}%
\pgfpathlineto{\pgfqpoint{0.000000in}{-0.048611in}}%
\pgfusepath{stroke,fill}%
}%
\begin{pgfscope}%
\pgfsys@transformshift{2.803565in}{0.469412in}%
\pgfsys@useobject{currentmarker}{}%
\end{pgfscope}%
\end{pgfscope}%
\begin{pgfscope}%
\definecolor{textcolor}{rgb}{0.000000,0.000000,0.000000}%
\pgfsetstrokecolor{textcolor}%
\pgfsetfillcolor{textcolor}%
\pgftext[x=2.803565in,y=0.372189in,,top]{\color{textcolor}\rmfamily\fontsize{8.000000}{9.600000}\selectfont \(\displaystyle 1000000\)}%
\end{pgfscope}%
\begin{pgfscope}%
\pgfsetbuttcap%
\pgfsetroundjoin%
\definecolor{currentfill}{rgb}{0.000000,0.000000,0.000000}%
\pgfsetfillcolor{currentfill}%
\pgfsetlinewidth{0.803000pt}%
\definecolor{currentstroke}{rgb}{0.000000,0.000000,0.000000}%
\pgfsetstrokecolor{currentstroke}%
\pgfsetdash{}{0pt}%
\pgfsys@defobject{currentmarker}{\pgfqpoint{0.000000in}{-0.048611in}}{\pgfqpoint{0.000000in}{0.000000in}}{%
\pgfpathmoveto{\pgfqpoint{0.000000in}{0.000000in}}%
\pgfpathlineto{\pgfqpoint{0.000000in}{-0.048611in}}%
\pgfusepath{stroke,fill}%
}%
\begin{pgfscope}%
\pgfsys@transformshift{3.201993in}{0.469412in}%
\pgfsys@useobject{currentmarker}{}%
\end{pgfscope}%
\end{pgfscope}%
\begin{pgfscope}%
\definecolor{textcolor}{rgb}{0.000000,0.000000,0.000000}%
\pgfsetstrokecolor{textcolor}%
\pgfsetfillcolor{textcolor}%
\pgftext[x=3.201993in,y=0.372189in,,top]{\color{textcolor}\rmfamily\fontsize{8.000000}{9.600000}\selectfont \(\displaystyle 1200000\)}%
\end{pgfscope}%
\begin{pgfscope}%
\pgfsetbuttcap%
\pgfsetroundjoin%
\definecolor{currentfill}{rgb}{0.000000,0.000000,0.000000}%
\pgfsetfillcolor{currentfill}%
\pgfsetlinewidth{0.803000pt}%
\definecolor{currentstroke}{rgb}{0.000000,0.000000,0.000000}%
\pgfsetstrokecolor{currentstroke}%
\pgfsetdash{}{0pt}%
\pgfsys@defobject{currentmarker}{\pgfqpoint{0.000000in}{-0.048611in}}{\pgfqpoint{0.000000in}{0.000000in}}{%
\pgfpathmoveto{\pgfqpoint{0.000000in}{0.000000in}}%
\pgfpathlineto{\pgfqpoint{0.000000in}{-0.048611in}}%
\pgfusepath{stroke,fill}%
}%
\begin{pgfscope}%
\pgfsys@transformshift{3.600422in}{0.469412in}%
\pgfsys@useobject{currentmarker}{}%
\end{pgfscope}%
\end{pgfscope}%
\begin{pgfscope}%
\definecolor{textcolor}{rgb}{0.000000,0.000000,0.000000}%
\pgfsetstrokecolor{textcolor}%
\pgfsetfillcolor{textcolor}%
\pgftext[x=3.600422in,y=0.372189in,,top]{\color{textcolor}\rmfamily\fontsize{8.000000}{9.600000}\selectfont \(\displaystyle 1400000\)}%
\end{pgfscope}%
\begin{pgfscope}%
\definecolor{textcolor}{rgb}{0.000000,0.000000,0.000000}%
\pgfsetstrokecolor{textcolor}%
\pgfsetfillcolor{textcolor}%
\pgftext[x=2.305529in,y=0.209104in,,top]{\color{textcolor}\rmfamily\fontsize{8.000000}{9.600000}\selectfont Time (years)}%
\end{pgfscope}%
\begin{pgfscope}%
\pgfsetbuttcap%
\pgfsetroundjoin%
\definecolor{currentfill}{rgb}{0.000000,0.000000,0.000000}%
\pgfsetfillcolor{currentfill}%
\pgfsetlinewidth{0.803000pt}%
\definecolor{currentstroke}{rgb}{0.000000,0.000000,0.000000}%
\pgfsetstrokecolor{currentstroke}%
\pgfsetdash{}{0pt}%
\pgfsys@defobject{currentmarker}{\pgfqpoint{-0.048611in}{0.000000in}}{\pgfqpoint{0.000000in}{0.000000in}}{%
\pgfpathmoveto{\pgfqpoint{0.000000in}{0.000000in}}%
\pgfpathlineto{\pgfqpoint{-0.048611in}{0.000000in}}%
\pgfusepath{stroke,fill}%
}%
\begin{pgfscope}%
\pgfsys@transformshift{0.662010in}{0.803417in}%
\pgfsys@useobject{currentmarker}{}%
\end{pgfscope}%
\end{pgfscope}%
\begin{pgfscope}%
\definecolor{textcolor}{rgb}{0.000000,0.000000,0.000000}%
\pgfsetstrokecolor{textcolor}%
\pgfsetfillcolor{textcolor}%
\pgftext[x=0.263086in,y=0.761208in,left,base]{\color{textcolor}\rmfamily\fontsize{8.000000}{9.600000}\selectfont \(\displaystyle -0.04\)}%
\end{pgfscope}%
\begin{pgfscope}%
\pgfsetbuttcap%
\pgfsetroundjoin%
\definecolor{currentfill}{rgb}{0.000000,0.000000,0.000000}%
\pgfsetfillcolor{currentfill}%
\pgfsetlinewidth{0.803000pt}%
\definecolor{currentstroke}{rgb}{0.000000,0.000000,0.000000}%
\pgfsetstrokecolor{currentstroke}%
\pgfsetdash{}{0pt}%
\pgfsys@defobject{currentmarker}{\pgfqpoint{-0.048611in}{0.000000in}}{\pgfqpoint{0.000000in}{0.000000in}}{%
\pgfpathmoveto{\pgfqpoint{0.000000in}{0.000000in}}%
\pgfpathlineto{\pgfqpoint{-0.048611in}{0.000000in}}%
\pgfusepath{stroke,fill}%
}%
\begin{pgfscope}%
\pgfsys@transformshift{0.662010in}{1.248758in}%
\pgfsys@useobject{currentmarker}{}%
\end{pgfscope}%
\end{pgfscope}%
\begin{pgfscope}%
\definecolor{textcolor}{rgb}{0.000000,0.000000,0.000000}%
\pgfsetstrokecolor{textcolor}%
\pgfsetfillcolor{textcolor}%
\pgftext[x=0.263086in,y=1.206548in,left,base]{\color{textcolor}\rmfamily\fontsize{8.000000}{9.600000}\selectfont \(\displaystyle -0.02\)}%
\end{pgfscope}%
\begin{pgfscope}%
\pgfsetbuttcap%
\pgfsetroundjoin%
\definecolor{currentfill}{rgb}{0.000000,0.000000,0.000000}%
\pgfsetfillcolor{currentfill}%
\pgfsetlinewidth{0.803000pt}%
\definecolor{currentstroke}{rgb}{0.000000,0.000000,0.000000}%
\pgfsetstrokecolor{currentstroke}%
\pgfsetdash{}{0pt}%
\pgfsys@defobject{currentmarker}{\pgfqpoint{-0.048611in}{0.000000in}}{\pgfqpoint{0.000000in}{0.000000in}}{%
\pgfpathmoveto{\pgfqpoint{0.000000in}{0.000000in}}%
\pgfpathlineto{\pgfqpoint{-0.048611in}{0.000000in}}%
\pgfusepath{stroke,fill}%
}%
\begin{pgfscope}%
\pgfsys@transformshift{0.662010in}{1.694098in}%
\pgfsys@useobject{currentmarker}{}%
\end{pgfscope}%
\end{pgfscope}%
\begin{pgfscope}%
\definecolor{textcolor}{rgb}{0.000000,0.000000,0.000000}%
\pgfsetstrokecolor{textcolor}%
\pgfsetfillcolor{textcolor}%
\pgftext[x=0.354908in,y=1.651889in,left,base]{\color{textcolor}\rmfamily\fontsize{8.000000}{9.600000}\selectfont \(\displaystyle 0.00\)}%
\end{pgfscope}%
\begin{pgfscope}%
\pgfsetbuttcap%
\pgfsetroundjoin%
\definecolor{currentfill}{rgb}{0.000000,0.000000,0.000000}%
\pgfsetfillcolor{currentfill}%
\pgfsetlinewidth{0.803000pt}%
\definecolor{currentstroke}{rgb}{0.000000,0.000000,0.000000}%
\pgfsetstrokecolor{currentstroke}%
\pgfsetdash{}{0pt}%
\pgfsys@defobject{currentmarker}{\pgfqpoint{-0.048611in}{0.000000in}}{\pgfqpoint{0.000000in}{0.000000in}}{%
\pgfpathmoveto{\pgfqpoint{0.000000in}{0.000000in}}%
\pgfpathlineto{\pgfqpoint{-0.048611in}{0.000000in}}%
\pgfusepath{stroke,fill}%
}%
\begin{pgfscope}%
\pgfsys@transformshift{0.662010in}{2.139439in}%
\pgfsys@useobject{currentmarker}{}%
\end{pgfscope}%
\end{pgfscope}%
\begin{pgfscope}%
\definecolor{textcolor}{rgb}{0.000000,0.000000,0.000000}%
\pgfsetstrokecolor{textcolor}%
\pgfsetfillcolor{textcolor}%
\pgftext[x=0.354908in,y=2.097230in,left,base]{\color{textcolor}\rmfamily\fontsize{8.000000}{9.600000}\selectfont \(\displaystyle 0.02\)}%
\end{pgfscope}%
\begin{pgfscope}%
\pgfsetbuttcap%
\pgfsetroundjoin%
\definecolor{currentfill}{rgb}{0.000000,0.000000,0.000000}%
\pgfsetfillcolor{currentfill}%
\pgfsetlinewidth{0.803000pt}%
\definecolor{currentstroke}{rgb}{0.000000,0.000000,0.000000}%
\pgfsetstrokecolor{currentstroke}%
\pgfsetdash{}{0pt}%
\pgfsys@defobject{currentmarker}{\pgfqpoint{-0.048611in}{0.000000in}}{\pgfqpoint{0.000000in}{0.000000in}}{%
\pgfpathmoveto{\pgfqpoint{0.000000in}{0.000000in}}%
\pgfpathlineto{\pgfqpoint{-0.048611in}{0.000000in}}%
\pgfusepath{stroke,fill}%
}%
\begin{pgfscope}%
\pgfsys@transformshift{0.662010in}{2.584779in}%
\pgfsys@useobject{currentmarker}{}%
\end{pgfscope}%
\end{pgfscope}%
\begin{pgfscope}%
\definecolor{textcolor}{rgb}{0.000000,0.000000,0.000000}%
\pgfsetstrokecolor{textcolor}%
\pgfsetfillcolor{textcolor}%
\pgftext[x=0.354908in,y=2.542570in,left,base]{\color{textcolor}\rmfamily\fontsize{8.000000}{9.600000}\selectfont \(\displaystyle 0.04\)}%
\end{pgfscope}%
\begin{pgfscope}%
\definecolor{textcolor}{rgb}{0.000000,0.000000,0.000000}%
\pgfsetstrokecolor{textcolor}%
\pgfsetfillcolor{textcolor}%
\pgftext[x=0.207530in,y=1.694098in,,bottom,rotate=90.000000]{\color{textcolor}\rmfamily\fontsize{8.000000}{9.600000}\selectfont ???}%
\end{pgfscope}%
\begin{pgfscope}%
\pgfpathrectangle{\pgfqpoint{0.662010in}{0.469412in}}{\pgfqpoint{3.287038in}{2.449373in}}%
\pgfusepath{clip}%
\pgfsetrectcap%
\pgfsetroundjoin%
\pgfsetlinewidth{1.505625pt}%
\definecolor{currentstroke}{rgb}{0.121569,0.466667,0.705882}%
\pgfsetstrokecolor{currentstroke}%
\pgfsetdash{}{0pt}%
\pgfpathmoveto{\pgfqpoint{0.811421in}{1.694098in}}%
\pgfpathlineto{\pgfqpoint{1.010635in}{1.694098in}}%
\pgfpathlineto{\pgfqpoint{1.209849in}{1.694098in}}%
\pgfpathlineto{\pgfqpoint{1.409064in}{1.694098in}}%
\pgfpathlineto{\pgfqpoint{1.608278in}{1.694098in}}%
\pgfpathlineto{\pgfqpoint{1.709143in}{1.694098in}}%
\pgfpathlineto{\pgfqpoint{1.770131in}{1.694098in}}%
\pgfpathlineto{\pgfqpoint{1.816462in}{1.694098in}}%
\pgfpathlineto{\pgfqpoint{1.854651in}{1.694098in}}%
\pgfpathlineto{\pgfqpoint{1.887629in}{1.694098in}}%
\pgfpathlineto{\pgfqpoint{1.917023in}{1.694098in}}%
\pgfpathlineto{\pgfqpoint{1.943852in}{1.694098in}}%
\pgfpathlineto{\pgfqpoint{1.968553in}{1.694098in}}%
\pgfpathlineto{\pgfqpoint{1.991342in}{1.694098in}}%
\pgfpathlineto{\pgfqpoint{2.012721in}{1.694098in}}%
\pgfpathlineto{\pgfqpoint{2.033063in}{1.694098in}}%
\pgfpathlineto{\pgfqpoint{2.052650in}{1.694098in}}%
\pgfpathlineto{\pgfqpoint{2.071709in}{1.694098in}}%
\pgfpathlineto{\pgfqpoint{2.090428in}{1.694098in}}%
\pgfpathlineto{\pgfqpoint{2.108965in}{1.694098in}}%
\pgfpathlineto{\pgfqpoint{2.127464in}{1.694098in}}%
\pgfpathlineto{\pgfqpoint{2.146055in}{1.694098in}}%
\pgfpathlineto{\pgfqpoint{2.164858in}{1.694098in}}%
\pgfpathlineto{\pgfqpoint{2.183478in}{1.694098in}}%
\pgfpathlineto{\pgfqpoint{2.201601in}{1.694098in}}%
\pgfpathlineto{\pgfqpoint{2.219459in}{1.694098in}}%
\pgfpathlineto{\pgfqpoint{2.237187in}{1.694098in}}%
\pgfpathlineto{\pgfqpoint{2.254955in}{1.694098in}}%
\pgfpathlineto{\pgfqpoint{2.272917in}{1.694098in}}%
\pgfpathlineto{\pgfqpoint{2.291213in}{1.694098in}}%
\pgfpathlineto{\pgfqpoint{2.309970in}{1.694098in}}%
\pgfpathlineto{\pgfqpoint{2.329379in}{1.694098in}}%
\pgfpathlineto{\pgfqpoint{2.349673in}{1.694098in}}%
\pgfpathlineto{\pgfqpoint{2.370907in}{1.694098in}}%
\pgfpathlineto{\pgfqpoint{2.393162in}{1.694098in}}%
\pgfpathlineto{\pgfqpoint{2.416469in}{1.694098in}}%
\pgfpathlineto{\pgfqpoint{2.440991in}{1.694098in}}%
\pgfpathlineto{\pgfqpoint{2.466648in}{1.694098in}}%
\pgfpathlineto{\pgfqpoint{2.492814in}{1.694098in}}%
\pgfpathlineto{\pgfqpoint{2.518889in}{1.694098in}}%
\pgfpathlineto{\pgfqpoint{2.546137in}{1.694098in}}%
\pgfpathlineto{\pgfqpoint{2.574277in}{1.694098in}}%
\pgfpathlineto{\pgfqpoint{2.594885in}{1.694098in}}%
\pgfpathlineto{\pgfqpoint{2.628552in}{1.694098in}}%
\pgfpathlineto{\pgfqpoint{2.667197in}{1.694098in}}%
\pgfpathlineto{\pgfqpoint{2.708562in}{1.694098in}}%
\pgfpathlineto{\pgfqpoint{2.752370in}{1.694098in}}%
\pgfpathlineto{\pgfqpoint{2.798572in}{1.694098in}}%
\pgfpathlineto{\pgfqpoint{2.847428in}{1.694098in}}%
\pgfpathlineto{\pgfqpoint{2.898971in}{1.694098in}}%
\pgfpathlineto{\pgfqpoint{2.953667in}{1.694098in}}%
\pgfpathlineto{\pgfqpoint{3.011990in}{1.694098in}}%
\pgfpathlineto{\pgfqpoint{3.074016in}{1.694098in}}%
\pgfpathlineto{\pgfqpoint{3.140112in}{1.694098in}}%
\pgfpathlineto{\pgfqpoint{3.210583in}{1.694098in}}%
\pgfpathlineto{\pgfqpoint{3.285670in}{1.694098in}}%
\pgfpathlineto{\pgfqpoint{3.365806in}{1.694098in}}%
\pgfpathlineto{\pgfqpoint{3.451235in}{1.694098in}}%
\pgfpathlineto{\pgfqpoint{3.541578in}{1.694098in}}%
\pgfpathlineto{\pgfqpoint{3.636522in}{1.694098in}}%
\pgfpathlineto{\pgfqpoint{3.736189in}{1.694098in}}%
\pgfpathlineto{\pgfqpoint{3.799637in}{1.694098in}}%
\pgfusepath{stroke}%
\end{pgfscope}%
\begin{pgfscope}%
\pgfsetrectcap%
\pgfsetmiterjoin%
\pgfsetlinewidth{0.803000pt}%
\definecolor{currentstroke}{rgb}{0.000000,0.000000,0.000000}%
\pgfsetstrokecolor{currentstroke}%
\pgfsetdash{}{0pt}%
\pgfpathmoveto{\pgfqpoint{0.662010in}{0.469412in}}%
\pgfpathlineto{\pgfqpoint{0.662010in}{2.918785in}}%
\pgfusepath{stroke}%
\end{pgfscope}%
\begin{pgfscope}%
\pgfsetrectcap%
\pgfsetmiterjoin%
\pgfsetlinewidth{0.803000pt}%
\definecolor{currentstroke}{rgb}{0.000000,0.000000,0.000000}%
\pgfsetstrokecolor{currentstroke}%
\pgfsetdash{}{0pt}%
\pgfpathmoveto{\pgfqpoint{3.949047in}{0.469412in}}%
\pgfpathlineto{\pgfqpoint{3.949047in}{2.918785in}}%
\pgfusepath{stroke}%
\end{pgfscope}%
\begin{pgfscope}%
\pgfsetrectcap%
\pgfsetmiterjoin%
\pgfsetlinewidth{0.803000pt}%
\definecolor{currentstroke}{rgb}{0.000000,0.000000,0.000000}%
\pgfsetstrokecolor{currentstroke}%
\pgfsetdash{}{0pt}%
\pgfpathmoveto{\pgfqpoint{0.662010in}{0.469412in}}%
\pgfpathlineto{\pgfqpoint{3.949047in}{0.469412in}}%
\pgfusepath{stroke}%
\end{pgfscope}%
\begin{pgfscope}%
\pgfsetrectcap%
\pgfsetmiterjoin%
\pgfsetlinewidth{0.803000pt}%
\definecolor{currentstroke}{rgb}{0.000000,0.000000,0.000000}%
\pgfsetstrokecolor{currentstroke}%
\pgfsetdash{}{0pt}%
\pgfpathmoveto{\pgfqpoint{0.662010in}{2.918785in}}%
\pgfpathlineto{\pgfqpoint{3.949047in}{2.918785in}}%
\pgfusepath{stroke}%
\end{pgfscope}%
\end{pgfpicture}%
\makeatother%
\endgroup%

    \caption{Caption}
    \label{fig:load_balancing_material_model}
\end{figure}

\subsubsection{Scaling}

\begin{figure}
    \centering
    %% Creator: Matplotlib, PGF backend
%%
%% To include the figure in your LaTeX document, write
%%   \input{<filename>.pgf}
%%
%% Make sure the required packages are loaded in your preamble
%%   \usepackage{pgf}
%%
%% Figures using additional raster images can only be included by \input if
%% they are in the same directory as the main LaTeX file. For loading figures
%% from other directories you can use the `import` package
%%   \usepackage{import}
%% and then include the figures with
%%   \import{<path to file>}{<filename>.pgf}
%%
%% Matplotlib used the following preamble
%%   \usepackage{fontspec}
%%   \setmainfont{DejaVuSerif.ttf}[Path=/home/connor/.local/lib/python3.8/site-packages/matplotlib/mpl-data/fonts/ttf/]
%%   \setsansfont{DejaVuSans.ttf}[Path=/home/connor/.local/lib/python3.8/site-packages/matplotlib/mpl-data/fonts/ttf/]
%%   \setmonofont{DejaVuSansMono.ttf}[Path=/home/connor/.local/lib/python3.8/site-packages/matplotlib/mpl-data/fonts/ttf/]
%%
\begingroup%
\makeatletter%
\begin{pgfpicture}%
\pgfpathrectangle{\pgfpointorigin}{\pgfqpoint{2.764972in}{2.197213in}}%
\pgfusepath{use as bounding box, clip}%
\begin{pgfscope}%
\pgfsetbuttcap%
\pgfsetmiterjoin%
\definecolor{currentfill}{rgb}{1.000000,1.000000,1.000000}%
\pgfsetfillcolor{currentfill}%
\pgfsetlinewidth{0.000000pt}%
\definecolor{currentstroke}{rgb}{1.000000,1.000000,1.000000}%
\pgfsetstrokecolor{currentstroke}%
\pgfsetdash{}{0pt}%
\pgfpathmoveto{\pgfqpoint{0.000000in}{0.000000in}}%
\pgfpathlineto{\pgfqpoint{2.764972in}{0.000000in}}%
\pgfpathlineto{\pgfqpoint{2.764972in}{2.197213in}}%
\pgfpathlineto{\pgfqpoint{0.000000in}{2.197213in}}%
\pgfpathclose%
\pgfusepath{fill}%
\end{pgfscope}%
\begin{pgfscope}%
\pgfsetbuttcap%
\pgfsetmiterjoin%
\definecolor{currentfill}{rgb}{1.000000,1.000000,1.000000}%
\pgfsetfillcolor{currentfill}%
\pgfsetlinewidth{0.000000pt}%
\definecolor{currentstroke}{rgb}{0.000000,0.000000,0.000000}%
\pgfsetstrokecolor{currentstroke}%
\pgfsetstrokeopacity{0.000000}%
\pgfsetdash{}{0pt}%
\pgfpathmoveto{\pgfqpoint{0.478365in}{0.467838in}}%
\pgfpathlineto{\pgfqpoint{2.664972in}{0.467838in}}%
\pgfpathlineto{\pgfqpoint{2.664972in}{2.097213in}}%
\pgfpathlineto{\pgfqpoint{0.478365in}{2.097213in}}%
\pgfpathclose%
\pgfusepath{fill}%
\end{pgfscope}%
\begin{pgfscope}%
\pgfsetbuttcap%
\pgfsetroundjoin%
\definecolor{currentfill}{rgb}{0.000000,0.000000,0.000000}%
\pgfsetfillcolor{currentfill}%
\pgfsetlinewidth{0.803000pt}%
\definecolor{currentstroke}{rgb}{0.000000,0.000000,0.000000}%
\pgfsetstrokecolor{currentstroke}%
\pgfsetdash{}{0pt}%
\pgfsys@defobject{currentmarker}{\pgfqpoint{0.000000in}{-0.048611in}}{\pgfqpoint{0.000000in}{0.000000in}}{%
\pgfpathmoveto{\pgfqpoint{0.000000in}{0.000000in}}%
\pgfpathlineto{\pgfqpoint{0.000000in}{-0.048611in}}%
\pgfusepath{stroke,fill}%
}%
\begin{pgfscope}%
\pgfsys@transformshift{0.577757in}{0.467838in}%
\pgfsys@useobject{currentmarker}{}%
\end{pgfscope}%
\end{pgfscope}%
\begin{pgfscope}%
\definecolor{textcolor}{rgb}{0.000000,0.000000,0.000000}%
\pgfsetstrokecolor{textcolor}%
\pgfsetfillcolor{textcolor}%
\pgftext[x=0.577757in,y=0.370616in,,top]{\color{textcolor}\rmfamily\fontsize{8.000000}{9.600000}\selectfont \(\displaystyle 1\)}%
\end{pgfscope}%
\begin{pgfscope}%
\pgfsetbuttcap%
\pgfsetroundjoin%
\definecolor{currentfill}{rgb}{0.000000,0.000000,0.000000}%
\pgfsetfillcolor{currentfill}%
\pgfsetlinewidth{0.803000pt}%
\definecolor{currentstroke}{rgb}{0.000000,0.000000,0.000000}%
\pgfsetstrokecolor{currentstroke}%
\pgfsetdash{}{0pt}%
\pgfsys@defobject{currentmarker}{\pgfqpoint{0.000000in}{-0.048611in}}{\pgfqpoint{0.000000in}{0.000000in}}{%
\pgfpathmoveto{\pgfqpoint{0.000000in}{0.000000in}}%
\pgfpathlineto{\pgfqpoint{0.000000in}{-0.048611in}}%
\pgfusepath{stroke,fill}%
}%
\begin{pgfscope}%
\pgfsys@transformshift{0.758468in}{0.467838in}%
\pgfsys@useobject{currentmarker}{}%
\end{pgfscope}%
\end{pgfscope}%
\begin{pgfscope}%
\definecolor{textcolor}{rgb}{0.000000,0.000000,0.000000}%
\pgfsetstrokecolor{textcolor}%
\pgfsetfillcolor{textcolor}%
\pgftext[x=0.758468in,y=0.370616in,,top]{\color{textcolor}\rmfamily\fontsize{8.000000}{9.600000}\selectfont \(\displaystyle 2\)}%
\end{pgfscope}%
\begin{pgfscope}%
\pgfsetbuttcap%
\pgfsetroundjoin%
\definecolor{currentfill}{rgb}{0.000000,0.000000,0.000000}%
\pgfsetfillcolor{currentfill}%
\pgfsetlinewidth{0.803000pt}%
\definecolor{currentstroke}{rgb}{0.000000,0.000000,0.000000}%
\pgfsetstrokecolor{currentstroke}%
\pgfsetdash{}{0pt}%
\pgfsys@defobject{currentmarker}{\pgfqpoint{0.000000in}{-0.048611in}}{\pgfqpoint{0.000000in}{0.000000in}}{%
\pgfpathmoveto{\pgfqpoint{0.000000in}{0.000000in}}%
\pgfpathlineto{\pgfqpoint{0.000000in}{-0.048611in}}%
\pgfusepath{stroke,fill}%
}%
\begin{pgfscope}%
\pgfsys@transformshift{0.939179in}{0.467838in}%
\pgfsys@useobject{currentmarker}{}%
\end{pgfscope}%
\end{pgfscope}%
\begin{pgfscope}%
\definecolor{textcolor}{rgb}{0.000000,0.000000,0.000000}%
\pgfsetstrokecolor{textcolor}%
\pgfsetfillcolor{textcolor}%
\pgftext[x=0.939179in,y=0.370616in,,top]{\color{textcolor}\rmfamily\fontsize{8.000000}{9.600000}\selectfont \(\displaystyle 3\)}%
\end{pgfscope}%
\begin{pgfscope}%
\pgfsetbuttcap%
\pgfsetroundjoin%
\definecolor{currentfill}{rgb}{0.000000,0.000000,0.000000}%
\pgfsetfillcolor{currentfill}%
\pgfsetlinewidth{0.803000pt}%
\definecolor{currentstroke}{rgb}{0.000000,0.000000,0.000000}%
\pgfsetstrokecolor{currentstroke}%
\pgfsetdash{}{0pt}%
\pgfsys@defobject{currentmarker}{\pgfqpoint{0.000000in}{-0.048611in}}{\pgfqpoint{0.000000in}{0.000000in}}{%
\pgfpathmoveto{\pgfqpoint{0.000000in}{0.000000in}}%
\pgfpathlineto{\pgfqpoint{0.000000in}{-0.048611in}}%
\pgfusepath{stroke,fill}%
}%
\begin{pgfscope}%
\pgfsys@transformshift{1.119890in}{0.467838in}%
\pgfsys@useobject{currentmarker}{}%
\end{pgfscope}%
\end{pgfscope}%
\begin{pgfscope}%
\definecolor{textcolor}{rgb}{0.000000,0.000000,0.000000}%
\pgfsetstrokecolor{textcolor}%
\pgfsetfillcolor{textcolor}%
\pgftext[x=1.119890in,y=0.370616in,,top]{\color{textcolor}\rmfamily\fontsize{8.000000}{9.600000}\selectfont \(\displaystyle 4\)}%
\end{pgfscope}%
\begin{pgfscope}%
\pgfsetbuttcap%
\pgfsetroundjoin%
\definecolor{currentfill}{rgb}{0.000000,0.000000,0.000000}%
\pgfsetfillcolor{currentfill}%
\pgfsetlinewidth{0.803000pt}%
\definecolor{currentstroke}{rgb}{0.000000,0.000000,0.000000}%
\pgfsetstrokecolor{currentstroke}%
\pgfsetdash{}{0pt}%
\pgfsys@defobject{currentmarker}{\pgfqpoint{0.000000in}{-0.048611in}}{\pgfqpoint{0.000000in}{0.000000in}}{%
\pgfpathmoveto{\pgfqpoint{0.000000in}{0.000000in}}%
\pgfpathlineto{\pgfqpoint{0.000000in}{-0.048611in}}%
\pgfusepath{stroke,fill}%
}%
\begin{pgfscope}%
\pgfsys@transformshift{1.300602in}{0.467838in}%
\pgfsys@useobject{currentmarker}{}%
\end{pgfscope}%
\end{pgfscope}%
\begin{pgfscope}%
\definecolor{textcolor}{rgb}{0.000000,0.000000,0.000000}%
\pgfsetstrokecolor{textcolor}%
\pgfsetfillcolor{textcolor}%
\pgftext[x=1.300602in,y=0.370616in,,top]{\color{textcolor}\rmfamily\fontsize{8.000000}{9.600000}\selectfont \(\displaystyle 5\)}%
\end{pgfscope}%
\begin{pgfscope}%
\pgfsetbuttcap%
\pgfsetroundjoin%
\definecolor{currentfill}{rgb}{0.000000,0.000000,0.000000}%
\pgfsetfillcolor{currentfill}%
\pgfsetlinewidth{0.803000pt}%
\definecolor{currentstroke}{rgb}{0.000000,0.000000,0.000000}%
\pgfsetstrokecolor{currentstroke}%
\pgfsetdash{}{0pt}%
\pgfsys@defobject{currentmarker}{\pgfqpoint{0.000000in}{-0.048611in}}{\pgfqpoint{0.000000in}{0.000000in}}{%
\pgfpathmoveto{\pgfqpoint{0.000000in}{0.000000in}}%
\pgfpathlineto{\pgfqpoint{0.000000in}{-0.048611in}}%
\pgfusepath{stroke,fill}%
}%
\begin{pgfscope}%
\pgfsys@transformshift{1.481313in}{0.467838in}%
\pgfsys@useobject{currentmarker}{}%
\end{pgfscope}%
\end{pgfscope}%
\begin{pgfscope}%
\definecolor{textcolor}{rgb}{0.000000,0.000000,0.000000}%
\pgfsetstrokecolor{textcolor}%
\pgfsetfillcolor{textcolor}%
\pgftext[x=1.481313in,y=0.370616in,,top]{\color{textcolor}\rmfamily\fontsize{8.000000}{9.600000}\selectfont \(\displaystyle 6\)}%
\end{pgfscope}%
\begin{pgfscope}%
\pgfsetbuttcap%
\pgfsetroundjoin%
\definecolor{currentfill}{rgb}{0.000000,0.000000,0.000000}%
\pgfsetfillcolor{currentfill}%
\pgfsetlinewidth{0.803000pt}%
\definecolor{currentstroke}{rgb}{0.000000,0.000000,0.000000}%
\pgfsetstrokecolor{currentstroke}%
\pgfsetdash{}{0pt}%
\pgfsys@defobject{currentmarker}{\pgfqpoint{0.000000in}{-0.048611in}}{\pgfqpoint{0.000000in}{0.000000in}}{%
\pgfpathmoveto{\pgfqpoint{0.000000in}{0.000000in}}%
\pgfpathlineto{\pgfqpoint{0.000000in}{-0.048611in}}%
\pgfusepath{stroke,fill}%
}%
\begin{pgfscope}%
\pgfsys@transformshift{1.662024in}{0.467838in}%
\pgfsys@useobject{currentmarker}{}%
\end{pgfscope}%
\end{pgfscope}%
\begin{pgfscope}%
\definecolor{textcolor}{rgb}{0.000000,0.000000,0.000000}%
\pgfsetstrokecolor{textcolor}%
\pgfsetfillcolor{textcolor}%
\pgftext[x=1.662024in,y=0.370616in,,top]{\color{textcolor}\rmfamily\fontsize{8.000000}{9.600000}\selectfont \(\displaystyle 7\)}%
\end{pgfscope}%
\begin{pgfscope}%
\pgfsetbuttcap%
\pgfsetroundjoin%
\definecolor{currentfill}{rgb}{0.000000,0.000000,0.000000}%
\pgfsetfillcolor{currentfill}%
\pgfsetlinewidth{0.803000pt}%
\definecolor{currentstroke}{rgb}{0.000000,0.000000,0.000000}%
\pgfsetstrokecolor{currentstroke}%
\pgfsetdash{}{0pt}%
\pgfsys@defobject{currentmarker}{\pgfqpoint{0.000000in}{-0.048611in}}{\pgfqpoint{0.000000in}{0.000000in}}{%
\pgfpathmoveto{\pgfqpoint{0.000000in}{0.000000in}}%
\pgfpathlineto{\pgfqpoint{0.000000in}{-0.048611in}}%
\pgfusepath{stroke,fill}%
}%
\begin{pgfscope}%
\pgfsys@transformshift{1.842736in}{0.467838in}%
\pgfsys@useobject{currentmarker}{}%
\end{pgfscope}%
\end{pgfscope}%
\begin{pgfscope}%
\definecolor{textcolor}{rgb}{0.000000,0.000000,0.000000}%
\pgfsetstrokecolor{textcolor}%
\pgfsetfillcolor{textcolor}%
\pgftext[x=1.842736in,y=0.370616in,,top]{\color{textcolor}\rmfamily\fontsize{8.000000}{9.600000}\selectfont \(\displaystyle 8\)}%
\end{pgfscope}%
\begin{pgfscope}%
\pgfsetbuttcap%
\pgfsetroundjoin%
\definecolor{currentfill}{rgb}{0.000000,0.000000,0.000000}%
\pgfsetfillcolor{currentfill}%
\pgfsetlinewidth{0.803000pt}%
\definecolor{currentstroke}{rgb}{0.000000,0.000000,0.000000}%
\pgfsetstrokecolor{currentstroke}%
\pgfsetdash{}{0pt}%
\pgfsys@defobject{currentmarker}{\pgfqpoint{0.000000in}{-0.048611in}}{\pgfqpoint{0.000000in}{0.000000in}}{%
\pgfpathmoveto{\pgfqpoint{0.000000in}{0.000000in}}%
\pgfpathlineto{\pgfqpoint{0.000000in}{-0.048611in}}%
\pgfusepath{stroke,fill}%
}%
\begin{pgfscope}%
\pgfsys@transformshift{2.023447in}{0.467838in}%
\pgfsys@useobject{currentmarker}{}%
\end{pgfscope}%
\end{pgfscope}%
\begin{pgfscope}%
\definecolor{textcolor}{rgb}{0.000000,0.000000,0.000000}%
\pgfsetstrokecolor{textcolor}%
\pgfsetfillcolor{textcolor}%
\pgftext[x=2.023447in,y=0.370616in,,top]{\color{textcolor}\rmfamily\fontsize{8.000000}{9.600000}\selectfont \(\displaystyle 9\)}%
\end{pgfscope}%
\begin{pgfscope}%
\pgfsetbuttcap%
\pgfsetroundjoin%
\definecolor{currentfill}{rgb}{0.000000,0.000000,0.000000}%
\pgfsetfillcolor{currentfill}%
\pgfsetlinewidth{0.803000pt}%
\definecolor{currentstroke}{rgb}{0.000000,0.000000,0.000000}%
\pgfsetstrokecolor{currentstroke}%
\pgfsetdash{}{0pt}%
\pgfsys@defobject{currentmarker}{\pgfqpoint{0.000000in}{-0.048611in}}{\pgfqpoint{0.000000in}{0.000000in}}{%
\pgfpathmoveto{\pgfqpoint{0.000000in}{0.000000in}}%
\pgfpathlineto{\pgfqpoint{0.000000in}{-0.048611in}}%
\pgfusepath{stroke,fill}%
}%
\begin{pgfscope}%
\pgfsys@transformshift{2.204158in}{0.467838in}%
\pgfsys@useobject{currentmarker}{}%
\end{pgfscope}%
\end{pgfscope}%
\begin{pgfscope}%
\definecolor{textcolor}{rgb}{0.000000,0.000000,0.000000}%
\pgfsetstrokecolor{textcolor}%
\pgfsetfillcolor{textcolor}%
\pgftext[x=2.204158in,y=0.370616in,,top]{\color{textcolor}\rmfamily\fontsize{8.000000}{9.600000}\selectfont \(\displaystyle 10\)}%
\end{pgfscope}%
\begin{pgfscope}%
\pgfsetbuttcap%
\pgfsetroundjoin%
\definecolor{currentfill}{rgb}{0.000000,0.000000,0.000000}%
\pgfsetfillcolor{currentfill}%
\pgfsetlinewidth{0.803000pt}%
\definecolor{currentstroke}{rgb}{0.000000,0.000000,0.000000}%
\pgfsetstrokecolor{currentstroke}%
\pgfsetdash{}{0pt}%
\pgfsys@defobject{currentmarker}{\pgfqpoint{0.000000in}{-0.048611in}}{\pgfqpoint{0.000000in}{0.000000in}}{%
\pgfpathmoveto{\pgfqpoint{0.000000in}{0.000000in}}%
\pgfpathlineto{\pgfqpoint{0.000000in}{-0.048611in}}%
\pgfusepath{stroke,fill}%
}%
\begin{pgfscope}%
\pgfsys@transformshift{2.384870in}{0.467838in}%
\pgfsys@useobject{currentmarker}{}%
\end{pgfscope}%
\end{pgfscope}%
\begin{pgfscope}%
\definecolor{textcolor}{rgb}{0.000000,0.000000,0.000000}%
\pgfsetstrokecolor{textcolor}%
\pgfsetfillcolor{textcolor}%
\pgftext[x=2.384870in,y=0.370616in,,top]{\color{textcolor}\rmfamily\fontsize{8.000000}{9.600000}\selectfont \(\displaystyle 11\)}%
\end{pgfscope}%
\begin{pgfscope}%
\pgfsetbuttcap%
\pgfsetroundjoin%
\definecolor{currentfill}{rgb}{0.000000,0.000000,0.000000}%
\pgfsetfillcolor{currentfill}%
\pgfsetlinewidth{0.803000pt}%
\definecolor{currentstroke}{rgb}{0.000000,0.000000,0.000000}%
\pgfsetstrokecolor{currentstroke}%
\pgfsetdash{}{0pt}%
\pgfsys@defobject{currentmarker}{\pgfqpoint{0.000000in}{-0.048611in}}{\pgfqpoint{0.000000in}{0.000000in}}{%
\pgfpathmoveto{\pgfqpoint{0.000000in}{0.000000in}}%
\pgfpathlineto{\pgfqpoint{0.000000in}{-0.048611in}}%
\pgfusepath{stroke,fill}%
}%
\begin{pgfscope}%
\pgfsys@transformshift{2.565581in}{0.467838in}%
\pgfsys@useobject{currentmarker}{}%
\end{pgfscope}%
\end{pgfscope}%
\begin{pgfscope}%
\definecolor{textcolor}{rgb}{0.000000,0.000000,0.000000}%
\pgfsetstrokecolor{textcolor}%
\pgfsetfillcolor{textcolor}%
\pgftext[x=2.565581in,y=0.370616in,,top]{\color{textcolor}\rmfamily\fontsize{8.000000}{9.600000}\selectfont \(\displaystyle 12\)}%
\end{pgfscope}%
\begin{pgfscope}%
\definecolor{textcolor}{rgb}{0.000000,0.000000,0.000000}%
\pgfsetstrokecolor{textcolor}%
\pgfsetfillcolor{textcolor}%
\pgftext[x=1.571669in,y=0.207530in,,top]{\color{textcolor}\rmfamily\fontsize{8.000000}{9.600000}\selectfont Number of processors}%
\end{pgfscope}%
\begin{pgfscope}%
\pgfsetbuttcap%
\pgfsetroundjoin%
\definecolor{currentfill}{rgb}{0.000000,0.000000,0.000000}%
\pgfsetfillcolor{currentfill}%
\pgfsetlinewidth{0.803000pt}%
\definecolor{currentstroke}{rgb}{0.000000,0.000000,0.000000}%
\pgfsetstrokecolor{currentstroke}%
\pgfsetdash{}{0pt}%
\pgfsys@defobject{currentmarker}{\pgfqpoint{-0.048611in}{0.000000in}}{\pgfqpoint{0.000000in}{0.000000in}}{%
\pgfpathmoveto{\pgfqpoint{0.000000in}{0.000000in}}%
\pgfpathlineto{\pgfqpoint{-0.048611in}{0.000000in}}%
\pgfusepath{stroke,fill}%
}%
\begin{pgfscope}%
\pgfsys@transformshift{0.478365in}{0.692882in}%
\pgfsys@useobject{currentmarker}{}%
\end{pgfscope}%
\end{pgfscope}%
\begin{pgfscope}%
\definecolor{textcolor}{rgb}{0.000000,0.000000,0.000000}%
\pgfsetstrokecolor{textcolor}%
\pgfsetfillcolor{textcolor}%
\pgftext[x=0.322114in,y=0.650673in,left,base]{\color{textcolor}\rmfamily\fontsize{8.000000}{9.600000}\selectfont \(\displaystyle 2\)}%
\end{pgfscope}%
\begin{pgfscope}%
\pgfsetbuttcap%
\pgfsetroundjoin%
\definecolor{currentfill}{rgb}{0.000000,0.000000,0.000000}%
\pgfsetfillcolor{currentfill}%
\pgfsetlinewidth{0.803000pt}%
\definecolor{currentstroke}{rgb}{0.000000,0.000000,0.000000}%
\pgfsetstrokecolor{currentstroke}%
\pgfsetdash{}{0pt}%
\pgfsys@defobject{currentmarker}{\pgfqpoint{-0.048611in}{0.000000in}}{\pgfqpoint{0.000000in}{0.000000in}}{%
\pgfpathmoveto{\pgfqpoint{0.000000in}{0.000000in}}%
\pgfpathlineto{\pgfqpoint{-0.048611in}{0.000000in}}%
\pgfusepath{stroke,fill}%
}%
\begin{pgfscope}%
\pgfsys@transformshift{0.478365in}{0.994845in}%
\pgfsys@useobject{currentmarker}{}%
\end{pgfscope}%
\end{pgfscope}%
\begin{pgfscope}%
\definecolor{textcolor}{rgb}{0.000000,0.000000,0.000000}%
\pgfsetstrokecolor{textcolor}%
\pgfsetfillcolor{textcolor}%
\pgftext[x=0.322114in,y=0.952636in,left,base]{\color{textcolor}\rmfamily\fontsize{8.000000}{9.600000}\selectfont \(\displaystyle 4\)}%
\end{pgfscope}%
\begin{pgfscope}%
\pgfsetbuttcap%
\pgfsetroundjoin%
\definecolor{currentfill}{rgb}{0.000000,0.000000,0.000000}%
\pgfsetfillcolor{currentfill}%
\pgfsetlinewidth{0.803000pt}%
\definecolor{currentstroke}{rgb}{0.000000,0.000000,0.000000}%
\pgfsetstrokecolor{currentstroke}%
\pgfsetdash{}{0pt}%
\pgfsys@defobject{currentmarker}{\pgfqpoint{-0.048611in}{0.000000in}}{\pgfqpoint{0.000000in}{0.000000in}}{%
\pgfpathmoveto{\pgfqpoint{0.000000in}{0.000000in}}%
\pgfpathlineto{\pgfqpoint{-0.048611in}{0.000000in}}%
\pgfusepath{stroke,fill}%
}%
\begin{pgfscope}%
\pgfsys@transformshift{0.478365in}{1.296808in}%
\pgfsys@useobject{currentmarker}{}%
\end{pgfscope}%
\end{pgfscope}%
\begin{pgfscope}%
\definecolor{textcolor}{rgb}{0.000000,0.000000,0.000000}%
\pgfsetstrokecolor{textcolor}%
\pgfsetfillcolor{textcolor}%
\pgftext[x=0.322114in,y=1.254599in,left,base]{\color{textcolor}\rmfamily\fontsize{8.000000}{9.600000}\selectfont \(\displaystyle 6\)}%
\end{pgfscope}%
\begin{pgfscope}%
\pgfsetbuttcap%
\pgfsetroundjoin%
\definecolor{currentfill}{rgb}{0.000000,0.000000,0.000000}%
\pgfsetfillcolor{currentfill}%
\pgfsetlinewidth{0.803000pt}%
\definecolor{currentstroke}{rgb}{0.000000,0.000000,0.000000}%
\pgfsetstrokecolor{currentstroke}%
\pgfsetdash{}{0pt}%
\pgfsys@defobject{currentmarker}{\pgfqpoint{-0.048611in}{0.000000in}}{\pgfqpoint{0.000000in}{0.000000in}}{%
\pgfpathmoveto{\pgfqpoint{0.000000in}{0.000000in}}%
\pgfpathlineto{\pgfqpoint{-0.048611in}{0.000000in}}%
\pgfusepath{stroke,fill}%
}%
\begin{pgfscope}%
\pgfsys@transformshift{0.478365in}{1.598771in}%
\pgfsys@useobject{currentmarker}{}%
\end{pgfscope}%
\end{pgfscope}%
\begin{pgfscope}%
\definecolor{textcolor}{rgb}{0.000000,0.000000,0.000000}%
\pgfsetstrokecolor{textcolor}%
\pgfsetfillcolor{textcolor}%
\pgftext[x=0.322114in,y=1.556561in,left,base]{\color{textcolor}\rmfamily\fontsize{8.000000}{9.600000}\selectfont \(\displaystyle 8\)}%
\end{pgfscope}%
\begin{pgfscope}%
\pgfsetbuttcap%
\pgfsetroundjoin%
\definecolor{currentfill}{rgb}{0.000000,0.000000,0.000000}%
\pgfsetfillcolor{currentfill}%
\pgfsetlinewidth{0.803000pt}%
\definecolor{currentstroke}{rgb}{0.000000,0.000000,0.000000}%
\pgfsetstrokecolor{currentstroke}%
\pgfsetdash{}{0pt}%
\pgfsys@defobject{currentmarker}{\pgfqpoint{-0.048611in}{0.000000in}}{\pgfqpoint{0.000000in}{0.000000in}}{%
\pgfpathmoveto{\pgfqpoint{0.000000in}{0.000000in}}%
\pgfpathlineto{\pgfqpoint{-0.048611in}{0.000000in}}%
\pgfusepath{stroke,fill}%
}%
\begin{pgfscope}%
\pgfsys@transformshift{0.478365in}{1.900733in}%
\pgfsys@useobject{currentmarker}{}%
\end{pgfscope}%
\end{pgfscope}%
\begin{pgfscope}%
\definecolor{textcolor}{rgb}{0.000000,0.000000,0.000000}%
\pgfsetstrokecolor{textcolor}%
\pgfsetfillcolor{textcolor}%
\pgftext[x=0.263086in,y=1.858524in,left,base]{\color{textcolor}\rmfamily\fontsize{8.000000}{9.600000}\selectfont \(\displaystyle 10\)}%
\end{pgfscope}%
\begin{pgfscope}%
\definecolor{textcolor}{rgb}{0.000000,0.000000,0.000000}%
\pgfsetstrokecolor{textcolor}%
\pgfsetfillcolor{textcolor}%
\pgftext[x=0.207530in,y=1.282526in,,bottom,rotate=90.000000]{\color{textcolor}\rmfamily\fontsize{8.000000}{9.600000}\selectfont Speedup}%
\end{pgfscope}%
\begin{pgfscope}%
\pgfpathrectangle{\pgfqpoint{0.478365in}{0.467838in}}{\pgfqpoint{2.186607in}{1.629375in}}%
\pgfusepath{clip}%
\pgfsetrectcap%
\pgfsetroundjoin%
\pgfsetlinewidth{1.505625pt}%
\definecolor{currentstroke}{rgb}{0.121569,0.466667,0.705882}%
\pgfsetstrokecolor{currentstroke}%
\pgfsetdash{}{0pt}%
\pgfpathmoveto{\pgfqpoint{0.577757in}{0.541901in}}%
\pgfpathlineto{\pgfqpoint{0.758468in}{0.670945in}}%
\pgfpathlineto{\pgfqpoint{0.939179in}{0.798926in}}%
\pgfpathlineto{\pgfqpoint{1.119890in}{0.901246in}}%
\pgfpathlineto{\pgfqpoint{1.300602in}{0.998767in}}%
\pgfpathlineto{\pgfqpoint{1.481313in}{1.086524in}}%
\pgfpathlineto{\pgfqpoint{1.662024in}{1.118985in}}%
\pgfpathlineto{\pgfqpoint{1.842736in}{1.226709in}}%
\pgfpathlineto{\pgfqpoint{2.023447in}{1.264598in}}%
\pgfpathlineto{\pgfqpoint{2.204158in}{1.348901in}}%
\pgfpathlineto{\pgfqpoint{2.384870in}{1.431013in}}%
\pgfpathlineto{\pgfqpoint{2.565581in}{1.486671in}}%
\pgfusepath{stroke}%
\end{pgfscope}%
\begin{pgfscope}%
\pgfpathrectangle{\pgfqpoint{0.478365in}{0.467838in}}{\pgfqpoint{2.186607in}{1.629375in}}%
\pgfusepath{clip}%
\pgfsetrectcap%
\pgfsetroundjoin%
\pgfsetlinewidth{1.505625pt}%
\definecolor{currentstroke}{rgb}{1.000000,0.498039,0.054902}%
\pgfsetstrokecolor{currentstroke}%
\pgfsetdash{}{0pt}%
\pgfpathmoveto{\pgfqpoint{0.577757in}{0.541901in}}%
\pgfpathlineto{\pgfqpoint{0.758468in}{0.594602in}}%
\pgfpathlineto{\pgfqpoint{0.939179in}{0.634878in}}%
\pgfpathlineto{\pgfqpoint{1.119890in}{0.652938in}}%
\pgfpathlineto{\pgfqpoint{1.300602in}{0.658458in}}%
\pgfpathlineto{\pgfqpoint{1.481313in}{0.680334in}}%
\pgfpathlineto{\pgfqpoint{1.662024in}{0.687490in}}%
\pgfpathlineto{\pgfqpoint{1.842736in}{0.686273in}}%
\pgfpathlineto{\pgfqpoint{2.023447in}{0.663869in}}%
\pgfpathlineto{\pgfqpoint{2.204158in}{0.665257in}}%
\pgfpathlineto{\pgfqpoint{2.384870in}{0.666658in}}%
\pgfpathlineto{\pgfqpoint{2.565581in}{0.660126in}}%
\pgfusepath{stroke}%
\end{pgfscope}%
\begin{pgfscope}%
\pgfpathrectangle{\pgfqpoint{0.478365in}{0.467838in}}{\pgfqpoint{2.186607in}{1.629375in}}%
\pgfusepath{clip}%
\pgfsetbuttcap%
\pgfsetroundjoin%
\pgfsetlinewidth{0.501875pt}%
\definecolor{currentstroke}{rgb}{0.000000,0.000000,0.000000}%
\pgfsetstrokecolor{currentstroke}%
\pgfsetdash{{1.850000pt}{0.800000pt}}{0.000000pt}%
\pgfpathmoveto{\pgfqpoint{0.577757in}{0.541901in}}%
\pgfpathlineto{\pgfqpoint{0.758468in}{0.689892in}}%
\pgfpathlineto{\pgfqpoint{0.939179in}{0.834982in}}%
\pgfpathlineto{\pgfqpoint{1.119890in}{0.977255in}}%
\pgfpathlineto{\pgfqpoint{1.300602in}{1.116791in}}%
\pgfpathlineto{\pgfqpoint{1.481313in}{1.253670in}}%
\pgfpathlineto{\pgfqpoint{1.662024in}{1.387966in}}%
\pgfpathlineto{\pgfqpoint{1.842736in}{1.519752in}}%
\pgfpathlineto{\pgfqpoint{2.023447in}{1.649098in}}%
\pgfpathlineto{\pgfqpoint{2.204158in}{1.776070in}}%
\pgfpathlineto{\pgfqpoint{2.384870in}{1.900733in}}%
\pgfpathlineto{\pgfqpoint{2.565581in}{2.023151in}}%
\pgfusepath{stroke}%
\end{pgfscope}%
\begin{pgfscope}%
\pgfpathrectangle{\pgfqpoint{0.478365in}{0.467838in}}{\pgfqpoint{2.186607in}{1.629375in}}%
\pgfusepath{clip}%
\pgfsetbuttcap%
\pgfsetroundjoin%
\pgfsetlinewidth{0.501875pt}%
\definecolor{currentstroke}{rgb}{0.000000,0.000000,0.000000}%
\pgfsetstrokecolor{currentstroke}%
\pgfsetdash{{1.850000pt}{0.800000pt}}{0.000000pt}%
\pgfpathmoveto{\pgfqpoint{0.577757in}{0.541901in}}%
\pgfpathlineto{\pgfqpoint{0.758468in}{0.665431in}}%
\pgfpathlineto{\pgfqpoint{0.939179in}{0.768373in}}%
\pgfpathlineto{\pgfqpoint{1.119890in}{0.855478in}}%
\pgfpathlineto{\pgfqpoint{1.300602in}{0.930139in}}%
\pgfpathlineto{\pgfqpoint{1.481313in}{0.994845in}}%
\pgfpathlineto{\pgfqpoint{1.662024in}{1.051463in}}%
\pgfpathlineto{\pgfqpoint{1.842736in}{1.101420in}}%
\pgfpathlineto{\pgfqpoint{2.023447in}{1.145826in}}%
\pgfpathlineto{\pgfqpoint{2.204158in}{1.185558in}}%
\pgfpathlineto{\pgfqpoint{2.384870in}{1.221317in}}%
\pgfpathlineto{\pgfqpoint{2.565581in}{1.253670in}}%
\pgfusepath{stroke}%
\end{pgfscope}%
\begin{pgfscope}%
\pgfpathrectangle{\pgfqpoint{0.478365in}{0.467838in}}{\pgfqpoint{2.186607in}{1.629375in}}%
\pgfusepath{clip}%
\pgfsetbuttcap%
\pgfsetroundjoin%
\pgfsetlinewidth{0.501875pt}%
\definecolor{currentstroke}{rgb}{0.000000,0.000000,0.000000}%
\pgfsetstrokecolor{currentstroke}%
\pgfsetdash{{1.850000pt}{0.800000pt}}{0.000000pt}%
\pgfpathmoveto{\pgfqpoint{0.577757in}{0.541901in}}%
\pgfpathlineto{\pgfqpoint{0.758468in}{0.592228in}}%
\pgfpathlineto{\pgfqpoint{0.939179in}{0.617392in}}%
\pgfpathlineto{\pgfqpoint{1.119890in}{0.632490in}}%
\pgfpathlineto{\pgfqpoint{1.300602in}{0.642555in}}%
\pgfpathlineto{\pgfqpoint{1.481313in}{0.649745in}}%
\pgfpathlineto{\pgfqpoint{1.662024in}{0.655137in}}%
\pgfpathlineto{\pgfqpoint{1.842736in}{0.659331in}}%
\pgfpathlineto{\pgfqpoint{2.023447in}{0.662686in}}%
\pgfpathlineto{\pgfqpoint{2.204158in}{0.665431in}}%
\pgfpathlineto{\pgfqpoint{2.384870in}{0.667719in}}%
\pgfpathlineto{\pgfqpoint{2.565581in}{0.669654in}}%
\pgfusepath{stroke}%
\end{pgfscope}%
\begin{pgfscope}%
\pgfsetrectcap%
\pgfsetmiterjoin%
\pgfsetlinewidth{0.803000pt}%
\definecolor{currentstroke}{rgb}{0.000000,0.000000,0.000000}%
\pgfsetstrokecolor{currentstroke}%
\pgfsetdash{}{0pt}%
\pgfpathmoveto{\pgfqpoint{0.478365in}{0.467838in}}%
\pgfpathlineto{\pgfqpoint{0.478365in}{2.097213in}}%
\pgfusepath{stroke}%
\end{pgfscope}%
\begin{pgfscope}%
\pgfsetrectcap%
\pgfsetmiterjoin%
\pgfsetlinewidth{0.803000pt}%
\definecolor{currentstroke}{rgb}{0.000000,0.000000,0.000000}%
\pgfsetstrokecolor{currentstroke}%
\pgfsetdash{}{0pt}%
\pgfpathmoveto{\pgfqpoint{2.664972in}{0.467838in}}%
\pgfpathlineto{\pgfqpoint{2.664972in}{2.097213in}}%
\pgfusepath{stroke}%
\end{pgfscope}%
\begin{pgfscope}%
\pgfsetrectcap%
\pgfsetmiterjoin%
\pgfsetlinewidth{0.803000pt}%
\definecolor{currentstroke}{rgb}{0.000000,0.000000,0.000000}%
\pgfsetstrokecolor{currentstroke}%
\pgfsetdash{}{0pt}%
\pgfpathmoveto{\pgfqpoint{0.478365in}{0.467838in}}%
\pgfpathlineto{\pgfqpoint{2.664972in}{0.467838in}}%
\pgfusepath{stroke}%
\end{pgfscope}%
\begin{pgfscope}%
\pgfsetrectcap%
\pgfsetmiterjoin%
\pgfsetlinewidth{0.803000pt}%
\definecolor{currentstroke}{rgb}{0.000000,0.000000,0.000000}%
\pgfsetstrokecolor{currentstroke}%
\pgfsetdash{}{0pt}%
\pgfpathmoveto{\pgfqpoint{0.478365in}{2.097213in}}%
\pgfpathlineto{\pgfqpoint{2.664972in}{2.097213in}}%
\pgfusepath{stroke}%
\end{pgfscope}%
\begin{pgfscope}%
\definecolor{textcolor}{rgb}{0.000000,0.000000,0.000000}%
\pgfsetstrokecolor{textcolor}%
\pgfsetfillcolor{textcolor}%
\pgftext[x=1.759247in,y=1.327894in,left,base]{\color{textcolor}\rmfamily\fontsize{8.000000}{9.600000}\selectfont \(\displaystyle f=0.01\)}%
\end{pgfscope}%
\begin{pgfscope}%
\definecolor{textcolor}{rgb}{0.000000,0.000000,0.000000}%
\pgfsetstrokecolor{textcolor}%
\pgfsetfillcolor{textcolor}%
\pgftext[x=1.759247in,y=0.954241in,left,base]{\color{textcolor}\rmfamily\fontsize{8.000000}{9.600000}\selectfont \(\displaystyle f=0.1\)}%
\end{pgfscope}%
\begin{pgfscope}%
\definecolor{textcolor}{rgb}{0.000000,0.000000,0.000000}%
\pgfsetstrokecolor{textcolor}%
\pgfsetfillcolor{textcolor}%
\pgftext[x=1.759247in,y=0.530137in,left,base]{\color{textcolor}\rmfamily\fontsize{8.000000}{9.600000}\selectfont \(\displaystyle f=0.5\)}%
\end{pgfscope}%
\begin{pgfscope}%
\pgfsetrectcap%
\pgfsetroundjoin%
\pgfsetlinewidth{1.505625pt}%
\definecolor{currentstroke}{rgb}{0.121569,0.466667,0.705882}%
\pgfsetstrokecolor{currentstroke}%
\pgfsetdash{}{0pt}%
\pgfpathmoveto{\pgfqpoint{0.578365in}{1.951684in}}%
\pgfpathlineto{\pgfqpoint{0.800588in}{1.951684in}}%
\pgfusepath{stroke}%
\end{pgfscope}%
\begin{pgfscope}%
\definecolor{textcolor}{rgb}{0.000000,0.000000,0.000000}%
\pgfsetstrokecolor{textcolor}%
\pgfsetfillcolor{textcolor}%
\pgftext[x=0.889476in,y=1.912795in,left,base]{\color{textcolor}\rmfamily\fontsize{8.000000}{9.600000}\selectfont With Perple\_X}%
\end{pgfscope}%
\begin{pgfscope}%
\pgfsetrectcap%
\pgfsetroundjoin%
\pgfsetlinewidth{1.505625pt}%
\definecolor{currentstroke}{rgb}{1.000000,0.498039,0.054902}%
\pgfsetstrokecolor{currentstroke}%
\pgfsetdash{}{0pt}%
\pgfpathmoveto{\pgfqpoint{0.578365in}{1.785505in}}%
\pgfpathlineto{\pgfqpoint{0.800588in}{1.785505in}}%
\pgfusepath{stroke}%
\end{pgfscope}%
\begin{pgfscope}%
\definecolor{textcolor}{rgb}{0.000000,0.000000,0.000000}%
\pgfsetstrokecolor{textcolor}%
\pgfsetfillcolor{textcolor}%
\pgftext[x=0.889476in,y=1.746616in,left,base]{\color{textcolor}\rmfamily\fontsize{8.000000}{9.600000}\selectfont Without Perple\_X}%
\end{pgfscope}%
\end{pgfpicture}%
\makeatother%
\endgroup%

    \caption{Caption}
    \label{fig:my_label}
\end{figure}

\begin{figure}
    \centering
    %% Creator: Matplotlib, PGF backend
%%
%% To include the figure in your LaTeX document, write
%%   \input{<filename>.pgf}
%%
%% Make sure the required packages are loaded in your preamble
%%   \usepackage{pgf}
%%
%% Figures using additional raster images can only be included by \input if
%% they are in the same directory as the main LaTeX file. For loading figures
%% from other directories you can use the `import` package
%%   \usepackage{import}
%% and then include the figures with
%%   \import{<path to file>}{<filename>.pgf}
%%
%% Matplotlib used the following preamble
%%   \usepackage{fontspec}
%%   \setmainfont{DejaVuSerif.ttf}[Path=/home/connor/.local/lib/python3.8/site-packages/matplotlib/mpl-data/fonts/ttf/]
%%   \setsansfont{DejaVuSans.ttf}[Path=/home/connor/.local/lib/python3.8/site-packages/matplotlib/mpl-data/fonts/ttf/]
%%   \setmonofont{DejaVuSansMono.ttf}[Path=/home/connor/.local/lib/python3.8/site-packages/matplotlib/mpl-data/fonts/ttf/]
%%
\begingroup%
\makeatletter%
\begin{pgfpicture}%
\pgfpathrectangle{\pgfpointorigin}{\pgfqpoint{3.898197in}{3.017211in}}%
\pgfusepath{use as bounding box, clip}%
\begin{pgfscope}%
\pgfsetbuttcap%
\pgfsetmiterjoin%
\definecolor{currentfill}{rgb}{1.000000,1.000000,1.000000}%
\pgfsetfillcolor{currentfill}%
\pgfsetlinewidth{0.000000pt}%
\definecolor{currentstroke}{rgb}{1.000000,1.000000,1.000000}%
\pgfsetstrokecolor{currentstroke}%
\pgfsetdash{}{0pt}%
\pgfpathmoveto{\pgfqpoint{0.000000in}{-0.000000in}}%
\pgfpathlineto{\pgfqpoint{3.898197in}{-0.000000in}}%
\pgfpathlineto{\pgfqpoint{3.898197in}{3.017211in}}%
\pgfpathlineto{\pgfqpoint{0.000000in}{3.017211in}}%
\pgfpathclose%
\pgfusepath{fill}%
\end{pgfscope}%
\begin{pgfscope}%
\pgfsetbuttcap%
\pgfsetmiterjoin%
\definecolor{currentfill}{rgb}{1.000000,1.000000,1.000000}%
\pgfsetfillcolor{currentfill}%
\pgfsetlinewidth{0.000000pt}%
\definecolor{currentstroke}{rgb}{0.000000,0.000000,0.000000}%
\pgfsetstrokecolor{currentstroke}%
\pgfsetstrokeopacity{0.000000}%
\pgfsetdash{}{0pt}%
\pgfpathmoveto{\pgfqpoint{0.511159in}{0.467838in}}%
\pgfpathlineto{\pgfqpoint{3.798197in}{0.467838in}}%
\pgfpathlineto{\pgfqpoint{3.798197in}{2.917211in}}%
\pgfpathlineto{\pgfqpoint{0.511159in}{2.917211in}}%
\pgfpathclose%
\pgfusepath{fill}%
\end{pgfscope}%
\begin{pgfscope}%
\pgfsetbuttcap%
\pgfsetroundjoin%
\definecolor{currentfill}{rgb}{0.000000,0.000000,0.000000}%
\pgfsetfillcolor{currentfill}%
\pgfsetlinewidth{0.803000pt}%
\definecolor{currentstroke}{rgb}{0.000000,0.000000,0.000000}%
\pgfsetstrokecolor{currentstroke}%
\pgfsetdash{}{0pt}%
\pgfsys@defobject{currentmarker}{\pgfqpoint{0.000000in}{-0.048611in}}{\pgfqpoint{0.000000in}{0.000000in}}{%
\pgfpathmoveto{\pgfqpoint{0.000000in}{0.000000in}}%
\pgfpathlineto{\pgfqpoint{0.000000in}{-0.048611in}}%
\pgfusepath{stroke,fill}%
}%
\begin{pgfscope}%
\pgfsys@transformshift{0.932226in}{0.467838in}%
\pgfsys@useobject{currentmarker}{}%
\end{pgfscope}%
\end{pgfscope}%
\begin{pgfscope}%
\definecolor{textcolor}{rgb}{0.000000,0.000000,0.000000}%
\pgfsetstrokecolor{textcolor}%
\pgfsetfillcolor{textcolor}%
\pgftext[x=0.932226in,y=0.370616in,,top]{\color{textcolor}\rmfamily\fontsize{8.000000}{9.600000}\selectfont \(\displaystyle 2\)}%
\end{pgfscope}%
\begin{pgfscope}%
\pgfsetbuttcap%
\pgfsetroundjoin%
\definecolor{currentfill}{rgb}{0.000000,0.000000,0.000000}%
\pgfsetfillcolor{currentfill}%
\pgfsetlinewidth{0.803000pt}%
\definecolor{currentstroke}{rgb}{0.000000,0.000000,0.000000}%
\pgfsetstrokecolor{currentstroke}%
\pgfsetdash{}{0pt}%
\pgfsys@defobject{currentmarker}{\pgfqpoint{0.000000in}{-0.048611in}}{\pgfqpoint{0.000000in}{0.000000in}}{%
\pgfpathmoveto{\pgfqpoint{0.000000in}{0.000000in}}%
\pgfpathlineto{\pgfqpoint{0.000000in}{-0.048611in}}%
\pgfusepath{stroke,fill}%
}%
\begin{pgfscope}%
\pgfsys@transformshift{1.475538in}{0.467838in}%
\pgfsys@useobject{currentmarker}{}%
\end{pgfscope}%
\end{pgfscope}%
\begin{pgfscope}%
\definecolor{textcolor}{rgb}{0.000000,0.000000,0.000000}%
\pgfsetstrokecolor{textcolor}%
\pgfsetfillcolor{textcolor}%
\pgftext[x=1.475538in,y=0.370616in,,top]{\color{textcolor}\rmfamily\fontsize{8.000000}{9.600000}\selectfont \(\displaystyle 4\)}%
\end{pgfscope}%
\begin{pgfscope}%
\pgfsetbuttcap%
\pgfsetroundjoin%
\definecolor{currentfill}{rgb}{0.000000,0.000000,0.000000}%
\pgfsetfillcolor{currentfill}%
\pgfsetlinewidth{0.803000pt}%
\definecolor{currentstroke}{rgb}{0.000000,0.000000,0.000000}%
\pgfsetstrokecolor{currentstroke}%
\pgfsetdash{}{0pt}%
\pgfsys@defobject{currentmarker}{\pgfqpoint{0.000000in}{-0.048611in}}{\pgfqpoint{0.000000in}{0.000000in}}{%
\pgfpathmoveto{\pgfqpoint{0.000000in}{0.000000in}}%
\pgfpathlineto{\pgfqpoint{0.000000in}{-0.048611in}}%
\pgfusepath{stroke,fill}%
}%
\begin{pgfscope}%
\pgfsys@transformshift{2.018850in}{0.467838in}%
\pgfsys@useobject{currentmarker}{}%
\end{pgfscope}%
\end{pgfscope}%
\begin{pgfscope}%
\definecolor{textcolor}{rgb}{0.000000,0.000000,0.000000}%
\pgfsetstrokecolor{textcolor}%
\pgfsetfillcolor{textcolor}%
\pgftext[x=2.018850in,y=0.370616in,,top]{\color{textcolor}\rmfamily\fontsize{8.000000}{9.600000}\selectfont \(\displaystyle 6\)}%
\end{pgfscope}%
\begin{pgfscope}%
\pgfsetbuttcap%
\pgfsetroundjoin%
\definecolor{currentfill}{rgb}{0.000000,0.000000,0.000000}%
\pgfsetfillcolor{currentfill}%
\pgfsetlinewidth{0.803000pt}%
\definecolor{currentstroke}{rgb}{0.000000,0.000000,0.000000}%
\pgfsetstrokecolor{currentstroke}%
\pgfsetdash{}{0pt}%
\pgfsys@defobject{currentmarker}{\pgfqpoint{0.000000in}{-0.048611in}}{\pgfqpoint{0.000000in}{0.000000in}}{%
\pgfpathmoveto{\pgfqpoint{0.000000in}{0.000000in}}%
\pgfpathlineto{\pgfqpoint{0.000000in}{-0.048611in}}%
\pgfusepath{stroke,fill}%
}%
\begin{pgfscope}%
\pgfsys@transformshift{2.562162in}{0.467838in}%
\pgfsys@useobject{currentmarker}{}%
\end{pgfscope}%
\end{pgfscope}%
\begin{pgfscope}%
\definecolor{textcolor}{rgb}{0.000000,0.000000,0.000000}%
\pgfsetstrokecolor{textcolor}%
\pgfsetfillcolor{textcolor}%
\pgftext[x=2.562162in,y=0.370616in,,top]{\color{textcolor}\rmfamily\fontsize{8.000000}{9.600000}\selectfont \(\displaystyle 8\)}%
\end{pgfscope}%
\begin{pgfscope}%
\pgfsetbuttcap%
\pgfsetroundjoin%
\definecolor{currentfill}{rgb}{0.000000,0.000000,0.000000}%
\pgfsetfillcolor{currentfill}%
\pgfsetlinewidth{0.803000pt}%
\definecolor{currentstroke}{rgb}{0.000000,0.000000,0.000000}%
\pgfsetstrokecolor{currentstroke}%
\pgfsetdash{}{0pt}%
\pgfsys@defobject{currentmarker}{\pgfqpoint{0.000000in}{-0.048611in}}{\pgfqpoint{0.000000in}{0.000000in}}{%
\pgfpathmoveto{\pgfqpoint{0.000000in}{0.000000in}}%
\pgfpathlineto{\pgfqpoint{0.000000in}{-0.048611in}}%
\pgfusepath{stroke,fill}%
}%
\begin{pgfscope}%
\pgfsys@transformshift{3.105474in}{0.467838in}%
\pgfsys@useobject{currentmarker}{}%
\end{pgfscope}%
\end{pgfscope}%
\begin{pgfscope}%
\definecolor{textcolor}{rgb}{0.000000,0.000000,0.000000}%
\pgfsetstrokecolor{textcolor}%
\pgfsetfillcolor{textcolor}%
\pgftext[x=3.105474in,y=0.370616in,,top]{\color{textcolor}\rmfamily\fontsize{8.000000}{9.600000}\selectfont \(\displaystyle 10\)}%
\end{pgfscope}%
\begin{pgfscope}%
\pgfsetbuttcap%
\pgfsetroundjoin%
\definecolor{currentfill}{rgb}{0.000000,0.000000,0.000000}%
\pgfsetfillcolor{currentfill}%
\pgfsetlinewidth{0.803000pt}%
\definecolor{currentstroke}{rgb}{0.000000,0.000000,0.000000}%
\pgfsetstrokecolor{currentstroke}%
\pgfsetdash{}{0pt}%
\pgfsys@defobject{currentmarker}{\pgfqpoint{0.000000in}{-0.048611in}}{\pgfqpoint{0.000000in}{0.000000in}}{%
\pgfpathmoveto{\pgfqpoint{0.000000in}{0.000000in}}%
\pgfpathlineto{\pgfqpoint{0.000000in}{-0.048611in}}%
\pgfusepath{stroke,fill}%
}%
\begin{pgfscope}%
\pgfsys@transformshift{3.648786in}{0.467838in}%
\pgfsys@useobject{currentmarker}{}%
\end{pgfscope}%
\end{pgfscope}%
\begin{pgfscope}%
\definecolor{textcolor}{rgb}{0.000000,0.000000,0.000000}%
\pgfsetstrokecolor{textcolor}%
\pgfsetfillcolor{textcolor}%
\pgftext[x=3.648786in,y=0.370616in,,top]{\color{textcolor}\rmfamily\fontsize{8.000000}{9.600000}\selectfont \(\displaystyle 12\)}%
\end{pgfscope}%
\begin{pgfscope}%
\definecolor{textcolor}{rgb}{0.000000,0.000000,0.000000}%
\pgfsetstrokecolor{textcolor}%
\pgfsetfillcolor{textcolor}%
\pgftext[x=2.154678in,y=0.207530in,,top]{\color{textcolor}\rmfamily\fontsize{8.000000}{9.600000}\selectfont Number of processors}%
\end{pgfscope}%
\begin{pgfscope}%
\pgfsetbuttcap%
\pgfsetroundjoin%
\definecolor{currentfill}{rgb}{0.000000,0.000000,0.000000}%
\pgfsetfillcolor{currentfill}%
\pgfsetlinewidth{0.803000pt}%
\definecolor{currentstroke}{rgb}{0.000000,0.000000,0.000000}%
\pgfsetstrokecolor{currentstroke}%
\pgfsetdash{}{0pt}%
\pgfsys@defobject{currentmarker}{\pgfqpoint{-0.048611in}{0.000000in}}{\pgfqpoint{0.000000in}{0.000000in}}{%
\pgfpathmoveto{\pgfqpoint{0.000000in}{0.000000in}}%
\pgfpathlineto{\pgfqpoint{-0.048611in}{0.000000in}}%
\pgfusepath{stroke,fill}%
}%
\begin{pgfscope}%
\pgfsys@transformshift{0.511159in}{0.879056in}%
\pgfsys@useobject{currentmarker}{}%
\end{pgfscope}%
\end{pgfscope}%
\begin{pgfscope}%
\definecolor{textcolor}{rgb}{0.000000,0.000000,0.000000}%
\pgfsetstrokecolor{textcolor}%
\pgfsetfillcolor{textcolor}%
\pgftext[x=0.263086in,y=0.836846in,left,base]{\color{textcolor}\rmfamily\fontsize{8.000000}{9.600000}\selectfont \(\displaystyle 1.0\)}%
\end{pgfscope}%
\begin{pgfscope}%
\pgfsetbuttcap%
\pgfsetroundjoin%
\definecolor{currentfill}{rgb}{0.000000,0.000000,0.000000}%
\pgfsetfillcolor{currentfill}%
\pgfsetlinewidth{0.803000pt}%
\definecolor{currentstroke}{rgb}{0.000000,0.000000,0.000000}%
\pgfsetstrokecolor{currentstroke}%
\pgfsetdash{}{0pt}%
\pgfsys@defobject{currentmarker}{\pgfqpoint{-0.048611in}{0.000000in}}{\pgfqpoint{0.000000in}{0.000000in}}{%
\pgfpathmoveto{\pgfqpoint{0.000000in}{0.000000in}}%
\pgfpathlineto{\pgfqpoint{-0.048611in}{0.000000in}}%
\pgfusepath{stroke,fill}%
}%
\begin{pgfscope}%
\pgfsys@transformshift{0.511159in}{1.334050in}%
\pgfsys@useobject{currentmarker}{}%
\end{pgfscope}%
\end{pgfscope}%
\begin{pgfscope}%
\definecolor{textcolor}{rgb}{0.000000,0.000000,0.000000}%
\pgfsetstrokecolor{textcolor}%
\pgfsetfillcolor{textcolor}%
\pgftext[x=0.263086in,y=1.291840in,left,base]{\color{textcolor}\rmfamily\fontsize{8.000000}{9.600000}\selectfont \(\displaystyle 1.2\)}%
\end{pgfscope}%
\begin{pgfscope}%
\pgfsetbuttcap%
\pgfsetroundjoin%
\definecolor{currentfill}{rgb}{0.000000,0.000000,0.000000}%
\pgfsetfillcolor{currentfill}%
\pgfsetlinewidth{0.803000pt}%
\definecolor{currentstroke}{rgb}{0.000000,0.000000,0.000000}%
\pgfsetstrokecolor{currentstroke}%
\pgfsetdash{}{0pt}%
\pgfsys@defobject{currentmarker}{\pgfqpoint{-0.048611in}{0.000000in}}{\pgfqpoint{0.000000in}{0.000000in}}{%
\pgfpathmoveto{\pgfqpoint{0.000000in}{0.000000in}}%
\pgfpathlineto{\pgfqpoint{-0.048611in}{0.000000in}}%
\pgfusepath{stroke,fill}%
}%
\begin{pgfscope}%
\pgfsys@transformshift{0.511159in}{1.789043in}%
\pgfsys@useobject{currentmarker}{}%
\end{pgfscope}%
\end{pgfscope}%
\begin{pgfscope}%
\definecolor{textcolor}{rgb}{0.000000,0.000000,0.000000}%
\pgfsetstrokecolor{textcolor}%
\pgfsetfillcolor{textcolor}%
\pgftext[x=0.263086in,y=1.746834in,left,base]{\color{textcolor}\rmfamily\fontsize{8.000000}{9.600000}\selectfont \(\displaystyle 1.4\)}%
\end{pgfscope}%
\begin{pgfscope}%
\pgfsetbuttcap%
\pgfsetroundjoin%
\definecolor{currentfill}{rgb}{0.000000,0.000000,0.000000}%
\pgfsetfillcolor{currentfill}%
\pgfsetlinewidth{0.803000pt}%
\definecolor{currentstroke}{rgb}{0.000000,0.000000,0.000000}%
\pgfsetstrokecolor{currentstroke}%
\pgfsetdash{}{0pt}%
\pgfsys@defobject{currentmarker}{\pgfqpoint{-0.048611in}{0.000000in}}{\pgfqpoint{0.000000in}{0.000000in}}{%
\pgfpathmoveto{\pgfqpoint{0.000000in}{0.000000in}}%
\pgfpathlineto{\pgfqpoint{-0.048611in}{0.000000in}}%
\pgfusepath{stroke,fill}%
}%
\begin{pgfscope}%
\pgfsys@transformshift{0.511159in}{2.244037in}%
\pgfsys@useobject{currentmarker}{}%
\end{pgfscope}%
\end{pgfscope}%
\begin{pgfscope}%
\definecolor{textcolor}{rgb}{0.000000,0.000000,0.000000}%
\pgfsetstrokecolor{textcolor}%
\pgfsetfillcolor{textcolor}%
\pgftext[x=0.263086in,y=2.201828in,left,base]{\color{textcolor}\rmfamily\fontsize{8.000000}{9.600000}\selectfont \(\displaystyle 1.6\)}%
\end{pgfscope}%
\begin{pgfscope}%
\pgfsetbuttcap%
\pgfsetroundjoin%
\definecolor{currentfill}{rgb}{0.000000,0.000000,0.000000}%
\pgfsetfillcolor{currentfill}%
\pgfsetlinewidth{0.803000pt}%
\definecolor{currentstroke}{rgb}{0.000000,0.000000,0.000000}%
\pgfsetstrokecolor{currentstroke}%
\pgfsetdash{}{0pt}%
\pgfsys@defobject{currentmarker}{\pgfqpoint{-0.048611in}{0.000000in}}{\pgfqpoint{0.000000in}{0.000000in}}{%
\pgfpathmoveto{\pgfqpoint{0.000000in}{0.000000in}}%
\pgfpathlineto{\pgfqpoint{-0.048611in}{0.000000in}}%
\pgfusepath{stroke,fill}%
}%
\begin{pgfscope}%
\pgfsys@transformshift{0.511159in}{2.699031in}%
\pgfsys@useobject{currentmarker}{}%
\end{pgfscope}%
\end{pgfscope}%
\begin{pgfscope}%
\definecolor{textcolor}{rgb}{0.000000,0.000000,0.000000}%
\pgfsetstrokecolor{textcolor}%
\pgfsetfillcolor{textcolor}%
\pgftext[x=0.263086in,y=2.656822in,left,base]{\color{textcolor}\rmfamily\fontsize{8.000000}{9.600000}\selectfont \(\displaystyle 1.8\)}%
\end{pgfscope}%
\begin{pgfscope}%
\definecolor{textcolor}{rgb}{0.000000,0.000000,0.000000}%
\pgfsetstrokecolor{textcolor}%
\pgfsetfillcolor{textcolor}%
\pgftext[x=0.207530in,y=1.692525in,,bottom,rotate=90.000000]{\color{textcolor}\rmfamily\fontsize{8.000000}{9.600000}\selectfont Speedup}%
\end{pgfscope}%
\begin{pgfscope}%
\pgfpathrectangle{\pgfqpoint{0.511159in}{0.467838in}}{\pgfqpoint{3.287038in}{2.449373in}}%
\pgfusepath{clip}%
\pgfsetrectcap%
\pgfsetroundjoin%
\pgfsetlinewidth{1.505625pt}%
\definecolor{currentstroke}{rgb}{0.121569,0.466667,0.705882}%
\pgfsetstrokecolor{currentstroke}%
\pgfsetdash{}{0pt}%
\pgfpathmoveto{\pgfqpoint{0.660570in}{0.879056in}}%
\pgfpathlineto{\pgfqpoint{0.932226in}{0.785407in}}%
\pgfpathlineto{\pgfqpoint{1.203882in}{0.922758in}}%
\pgfpathlineto{\pgfqpoint{1.475538in}{0.860157in}}%
\pgfpathlineto{\pgfqpoint{1.747194in}{0.798628in}}%
\pgfpathlineto{\pgfqpoint{2.018850in}{0.693121in}}%
\pgfpathlineto{\pgfqpoint{2.290506in}{0.713403in}}%
\pgfpathlineto{\pgfqpoint{2.562162in}{0.673225in}}%
\pgfpathlineto{\pgfqpoint{2.833818in}{0.615749in}}%
\pgfpathlineto{\pgfqpoint{3.105474in}{0.597294in}}%
\pgfpathlineto{\pgfqpoint{3.377130in}{0.597294in}}%
\pgfpathlineto{\pgfqpoint{3.648786in}{0.579173in}}%
\pgfusepath{stroke}%
\end{pgfscope}%
\begin{pgfscope}%
\pgfpathrectangle{\pgfqpoint{0.511159in}{0.467838in}}{\pgfqpoint{3.287038in}{2.449373in}}%
\pgfusepath{clip}%
\pgfsetrectcap%
\pgfsetroundjoin%
\pgfsetlinewidth{1.505625pt}%
\definecolor{currentstroke}{rgb}{1.000000,0.498039,0.054902}%
\pgfsetstrokecolor{currentstroke}%
\pgfsetdash{}{0pt}%
\pgfpathmoveto{\pgfqpoint{0.660570in}{0.879056in}}%
\pgfpathlineto{\pgfqpoint{0.932226in}{1.580682in}}%
\pgfpathlineto{\pgfqpoint{1.203882in}{2.194793in}}%
\pgfpathlineto{\pgfqpoint{1.475538in}{2.436766in}}%
\pgfpathlineto{\pgfqpoint{1.747194in}{2.590265in}}%
\pgfpathlineto{\pgfqpoint{2.018850in}{2.610321in}}%
\pgfpathlineto{\pgfqpoint{2.290506in}{2.382208in}}%
\pgfpathlineto{\pgfqpoint{2.562162in}{2.805876in}}%
\pgfpathlineto{\pgfqpoint{2.833818in}{2.555646in}}%
\pgfpathlineto{\pgfqpoint{3.105474in}{2.531283in}}%
\pgfpathlineto{\pgfqpoint{3.377130in}{2.386695in}}%
\pgfpathlineto{\pgfqpoint{3.648786in}{2.502442in}}%
\pgfusepath{stroke}%
\end{pgfscope}%
\begin{pgfscope}%
\pgfsetrectcap%
\pgfsetmiterjoin%
\pgfsetlinewidth{0.803000pt}%
\definecolor{currentstroke}{rgb}{0.000000,0.000000,0.000000}%
\pgfsetstrokecolor{currentstroke}%
\pgfsetdash{}{0pt}%
\pgfpathmoveto{\pgfqpoint{0.511159in}{0.467838in}}%
\pgfpathlineto{\pgfqpoint{0.511159in}{2.917211in}}%
\pgfusepath{stroke}%
\end{pgfscope}%
\begin{pgfscope}%
\pgfsetrectcap%
\pgfsetmiterjoin%
\pgfsetlinewidth{0.803000pt}%
\definecolor{currentstroke}{rgb}{0.000000,0.000000,0.000000}%
\pgfsetstrokecolor{currentstroke}%
\pgfsetdash{}{0pt}%
\pgfpathmoveto{\pgfqpoint{3.798197in}{0.467838in}}%
\pgfpathlineto{\pgfqpoint{3.798197in}{2.917211in}}%
\pgfusepath{stroke}%
\end{pgfscope}%
\begin{pgfscope}%
\pgfsetrectcap%
\pgfsetmiterjoin%
\pgfsetlinewidth{0.803000pt}%
\definecolor{currentstroke}{rgb}{0.000000,0.000000,0.000000}%
\pgfsetstrokecolor{currentstroke}%
\pgfsetdash{}{0pt}%
\pgfpathmoveto{\pgfqpoint{0.511159in}{0.467838in}}%
\pgfpathlineto{\pgfqpoint{3.798197in}{0.467838in}}%
\pgfusepath{stroke}%
\end{pgfscope}%
\begin{pgfscope}%
\pgfsetrectcap%
\pgfsetmiterjoin%
\pgfsetlinewidth{0.803000pt}%
\definecolor{currentstroke}{rgb}{0.000000,0.000000,0.000000}%
\pgfsetstrokecolor{currentstroke}%
\pgfsetdash{}{0pt}%
\pgfpathmoveto{\pgfqpoint{0.511159in}{2.917211in}}%
\pgfpathlineto{\pgfqpoint{3.798197in}{2.917211in}}%
\pgfusepath{stroke}%
\end{pgfscope}%
\begin{pgfscope}%
\pgfsetrectcap%
\pgfsetroundjoin%
\pgfsetlinewidth{1.505625pt}%
\definecolor{currentstroke}{rgb}{0.121569,0.466667,0.705882}%
\pgfsetstrokecolor{currentstroke}%
\pgfsetdash{}{0pt}%
\pgfpathmoveto{\pgfqpoint{2.408950in}{1.785396in}}%
\pgfpathlineto{\pgfqpoint{2.631172in}{1.785396in}}%
\pgfusepath{stroke}%
\end{pgfscope}%
\begin{pgfscope}%
\definecolor{textcolor}{rgb}{0.000000,0.000000,0.000000}%
\pgfsetstrokecolor{textcolor}%
\pgfsetfillcolor{textcolor}%
\pgftext[x=2.720061in,y=1.746507in,left,base]{\color{textcolor}\rmfamily\fontsize{8.000000}{9.600000}\selectfont With Perple\_X}%
\end{pgfscope}%
\begin{pgfscope}%
\pgfsetrectcap%
\pgfsetroundjoin%
\pgfsetlinewidth{1.505625pt}%
\definecolor{currentstroke}{rgb}{1.000000,0.498039,0.054902}%
\pgfsetstrokecolor{currentstroke}%
\pgfsetdash{}{0pt}%
\pgfpathmoveto{\pgfqpoint{2.408950in}{1.619218in}}%
\pgfpathlineto{\pgfqpoint{2.631172in}{1.619218in}}%
\pgfusepath{stroke}%
\end{pgfscope}%
\begin{pgfscope}%
\definecolor{textcolor}{rgb}{0.000000,0.000000,0.000000}%
\pgfsetstrokecolor{textcolor}%
\pgfsetfillcolor{textcolor}%
\pgftext[x=2.720061in,y=1.580329in,left,base]{\color{textcolor}\rmfamily\fontsize{8.000000}{9.600000}\selectfont Without Perple\_X}%
\end{pgfscope}%
\end{pgfpicture}%
\makeatother%
\endgroup%

    \caption{Caption}
    \label{fig:my_label}
\end{figure}

\begin{figure}
    \centering
    %% Creator: Matplotlib, PGF backend
%%
%% To include the figure in your LaTeX document, write
%%   \input{<filename>.pgf}
%%
%% Make sure the required packages are loaded in your preamble
%%   \usepackage{pgf}
%%
%% Figures using additional raster images can only be included by \input if
%% they are in the same directory as the main LaTeX file. For loading figures
%% from other directories you can use the `import` package
%%   \usepackage{import}
%% and then include the figures with
%%   \import{<path to file>}{<filename>.pgf}
%%
%% Matplotlib used the following preamble
%%   \usepackage{fontspec}
%%   \setmainfont{DejaVuSerif.ttf}[Path=/home/connor/.local/lib/python3.8/site-packages/matplotlib/mpl-data/fonts/ttf/]
%%   \setsansfont{DejaVuSans.ttf}[Path=/home/connor/.local/lib/python3.8/site-packages/matplotlib/mpl-data/fonts/ttf/]
%%   \setmonofont{DejaVuSansMono.ttf}[Path=/home/connor/.local/lib/python3.8/site-packages/matplotlib/mpl-data/fonts/ttf/]
%%
\begingroup%
\makeatletter%
\begin{pgfpicture}%
\pgfpathrectangle{\pgfpointorigin}{\pgfqpoint{3.898197in}{3.017211in}}%
\pgfusepath{use as bounding box, clip}%
\begin{pgfscope}%
\pgfsetbuttcap%
\pgfsetmiterjoin%
\definecolor{currentfill}{rgb}{1.000000,1.000000,1.000000}%
\pgfsetfillcolor{currentfill}%
\pgfsetlinewidth{0.000000pt}%
\definecolor{currentstroke}{rgb}{1.000000,1.000000,1.000000}%
\pgfsetstrokecolor{currentstroke}%
\pgfsetdash{}{0pt}%
\pgfpathmoveto{\pgfqpoint{0.000000in}{-0.000000in}}%
\pgfpathlineto{\pgfqpoint{3.898197in}{-0.000000in}}%
\pgfpathlineto{\pgfqpoint{3.898197in}{3.017211in}}%
\pgfpathlineto{\pgfqpoint{0.000000in}{3.017211in}}%
\pgfpathclose%
\pgfusepath{fill}%
\end{pgfscope}%
\begin{pgfscope}%
\pgfsetbuttcap%
\pgfsetmiterjoin%
\definecolor{currentfill}{rgb}{1.000000,1.000000,1.000000}%
\pgfsetfillcolor{currentfill}%
\pgfsetlinewidth{0.000000pt}%
\definecolor{currentstroke}{rgb}{0.000000,0.000000,0.000000}%
\pgfsetstrokecolor{currentstroke}%
\pgfsetstrokeopacity{0.000000}%
\pgfsetdash{}{0pt}%
\pgfpathmoveto{\pgfqpoint{0.511159in}{0.467838in}}%
\pgfpathlineto{\pgfqpoint{3.798197in}{0.467838in}}%
\pgfpathlineto{\pgfqpoint{3.798197in}{2.917211in}}%
\pgfpathlineto{\pgfqpoint{0.511159in}{2.917211in}}%
\pgfpathclose%
\pgfusepath{fill}%
\end{pgfscope}%
\begin{pgfscope}%
\pgfsetbuttcap%
\pgfsetroundjoin%
\definecolor{currentfill}{rgb}{0.000000,0.000000,0.000000}%
\pgfsetfillcolor{currentfill}%
\pgfsetlinewidth{0.803000pt}%
\definecolor{currentstroke}{rgb}{0.000000,0.000000,0.000000}%
\pgfsetstrokecolor{currentstroke}%
\pgfsetdash{}{0pt}%
\pgfsys@defobject{currentmarker}{\pgfqpoint{0.000000in}{-0.048611in}}{\pgfqpoint{0.000000in}{0.000000in}}{%
\pgfpathmoveto{\pgfqpoint{0.000000in}{0.000000in}}%
\pgfpathlineto{\pgfqpoint{0.000000in}{-0.048611in}}%
\pgfusepath{stroke,fill}%
}%
\begin{pgfscope}%
\pgfsys@transformshift{0.932226in}{0.467838in}%
\pgfsys@useobject{currentmarker}{}%
\end{pgfscope}%
\end{pgfscope}%
\begin{pgfscope}%
\definecolor{textcolor}{rgb}{0.000000,0.000000,0.000000}%
\pgfsetstrokecolor{textcolor}%
\pgfsetfillcolor{textcolor}%
\pgftext[x=0.932226in,y=0.370616in,,top]{\color{textcolor}\rmfamily\fontsize{8.000000}{9.600000}\selectfont \(\displaystyle 2\)}%
\end{pgfscope}%
\begin{pgfscope}%
\pgfsetbuttcap%
\pgfsetroundjoin%
\definecolor{currentfill}{rgb}{0.000000,0.000000,0.000000}%
\pgfsetfillcolor{currentfill}%
\pgfsetlinewidth{0.803000pt}%
\definecolor{currentstroke}{rgb}{0.000000,0.000000,0.000000}%
\pgfsetstrokecolor{currentstroke}%
\pgfsetdash{}{0pt}%
\pgfsys@defobject{currentmarker}{\pgfqpoint{0.000000in}{-0.048611in}}{\pgfqpoint{0.000000in}{0.000000in}}{%
\pgfpathmoveto{\pgfqpoint{0.000000in}{0.000000in}}%
\pgfpathlineto{\pgfqpoint{0.000000in}{-0.048611in}}%
\pgfusepath{stroke,fill}%
}%
\begin{pgfscope}%
\pgfsys@transformshift{1.475538in}{0.467838in}%
\pgfsys@useobject{currentmarker}{}%
\end{pgfscope}%
\end{pgfscope}%
\begin{pgfscope}%
\definecolor{textcolor}{rgb}{0.000000,0.000000,0.000000}%
\pgfsetstrokecolor{textcolor}%
\pgfsetfillcolor{textcolor}%
\pgftext[x=1.475538in,y=0.370616in,,top]{\color{textcolor}\rmfamily\fontsize{8.000000}{9.600000}\selectfont \(\displaystyle 4\)}%
\end{pgfscope}%
\begin{pgfscope}%
\pgfsetbuttcap%
\pgfsetroundjoin%
\definecolor{currentfill}{rgb}{0.000000,0.000000,0.000000}%
\pgfsetfillcolor{currentfill}%
\pgfsetlinewidth{0.803000pt}%
\definecolor{currentstroke}{rgb}{0.000000,0.000000,0.000000}%
\pgfsetstrokecolor{currentstroke}%
\pgfsetdash{}{0pt}%
\pgfsys@defobject{currentmarker}{\pgfqpoint{0.000000in}{-0.048611in}}{\pgfqpoint{0.000000in}{0.000000in}}{%
\pgfpathmoveto{\pgfqpoint{0.000000in}{0.000000in}}%
\pgfpathlineto{\pgfqpoint{0.000000in}{-0.048611in}}%
\pgfusepath{stroke,fill}%
}%
\begin{pgfscope}%
\pgfsys@transformshift{2.018850in}{0.467838in}%
\pgfsys@useobject{currentmarker}{}%
\end{pgfscope}%
\end{pgfscope}%
\begin{pgfscope}%
\definecolor{textcolor}{rgb}{0.000000,0.000000,0.000000}%
\pgfsetstrokecolor{textcolor}%
\pgfsetfillcolor{textcolor}%
\pgftext[x=2.018850in,y=0.370616in,,top]{\color{textcolor}\rmfamily\fontsize{8.000000}{9.600000}\selectfont \(\displaystyle 6\)}%
\end{pgfscope}%
\begin{pgfscope}%
\pgfsetbuttcap%
\pgfsetroundjoin%
\definecolor{currentfill}{rgb}{0.000000,0.000000,0.000000}%
\pgfsetfillcolor{currentfill}%
\pgfsetlinewidth{0.803000pt}%
\definecolor{currentstroke}{rgb}{0.000000,0.000000,0.000000}%
\pgfsetstrokecolor{currentstroke}%
\pgfsetdash{}{0pt}%
\pgfsys@defobject{currentmarker}{\pgfqpoint{0.000000in}{-0.048611in}}{\pgfqpoint{0.000000in}{0.000000in}}{%
\pgfpathmoveto{\pgfqpoint{0.000000in}{0.000000in}}%
\pgfpathlineto{\pgfqpoint{0.000000in}{-0.048611in}}%
\pgfusepath{stroke,fill}%
}%
\begin{pgfscope}%
\pgfsys@transformshift{2.562162in}{0.467838in}%
\pgfsys@useobject{currentmarker}{}%
\end{pgfscope}%
\end{pgfscope}%
\begin{pgfscope}%
\definecolor{textcolor}{rgb}{0.000000,0.000000,0.000000}%
\pgfsetstrokecolor{textcolor}%
\pgfsetfillcolor{textcolor}%
\pgftext[x=2.562162in,y=0.370616in,,top]{\color{textcolor}\rmfamily\fontsize{8.000000}{9.600000}\selectfont \(\displaystyle 8\)}%
\end{pgfscope}%
\begin{pgfscope}%
\pgfsetbuttcap%
\pgfsetroundjoin%
\definecolor{currentfill}{rgb}{0.000000,0.000000,0.000000}%
\pgfsetfillcolor{currentfill}%
\pgfsetlinewidth{0.803000pt}%
\definecolor{currentstroke}{rgb}{0.000000,0.000000,0.000000}%
\pgfsetstrokecolor{currentstroke}%
\pgfsetdash{}{0pt}%
\pgfsys@defobject{currentmarker}{\pgfqpoint{0.000000in}{-0.048611in}}{\pgfqpoint{0.000000in}{0.000000in}}{%
\pgfpathmoveto{\pgfqpoint{0.000000in}{0.000000in}}%
\pgfpathlineto{\pgfqpoint{0.000000in}{-0.048611in}}%
\pgfusepath{stroke,fill}%
}%
\begin{pgfscope}%
\pgfsys@transformshift{3.105474in}{0.467838in}%
\pgfsys@useobject{currentmarker}{}%
\end{pgfscope}%
\end{pgfscope}%
\begin{pgfscope}%
\definecolor{textcolor}{rgb}{0.000000,0.000000,0.000000}%
\pgfsetstrokecolor{textcolor}%
\pgfsetfillcolor{textcolor}%
\pgftext[x=3.105474in,y=0.370616in,,top]{\color{textcolor}\rmfamily\fontsize{8.000000}{9.600000}\selectfont \(\displaystyle 10\)}%
\end{pgfscope}%
\begin{pgfscope}%
\pgfsetbuttcap%
\pgfsetroundjoin%
\definecolor{currentfill}{rgb}{0.000000,0.000000,0.000000}%
\pgfsetfillcolor{currentfill}%
\pgfsetlinewidth{0.803000pt}%
\definecolor{currentstroke}{rgb}{0.000000,0.000000,0.000000}%
\pgfsetstrokecolor{currentstroke}%
\pgfsetdash{}{0pt}%
\pgfsys@defobject{currentmarker}{\pgfqpoint{0.000000in}{-0.048611in}}{\pgfqpoint{0.000000in}{0.000000in}}{%
\pgfpathmoveto{\pgfqpoint{0.000000in}{0.000000in}}%
\pgfpathlineto{\pgfqpoint{0.000000in}{-0.048611in}}%
\pgfusepath{stroke,fill}%
}%
\begin{pgfscope}%
\pgfsys@transformshift{3.648786in}{0.467838in}%
\pgfsys@useobject{currentmarker}{}%
\end{pgfscope}%
\end{pgfscope}%
\begin{pgfscope}%
\definecolor{textcolor}{rgb}{0.000000,0.000000,0.000000}%
\pgfsetstrokecolor{textcolor}%
\pgfsetfillcolor{textcolor}%
\pgftext[x=3.648786in,y=0.370616in,,top]{\color{textcolor}\rmfamily\fontsize{8.000000}{9.600000}\selectfont \(\displaystyle 12\)}%
\end{pgfscope}%
\begin{pgfscope}%
\definecolor{textcolor}{rgb}{0.000000,0.000000,0.000000}%
\pgfsetstrokecolor{textcolor}%
\pgfsetfillcolor{textcolor}%
\pgftext[x=2.154678in,y=0.207530in,,top]{\color{textcolor}\rmfamily\fontsize{8.000000}{9.600000}\selectfont Number of processors}%
\end{pgfscope}%
\begin{pgfscope}%
\pgfsetbuttcap%
\pgfsetroundjoin%
\definecolor{currentfill}{rgb}{0.000000,0.000000,0.000000}%
\pgfsetfillcolor{currentfill}%
\pgfsetlinewidth{0.803000pt}%
\definecolor{currentstroke}{rgb}{0.000000,0.000000,0.000000}%
\pgfsetstrokecolor{currentstroke}%
\pgfsetdash{}{0pt}%
\pgfsys@defobject{currentmarker}{\pgfqpoint{-0.048611in}{0.000000in}}{\pgfqpoint{0.000000in}{0.000000in}}{%
\pgfpathmoveto{\pgfqpoint{0.000000in}{0.000000in}}%
\pgfpathlineto{\pgfqpoint{-0.048611in}{0.000000in}}%
\pgfusepath{stroke,fill}%
}%
\begin{pgfscope}%
\pgfsys@transformshift{0.511159in}{0.579173in}%
\pgfsys@useobject{currentmarker}{}%
\end{pgfscope}%
\end{pgfscope}%
\begin{pgfscope}%
\definecolor{textcolor}{rgb}{0.000000,0.000000,0.000000}%
\pgfsetstrokecolor{textcolor}%
\pgfsetfillcolor{textcolor}%
\pgftext[x=0.263086in,y=0.536964in,left,base]{\color{textcolor}\rmfamily\fontsize{8.000000}{9.600000}\selectfont \(\displaystyle 1.0\)}%
\end{pgfscope}%
\begin{pgfscope}%
\pgfsetbuttcap%
\pgfsetroundjoin%
\definecolor{currentfill}{rgb}{0.000000,0.000000,0.000000}%
\pgfsetfillcolor{currentfill}%
\pgfsetlinewidth{0.803000pt}%
\definecolor{currentstroke}{rgb}{0.000000,0.000000,0.000000}%
\pgfsetstrokecolor{currentstroke}%
\pgfsetdash{}{0pt}%
\pgfsys@defobject{currentmarker}{\pgfqpoint{-0.048611in}{0.000000in}}{\pgfqpoint{0.000000in}{0.000000in}}{%
\pgfpathmoveto{\pgfqpoint{0.000000in}{0.000000in}}%
\pgfpathlineto{\pgfqpoint{-0.048611in}{0.000000in}}%
\pgfusepath{stroke,fill}%
}%
\begin{pgfscope}%
\pgfsys@transformshift{0.511159in}{1.073758in}%
\pgfsys@useobject{currentmarker}{}%
\end{pgfscope}%
\end{pgfscope}%
\begin{pgfscope}%
\definecolor{textcolor}{rgb}{0.000000,0.000000,0.000000}%
\pgfsetstrokecolor{textcolor}%
\pgfsetfillcolor{textcolor}%
\pgftext[x=0.263086in,y=1.031549in,left,base]{\color{textcolor}\rmfamily\fontsize{8.000000}{9.600000}\selectfont \(\displaystyle 1.2\)}%
\end{pgfscope}%
\begin{pgfscope}%
\pgfsetbuttcap%
\pgfsetroundjoin%
\definecolor{currentfill}{rgb}{0.000000,0.000000,0.000000}%
\pgfsetfillcolor{currentfill}%
\pgfsetlinewidth{0.803000pt}%
\definecolor{currentstroke}{rgb}{0.000000,0.000000,0.000000}%
\pgfsetstrokecolor{currentstroke}%
\pgfsetdash{}{0pt}%
\pgfsys@defobject{currentmarker}{\pgfqpoint{-0.048611in}{0.000000in}}{\pgfqpoint{0.000000in}{0.000000in}}{%
\pgfpathmoveto{\pgfqpoint{0.000000in}{0.000000in}}%
\pgfpathlineto{\pgfqpoint{-0.048611in}{0.000000in}}%
\pgfusepath{stroke,fill}%
}%
\begin{pgfscope}%
\pgfsys@transformshift{0.511159in}{1.568343in}%
\pgfsys@useobject{currentmarker}{}%
\end{pgfscope}%
\end{pgfscope}%
\begin{pgfscope}%
\definecolor{textcolor}{rgb}{0.000000,0.000000,0.000000}%
\pgfsetstrokecolor{textcolor}%
\pgfsetfillcolor{textcolor}%
\pgftext[x=0.263086in,y=1.526134in,left,base]{\color{textcolor}\rmfamily\fontsize{8.000000}{9.600000}\selectfont \(\displaystyle 1.4\)}%
\end{pgfscope}%
\begin{pgfscope}%
\pgfsetbuttcap%
\pgfsetroundjoin%
\definecolor{currentfill}{rgb}{0.000000,0.000000,0.000000}%
\pgfsetfillcolor{currentfill}%
\pgfsetlinewidth{0.803000pt}%
\definecolor{currentstroke}{rgb}{0.000000,0.000000,0.000000}%
\pgfsetstrokecolor{currentstroke}%
\pgfsetdash{}{0pt}%
\pgfsys@defobject{currentmarker}{\pgfqpoint{-0.048611in}{0.000000in}}{\pgfqpoint{0.000000in}{0.000000in}}{%
\pgfpathmoveto{\pgfqpoint{0.000000in}{0.000000in}}%
\pgfpathlineto{\pgfqpoint{-0.048611in}{0.000000in}}%
\pgfusepath{stroke,fill}%
}%
\begin{pgfscope}%
\pgfsys@transformshift{0.511159in}{2.062928in}%
\pgfsys@useobject{currentmarker}{}%
\end{pgfscope}%
\end{pgfscope}%
\begin{pgfscope}%
\definecolor{textcolor}{rgb}{0.000000,0.000000,0.000000}%
\pgfsetstrokecolor{textcolor}%
\pgfsetfillcolor{textcolor}%
\pgftext[x=0.263086in,y=2.020719in,left,base]{\color{textcolor}\rmfamily\fontsize{8.000000}{9.600000}\selectfont \(\displaystyle 1.6\)}%
\end{pgfscope}%
\begin{pgfscope}%
\pgfsetbuttcap%
\pgfsetroundjoin%
\definecolor{currentfill}{rgb}{0.000000,0.000000,0.000000}%
\pgfsetfillcolor{currentfill}%
\pgfsetlinewidth{0.803000pt}%
\definecolor{currentstroke}{rgb}{0.000000,0.000000,0.000000}%
\pgfsetstrokecolor{currentstroke}%
\pgfsetdash{}{0pt}%
\pgfsys@defobject{currentmarker}{\pgfqpoint{-0.048611in}{0.000000in}}{\pgfqpoint{0.000000in}{0.000000in}}{%
\pgfpathmoveto{\pgfqpoint{0.000000in}{0.000000in}}%
\pgfpathlineto{\pgfqpoint{-0.048611in}{0.000000in}}%
\pgfusepath{stroke,fill}%
}%
\begin{pgfscope}%
\pgfsys@transformshift{0.511159in}{2.557513in}%
\pgfsys@useobject{currentmarker}{}%
\end{pgfscope}%
\end{pgfscope}%
\begin{pgfscope}%
\definecolor{textcolor}{rgb}{0.000000,0.000000,0.000000}%
\pgfsetstrokecolor{textcolor}%
\pgfsetfillcolor{textcolor}%
\pgftext[x=0.263086in,y=2.515304in,left,base]{\color{textcolor}\rmfamily\fontsize{8.000000}{9.600000}\selectfont \(\displaystyle 1.8\)}%
\end{pgfscope}%
\begin{pgfscope}%
\definecolor{textcolor}{rgb}{0.000000,0.000000,0.000000}%
\pgfsetstrokecolor{textcolor}%
\pgfsetfillcolor{textcolor}%
\pgftext[x=0.207530in,y=1.692525in,,bottom,rotate=90.000000]{\color{textcolor}\rmfamily\fontsize{8.000000}{9.600000}\selectfont Speedup}%
\end{pgfscope}%
\begin{pgfscope}%
\pgfpathrectangle{\pgfqpoint{0.511159in}{0.467838in}}{\pgfqpoint{3.287038in}{2.449373in}}%
\pgfusepath{clip}%
\pgfsetrectcap%
\pgfsetroundjoin%
\pgfsetlinewidth{1.505625pt}%
\definecolor{currentstroke}{rgb}{0.121569,0.466667,0.705882}%
\pgfsetstrokecolor{currentstroke}%
\pgfsetdash{}{0pt}%
\pgfpathmoveto{\pgfqpoint{0.660570in}{0.579173in}}%
\pgfpathlineto{\pgfqpoint{0.932226in}{0.820284in}}%
\pgfpathlineto{\pgfqpoint{1.203882in}{1.477721in}}%
\pgfpathlineto{\pgfqpoint{1.475538in}{1.955944in}}%
\pgfpathlineto{\pgfqpoint{1.747194in}{2.025433in}}%
\pgfpathlineto{\pgfqpoint{2.018850in}{2.265689in}}%
\pgfpathlineto{\pgfqpoint{2.290506in}{2.249813in}}%
\pgfpathlineto{\pgfqpoint{2.562162in}{2.501447in}}%
\pgfpathlineto{\pgfqpoint{2.833818in}{2.265689in}}%
\pgfpathlineto{\pgfqpoint{3.105474in}{2.483724in}}%
\pgfpathlineto{\pgfqpoint{3.377130in}{2.414241in}}%
\pgfpathlineto{\pgfqpoint{3.648786in}{2.805876in}}%
\pgfusepath{stroke}%
\end{pgfscope}%
\begin{pgfscope}%
\pgfsetrectcap%
\pgfsetmiterjoin%
\pgfsetlinewidth{0.803000pt}%
\definecolor{currentstroke}{rgb}{0.000000,0.000000,0.000000}%
\pgfsetstrokecolor{currentstroke}%
\pgfsetdash{}{0pt}%
\pgfpathmoveto{\pgfqpoint{0.511159in}{0.467838in}}%
\pgfpathlineto{\pgfqpoint{0.511159in}{2.917211in}}%
\pgfusepath{stroke}%
\end{pgfscope}%
\begin{pgfscope}%
\pgfsetrectcap%
\pgfsetmiterjoin%
\pgfsetlinewidth{0.803000pt}%
\definecolor{currentstroke}{rgb}{0.000000,0.000000,0.000000}%
\pgfsetstrokecolor{currentstroke}%
\pgfsetdash{}{0pt}%
\pgfpathmoveto{\pgfqpoint{3.798197in}{0.467838in}}%
\pgfpathlineto{\pgfqpoint{3.798197in}{2.917211in}}%
\pgfusepath{stroke}%
\end{pgfscope}%
\begin{pgfscope}%
\pgfsetrectcap%
\pgfsetmiterjoin%
\pgfsetlinewidth{0.803000pt}%
\definecolor{currentstroke}{rgb}{0.000000,0.000000,0.000000}%
\pgfsetstrokecolor{currentstroke}%
\pgfsetdash{}{0pt}%
\pgfpathmoveto{\pgfqpoint{0.511159in}{0.467838in}}%
\pgfpathlineto{\pgfqpoint{3.798197in}{0.467838in}}%
\pgfusepath{stroke}%
\end{pgfscope}%
\begin{pgfscope}%
\pgfsetrectcap%
\pgfsetmiterjoin%
\pgfsetlinewidth{0.803000pt}%
\definecolor{currentstroke}{rgb}{0.000000,0.000000,0.000000}%
\pgfsetstrokecolor{currentstroke}%
\pgfsetdash{}{0pt}%
\pgfpathmoveto{\pgfqpoint{0.511159in}{2.917211in}}%
\pgfpathlineto{\pgfqpoint{3.798197in}{2.917211in}}%
\pgfusepath{stroke}%
\end{pgfscope}%
\begin{pgfscope}%
\pgfsetrectcap%
\pgfsetroundjoin%
\pgfsetlinewidth{1.505625pt}%
\definecolor{currentstroke}{rgb}{0.121569,0.466667,0.705882}%
\pgfsetstrokecolor{currentstroke}%
\pgfsetdash{}{0pt}%
\pgfpathmoveto{\pgfqpoint{0.611159in}{2.771682in}}%
\pgfpathlineto{\pgfqpoint{0.833381in}{2.771682in}}%
\pgfusepath{stroke}%
\end{pgfscope}%
\begin{pgfscope}%
\definecolor{textcolor}{rgb}{0.000000,0.000000,0.000000}%
\pgfsetstrokecolor{textcolor}%
\pgfsetfillcolor{textcolor}%
\pgftext[x=0.922270in,y=2.732793in,left,base]{\color{textcolor}\rmfamily\fontsize{8.000000}{9.600000}\selectfont With Perple\_X}%
\end{pgfscope}%
\end{pgfpicture}%
\makeatother%
\endgroup%

    \caption{Caption}
    \label{fig:my_label}
\end{figure}

\subsubsection{Cache usage}

\begin{figure}
    \centering
    %% Creator: Matplotlib, PGF backend
%%
%% To include the figure in your LaTeX document, write
%%   \input{<filename>.pgf}
%%
%% Make sure the required packages are loaded in your preamble
%%   \usepackage{pgf}
%%
%% Figures using additional raster images can only be included by \input if
%% they are in the same directory as the main LaTeX file. For loading figures
%% from other directories you can use the `import` package
%%   \usepackage{import}
%% and then include the figures with
%%   \import{<path to file>}{<filename>.pgf}
%%
%% Matplotlib used the following preamble
%%   \usepackage{fontspec}
%%   \setmainfont{DejaVuSerif.ttf}[Path=/home/connor/.local/lib/python3.8/site-packages/matplotlib/mpl-data/fonts/ttf/]
%%   \setsansfont{DejaVuSans.ttf}[Path=/home/connor/.local/lib/python3.8/site-packages/matplotlib/mpl-data/fonts/ttf/]
%%   \setmonofont{DejaVuSansMono.ttf}[Path=/home/connor/.local/lib/python3.8/site-packages/matplotlib/mpl-data/fonts/ttf/]
%%
\begingroup%
\makeatletter%
\begin{pgfpicture}%
\pgfpathrectangle{\pgfpointorigin}{\pgfqpoint{3.898197in}{3.018785in}}%
\pgfusepath{use as bounding box, clip}%
\begin{pgfscope}%
\pgfsetbuttcap%
\pgfsetmiterjoin%
\definecolor{currentfill}{rgb}{1.000000,1.000000,1.000000}%
\pgfsetfillcolor{currentfill}%
\pgfsetlinewidth{0.000000pt}%
\definecolor{currentstroke}{rgb}{1.000000,1.000000,1.000000}%
\pgfsetstrokecolor{currentstroke}%
\pgfsetdash{}{0pt}%
\pgfpathmoveto{\pgfqpoint{0.000000in}{0.000000in}}%
\pgfpathlineto{\pgfqpoint{3.898197in}{0.000000in}}%
\pgfpathlineto{\pgfqpoint{3.898197in}{3.018785in}}%
\pgfpathlineto{\pgfqpoint{0.000000in}{3.018785in}}%
\pgfpathclose%
\pgfusepath{fill}%
\end{pgfscope}%
\begin{pgfscope}%
\pgfsetbuttcap%
\pgfsetmiterjoin%
\definecolor{currentfill}{rgb}{1.000000,1.000000,1.000000}%
\pgfsetfillcolor{currentfill}%
\pgfsetlinewidth{0.000000pt}%
\definecolor{currentstroke}{rgb}{0.000000,0.000000,0.000000}%
\pgfsetstrokecolor{currentstroke}%
\pgfsetstrokeopacity{0.000000}%
\pgfsetdash{}{0pt}%
\pgfpathmoveto{\pgfqpoint{0.511159in}{0.469412in}}%
\pgfpathlineto{\pgfqpoint{3.798197in}{0.469412in}}%
\pgfpathlineto{\pgfqpoint{3.798197in}{2.918785in}}%
\pgfpathlineto{\pgfqpoint{0.511159in}{2.918785in}}%
\pgfpathclose%
\pgfusepath{fill}%
\end{pgfscope}%
\begin{pgfscope}%
\pgfsetbuttcap%
\pgfsetroundjoin%
\definecolor{currentfill}{rgb}{0.000000,0.000000,0.000000}%
\pgfsetfillcolor{currentfill}%
\pgfsetlinewidth{0.803000pt}%
\definecolor{currentstroke}{rgb}{0.000000,0.000000,0.000000}%
\pgfsetstrokecolor{currentstroke}%
\pgfsetdash{}{0pt}%
\pgfsys@defobject{currentmarker}{\pgfqpoint{0.000000in}{-0.048611in}}{\pgfqpoint{0.000000in}{0.000000in}}{%
\pgfpathmoveto{\pgfqpoint{0.000000in}{0.000000in}}%
\pgfpathlineto{\pgfqpoint{0.000000in}{-0.048611in}}%
\pgfusepath{stroke,fill}%
}%
\begin{pgfscope}%
\pgfsys@transformshift{0.660570in}{0.469412in}%
\pgfsys@useobject{currentmarker}{}%
\end{pgfscope}%
\end{pgfscope}%
\begin{pgfscope}%
\definecolor{textcolor}{rgb}{0.000000,0.000000,0.000000}%
\pgfsetstrokecolor{textcolor}%
\pgfsetfillcolor{textcolor}%
\pgftext[x=0.660570in,y=0.372189in,,top]{\color{textcolor}\rmfamily\fontsize{8.000000}{9.600000}\selectfont \(\displaystyle 0\)}%
\end{pgfscope}%
\begin{pgfscope}%
\pgfsetbuttcap%
\pgfsetroundjoin%
\definecolor{currentfill}{rgb}{0.000000,0.000000,0.000000}%
\pgfsetfillcolor{currentfill}%
\pgfsetlinewidth{0.803000pt}%
\definecolor{currentstroke}{rgb}{0.000000,0.000000,0.000000}%
\pgfsetstrokecolor{currentstroke}%
\pgfsetdash{}{0pt}%
\pgfsys@defobject{currentmarker}{\pgfqpoint{0.000000in}{-0.048611in}}{\pgfqpoint{0.000000in}{0.000000in}}{%
\pgfpathmoveto{\pgfqpoint{0.000000in}{0.000000in}}%
\pgfpathlineto{\pgfqpoint{0.000000in}{-0.048611in}}%
\pgfusepath{stroke,fill}%
}%
\begin{pgfscope}%
\pgfsys@transformshift{1.058999in}{0.469412in}%
\pgfsys@useobject{currentmarker}{}%
\end{pgfscope}%
\end{pgfscope}%
\begin{pgfscope}%
\definecolor{textcolor}{rgb}{0.000000,0.000000,0.000000}%
\pgfsetstrokecolor{textcolor}%
\pgfsetfillcolor{textcolor}%
\pgftext[x=1.058999in,y=0.372189in,,top]{\color{textcolor}\rmfamily\fontsize{8.000000}{9.600000}\selectfont \(\displaystyle 200000\)}%
\end{pgfscope}%
\begin{pgfscope}%
\pgfsetbuttcap%
\pgfsetroundjoin%
\definecolor{currentfill}{rgb}{0.000000,0.000000,0.000000}%
\pgfsetfillcolor{currentfill}%
\pgfsetlinewidth{0.803000pt}%
\definecolor{currentstroke}{rgb}{0.000000,0.000000,0.000000}%
\pgfsetstrokecolor{currentstroke}%
\pgfsetdash{}{0pt}%
\pgfsys@defobject{currentmarker}{\pgfqpoint{0.000000in}{-0.048611in}}{\pgfqpoint{0.000000in}{0.000000in}}{%
\pgfpathmoveto{\pgfqpoint{0.000000in}{0.000000in}}%
\pgfpathlineto{\pgfqpoint{0.000000in}{-0.048611in}}%
\pgfusepath{stroke,fill}%
}%
\begin{pgfscope}%
\pgfsys@transformshift{1.457427in}{0.469412in}%
\pgfsys@useobject{currentmarker}{}%
\end{pgfscope}%
\end{pgfscope}%
\begin{pgfscope}%
\definecolor{textcolor}{rgb}{0.000000,0.000000,0.000000}%
\pgfsetstrokecolor{textcolor}%
\pgfsetfillcolor{textcolor}%
\pgftext[x=1.457427in,y=0.372189in,,top]{\color{textcolor}\rmfamily\fontsize{8.000000}{9.600000}\selectfont \(\displaystyle 400000\)}%
\end{pgfscope}%
\begin{pgfscope}%
\pgfsetbuttcap%
\pgfsetroundjoin%
\definecolor{currentfill}{rgb}{0.000000,0.000000,0.000000}%
\pgfsetfillcolor{currentfill}%
\pgfsetlinewidth{0.803000pt}%
\definecolor{currentstroke}{rgb}{0.000000,0.000000,0.000000}%
\pgfsetstrokecolor{currentstroke}%
\pgfsetdash{}{0pt}%
\pgfsys@defobject{currentmarker}{\pgfqpoint{0.000000in}{-0.048611in}}{\pgfqpoint{0.000000in}{0.000000in}}{%
\pgfpathmoveto{\pgfqpoint{0.000000in}{0.000000in}}%
\pgfpathlineto{\pgfqpoint{0.000000in}{-0.048611in}}%
\pgfusepath{stroke,fill}%
}%
\begin{pgfscope}%
\pgfsys@transformshift{1.855856in}{0.469412in}%
\pgfsys@useobject{currentmarker}{}%
\end{pgfscope}%
\end{pgfscope}%
\begin{pgfscope}%
\definecolor{textcolor}{rgb}{0.000000,0.000000,0.000000}%
\pgfsetstrokecolor{textcolor}%
\pgfsetfillcolor{textcolor}%
\pgftext[x=1.855856in,y=0.372189in,,top]{\color{textcolor}\rmfamily\fontsize{8.000000}{9.600000}\selectfont \(\displaystyle 600000\)}%
\end{pgfscope}%
\begin{pgfscope}%
\pgfsetbuttcap%
\pgfsetroundjoin%
\definecolor{currentfill}{rgb}{0.000000,0.000000,0.000000}%
\pgfsetfillcolor{currentfill}%
\pgfsetlinewidth{0.803000pt}%
\definecolor{currentstroke}{rgb}{0.000000,0.000000,0.000000}%
\pgfsetstrokecolor{currentstroke}%
\pgfsetdash{}{0pt}%
\pgfsys@defobject{currentmarker}{\pgfqpoint{0.000000in}{-0.048611in}}{\pgfqpoint{0.000000in}{0.000000in}}{%
\pgfpathmoveto{\pgfqpoint{0.000000in}{0.000000in}}%
\pgfpathlineto{\pgfqpoint{0.000000in}{-0.048611in}}%
\pgfusepath{stroke,fill}%
}%
\begin{pgfscope}%
\pgfsys@transformshift{2.254285in}{0.469412in}%
\pgfsys@useobject{currentmarker}{}%
\end{pgfscope}%
\end{pgfscope}%
\begin{pgfscope}%
\definecolor{textcolor}{rgb}{0.000000,0.000000,0.000000}%
\pgfsetstrokecolor{textcolor}%
\pgfsetfillcolor{textcolor}%
\pgftext[x=2.254285in,y=0.372189in,,top]{\color{textcolor}\rmfamily\fontsize{8.000000}{9.600000}\selectfont \(\displaystyle 800000\)}%
\end{pgfscope}%
\begin{pgfscope}%
\pgfsetbuttcap%
\pgfsetroundjoin%
\definecolor{currentfill}{rgb}{0.000000,0.000000,0.000000}%
\pgfsetfillcolor{currentfill}%
\pgfsetlinewidth{0.803000pt}%
\definecolor{currentstroke}{rgb}{0.000000,0.000000,0.000000}%
\pgfsetstrokecolor{currentstroke}%
\pgfsetdash{}{0pt}%
\pgfsys@defobject{currentmarker}{\pgfqpoint{0.000000in}{-0.048611in}}{\pgfqpoint{0.000000in}{0.000000in}}{%
\pgfpathmoveto{\pgfqpoint{0.000000in}{0.000000in}}%
\pgfpathlineto{\pgfqpoint{0.000000in}{-0.048611in}}%
\pgfusepath{stroke,fill}%
}%
\begin{pgfscope}%
\pgfsys@transformshift{2.652714in}{0.469412in}%
\pgfsys@useobject{currentmarker}{}%
\end{pgfscope}%
\end{pgfscope}%
\begin{pgfscope}%
\definecolor{textcolor}{rgb}{0.000000,0.000000,0.000000}%
\pgfsetstrokecolor{textcolor}%
\pgfsetfillcolor{textcolor}%
\pgftext[x=2.652714in,y=0.372189in,,top]{\color{textcolor}\rmfamily\fontsize{8.000000}{9.600000}\selectfont \(\displaystyle 1000000\)}%
\end{pgfscope}%
\begin{pgfscope}%
\pgfsetbuttcap%
\pgfsetroundjoin%
\definecolor{currentfill}{rgb}{0.000000,0.000000,0.000000}%
\pgfsetfillcolor{currentfill}%
\pgfsetlinewidth{0.803000pt}%
\definecolor{currentstroke}{rgb}{0.000000,0.000000,0.000000}%
\pgfsetstrokecolor{currentstroke}%
\pgfsetdash{}{0pt}%
\pgfsys@defobject{currentmarker}{\pgfqpoint{0.000000in}{-0.048611in}}{\pgfqpoint{0.000000in}{0.000000in}}{%
\pgfpathmoveto{\pgfqpoint{0.000000in}{0.000000in}}%
\pgfpathlineto{\pgfqpoint{0.000000in}{-0.048611in}}%
\pgfusepath{stroke,fill}%
}%
\begin{pgfscope}%
\pgfsys@transformshift{3.051143in}{0.469412in}%
\pgfsys@useobject{currentmarker}{}%
\end{pgfscope}%
\end{pgfscope}%
\begin{pgfscope}%
\definecolor{textcolor}{rgb}{0.000000,0.000000,0.000000}%
\pgfsetstrokecolor{textcolor}%
\pgfsetfillcolor{textcolor}%
\pgftext[x=3.051143in,y=0.372189in,,top]{\color{textcolor}\rmfamily\fontsize{8.000000}{9.600000}\selectfont \(\displaystyle 1200000\)}%
\end{pgfscope}%
\begin{pgfscope}%
\pgfsetbuttcap%
\pgfsetroundjoin%
\definecolor{currentfill}{rgb}{0.000000,0.000000,0.000000}%
\pgfsetfillcolor{currentfill}%
\pgfsetlinewidth{0.803000pt}%
\definecolor{currentstroke}{rgb}{0.000000,0.000000,0.000000}%
\pgfsetstrokecolor{currentstroke}%
\pgfsetdash{}{0pt}%
\pgfsys@defobject{currentmarker}{\pgfqpoint{0.000000in}{-0.048611in}}{\pgfqpoint{0.000000in}{0.000000in}}{%
\pgfpathmoveto{\pgfqpoint{0.000000in}{0.000000in}}%
\pgfpathlineto{\pgfqpoint{0.000000in}{-0.048611in}}%
\pgfusepath{stroke,fill}%
}%
\begin{pgfscope}%
\pgfsys@transformshift{3.449571in}{0.469412in}%
\pgfsys@useobject{currentmarker}{}%
\end{pgfscope}%
\end{pgfscope}%
\begin{pgfscope}%
\definecolor{textcolor}{rgb}{0.000000,0.000000,0.000000}%
\pgfsetstrokecolor{textcolor}%
\pgfsetfillcolor{textcolor}%
\pgftext[x=3.449571in,y=0.372189in,,top]{\color{textcolor}\rmfamily\fontsize{8.000000}{9.600000}\selectfont \(\displaystyle 1400000\)}%
\end{pgfscope}%
\begin{pgfscope}%
\definecolor{textcolor}{rgb}{0.000000,0.000000,0.000000}%
\pgfsetstrokecolor{textcolor}%
\pgfsetfillcolor{textcolor}%
\pgftext[x=2.154678in,y=0.209104in,,top]{\color{textcolor}\rmfamily\fontsize{8.000000}{9.600000}\selectfont Time (years)}%
\end{pgfscope}%
\begin{pgfscope}%
\pgfsetbuttcap%
\pgfsetroundjoin%
\definecolor{currentfill}{rgb}{0.000000,0.000000,0.000000}%
\pgfsetfillcolor{currentfill}%
\pgfsetlinewidth{0.803000pt}%
\definecolor{currentstroke}{rgb}{0.000000,0.000000,0.000000}%
\pgfsetstrokecolor{currentstroke}%
\pgfsetdash{}{0pt}%
\pgfsys@defobject{currentmarker}{\pgfqpoint{-0.048611in}{0.000000in}}{\pgfqpoint{0.000000in}{0.000000in}}{%
\pgfpathmoveto{\pgfqpoint{0.000000in}{0.000000in}}%
\pgfpathlineto{\pgfqpoint{-0.048611in}{0.000000in}}%
\pgfusepath{stroke,fill}%
}%
\begin{pgfscope}%
\pgfsys@transformshift{0.511159in}{0.578513in}%
\pgfsys@useobject{currentmarker}{}%
\end{pgfscope}%
\end{pgfscope}%
\begin{pgfscope}%
\definecolor{textcolor}{rgb}{0.000000,0.000000,0.000000}%
\pgfsetstrokecolor{textcolor}%
\pgfsetfillcolor{textcolor}%
\pgftext[x=0.263086in,y=0.536304in,left,base]{\color{textcolor}\rmfamily\fontsize{8.000000}{9.600000}\selectfont \(\displaystyle 0.0\)}%
\end{pgfscope}%
\begin{pgfscope}%
\pgfsetbuttcap%
\pgfsetroundjoin%
\definecolor{currentfill}{rgb}{0.000000,0.000000,0.000000}%
\pgfsetfillcolor{currentfill}%
\pgfsetlinewidth{0.803000pt}%
\definecolor{currentstroke}{rgb}{0.000000,0.000000,0.000000}%
\pgfsetstrokecolor{currentstroke}%
\pgfsetdash{}{0pt}%
\pgfsys@defobject{currentmarker}{\pgfqpoint{-0.048611in}{0.000000in}}{\pgfqpoint{0.000000in}{0.000000in}}{%
\pgfpathmoveto{\pgfqpoint{0.000000in}{0.000000in}}%
\pgfpathlineto{\pgfqpoint{-0.048611in}{0.000000in}}%
\pgfusepath{stroke,fill}%
}%
\begin{pgfscope}%
\pgfsys@transformshift{0.511159in}{1.025194in}%
\pgfsys@useobject{currentmarker}{}%
\end{pgfscope}%
\end{pgfscope}%
\begin{pgfscope}%
\definecolor{textcolor}{rgb}{0.000000,0.000000,0.000000}%
\pgfsetstrokecolor{textcolor}%
\pgfsetfillcolor{textcolor}%
\pgftext[x=0.263086in,y=0.982985in,left,base]{\color{textcolor}\rmfamily\fontsize{8.000000}{9.600000}\selectfont \(\displaystyle 0.2\)}%
\end{pgfscope}%
\begin{pgfscope}%
\pgfsetbuttcap%
\pgfsetroundjoin%
\definecolor{currentfill}{rgb}{0.000000,0.000000,0.000000}%
\pgfsetfillcolor{currentfill}%
\pgfsetlinewidth{0.803000pt}%
\definecolor{currentstroke}{rgb}{0.000000,0.000000,0.000000}%
\pgfsetstrokecolor{currentstroke}%
\pgfsetdash{}{0pt}%
\pgfsys@defobject{currentmarker}{\pgfqpoint{-0.048611in}{0.000000in}}{\pgfqpoint{0.000000in}{0.000000in}}{%
\pgfpathmoveto{\pgfqpoint{0.000000in}{0.000000in}}%
\pgfpathlineto{\pgfqpoint{-0.048611in}{0.000000in}}%
\pgfusepath{stroke,fill}%
}%
\begin{pgfscope}%
\pgfsys@transformshift{0.511159in}{1.471875in}%
\pgfsys@useobject{currentmarker}{}%
\end{pgfscope}%
\end{pgfscope}%
\begin{pgfscope}%
\definecolor{textcolor}{rgb}{0.000000,0.000000,0.000000}%
\pgfsetstrokecolor{textcolor}%
\pgfsetfillcolor{textcolor}%
\pgftext[x=0.263086in,y=1.429665in,left,base]{\color{textcolor}\rmfamily\fontsize{8.000000}{9.600000}\selectfont \(\displaystyle 0.4\)}%
\end{pgfscope}%
\begin{pgfscope}%
\pgfsetbuttcap%
\pgfsetroundjoin%
\definecolor{currentfill}{rgb}{0.000000,0.000000,0.000000}%
\pgfsetfillcolor{currentfill}%
\pgfsetlinewidth{0.803000pt}%
\definecolor{currentstroke}{rgb}{0.000000,0.000000,0.000000}%
\pgfsetstrokecolor{currentstroke}%
\pgfsetdash{}{0pt}%
\pgfsys@defobject{currentmarker}{\pgfqpoint{-0.048611in}{0.000000in}}{\pgfqpoint{0.000000in}{0.000000in}}{%
\pgfpathmoveto{\pgfqpoint{0.000000in}{0.000000in}}%
\pgfpathlineto{\pgfqpoint{-0.048611in}{0.000000in}}%
\pgfusepath{stroke,fill}%
}%
\begin{pgfscope}%
\pgfsys@transformshift{0.511159in}{1.918555in}%
\pgfsys@useobject{currentmarker}{}%
\end{pgfscope}%
\end{pgfscope}%
\begin{pgfscope}%
\definecolor{textcolor}{rgb}{0.000000,0.000000,0.000000}%
\pgfsetstrokecolor{textcolor}%
\pgfsetfillcolor{textcolor}%
\pgftext[x=0.263086in,y=1.876346in,left,base]{\color{textcolor}\rmfamily\fontsize{8.000000}{9.600000}\selectfont \(\displaystyle 0.6\)}%
\end{pgfscope}%
\begin{pgfscope}%
\pgfsetbuttcap%
\pgfsetroundjoin%
\definecolor{currentfill}{rgb}{0.000000,0.000000,0.000000}%
\pgfsetfillcolor{currentfill}%
\pgfsetlinewidth{0.803000pt}%
\definecolor{currentstroke}{rgb}{0.000000,0.000000,0.000000}%
\pgfsetstrokecolor{currentstroke}%
\pgfsetdash{}{0pt}%
\pgfsys@defobject{currentmarker}{\pgfqpoint{-0.048611in}{0.000000in}}{\pgfqpoint{0.000000in}{0.000000in}}{%
\pgfpathmoveto{\pgfqpoint{0.000000in}{0.000000in}}%
\pgfpathlineto{\pgfqpoint{-0.048611in}{0.000000in}}%
\pgfusepath{stroke,fill}%
}%
\begin{pgfscope}%
\pgfsys@transformshift{0.511159in}{2.365236in}%
\pgfsys@useobject{currentmarker}{}%
\end{pgfscope}%
\end{pgfscope}%
\begin{pgfscope}%
\definecolor{textcolor}{rgb}{0.000000,0.000000,0.000000}%
\pgfsetstrokecolor{textcolor}%
\pgfsetfillcolor{textcolor}%
\pgftext[x=0.263086in,y=2.323027in,left,base]{\color{textcolor}\rmfamily\fontsize{8.000000}{9.600000}\selectfont \(\displaystyle 0.8\)}%
\end{pgfscope}%
\begin{pgfscope}%
\pgfsetbuttcap%
\pgfsetroundjoin%
\definecolor{currentfill}{rgb}{0.000000,0.000000,0.000000}%
\pgfsetfillcolor{currentfill}%
\pgfsetlinewidth{0.803000pt}%
\definecolor{currentstroke}{rgb}{0.000000,0.000000,0.000000}%
\pgfsetstrokecolor{currentstroke}%
\pgfsetdash{}{0pt}%
\pgfsys@defobject{currentmarker}{\pgfqpoint{-0.048611in}{0.000000in}}{\pgfqpoint{0.000000in}{0.000000in}}{%
\pgfpathmoveto{\pgfqpoint{0.000000in}{0.000000in}}%
\pgfpathlineto{\pgfqpoint{-0.048611in}{0.000000in}}%
\pgfusepath{stroke,fill}%
}%
\begin{pgfscope}%
\pgfsys@transformshift{0.511159in}{2.811916in}%
\pgfsys@useobject{currentmarker}{}%
\end{pgfscope}%
\end{pgfscope}%
\begin{pgfscope}%
\definecolor{textcolor}{rgb}{0.000000,0.000000,0.000000}%
\pgfsetstrokecolor{textcolor}%
\pgfsetfillcolor{textcolor}%
\pgftext[x=0.263086in,y=2.769707in,left,base]{\color{textcolor}\rmfamily\fontsize{8.000000}{9.600000}\selectfont \(\displaystyle 1.0\)}%
\end{pgfscope}%
\begin{pgfscope}%
\definecolor{textcolor}{rgb}{0.000000,0.000000,0.000000}%
\pgfsetstrokecolor{textcolor}%
\pgfsetfillcolor{textcolor}%
\pgftext[x=0.207530in,y=1.694098in,,bottom,rotate=90.000000]{\color{textcolor}\rmfamily\fontsize{8.000000}{9.600000}\selectfont Hit rate}%
\end{pgfscope}%
\begin{pgfscope}%
\pgfpathrectangle{\pgfqpoint{0.511159in}{0.469412in}}{\pgfqpoint{3.287038in}{2.449373in}}%
\pgfusepath{clip}%
\pgfsetrectcap%
\pgfsetroundjoin%
\pgfsetlinewidth{1.505625pt}%
\definecolor{currentstroke}{rgb}{0.121569,0.466667,0.705882}%
\pgfsetstrokecolor{currentstroke}%
\pgfsetdash{}{0pt}%
\pgfpathmoveto{\pgfqpoint{0.660570in}{1.512076in}}%
\pgfpathlineto{\pgfqpoint{0.859784in}{2.805216in}}%
\pgfpathlineto{\pgfqpoint{1.058999in}{2.807450in}}%
\pgfpathlineto{\pgfqpoint{1.258213in}{2.789582in}}%
\pgfpathlineto{\pgfqpoint{1.457427in}{2.765015in}}%
\pgfpathlineto{\pgfqpoint{1.656642in}{2.720347in}}%
\pgfpathlineto{\pgfqpoint{1.855856in}{2.304934in}}%
\pgfpathlineto{\pgfqpoint{2.055071in}{2.182097in}}%
\pgfpathlineto{\pgfqpoint{2.187680in}{2.211131in}}%
\pgfpathlineto{\pgfqpoint{2.266052in}{2.255799in}}%
\pgfpathlineto{\pgfqpoint{2.322114in}{2.298234in}}%
\pgfpathlineto{\pgfqpoint{2.365890in}{2.318334in}}%
\pgfpathlineto{\pgfqpoint{2.401891in}{2.313868in}}%
\pgfpathlineto{\pgfqpoint{2.432529in}{2.380870in}}%
\pgfpathlineto{\pgfqpoint{2.459280in}{2.371936in}}%
\pgfpathlineto{\pgfqpoint{2.483095in}{2.376403in}}%
\pgfpathlineto{\pgfqpoint{2.504617in}{2.418838in}}%
\pgfpathlineto{\pgfqpoint{2.524307in}{2.392037in}}%
\pgfpathlineto{\pgfqpoint{2.542505in}{2.432238in}}%
\pgfpathlineto{\pgfqpoint{2.559469in}{2.418838in}}%
\pgfpathlineto{\pgfqpoint{2.575277in}{2.454572in}}%
\pgfpathlineto{\pgfqpoint{2.590066in}{2.467972in}}%
\pgfpathlineto{\pgfqpoint{2.603999in}{2.459039in}}%
\pgfpathlineto{\pgfqpoint{2.617204in}{2.505940in}}%
\pgfpathlineto{\pgfqpoint{2.629788in}{2.497007in}}%
\pgfpathlineto{\pgfqpoint{2.641838in}{2.508174in}}%
\pgfpathlineto{\pgfqpoint{2.653426in}{2.503707in}}%
\pgfpathlineto{\pgfqpoint{2.664614in}{2.510407in}}%
\pgfpathlineto{\pgfqpoint{2.675455in}{2.528274in}}%
\pgfpathlineto{\pgfqpoint{2.685993in}{2.519341in}}%
\pgfpathlineto{\pgfqpoint{2.696268in}{2.519341in}}%
\pgfpathlineto{\pgfqpoint{2.706314in}{2.530508in}}%
\pgfpathlineto{\pgfqpoint{2.716161in}{2.512640in}}%
\pgfpathlineto{\pgfqpoint{2.725834in}{2.550608in}}%
\pgfpathlineto{\pgfqpoint{2.735355in}{2.532741in}}%
\pgfpathlineto{\pgfqpoint{2.744748in}{2.561775in}}%
\pgfpathlineto{\pgfqpoint{2.754034in}{2.532741in}}%
\pgfpathlineto{\pgfqpoint{2.763232in}{2.552842in}}%
\pgfpathlineto{\pgfqpoint{2.772361in}{2.548375in}}%
\pgfpathlineto{\pgfqpoint{2.781438in}{2.559542in}}%
\pgfpathlineto{\pgfqpoint{2.790479in}{2.548375in}}%
\pgfpathlineto{\pgfqpoint{2.799499in}{2.537208in}}%
\pgfpathlineto{\pgfqpoint{2.808513in}{2.581876in}}%
\pgfpathlineto{\pgfqpoint{2.817536in}{2.550608in}}%
\pgfpathlineto{\pgfqpoint{2.826581in}{2.546141in}}%
\pgfpathlineto{\pgfqpoint{2.835662in}{2.561775in}}%
\pgfpathlineto{\pgfqpoint{2.844794in}{2.550608in}}%
\pgfpathlineto{\pgfqpoint{2.853989in}{2.568476in}}%
\pgfpathlineto{\pgfqpoint{2.863261in}{2.559542in}}%
\pgfpathlineto{\pgfqpoint{2.872623in}{2.550608in}}%
\pgfpathlineto{\pgfqpoint{2.882090in}{2.534974in}}%
\pgfpathlineto{\pgfqpoint{2.891676in}{2.534974in}}%
\pgfpathlineto{\pgfqpoint{2.901394in}{2.543908in}}%
\pgfpathlineto{\pgfqpoint{2.911259in}{2.530508in}}%
\pgfpathlineto{\pgfqpoint{2.921286in}{2.517107in}}%
\pgfpathlineto{\pgfqpoint{2.931490in}{2.492540in}}%
\pgfpathlineto{\pgfqpoint{2.941872in}{2.497007in}}%
\pgfpathlineto{\pgfqpoint{2.952152in}{2.503707in}}%
\pgfpathlineto{\pgfqpoint{2.962294in}{2.490306in}}%
\pgfpathlineto{\pgfqpoint{2.972342in}{2.505940in}}%
\pgfpathlineto{\pgfqpoint{2.982336in}{2.499240in}}%
\pgfpathlineto{\pgfqpoint{2.992313in}{2.508174in}}%
\pgfpathlineto{\pgfqpoint{3.002305in}{2.503707in}}%
\pgfpathlineto{\pgfqpoint{3.012345in}{2.490306in}}%
\pgfpathlineto{\pgfqpoint{3.022461in}{2.481373in}}%
\pgfpathlineto{\pgfqpoint{3.032682in}{2.476906in}}%
\pgfpathlineto{\pgfqpoint{3.043013in}{2.467972in}}%
\pgfpathlineto{\pgfqpoint{3.053481in}{2.492540in}}%
\pgfpathlineto{\pgfqpoint{3.064109in}{2.514874in}}%
\pgfpathlineto{\pgfqpoint{3.074923in}{2.497007in}}%
\pgfpathlineto{\pgfqpoint{3.085943in}{2.510407in}}%
\pgfpathlineto{\pgfqpoint{3.097192in}{2.476906in}}%
\pgfpathlineto{\pgfqpoint{3.108691in}{2.483606in}}%
\pgfpathlineto{\pgfqpoint{3.120460in}{2.472439in}}%
\pgfpathlineto{\pgfqpoint{3.132518in}{2.481373in}}%
\pgfpathlineto{\pgfqpoint{3.144884in}{2.452339in}}%
\pgfpathlineto{\pgfqpoint{3.157575in}{2.456805in}}%
\pgfpathlineto{\pgfqpoint{3.170607in}{2.472439in}}%
\pgfpathlineto{\pgfqpoint{3.183997in}{2.459039in}}%
\pgfpathlineto{\pgfqpoint{3.197758in}{2.434471in}}%
\pgfpathlineto{\pgfqpoint{3.211906in}{2.436705in}}%
\pgfpathlineto{\pgfqpoint{3.226453in}{2.438938in}}%
\pgfpathlineto{\pgfqpoint{3.241412in}{2.438938in}}%
\pgfpathlineto{\pgfqpoint{3.256797in}{2.423304in}}%
\pgfpathlineto{\pgfqpoint{3.272618in}{2.430005in}}%
\pgfpathlineto{\pgfqpoint{3.288887in}{2.423304in}}%
\pgfpathlineto{\pgfqpoint{3.305618in}{2.409904in}}%
\pgfpathlineto{\pgfqpoint{3.322821in}{2.398737in}}%
\pgfpathlineto{\pgfqpoint{3.340509in}{2.392037in}}%
\pgfpathlineto{\pgfqpoint{3.358683in}{2.409904in}}%
\pgfpathlineto{\pgfqpoint{3.377346in}{2.371936in}}%
\pgfpathlineto{\pgfqpoint{3.396510in}{2.414371in}}%
\pgfpathlineto{\pgfqpoint{3.416187in}{2.389803in}}%
\pgfpathlineto{\pgfqpoint{3.436392in}{2.416604in}}%
\pgfpathlineto{\pgfqpoint{3.457140in}{2.389803in}}%
\pgfpathlineto{\pgfqpoint{3.478447in}{2.394270in}}%
\pgfpathlineto{\pgfqpoint{3.500330in}{2.385336in}}%
\pgfpathlineto{\pgfqpoint{3.522807in}{2.400970in}}%
\pgfpathlineto{\pgfqpoint{3.545882in}{2.387570in}}%
\pgfpathlineto{\pgfqpoint{3.569559in}{2.396503in}}%
\pgfpathlineto{\pgfqpoint{3.593858in}{2.398737in}}%
\pgfpathlineto{\pgfqpoint{3.618798in}{2.389803in}}%
\pgfpathlineto{\pgfqpoint{3.644399in}{2.427771in}}%
\pgfpathlineto{\pgfqpoint{3.648786in}{2.735981in}}%
\pgfusepath{stroke}%
\end{pgfscope}%
\begin{pgfscope}%
\pgfpathrectangle{\pgfqpoint{0.511159in}{0.469412in}}{\pgfqpoint{3.287038in}{2.449373in}}%
\pgfusepath{clip}%
\pgfsetrectcap%
\pgfsetroundjoin%
\pgfsetlinewidth{1.505625pt}%
\definecolor{currentstroke}{rgb}{1.000000,0.498039,0.054902}%
\pgfsetstrokecolor{currentstroke}%
\pgfsetdash{}{0pt}%
\pgfpathmoveto{\pgfqpoint{0.660570in}{0.714751in}}%
\pgfpathlineto{\pgfqpoint{0.859784in}{2.767248in}}%
\pgfpathlineto{\pgfqpoint{1.058999in}{2.668979in}}%
\pgfpathlineto{\pgfqpoint{1.258213in}{1.670647in}}%
\pgfpathlineto{\pgfqpoint{1.457427in}{1.215033in}}%
\pgfpathlineto{\pgfqpoint{1.656642in}{1.018494in}}%
\pgfpathlineto{\pgfqpoint{1.855856in}{1.047528in}}%
\pgfpathlineto{\pgfqpoint{2.055071in}{1.027427in}}%
\pgfpathlineto{\pgfqpoint{2.187680in}{1.098896in}}%
\pgfpathlineto{\pgfqpoint{2.266052in}{1.125697in}}%
\pgfpathlineto{\pgfqpoint{2.322114in}{1.219500in}}%
\pgfpathlineto{\pgfqpoint{2.365890in}{1.223967in}}%
\pgfpathlineto{\pgfqpoint{2.401891in}{1.241834in}}%
\pgfpathlineto{\pgfqpoint{2.432529in}{1.286502in}}%
\pgfpathlineto{\pgfqpoint{2.459280in}{1.306603in}}%
\pgfpathlineto{\pgfqpoint{2.483095in}{1.364671in}}%
\pgfpathlineto{\pgfqpoint{2.504617in}{1.306603in}}%
\pgfpathlineto{\pgfqpoint{2.524307in}{1.393706in}}%
\pgfpathlineto{\pgfqpoint{2.542505in}{1.375838in}}%
\pgfpathlineto{\pgfqpoint{2.559469in}{1.442840in}}%
\pgfpathlineto{\pgfqpoint{2.575277in}{1.391472in}}%
\pgfpathlineto{\pgfqpoint{2.590066in}{1.424973in}}%
\pgfpathlineto{\pgfqpoint{2.603999in}{1.465174in}}%
\pgfpathlineto{\pgfqpoint{2.617204in}{1.440607in}}%
\pgfpathlineto{\pgfqpoint{2.629788in}{1.402639in}}%
\pgfpathlineto{\pgfqpoint{2.641838in}{1.418273in}}%
\pgfpathlineto{\pgfqpoint{2.653426in}{1.436140in}}%
\pgfpathlineto{\pgfqpoint{2.664614in}{1.469641in}}%
\pgfpathlineto{\pgfqpoint{2.675455in}{1.420506in}}%
\pgfpathlineto{\pgfqpoint{2.685993in}{1.451774in}}%
\pgfpathlineto{\pgfqpoint{2.696268in}{1.420506in}}%
\pgfpathlineto{\pgfqpoint{2.706314in}{1.400406in}}%
\pgfpathlineto{\pgfqpoint{2.716161in}{1.375838in}}%
\pgfpathlineto{\pgfqpoint{2.725834in}{1.429440in}}%
\pgfpathlineto{\pgfqpoint{2.735355in}{1.424973in}}%
\pgfpathlineto{\pgfqpoint{2.744748in}{1.445074in}}%
\pgfpathlineto{\pgfqpoint{2.754034in}{1.427207in}}%
\pgfpathlineto{\pgfqpoint{2.763232in}{1.433907in}}%
\pgfpathlineto{\pgfqpoint{2.772361in}{1.418273in}}%
\pgfpathlineto{\pgfqpoint{2.781438in}{1.445074in}}%
\pgfpathlineto{\pgfqpoint{2.790479in}{1.418273in}}%
\pgfpathlineto{\pgfqpoint{2.799499in}{1.395939in}}%
\pgfpathlineto{\pgfqpoint{2.808513in}{1.422740in}}%
\pgfpathlineto{\pgfqpoint{2.817536in}{1.402639in}}%
\pgfpathlineto{\pgfqpoint{2.826581in}{1.395939in}}%
\pgfpathlineto{\pgfqpoint{2.835662in}{1.395939in}}%
\pgfpathlineto{\pgfqpoint{2.844794in}{1.409339in}}%
\pgfpathlineto{\pgfqpoint{2.853989in}{1.413806in}}%
\pgfpathlineto{\pgfqpoint{2.863261in}{1.427207in}}%
\pgfpathlineto{\pgfqpoint{2.872623in}{1.433907in}}%
\pgfpathlineto{\pgfqpoint{2.882090in}{1.420506in}}%
\pgfpathlineto{\pgfqpoint{2.891676in}{1.424973in}}%
\pgfpathlineto{\pgfqpoint{2.901394in}{1.378072in}}%
\pgfpathlineto{\pgfqpoint{2.911259in}{1.433907in}}%
\pgfpathlineto{\pgfqpoint{2.921286in}{1.429440in}}%
\pgfpathlineto{\pgfqpoint{2.931490in}{1.476341in}}%
\pgfpathlineto{\pgfqpoint{2.941872in}{1.411573in}}%
\pgfpathlineto{\pgfqpoint{2.952152in}{1.391472in}}%
\pgfpathlineto{\pgfqpoint{2.962294in}{1.422740in}}%
\pgfpathlineto{\pgfqpoint{2.972342in}{1.447307in}}%
\pgfpathlineto{\pgfqpoint{2.982336in}{1.389239in}}%
\pgfpathlineto{\pgfqpoint{2.992313in}{1.445074in}}%
\pgfpathlineto{\pgfqpoint{3.002305in}{1.440607in}}%
\pgfpathlineto{\pgfqpoint{3.012345in}{1.445074in}}%
\pgfpathlineto{\pgfqpoint{3.022461in}{1.467408in}}%
\pgfpathlineto{\pgfqpoint{3.032682in}{1.451774in}}%
\pgfpathlineto{\pgfqpoint{3.043013in}{1.456241in}}%
\pgfpathlineto{\pgfqpoint{3.053481in}{1.476341in}}%
\pgfpathlineto{\pgfqpoint{3.064109in}{1.494209in}}%
\pgfpathlineto{\pgfqpoint{3.074923in}{1.467408in}}%
\pgfpathlineto{\pgfqpoint{3.085943in}{1.427207in}}%
\pgfpathlineto{\pgfqpoint{3.097192in}{1.445074in}}%
\pgfpathlineto{\pgfqpoint{3.108691in}{1.456241in}}%
\pgfpathlineto{\pgfqpoint{3.120460in}{1.451774in}}%
\pgfpathlineto{\pgfqpoint{3.132518in}{1.480808in}}%
\pgfpathlineto{\pgfqpoint{3.144884in}{1.467408in}}%
\pgfpathlineto{\pgfqpoint{3.157575in}{1.462941in}}%
\pgfpathlineto{\pgfqpoint{3.170607in}{1.489742in}}%
\pgfpathlineto{\pgfqpoint{3.183997in}{1.471875in}}%
\pgfpathlineto{\pgfqpoint{3.197758in}{1.509842in}}%
\pgfpathlineto{\pgfqpoint{3.211906in}{1.507609in}}%
\pgfpathlineto{\pgfqpoint{3.226453in}{1.476341in}}%
\pgfpathlineto{\pgfqpoint{3.241412in}{1.500909in}}%
\pgfpathlineto{\pgfqpoint{3.256797in}{1.465174in}}%
\pgfpathlineto{\pgfqpoint{3.272618in}{1.507609in}}%
\pgfpathlineto{\pgfqpoint{3.288887in}{1.505376in}}%
\pgfpathlineto{\pgfqpoint{3.305618in}{1.494209in}}%
\pgfpathlineto{\pgfqpoint{3.322821in}{1.474108in}}%
\pgfpathlineto{\pgfqpoint{3.340509in}{1.494209in}}%
\pgfpathlineto{\pgfqpoint{3.358683in}{1.478575in}}%
\pgfpathlineto{\pgfqpoint{3.377346in}{1.498675in}}%
\pgfpathlineto{\pgfqpoint{3.396510in}{1.500909in}}%
\pgfpathlineto{\pgfqpoint{3.416187in}{1.491975in}}%
\pgfpathlineto{\pgfqpoint{3.436392in}{1.491975in}}%
\pgfpathlineto{\pgfqpoint{3.457140in}{1.498675in}}%
\pgfpathlineto{\pgfqpoint{3.478447in}{1.483042in}}%
\pgfpathlineto{\pgfqpoint{3.500330in}{1.500909in}}%
\pgfpathlineto{\pgfqpoint{3.522807in}{1.487508in}}%
\pgfpathlineto{\pgfqpoint{3.545882in}{1.478575in}}%
\pgfpathlineto{\pgfqpoint{3.569559in}{1.516543in}}%
\pgfpathlineto{\pgfqpoint{3.593858in}{1.496442in}}%
\pgfpathlineto{\pgfqpoint{3.618798in}{1.478575in}}%
\pgfpathlineto{\pgfqpoint{3.644399in}{1.465174in}}%
\pgfpathlineto{\pgfqpoint{3.648786in}{1.710849in}}%
\pgfusepath{stroke}%
\end{pgfscope}%
\begin{pgfscope}%
\pgfpathrectangle{\pgfqpoint{0.511159in}{0.469412in}}{\pgfqpoint{3.287038in}{2.449373in}}%
\pgfusepath{clip}%
\pgfsetrectcap%
\pgfsetroundjoin%
\pgfsetlinewidth{1.505625pt}%
\definecolor{currentstroke}{rgb}{0.172549,0.627451,0.172549}%
\pgfsetstrokecolor{currentstroke}%
\pgfsetdash{}{0pt}%
\pgfpathmoveto{\pgfqpoint{0.660570in}{0.596381in}}%
\pgfpathlineto{\pgfqpoint{0.859784in}{1.387005in}}%
\pgfpathlineto{\pgfqpoint{1.058999in}{0.830888in}}%
\pgfpathlineto{\pgfqpoint{1.258213in}{0.723685in}}%
\pgfpathlineto{\pgfqpoint{1.457427in}{0.656683in}}%
\pgfpathlineto{\pgfqpoint{1.656642in}{0.641049in}}%
\pgfpathlineto{\pgfqpoint{1.855856in}{0.634348in}}%
\pgfpathlineto{\pgfqpoint{2.055071in}{0.632115in}}%
\pgfpathlineto{\pgfqpoint{2.187680in}{0.663383in}}%
\pgfpathlineto{\pgfqpoint{2.266052in}{0.663383in}}%
\pgfpathlineto{\pgfqpoint{2.322114in}{0.663383in}}%
\pgfpathlineto{\pgfqpoint{2.365890in}{0.683483in}}%
\pgfpathlineto{\pgfqpoint{2.401891in}{0.699117in}}%
\pgfpathlineto{\pgfqpoint{2.432529in}{0.694650in}}%
\pgfpathlineto{\pgfqpoint{2.459280in}{0.721451in}}%
\pgfpathlineto{\pgfqpoint{2.483095in}{0.701351in}}%
\pgfpathlineto{\pgfqpoint{2.504617in}{0.714751in}}%
\pgfpathlineto{\pgfqpoint{2.524307in}{0.692417in}}%
\pgfpathlineto{\pgfqpoint{2.542505in}{0.710284in}}%
\pgfpathlineto{\pgfqpoint{2.559469in}{0.701351in}}%
\pgfpathlineto{\pgfqpoint{2.575277in}{0.676783in}}%
\pgfpathlineto{\pgfqpoint{2.590066in}{0.676783in}}%
\pgfpathlineto{\pgfqpoint{2.603999in}{0.692417in}}%
\pgfpathlineto{\pgfqpoint{2.617204in}{0.703584in}}%
\pgfpathlineto{\pgfqpoint{2.629788in}{0.692417in}}%
\pgfpathlineto{\pgfqpoint{2.641838in}{0.696884in}}%
\pgfpathlineto{\pgfqpoint{2.653426in}{0.683483in}}%
\pgfpathlineto{\pgfqpoint{2.664614in}{0.701351in}}%
\pgfpathlineto{\pgfqpoint{2.675455in}{0.712518in}}%
\pgfpathlineto{\pgfqpoint{2.685993in}{0.699117in}}%
\pgfpathlineto{\pgfqpoint{2.696268in}{0.701351in}}%
\pgfpathlineto{\pgfqpoint{2.706314in}{0.687950in}}%
\pgfpathlineto{\pgfqpoint{2.716161in}{0.699117in}}%
\pgfpathlineto{\pgfqpoint{2.725834in}{0.683483in}}%
\pgfpathlineto{\pgfqpoint{2.735355in}{0.696884in}}%
\pgfpathlineto{\pgfqpoint{2.744748in}{0.705817in}}%
\pgfpathlineto{\pgfqpoint{2.754034in}{0.712518in}}%
\pgfpathlineto{\pgfqpoint{2.763232in}{0.714751in}}%
\pgfpathlineto{\pgfqpoint{2.772361in}{0.687950in}}%
\pgfpathlineto{\pgfqpoint{2.781438in}{0.701351in}}%
\pgfpathlineto{\pgfqpoint{2.790479in}{0.694650in}}%
\pgfpathlineto{\pgfqpoint{2.799499in}{0.690184in}}%
\pgfpathlineto{\pgfqpoint{2.808513in}{0.701351in}}%
\pgfpathlineto{\pgfqpoint{2.817536in}{0.703584in}}%
\pgfpathlineto{\pgfqpoint{2.826581in}{0.705817in}}%
\pgfpathlineto{\pgfqpoint{2.835662in}{0.687950in}}%
\pgfpathlineto{\pgfqpoint{2.844794in}{0.721451in}}%
\pgfpathlineto{\pgfqpoint{2.853989in}{0.694650in}}%
\pgfpathlineto{\pgfqpoint{2.863261in}{0.710284in}}%
\pgfpathlineto{\pgfqpoint{2.872623in}{0.750485in}}%
\pgfpathlineto{\pgfqpoint{2.882090in}{0.714751in}}%
\pgfpathlineto{\pgfqpoint{2.891676in}{0.725918in}}%
\pgfpathlineto{\pgfqpoint{2.901394in}{0.708051in}}%
\pgfpathlineto{\pgfqpoint{2.911259in}{0.725918in}}%
\pgfpathlineto{\pgfqpoint{2.921286in}{0.754952in}}%
\pgfpathlineto{\pgfqpoint{2.931490in}{0.730385in}}%
\pgfpathlineto{\pgfqpoint{2.941872in}{0.746019in}}%
\pgfpathlineto{\pgfqpoint{2.952152in}{0.748252in}}%
\pgfpathlineto{\pgfqpoint{2.962294in}{0.777286in}}%
\pgfpathlineto{\pgfqpoint{2.972342in}{0.746019in}}%
\pgfpathlineto{\pgfqpoint{2.982336in}{0.741552in}}%
\pgfpathlineto{\pgfqpoint{2.992313in}{0.743785in}}%
\pgfpathlineto{\pgfqpoint{3.002305in}{0.754952in}}%
\pgfpathlineto{\pgfqpoint{3.012345in}{0.777286in}}%
\pgfpathlineto{\pgfqpoint{3.022461in}{0.772819in}}%
\pgfpathlineto{\pgfqpoint{3.032682in}{0.792920in}}%
\pgfpathlineto{\pgfqpoint{3.043013in}{0.781753in}}%
\pgfpathlineto{\pgfqpoint{3.053481in}{0.772819in}}%
\pgfpathlineto{\pgfqpoint{3.064109in}{0.830888in}}%
\pgfpathlineto{\pgfqpoint{3.074923in}{0.801854in}}%
\pgfpathlineto{\pgfqpoint{3.085943in}{0.770586in}}%
\pgfpathlineto{\pgfqpoint{3.097192in}{0.790687in}}%
\pgfpathlineto{\pgfqpoint{3.108691in}{0.790687in}}%
\pgfpathlineto{\pgfqpoint{3.120460in}{0.759419in}}%
\pgfpathlineto{\pgfqpoint{3.132518in}{0.788453in}}%
\pgfpathlineto{\pgfqpoint{3.144884in}{0.781753in}}%
\pgfpathlineto{\pgfqpoint{3.157575in}{0.770586in}}%
\pgfpathlineto{\pgfqpoint{3.170607in}{0.790687in}}%
\pgfpathlineto{\pgfqpoint{3.183997in}{0.795154in}}%
\pgfpathlineto{\pgfqpoint{3.197758in}{0.824188in}}%
\pgfpathlineto{\pgfqpoint{3.211906in}{0.772819in}}%
\pgfpathlineto{\pgfqpoint{3.226453in}{0.824188in}}%
\pgfpathlineto{\pgfqpoint{3.241412in}{0.792920in}}%
\pgfpathlineto{\pgfqpoint{3.256797in}{0.817488in}}%
\pgfpathlineto{\pgfqpoint{3.272618in}{0.819721in}}%
\pgfpathlineto{\pgfqpoint{3.288887in}{0.806321in}}%
\pgfpathlineto{\pgfqpoint{3.305618in}{0.810787in}}%
\pgfpathlineto{\pgfqpoint{3.322821in}{0.766119in}}%
\pgfpathlineto{\pgfqpoint{3.340509in}{0.797387in}}%
\pgfpathlineto{\pgfqpoint{3.358683in}{0.801854in}}%
\pgfpathlineto{\pgfqpoint{3.377346in}{0.833121in}}%
\pgfpathlineto{\pgfqpoint{3.396510in}{0.817488in}}%
\pgfpathlineto{\pgfqpoint{3.416187in}{0.804087in}}%
\pgfpathlineto{\pgfqpoint{3.436392in}{0.819721in}}%
\pgfpathlineto{\pgfqpoint{3.457140in}{0.815254in}}%
\pgfpathlineto{\pgfqpoint{3.478447in}{0.817488in}}%
\pgfpathlineto{\pgfqpoint{3.500330in}{0.846522in}}%
\pgfpathlineto{\pgfqpoint{3.522807in}{0.786220in}}%
\pgfpathlineto{\pgfqpoint{3.545882in}{0.808554in}}%
\pgfpathlineto{\pgfqpoint{3.569559in}{0.839822in}}%
\pgfpathlineto{\pgfqpoint{3.593858in}{0.783986in}}%
\pgfpathlineto{\pgfqpoint{3.618798in}{0.815254in}}%
\pgfpathlineto{\pgfqpoint{3.644399in}{0.817488in}}%
\pgfpathlineto{\pgfqpoint{3.648786in}{0.893423in}}%
\pgfusepath{stroke}%
\end{pgfscope}%
\begin{pgfscope}%
\pgfpathrectangle{\pgfqpoint{0.511159in}{0.469412in}}{\pgfqpoint{3.287038in}{2.449373in}}%
\pgfusepath{clip}%
\pgfsetrectcap%
\pgfsetroundjoin%
\pgfsetlinewidth{1.505625pt}%
\definecolor{currentstroke}{rgb}{0.839216,0.152941,0.156863}%
\pgfsetstrokecolor{currentstroke}%
\pgfsetdash{}{0pt}%
\pgfpathmoveto{\pgfqpoint{0.660570in}{0.580747in}}%
\pgfpathlineto{\pgfqpoint{0.859784in}{0.667850in}}%
\pgfpathlineto{\pgfqpoint{1.058999in}{0.616481in}}%
\pgfpathlineto{\pgfqpoint{1.258213in}{0.600847in}}%
\pgfpathlineto{\pgfqpoint{1.457427in}{0.600847in}}%
\pgfpathlineto{\pgfqpoint{1.656642in}{0.580747in}}%
\pgfpathlineto{\pgfqpoint{1.855856in}{0.585214in}}%
\pgfpathlineto{\pgfqpoint{2.055071in}{0.585214in}}%
\pgfpathlineto{\pgfqpoint{2.187680in}{0.589680in}}%
\pgfpathlineto{\pgfqpoint{2.266052in}{0.587447in}}%
\pgfpathlineto{\pgfqpoint{2.322114in}{0.589680in}}%
\pgfpathlineto{\pgfqpoint{2.365890in}{0.589680in}}%
\pgfpathlineto{\pgfqpoint{2.401891in}{0.589680in}}%
\pgfpathlineto{\pgfqpoint{2.432529in}{0.589680in}}%
\pgfpathlineto{\pgfqpoint{2.459280in}{0.600847in}}%
\pgfpathlineto{\pgfqpoint{2.483095in}{0.587447in}}%
\pgfpathlineto{\pgfqpoint{2.504617in}{0.589680in}}%
\pgfpathlineto{\pgfqpoint{2.524307in}{0.589680in}}%
\pgfpathlineto{\pgfqpoint{2.542505in}{0.591914in}}%
\pgfpathlineto{\pgfqpoint{2.559469in}{0.603081in}}%
\pgfpathlineto{\pgfqpoint{2.575277in}{0.582980in}}%
\pgfpathlineto{\pgfqpoint{2.590066in}{0.589680in}}%
\pgfpathlineto{\pgfqpoint{2.603999in}{0.589680in}}%
\pgfpathlineto{\pgfqpoint{2.617204in}{0.594147in}}%
\pgfpathlineto{\pgfqpoint{2.629788in}{0.582980in}}%
\pgfpathlineto{\pgfqpoint{2.641838in}{0.589680in}}%
\pgfpathlineto{\pgfqpoint{2.653426in}{0.585214in}}%
\pgfpathlineto{\pgfqpoint{2.664614in}{0.591914in}}%
\pgfpathlineto{\pgfqpoint{2.675455in}{0.582980in}}%
\pgfpathlineto{\pgfqpoint{2.685993in}{0.589680in}}%
\pgfpathlineto{\pgfqpoint{2.696268in}{0.589680in}}%
\pgfpathlineto{\pgfqpoint{2.706314in}{0.585214in}}%
\pgfpathlineto{\pgfqpoint{2.716161in}{0.589680in}}%
\pgfpathlineto{\pgfqpoint{2.725834in}{0.582980in}}%
\pgfpathlineto{\pgfqpoint{2.735355in}{0.591914in}}%
\pgfpathlineto{\pgfqpoint{2.744748in}{0.585214in}}%
\pgfpathlineto{\pgfqpoint{2.754034in}{0.589680in}}%
\pgfpathlineto{\pgfqpoint{2.763232in}{0.585214in}}%
\pgfpathlineto{\pgfqpoint{2.772361in}{0.596381in}}%
\pgfpathlineto{\pgfqpoint{2.781438in}{0.594147in}}%
\pgfpathlineto{\pgfqpoint{2.790479in}{0.603081in}}%
\pgfpathlineto{\pgfqpoint{2.799499in}{0.596381in}}%
\pgfpathlineto{\pgfqpoint{2.808513in}{0.589680in}}%
\pgfpathlineto{\pgfqpoint{2.817536in}{0.594147in}}%
\pgfpathlineto{\pgfqpoint{2.826581in}{0.589680in}}%
\pgfpathlineto{\pgfqpoint{2.835662in}{0.580747in}}%
\pgfpathlineto{\pgfqpoint{2.844794in}{0.594147in}}%
\pgfpathlineto{\pgfqpoint{2.853989in}{0.585214in}}%
\pgfpathlineto{\pgfqpoint{2.863261in}{0.587447in}}%
\pgfpathlineto{\pgfqpoint{2.872623in}{0.596381in}}%
\pgfpathlineto{\pgfqpoint{2.882090in}{0.603081in}}%
\pgfpathlineto{\pgfqpoint{2.891676in}{0.596381in}}%
\pgfpathlineto{\pgfqpoint{2.901394in}{0.591914in}}%
\pgfpathlineto{\pgfqpoint{2.911259in}{0.591914in}}%
\pgfpathlineto{\pgfqpoint{2.921286in}{0.607548in}}%
\pgfpathlineto{\pgfqpoint{2.931490in}{0.594147in}}%
\pgfpathlineto{\pgfqpoint{2.941872in}{0.591914in}}%
\pgfpathlineto{\pgfqpoint{2.952152in}{0.591914in}}%
\pgfpathlineto{\pgfqpoint{2.962294in}{0.609781in}}%
\pgfpathlineto{\pgfqpoint{2.972342in}{0.594147in}}%
\pgfpathlineto{\pgfqpoint{2.982336in}{0.600847in}}%
\pgfpathlineto{\pgfqpoint{2.992313in}{0.607548in}}%
\pgfpathlineto{\pgfqpoint{3.002305in}{0.612014in}}%
\pgfpathlineto{\pgfqpoint{3.012345in}{0.594147in}}%
\pgfpathlineto{\pgfqpoint{3.022461in}{0.596381in}}%
\pgfpathlineto{\pgfqpoint{3.032682in}{0.609781in}}%
\pgfpathlineto{\pgfqpoint{3.043013in}{0.607548in}}%
\pgfpathlineto{\pgfqpoint{3.053481in}{0.603081in}}%
\pgfpathlineto{\pgfqpoint{3.064109in}{0.609781in}}%
\pgfpathlineto{\pgfqpoint{3.074923in}{0.596381in}}%
\pgfpathlineto{\pgfqpoint{3.085943in}{0.600847in}}%
\pgfpathlineto{\pgfqpoint{3.097192in}{0.609781in}}%
\pgfpathlineto{\pgfqpoint{3.108691in}{0.612014in}}%
\pgfpathlineto{\pgfqpoint{3.120460in}{0.609781in}}%
\pgfpathlineto{\pgfqpoint{3.132518in}{0.594147in}}%
\pgfpathlineto{\pgfqpoint{3.144884in}{0.603081in}}%
\pgfpathlineto{\pgfqpoint{3.157575in}{0.609781in}}%
\pgfpathlineto{\pgfqpoint{3.170607in}{0.609781in}}%
\pgfpathlineto{\pgfqpoint{3.183997in}{0.596381in}}%
\pgfpathlineto{\pgfqpoint{3.197758in}{0.632115in}}%
\pgfpathlineto{\pgfqpoint{3.211906in}{0.605314in}}%
\pgfpathlineto{\pgfqpoint{3.226453in}{0.612014in}}%
\pgfpathlineto{\pgfqpoint{3.241412in}{0.616481in}}%
\pgfpathlineto{\pgfqpoint{3.256797in}{0.607548in}}%
\pgfpathlineto{\pgfqpoint{3.272618in}{0.596381in}}%
\pgfpathlineto{\pgfqpoint{3.288887in}{0.612014in}}%
\pgfpathlineto{\pgfqpoint{3.305618in}{0.614248in}}%
\pgfpathlineto{\pgfqpoint{3.322821in}{0.591914in}}%
\pgfpathlineto{\pgfqpoint{3.340509in}{0.612014in}}%
\pgfpathlineto{\pgfqpoint{3.358683in}{0.612014in}}%
\pgfpathlineto{\pgfqpoint{3.377346in}{0.614248in}}%
\pgfpathlineto{\pgfqpoint{3.396510in}{0.603081in}}%
\pgfpathlineto{\pgfqpoint{3.416187in}{0.605314in}}%
\pgfpathlineto{\pgfqpoint{3.436392in}{0.612014in}}%
\pgfpathlineto{\pgfqpoint{3.457140in}{0.607548in}}%
\pgfpathlineto{\pgfqpoint{3.478447in}{0.623181in}}%
\pgfpathlineto{\pgfqpoint{3.500330in}{0.616481in}}%
\pgfpathlineto{\pgfqpoint{3.522807in}{0.607548in}}%
\pgfpathlineto{\pgfqpoint{3.545882in}{0.609781in}}%
\pgfpathlineto{\pgfqpoint{3.569559in}{0.607548in}}%
\pgfpathlineto{\pgfqpoint{3.593858in}{0.607548in}}%
\pgfpathlineto{\pgfqpoint{3.618798in}{0.605314in}}%
\pgfpathlineto{\pgfqpoint{3.644399in}{0.618715in}}%
\pgfpathlineto{\pgfqpoint{3.648786in}{0.643282in}}%
\pgfusepath{stroke}%
\end{pgfscope}%
\begin{pgfscope}%
\pgfsetrectcap%
\pgfsetmiterjoin%
\pgfsetlinewidth{0.803000pt}%
\definecolor{currentstroke}{rgb}{0.000000,0.000000,0.000000}%
\pgfsetstrokecolor{currentstroke}%
\pgfsetdash{}{0pt}%
\pgfpathmoveto{\pgfqpoint{0.511159in}{0.469412in}}%
\pgfpathlineto{\pgfqpoint{0.511159in}{2.918785in}}%
\pgfusepath{stroke}%
\end{pgfscope}%
\begin{pgfscope}%
\pgfsetrectcap%
\pgfsetmiterjoin%
\pgfsetlinewidth{0.803000pt}%
\definecolor{currentstroke}{rgb}{0.000000,0.000000,0.000000}%
\pgfsetstrokecolor{currentstroke}%
\pgfsetdash{}{0pt}%
\pgfpathmoveto{\pgfqpoint{3.798197in}{0.469412in}}%
\pgfpathlineto{\pgfqpoint{3.798197in}{2.918785in}}%
\pgfusepath{stroke}%
\end{pgfscope}%
\begin{pgfscope}%
\pgfsetrectcap%
\pgfsetmiterjoin%
\pgfsetlinewidth{0.803000pt}%
\definecolor{currentstroke}{rgb}{0.000000,0.000000,0.000000}%
\pgfsetstrokecolor{currentstroke}%
\pgfsetdash{}{0pt}%
\pgfpathmoveto{\pgfqpoint{0.511159in}{0.469412in}}%
\pgfpathlineto{\pgfqpoint{3.798197in}{0.469412in}}%
\pgfusepath{stroke}%
\end{pgfscope}%
\begin{pgfscope}%
\pgfsetrectcap%
\pgfsetmiterjoin%
\pgfsetlinewidth{0.803000pt}%
\definecolor{currentstroke}{rgb}{0.000000,0.000000,0.000000}%
\pgfsetstrokecolor{currentstroke}%
\pgfsetdash{}{0pt}%
\pgfpathmoveto{\pgfqpoint{0.511159in}{2.918785in}}%
\pgfpathlineto{\pgfqpoint{3.798197in}{2.918785in}}%
\pgfusepath{stroke}%
\end{pgfscope}%
\begin{pgfscope}%
\pgfsetrectcap%
\pgfsetroundjoin%
\pgfsetlinewidth{1.505625pt}%
\definecolor{currentstroke}{rgb}{0.121569,0.466667,0.705882}%
\pgfsetstrokecolor{currentstroke}%
\pgfsetdash{}{0pt}%
\pgfpathmoveto{\pgfqpoint{1.876781in}{1.946963in}}%
\pgfpathlineto{\pgfqpoint{2.099003in}{1.946963in}}%
\pgfusepath{stroke}%
\end{pgfscope}%
\begin{pgfscope}%
\definecolor{textcolor}{rgb}{0.000000,0.000000,0.000000}%
\pgfsetstrokecolor{textcolor}%
\pgfsetfillcolor{textcolor}%
\pgftext[x=2.187892in,y=1.908074in,left,base]{\color{textcolor}\rmfamily\fontsize{8.000000}{9.600000}\selectfont 1e-2}%
\end{pgfscope}%
\begin{pgfscope}%
\pgfsetrectcap%
\pgfsetroundjoin%
\pgfsetlinewidth{1.505625pt}%
\definecolor{currentstroke}{rgb}{1.000000,0.498039,0.054902}%
\pgfsetstrokecolor{currentstroke}%
\pgfsetdash{}{0pt}%
\pgfpathmoveto{\pgfqpoint{1.876781in}{1.783877in}}%
\pgfpathlineto{\pgfqpoint{2.099003in}{1.783877in}}%
\pgfusepath{stroke}%
\end{pgfscope}%
\begin{pgfscope}%
\definecolor{textcolor}{rgb}{0.000000,0.000000,0.000000}%
\pgfsetstrokecolor{textcolor}%
\pgfsetfillcolor{textcolor}%
\pgftext[x=2.187892in,y=1.744988in,left,base]{\color{textcolor}\rmfamily\fontsize{8.000000}{9.600000}\selectfont 1e-3}%
\end{pgfscope}%
\begin{pgfscope}%
\pgfsetrectcap%
\pgfsetroundjoin%
\pgfsetlinewidth{1.505625pt}%
\definecolor{currentstroke}{rgb}{0.172549,0.627451,0.172549}%
\pgfsetstrokecolor{currentstroke}%
\pgfsetdash{}{0pt}%
\pgfpathmoveto{\pgfqpoint{1.876781in}{1.620791in}}%
\pgfpathlineto{\pgfqpoint{2.099003in}{1.620791in}}%
\pgfusepath{stroke}%
\end{pgfscope}%
\begin{pgfscope}%
\definecolor{textcolor}{rgb}{0.000000,0.000000,0.000000}%
\pgfsetstrokecolor{textcolor}%
\pgfsetfillcolor{textcolor}%
\pgftext[x=2.187892in,y=1.581902in,left,base]{\color{textcolor}\rmfamily\fontsize{8.000000}{9.600000}\selectfont 1e-4}%
\end{pgfscope}%
\begin{pgfscope}%
\pgfsetrectcap%
\pgfsetroundjoin%
\pgfsetlinewidth{1.505625pt}%
\definecolor{currentstroke}{rgb}{0.839216,0.152941,0.156863}%
\pgfsetstrokecolor{currentstroke}%
\pgfsetdash{}{0pt}%
\pgfpathmoveto{\pgfqpoint{1.876781in}{1.457705in}}%
\pgfpathlineto{\pgfqpoint{2.099003in}{1.457705in}}%
\pgfusepath{stroke}%
\end{pgfscope}%
\begin{pgfscope}%
\definecolor{textcolor}{rgb}{0.000000,0.000000,0.000000}%
\pgfsetstrokecolor{textcolor}%
\pgfsetfillcolor{textcolor}%
\pgftext[x=2.187892in,y=1.418816in,left,base]{\color{textcolor}\rmfamily\fontsize{8.000000}{9.600000}\selectfont 1e-5}%
\end{pgfscope}%
\end{pgfpicture}%
\makeatother%
\endgroup%

    \caption{Caption}
    \label{fig:my_label}
\end{figure}

\begin{figure}
    \centering
    %% Creator: Matplotlib, PGF backend
%%
%% To include the figure in your LaTeX document, write
%%   \input{<filename>.pgf}
%%
%% Make sure the required packages are loaded in your preamble
%%   \usepackage{pgf}
%%
%% Figures using additional raster images can only be included by \input if
%% they are in the same directory as the main LaTeX file. For loading figures
%% from other directories you can use the `import` package
%%   \usepackage{import}
%% and then include the figures with
%%   \import{<path to file>}{<filename>.pgf}
%%
%% Matplotlib used the following preamble
%%   \usepackage{fontspec}
%%   \setmainfont{DejaVuSerif.ttf}[Path=/home/connor/.local/lib/python3.8/site-packages/matplotlib/mpl-data/fonts/ttf/]
%%   \setsansfont{DejaVuSans.ttf}[Path=/home/connor/.local/lib/python3.8/site-packages/matplotlib/mpl-data/fonts/ttf/]
%%   \setmonofont{DejaVuSansMono.ttf}[Path=/home/connor/.local/lib/python3.8/site-packages/matplotlib/mpl-data/fonts/ttf/]
%%
\begingroup%
\makeatletter%
\begin{pgfpicture}%
\pgfpathrectangle{\pgfpointorigin}{\pgfqpoint{2.986359in}{3.942354in}}%
\pgfusepath{use as bounding box, clip}%
\begin{pgfscope}%
\pgfsetbuttcap%
\pgfsetmiterjoin%
\definecolor{currentfill}{rgb}{1.000000,1.000000,1.000000}%
\pgfsetfillcolor{currentfill}%
\pgfsetlinewidth{0.000000pt}%
\definecolor{currentstroke}{rgb}{1.000000,1.000000,1.000000}%
\pgfsetstrokecolor{currentstroke}%
\pgfsetdash{}{0pt}%
\pgfpathmoveto{\pgfqpoint{0.000000in}{0.000000in}}%
\pgfpathlineto{\pgfqpoint{2.986359in}{0.000000in}}%
\pgfpathlineto{\pgfqpoint{2.986359in}{3.942354in}}%
\pgfpathlineto{\pgfqpoint{0.000000in}{3.942354in}}%
\pgfpathclose%
\pgfusepath{fill}%
\end{pgfscope}%
\begin{pgfscope}%
\pgfsetbuttcap%
\pgfsetmiterjoin%
\definecolor{currentfill}{rgb}{1.000000,1.000000,1.000000}%
\pgfsetfillcolor{currentfill}%
\pgfsetlinewidth{0.000000pt}%
\definecolor{currentstroke}{rgb}{0.000000,0.000000,0.000000}%
\pgfsetstrokecolor{currentstroke}%
\pgfsetstrokeopacity{0.000000}%
\pgfsetdash{}{0pt}%
\pgfpathmoveto{\pgfqpoint{0.629216in}{1.472354in}}%
\pgfpathlineto{\pgfqpoint{2.886359in}{1.472354in}}%
\pgfpathlineto{\pgfqpoint{2.886359in}{3.842354in}}%
\pgfpathlineto{\pgfqpoint{0.629216in}{3.842354in}}%
\pgfpathclose%
\pgfusepath{fill}%
\end{pgfscope}%
\begin{pgfscope}%
\pgfsetbuttcap%
\pgfsetroundjoin%
\definecolor{currentfill}{rgb}{0.000000,0.000000,0.000000}%
\pgfsetfillcolor{currentfill}%
\pgfsetlinewidth{0.803000pt}%
\definecolor{currentstroke}{rgb}{0.000000,0.000000,0.000000}%
\pgfsetstrokecolor{currentstroke}%
\pgfsetdash{}{0pt}%
\pgfsys@defobject{currentmarker}{\pgfqpoint{0.000000in}{-0.048611in}}{\pgfqpoint{0.000000in}{0.000000in}}{%
\pgfpathmoveto{\pgfqpoint{0.000000in}{0.000000in}}%
\pgfpathlineto{\pgfqpoint{0.000000in}{-0.048611in}}%
\pgfusepath{stroke,fill}%
}%
\begin{pgfscope}%
\pgfsys@transformshift{0.731814in}{1.472354in}%
\pgfsys@useobject{currentmarker}{}%
\end{pgfscope}%
\end{pgfscope}%
\begin{pgfscope}%
\definecolor{textcolor}{rgb}{0.000000,0.000000,0.000000}%
\pgfsetstrokecolor{textcolor}%
\pgfsetfillcolor{textcolor}%
\pgftext[x=0.731814in,y=1.375132in,,top]{\color{textcolor}\rmfamily\fontsize{8.000000}{9.600000}\selectfont \(\displaystyle 0.0\)}%
\end{pgfscope}%
\begin{pgfscope}%
\pgfsetbuttcap%
\pgfsetroundjoin%
\definecolor{currentfill}{rgb}{0.000000,0.000000,0.000000}%
\pgfsetfillcolor{currentfill}%
\pgfsetlinewidth{0.803000pt}%
\definecolor{currentstroke}{rgb}{0.000000,0.000000,0.000000}%
\pgfsetstrokecolor{currentstroke}%
\pgfsetdash{}{0pt}%
\pgfsys@defobject{currentmarker}{\pgfqpoint{0.000000in}{-0.048611in}}{\pgfqpoint{0.000000in}{0.000000in}}{%
\pgfpathmoveto{\pgfqpoint{0.000000in}{0.000000in}}%
\pgfpathlineto{\pgfqpoint{0.000000in}{-0.048611in}}%
\pgfusepath{stroke,fill}%
}%
\begin{pgfscope}%
\pgfsys@transformshift{1.415796in}{1.472354in}%
\pgfsys@useobject{currentmarker}{}%
\end{pgfscope}%
\end{pgfscope}%
\begin{pgfscope}%
\definecolor{textcolor}{rgb}{0.000000,0.000000,0.000000}%
\pgfsetstrokecolor{textcolor}%
\pgfsetfillcolor{textcolor}%
\pgftext[x=1.415796in,y=1.375132in,,top]{\color{textcolor}\rmfamily\fontsize{8.000000}{9.600000}\selectfont \(\displaystyle 0.5\)}%
\end{pgfscope}%
\begin{pgfscope}%
\pgfsetbuttcap%
\pgfsetroundjoin%
\definecolor{currentfill}{rgb}{0.000000,0.000000,0.000000}%
\pgfsetfillcolor{currentfill}%
\pgfsetlinewidth{0.803000pt}%
\definecolor{currentstroke}{rgb}{0.000000,0.000000,0.000000}%
\pgfsetstrokecolor{currentstroke}%
\pgfsetdash{}{0pt}%
\pgfsys@defobject{currentmarker}{\pgfqpoint{0.000000in}{-0.048611in}}{\pgfqpoint{0.000000in}{0.000000in}}{%
\pgfpathmoveto{\pgfqpoint{0.000000in}{0.000000in}}%
\pgfpathlineto{\pgfqpoint{0.000000in}{-0.048611in}}%
\pgfusepath{stroke,fill}%
}%
\begin{pgfscope}%
\pgfsys@transformshift{2.099779in}{1.472354in}%
\pgfsys@useobject{currentmarker}{}%
\end{pgfscope}%
\end{pgfscope}%
\begin{pgfscope}%
\definecolor{textcolor}{rgb}{0.000000,0.000000,0.000000}%
\pgfsetstrokecolor{textcolor}%
\pgfsetfillcolor{textcolor}%
\pgftext[x=2.099779in,y=1.375132in,,top]{\color{textcolor}\rmfamily\fontsize{8.000000}{9.600000}\selectfont \(\displaystyle 1.0\)}%
\end{pgfscope}%
\begin{pgfscope}%
\pgfsetbuttcap%
\pgfsetroundjoin%
\definecolor{currentfill}{rgb}{0.000000,0.000000,0.000000}%
\pgfsetfillcolor{currentfill}%
\pgfsetlinewidth{0.803000pt}%
\definecolor{currentstroke}{rgb}{0.000000,0.000000,0.000000}%
\pgfsetstrokecolor{currentstroke}%
\pgfsetdash{}{0pt}%
\pgfsys@defobject{currentmarker}{\pgfqpoint{0.000000in}{-0.048611in}}{\pgfqpoint{0.000000in}{0.000000in}}{%
\pgfpathmoveto{\pgfqpoint{0.000000in}{0.000000in}}%
\pgfpathlineto{\pgfqpoint{0.000000in}{-0.048611in}}%
\pgfusepath{stroke,fill}%
}%
\begin{pgfscope}%
\pgfsys@transformshift{2.783761in}{1.472354in}%
\pgfsys@useobject{currentmarker}{}%
\end{pgfscope}%
\end{pgfscope}%
\begin{pgfscope}%
\definecolor{textcolor}{rgb}{0.000000,0.000000,0.000000}%
\pgfsetstrokecolor{textcolor}%
\pgfsetfillcolor{textcolor}%
\pgftext[x=2.783761in,y=1.375132in,,top]{\color{textcolor}\rmfamily\fontsize{8.000000}{9.600000}\selectfont \(\displaystyle 1.5\)}%
\end{pgfscope}%
\begin{pgfscope}%
\pgfsetbuttcap%
\pgfsetroundjoin%
\definecolor{currentfill}{rgb}{0.000000,0.000000,0.000000}%
\pgfsetfillcolor{currentfill}%
\pgfsetlinewidth{0.803000pt}%
\definecolor{currentstroke}{rgb}{0.000000,0.000000,0.000000}%
\pgfsetstrokecolor{currentstroke}%
\pgfsetdash{}{0pt}%
\pgfsys@defobject{currentmarker}{\pgfqpoint{-0.048611in}{0.000000in}}{\pgfqpoint{0.000000in}{0.000000in}}{%
\pgfpathmoveto{\pgfqpoint{0.000000in}{0.000000in}}%
\pgfpathlineto{\pgfqpoint{-0.048611in}{0.000000in}}%
\pgfusepath{stroke,fill}%
}%
\begin{pgfscope}%
\pgfsys@transformshift{0.629216in}{1.743211in}%
\pgfsys@useobject{currentmarker}{}%
\end{pgfscope}%
\end{pgfscope}%
\begin{pgfscope}%
\definecolor{textcolor}{rgb}{0.000000,0.000000,0.000000}%
\pgfsetstrokecolor{textcolor}%
\pgfsetfillcolor{textcolor}%
\pgftext[x=0.322114in,y=1.701002in,left,base]{\color{textcolor}\rmfamily\fontsize{8.000000}{9.600000}\selectfont \(\displaystyle 98.5\)}%
\end{pgfscope}%
\begin{pgfscope}%
\pgfsetbuttcap%
\pgfsetroundjoin%
\definecolor{currentfill}{rgb}{0.000000,0.000000,0.000000}%
\pgfsetfillcolor{currentfill}%
\pgfsetlinewidth{0.803000pt}%
\definecolor{currentstroke}{rgb}{0.000000,0.000000,0.000000}%
\pgfsetstrokecolor{currentstroke}%
\pgfsetdash{}{0pt}%
\pgfsys@defobject{currentmarker}{\pgfqpoint{-0.048611in}{0.000000in}}{\pgfqpoint{0.000000in}{0.000000in}}{%
\pgfpathmoveto{\pgfqpoint{0.000000in}{0.000000in}}%
\pgfpathlineto{\pgfqpoint{-0.048611in}{0.000000in}}%
\pgfusepath{stroke,fill}%
}%
\begin{pgfscope}%
\pgfsys@transformshift{0.629216in}{2.420354in}%
\pgfsys@useobject{currentmarker}{}%
\end{pgfscope}%
\end{pgfscope}%
\begin{pgfscope}%
\definecolor{textcolor}{rgb}{0.000000,0.000000,0.000000}%
\pgfsetstrokecolor{textcolor}%
\pgfsetfillcolor{textcolor}%
\pgftext[x=0.322114in,y=2.378145in,left,base]{\color{textcolor}\rmfamily\fontsize{8.000000}{9.600000}\selectfont \(\displaystyle 99.0\)}%
\end{pgfscope}%
\begin{pgfscope}%
\pgfsetbuttcap%
\pgfsetroundjoin%
\definecolor{currentfill}{rgb}{0.000000,0.000000,0.000000}%
\pgfsetfillcolor{currentfill}%
\pgfsetlinewidth{0.803000pt}%
\definecolor{currentstroke}{rgb}{0.000000,0.000000,0.000000}%
\pgfsetstrokecolor{currentstroke}%
\pgfsetdash{}{0pt}%
\pgfsys@defobject{currentmarker}{\pgfqpoint{-0.048611in}{0.000000in}}{\pgfqpoint{0.000000in}{0.000000in}}{%
\pgfpathmoveto{\pgfqpoint{0.000000in}{0.000000in}}%
\pgfpathlineto{\pgfqpoint{-0.048611in}{0.000000in}}%
\pgfusepath{stroke,fill}%
}%
\begin{pgfscope}%
\pgfsys@transformshift{0.629216in}{3.097497in}%
\pgfsys@useobject{currentmarker}{}%
\end{pgfscope}%
\end{pgfscope}%
\begin{pgfscope}%
\definecolor{textcolor}{rgb}{0.000000,0.000000,0.000000}%
\pgfsetstrokecolor{textcolor}%
\pgfsetfillcolor{textcolor}%
\pgftext[x=0.322114in,y=3.055288in,left,base]{\color{textcolor}\rmfamily\fontsize{8.000000}{9.600000}\selectfont \(\displaystyle 99.5\)}%
\end{pgfscope}%
\begin{pgfscope}%
\pgfsetbuttcap%
\pgfsetroundjoin%
\definecolor{currentfill}{rgb}{0.000000,0.000000,0.000000}%
\pgfsetfillcolor{currentfill}%
\pgfsetlinewidth{0.803000pt}%
\definecolor{currentstroke}{rgb}{0.000000,0.000000,0.000000}%
\pgfsetstrokecolor{currentstroke}%
\pgfsetdash{}{0pt}%
\pgfsys@defobject{currentmarker}{\pgfqpoint{-0.048611in}{0.000000in}}{\pgfqpoint{0.000000in}{0.000000in}}{%
\pgfpathmoveto{\pgfqpoint{0.000000in}{0.000000in}}%
\pgfpathlineto{\pgfqpoint{-0.048611in}{0.000000in}}%
\pgfusepath{stroke,fill}%
}%
\begin{pgfscope}%
\pgfsys@transformshift{0.629216in}{3.774640in}%
\pgfsys@useobject{currentmarker}{}%
\end{pgfscope}%
\end{pgfscope}%
\begin{pgfscope}%
\definecolor{textcolor}{rgb}{0.000000,0.000000,0.000000}%
\pgfsetstrokecolor{textcolor}%
\pgfsetfillcolor{textcolor}%
\pgftext[x=0.263086in,y=3.732430in,left,base]{\color{textcolor}\rmfamily\fontsize{8.000000}{9.600000}\selectfont \(\displaystyle 100.0\)}%
\end{pgfscope}%
\begin{pgfscope}%
\definecolor{textcolor}{rgb}{0.000000,0.000000,0.000000}%
\pgfsetstrokecolor{textcolor}%
\pgfsetfillcolor{textcolor}%
\pgftext[x=0.207530in,y=2.657354in,,bottom,rotate=90.000000]{\color{textcolor}\rmfamily\fontsize{8.000000}{9.600000}\selectfont Hit rate (\%)}%
\end{pgfscope}%
\begin{pgfscope}%
\pgfpathrectangle{\pgfqpoint{0.629216in}{1.472354in}}{\pgfqpoint{2.257143in}{2.370000in}}%
\pgfusepath{clip}%
\pgfsetrectcap%
\pgfsetroundjoin%
\pgfsetlinewidth{1.505625pt}%
\definecolor{currentstroke}{rgb}{0.121569,0.466667,0.705882}%
\pgfsetstrokecolor{currentstroke}%
\pgfsetdash{}{0pt}%
\pgfpathmoveto{\pgfqpoint{0.731814in}{2.420354in}}%
\pgfpathlineto{\pgfqpoint{0.868610in}{3.774640in}}%
\pgfpathlineto{\pgfqpoint{1.005407in}{3.774640in}}%
\pgfpathlineto{\pgfqpoint{1.142203in}{3.774640in}}%
\pgfpathlineto{\pgfqpoint{1.279000in}{3.774640in}}%
\pgfpathlineto{\pgfqpoint{1.348261in}{3.774640in}}%
\pgfpathlineto{\pgfqpoint{1.390141in}{3.774640in}}%
\pgfpathlineto{\pgfqpoint{1.421955in}{3.774640in}}%
\pgfpathlineto{\pgfqpoint{1.448179in}{3.639211in}}%
\pgfpathlineto{\pgfqpoint{1.470824in}{3.639211in}}%
\pgfpathlineto{\pgfqpoint{1.491009in}{3.639211in}}%
\pgfpathlineto{\pgfqpoint{1.509432in}{3.639211in}}%
\pgfpathlineto{\pgfqpoint{1.526393in}{3.639211in}}%
\pgfpathlineto{\pgfqpoint{1.542042in}{3.639211in}}%
\pgfpathlineto{\pgfqpoint{1.556723in}{3.639211in}}%
\pgfpathlineto{\pgfqpoint{1.570691in}{3.639211in}}%
\pgfpathlineto{\pgfqpoint{1.584141in}{3.639211in}}%
\pgfpathlineto{\pgfqpoint{1.597229in}{3.639211in}}%
\pgfpathlineto{\pgfqpoint{1.610082in}{3.639211in}}%
\pgfpathlineto{\pgfqpoint{1.622811in}{3.639211in}}%
\pgfpathlineto{\pgfqpoint{1.635514in}{3.639211in}}%
\pgfpathlineto{\pgfqpoint{1.648280in}{3.639211in}}%
\pgfpathlineto{\pgfqpoint{1.661192in}{3.639211in}}%
\pgfpathlineto{\pgfqpoint{1.673978in}{3.639211in}}%
\pgfpathlineto{\pgfqpoint{1.686423in}{3.774640in}}%
\pgfpathlineto{\pgfqpoint{1.698685in}{3.774640in}}%
\pgfpathlineto{\pgfqpoint{1.710859in}{3.639211in}}%
\pgfpathlineto{\pgfqpoint{1.723059in}{3.774640in}}%
\pgfpathlineto{\pgfqpoint{1.735394in}{3.774640in}}%
\pgfpathlineto{\pgfqpoint{1.747957in}{3.774640in}}%
\pgfpathlineto{\pgfqpoint{1.760837in}{3.639211in}}%
\pgfpathlineto{\pgfqpoint{1.774167in}{3.639211in}}%
\pgfpathlineto{\pgfqpoint{1.788108in}{3.639211in}}%
\pgfpathlineto{\pgfqpoint{1.802741in}{3.639211in}}%
\pgfpathlineto{\pgfqpoint{1.818026in}{3.639211in}}%
\pgfpathlineto{\pgfqpoint{1.832248in}{3.639211in}}%
\pgfpathlineto{\pgfqpoint{1.848979in}{3.639211in}}%
\pgfpathlineto{\pgfqpoint{1.866427in}{3.639211in}}%
\pgfpathlineto{\pgfqpoint{1.884060in}{3.639211in}}%
\pgfpathlineto{\pgfqpoint{1.901454in}{3.639211in}}%
\pgfpathlineto{\pgfqpoint{1.919929in}{3.639211in}}%
\pgfpathlineto{\pgfqpoint{1.933485in}{3.639211in}}%
\pgfpathlineto{\pgfqpoint{1.950853in}{3.639211in}}%
\pgfpathlineto{\pgfqpoint{1.973789in}{3.639211in}}%
\pgfpathlineto{\pgfqpoint{1.999776in}{3.639211in}}%
\pgfpathlineto{\pgfqpoint{2.027650in}{3.639211in}}%
\pgfpathlineto{\pgfqpoint{2.057076in}{3.774640in}}%
\pgfpathlineto{\pgfqpoint{2.088231in}{3.774640in}}%
\pgfpathlineto{\pgfqpoint{2.121099in}{3.774640in}}%
\pgfpathlineto{\pgfqpoint{2.155838in}{3.774640in}}%
\pgfpathlineto{\pgfqpoint{2.192705in}{3.774640in}}%
\pgfpathlineto{\pgfqpoint{2.231946in}{3.774640in}}%
\pgfpathlineto{\pgfqpoint{2.273831in}{3.774640in}}%
\pgfpathlineto{\pgfqpoint{2.318472in}{3.774640in}}%
\pgfpathlineto{\pgfqpoint{2.366069in}{3.774640in}}%
\pgfpathlineto{\pgfqpoint{2.416730in}{3.774640in}}%
\pgfpathlineto{\pgfqpoint{2.470753in}{3.774640in}}%
\pgfpathlineto{\pgfqpoint{2.528321in}{3.774640in}}%
\pgfpathlineto{\pgfqpoint{2.589345in}{3.774640in}}%
\pgfpathlineto{\pgfqpoint{2.653624in}{3.774640in}}%
\pgfpathlineto{\pgfqpoint{2.721219in}{3.774640in}}%
\pgfpathlineto{\pgfqpoint{2.783761in}{3.774640in}}%
\pgfusepath{stroke}%
\end{pgfscope}%
\begin{pgfscope}%
\pgfpathrectangle{\pgfqpoint{0.629216in}{1.472354in}}{\pgfqpoint{2.257143in}{2.370000in}}%
\pgfusepath{clip}%
\pgfsetrectcap%
\pgfsetroundjoin%
\pgfsetlinewidth{1.505625pt}%
\definecolor{currentstroke}{rgb}{1.000000,0.498039,0.054902}%
\pgfsetstrokecolor{currentstroke}%
\pgfsetdash{}{0pt}%
\pgfpathmoveto{\pgfqpoint{0.802022in}{1.467354in}}%
\pgfpathlineto{\pgfqpoint{0.868610in}{3.774640in}}%
\pgfpathlineto{\pgfqpoint{1.005407in}{3.774640in}}%
\pgfpathlineto{\pgfqpoint{1.142203in}{3.774640in}}%
\pgfpathlineto{\pgfqpoint{1.279000in}{3.774640in}}%
\pgfpathlineto{\pgfqpoint{1.348261in}{3.639211in}}%
\pgfpathlineto{\pgfqpoint{1.390141in}{3.639211in}}%
\pgfpathlineto{\pgfqpoint{1.421955in}{3.639211in}}%
\pgfpathlineto{\pgfqpoint{1.448179in}{3.503783in}}%
\pgfpathlineto{\pgfqpoint{1.470824in}{3.503783in}}%
\pgfpathlineto{\pgfqpoint{1.491009in}{3.503783in}}%
\pgfpathlineto{\pgfqpoint{1.509432in}{3.503783in}}%
\pgfpathlineto{\pgfqpoint{1.526393in}{3.503783in}}%
\pgfpathlineto{\pgfqpoint{1.542042in}{3.503783in}}%
\pgfpathlineto{\pgfqpoint{1.556723in}{3.368354in}}%
\pgfpathlineto{\pgfqpoint{1.570691in}{3.368354in}}%
\pgfpathlineto{\pgfqpoint{1.584141in}{3.368354in}}%
\pgfpathlineto{\pgfqpoint{1.597229in}{3.368354in}}%
\pgfpathlineto{\pgfqpoint{1.610082in}{3.368354in}}%
\pgfpathlineto{\pgfqpoint{1.622811in}{3.368354in}}%
\pgfpathlineto{\pgfqpoint{1.635514in}{3.368354in}}%
\pgfpathlineto{\pgfqpoint{1.648280in}{3.368354in}}%
\pgfpathlineto{\pgfqpoint{1.661192in}{3.368354in}}%
\pgfpathlineto{\pgfqpoint{1.673978in}{3.368354in}}%
\pgfpathlineto{\pgfqpoint{1.686423in}{3.368354in}}%
\pgfpathlineto{\pgfqpoint{1.698685in}{3.368354in}}%
\pgfpathlineto{\pgfqpoint{1.710859in}{3.368354in}}%
\pgfpathlineto{\pgfqpoint{1.723059in}{3.368354in}}%
\pgfpathlineto{\pgfqpoint{1.735394in}{3.368354in}}%
\pgfpathlineto{\pgfqpoint{1.747957in}{3.368354in}}%
\pgfpathlineto{\pgfqpoint{1.760837in}{3.368354in}}%
\pgfpathlineto{\pgfqpoint{1.774167in}{3.232925in}}%
\pgfpathlineto{\pgfqpoint{1.788103in}{3.232925in}}%
\pgfpathlineto{\pgfqpoint{1.802681in}{3.368354in}}%
\pgfpathlineto{\pgfqpoint{1.817954in}{3.368354in}}%
\pgfpathlineto{\pgfqpoint{1.833966in}{3.368354in}}%
\pgfpathlineto{\pgfqpoint{1.850017in}{3.368354in}}%
\pgfpathlineto{\pgfqpoint{1.867586in}{3.503783in}}%
\pgfpathlineto{\pgfqpoint{1.885596in}{3.368354in}}%
\pgfpathlineto{\pgfqpoint{1.903446in}{3.368354in}}%
\pgfpathlineto{\pgfqpoint{1.922076in}{3.368354in}}%
\pgfpathlineto{\pgfqpoint{1.941648in}{3.368354in}}%
\pgfpathlineto{\pgfqpoint{1.952398in}{3.232925in}}%
\pgfpathlineto{\pgfqpoint{1.974978in}{3.503783in}}%
\pgfpathlineto{\pgfqpoint{2.000776in}{3.503783in}}%
\pgfpathlineto{\pgfqpoint{2.028730in}{3.503783in}}%
\pgfpathlineto{\pgfqpoint{2.058429in}{3.503783in}}%
\pgfpathlineto{\pgfqpoint{2.089840in}{3.503783in}}%
\pgfpathlineto{\pgfqpoint{2.122980in}{3.503783in}}%
\pgfpathlineto{\pgfqpoint{2.158111in}{3.503783in}}%
\pgfpathlineto{\pgfqpoint{2.195289in}{3.503783in}}%
\pgfpathlineto{\pgfqpoint{2.234900in}{3.639211in}}%
\pgfpathlineto{\pgfqpoint{2.276921in}{3.639211in}}%
\pgfpathlineto{\pgfqpoint{2.321615in}{3.639211in}}%
\pgfpathlineto{\pgfqpoint{2.369253in}{3.639211in}}%
\pgfpathlineto{\pgfqpoint{2.419903in}{3.639211in}}%
\pgfpathlineto{\pgfqpoint{2.473908in}{3.639211in}}%
\pgfpathlineto{\pgfqpoint{2.531457in}{3.639211in}}%
\pgfpathlineto{\pgfqpoint{2.592446in}{3.639211in}}%
\pgfpathlineto{\pgfqpoint{2.656672in}{3.639211in}}%
\pgfpathlineto{\pgfqpoint{2.724203in}{3.639211in}}%
\pgfpathlineto{\pgfqpoint{2.783761in}{3.639211in}}%
\pgfusepath{stroke}%
\end{pgfscope}%
\begin{pgfscope}%
\pgfpathrectangle{\pgfqpoint{0.629216in}{1.472354in}}{\pgfqpoint{2.257143in}{2.370000in}}%
\pgfusepath{clip}%
\pgfsetrectcap%
\pgfsetroundjoin%
\pgfsetlinewidth{1.505625pt}%
\definecolor{currentstroke}{rgb}{0.172549,0.627451,0.172549}%
\pgfsetstrokecolor{currentstroke}%
\pgfsetdash{}{0pt}%
\pgfpathmoveto{\pgfqpoint{0.803872in}{1.467354in}}%
\pgfpathlineto{\pgfqpoint{0.868610in}{3.774640in}}%
\pgfpathlineto{\pgfqpoint{1.005407in}{3.774640in}}%
\pgfpathlineto{\pgfqpoint{1.142203in}{3.774640in}}%
\pgfpathlineto{\pgfqpoint{1.279000in}{3.774640in}}%
\pgfpathlineto{\pgfqpoint{1.348261in}{3.639211in}}%
\pgfpathlineto{\pgfqpoint{1.390141in}{3.503783in}}%
\pgfpathlineto{\pgfqpoint{1.421955in}{3.503783in}}%
\pgfpathlineto{\pgfqpoint{1.448179in}{3.368354in}}%
\pgfpathlineto{\pgfqpoint{1.470824in}{3.368354in}}%
\pgfpathlineto{\pgfqpoint{1.491009in}{3.368354in}}%
\pgfpathlineto{\pgfqpoint{1.509432in}{3.232925in}}%
\pgfpathlineto{\pgfqpoint{1.526393in}{3.232925in}}%
\pgfpathlineto{\pgfqpoint{1.542042in}{3.232925in}}%
\pgfpathlineto{\pgfqpoint{1.556723in}{3.232925in}}%
\pgfpathlineto{\pgfqpoint{1.570691in}{3.232925in}}%
\pgfpathlineto{\pgfqpoint{1.584141in}{3.232925in}}%
\pgfpathlineto{\pgfqpoint{1.597229in}{3.232925in}}%
\pgfpathlineto{\pgfqpoint{1.610082in}{3.097497in}}%
\pgfpathlineto{\pgfqpoint{1.622811in}{3.097497in}}%
\pgfpathlineto{\pgfqpoint{1.635514in}{3.097497in}}%
\pgfpathlineto{\pgfqpoint{1.648280in}{3.097497in}}%
\pgfpathlineto{\pgfqpoint{1.661192in}{3.232925in}}%
\pgfpathlineto{\pgfqpoint{1.673978in}{3.097497in}}%
\pgfpathlineto{\pgfqpoint{1.686423in}{3.097497in}}%
\pgfpathlineto{\pgfqpoint{1.698685in}{3.097497in}}%
\pgfpathlineto{\pgfqpoint{1.710859in}{3.097497in}}%
\pgfpathlineto{\pgfqpoint{1.723059in}{3.097497in}}%
\pgfpathlineto{\pgfqpoint{1.735394in}{3.097497in}}%
\pgfpathlineto{\pgfqpoint{1.747957in}{3.097497in}}%
\pgfpathlineto{\pgfqpoint{1.760837in}{3.097497in}}%
\pgfpathlineto{\pgfqpoint{1.774165in}{2.962068in}}%
\pgfpathlineto{\pgfqpoint{1.788101in}{3.097497in}}%
\pgfpathlineto{\pgfqpoint{1.802681in}{3.232925in}}%
\pgfpathlineto{\pgfqpoint{1.817964in}{3.097497in}}%
\pgfpathlineto{\pgfqpoint{1.833968in}{3.232925in}}%
\pgfpathlineto{\pgfqpoint{1.850807in}{3.232925in}}%
\pgfpathlineto{\pgfqpoint{1.868425in}{3.232925in}}%
\pgfpathlineto{\pgfqpoint{1.886393in}{3.232925in}}%
\pgfpathlineto{\pgfqpoint{1.904298in}{3.232925in}}%
\pgfpathlineto{\pgfqpoint{1.923008in}{3.232925in}}%
\pgfpathlineto{\pgfqpoint{1.942332in}{3.232925in}}%
\pgfpathlineto{\pgfqpoint{1.956483in}{3.097497in}}%
\pgfpathlineto{\pgfqpoint{1.979601in}{3.368354in}}%
\pgfpathlineto{\pgfqpoint{2.006138in}{3.368354in}}%
\pgfpathlineto{\pgfqpoint{2.034542in}{3.368354in}}%
\pgfpathlineto{\pgfqpoint{2.064625in}{3.368354in}}%
\pgfpathlineto{\pgfqpoint{2.096351in}{3.368354in}}%
\pgfpathlineto{\pgfqpoint{2.129899in}{3.368354in}}%
\pgfpathlineto{\pgfqpoint{2.165292in}{3.368354in}}%
\pgfpathlineto{\pgfqpoint{2.202851in}{3.503783in}}%
\pgfpathlineto{\pgfqpoint{2.242901in}{3.503783in}}%
\pgfpathlineto{\pgfqpoint{2.285492in}{3.503783in}}%
\pgfpathlineto{\pgfqpoint{2.330879in}{3.503783in}}%
\pgfpathlineto{\pgfqpoint{2.379270in}{3.503783in}}%
\pgfpathlineto{\pgfqpoint{2.430831in}{3.503783in}}%
\pgfpathlineto{\pgfqpoint{2.485859in}{3.503783in}}%
\pgfpathlineto{\pgfqpoint{2.544521in}{3.503783in}}%
\pgfpathlineto{\pgfqpoint{2.606558in}{3.503783in}}%
\pgfpathlineto{\pgfqpoint{2.671754in}{3.639211in}}%
\pgfpathlineto{\pgfqpoint{2.740194in}{3.639211in}}%
\pgfpathlineto{\pgfqpoint{2.783761in}{3.503783in}}%
\pgfusepath{stroke}%
\end{pgfscope}%
\begin{pgfscope}%
\pgfpathrectangle{\pgfqpoint{0.629216in}{1.472354in}}{\pgfqpoint{2.257143in}{2.370000in}}%
\pgfusepath{clip}%
\pgfsetrectcap%
\pgfsetroundjoin%
\pgfsetlinewidth{1.505625pt}%
\definecolor{currentstroke}{rgb}{0.839216,0.152941,0.156863}%
\pgfsetstrokecolor{currentstroke}%
\pgfsetdash{}{0pt}%
\pgfpathmoveto{\pgfqpoint{0.803872in}{1.467354in}}%
\pgfpathlineto{\pgfqpoint{0.868610in}{3.774640in}}%
\pgfpathlineto{\pgfqpoint{1.005407in}{3.774640in}}%
\pgfpathlineto{\pgfqpoint{1.142203in}{3.639211in}}%
\pgfpathlineto{\pgfqpoint{1.279000in}{3.639211in}}%
\pgfpathlineto{\pgfqpoint{1.348261in}{3.503783in}}%
\pgfpathlineto{\pgfqpoint{1.390141in}{3.368354in}}%
\pgfpathlineto{\pgfqpoint{1.421955in}{3.368354in}}%
\pgfpathlineto{\pgfqpoint{1.448179in}{3.232925in}}%
\pgfpathlineto{\pgfqpoint{1.470824in}{3.232925in}}%
\pgfpathlineto{\pgfqpoint{1.491009in}{3.097497in}}%
\pgfpathlineto{\pgfqpoint{1.509432in}{3.097497in}}%
\pgfpathlineto{\pgfqpoint{1.526393in}{3.097497in}}%
\pgfpathlineto{\pgfqpoint{1.542042in}{2.962068in}}%
\pgfpathlineto{\pgfqpoint{1.556723in}{2.962068in}}%
\pgfpathlineto{\pgfqpoint{1.570691in}{2.962068in}}%
\pgfpathlineto{\pgfqpoint{1.584141in}{2.962068in}}%
\pgfpathlineto{\pgfqpoint{1.597229in}{2.962068in}}%
\pgfpathlineto{\pgfqpoint{1.610082in}{2.962068in}}%
\pgfpathlineto{\pgfqpoint{1.622811in}{2.962068in}}%
\pgfpathlineto{\pgfqpoint{1.635514in}{2.962068in}}%
\pgfpathlineto{\pgfqpoint{1.648280in}{2.962068in}}%
\pgfpathlineto{\pgfqpoint{1.661192in}{2.962068in}}%
\pgfpathlineto{\pgfqpoint{1.673978in}{2.962068in}}%
\pgfpathlineto{\pgfqpoint{1.686423in}{2.962068in}}%
\pgfpathlineto{\pgfqpoint{1.698685in}{2.826640in}}%
\pgfpathlineto{\pgfqpoint{1.710859in}{2.962068in}}%
\pgfpathlineto{\pgfqpoint{1.723059in}{2.962068in}}%
\pgfpathlineto{\pgfqpoint{1.735394in}{2.962068in}}%
\pgfpathlineto{\pgfqpoint{1.747957in}{2.962068in}}%
\pgfpathlineto{\pgfqpoint{1.760837in}{2.962068in}}%
\pgfpathlineto{\pgfqpoint{1.774165in}{2.826640in}}%
\pgfpathlineto{\pgfqpoint{1.788101in}{2.826640in}}%
\pgfpathlineto{\pgfqpoint{1.802681in}{2.962068in}}%
\pgfpathlineto{\pgfqpoint{1.817964in}{2.826640in}}%
\pgfpathlineto{\pgfqpoint{1.833968in}{2.962068in}}%
\pgfpathlineto{\pgfqpoint{1.850807in}{2.962068in}}%
\pgfpathlineto{\pgfqpoint{1.868425in}{2.962068in}}%
\pgfpathlineto{\pgfqpoint{1.886393in}{2.962068in}}%
\pgfpathlineto{\pgfqpoint{1.904298in}{2.962068in}}%
\pgfpathlineto{\pgfqpoint{1.923008in}{3.097497in}}%
\pgfpathlineto{\pgfqpoint{1.942331in}{3.097497in}}%
\pgfpathlineto{\pgfqpoint{1.956480in}{2.826640in}}%
\pgfpathlineto{\pgfqpoint{1.979601in}{3.097497in}}%
\pgfpathlineto{\pgfqpoint{2.006137in}{3.232925in}}%
\pgfpathlineto{\pgfqpoint{2.034541in}{3.232925in}}%
\pgfpathlineto{\pgfqpoint{2.064625in}{3.232925in}}%
\pgfpathlineto{\pgfqpoint{2.096351in}{3.232925in}}%
\pgfpathlineto{\pgfqpoint{2.129897in}{3.232925in}}%
\pgfpathlineto{\pgfqpoint{2.165289in}{3.368354in}}%
\pgfpathlineto{\pgfqpoint{2.202843in}{3.368354in}}%
\pgfpathlineto{\pgfqpoint{2.242893in}{3.368354in}}%
\pgfpathlineto{\pgfqpoint{2.285479in}{3.368354in}}%
\pgfpathlineto{\pgfqpoint{2.330852in}{3.368354in}}%
\pgfpathlineto{\pgfqpoint{2.379242in}{3.368354in}}%
\pgfpathlineto{\pgfqpoint{2.430810in}{3.368354in}}%
\pgfpathlineto{\pgfqpoint{2.485822in}{3.503783in}}%
\pgfpathlineto{\pgfqpoint{2.544537in}{3.503783in}}%
\pgfpathlineto{\pgfqpoint{2.606519in}{3.503783in}}%
\pgfpathlineto{\pgfqpoint{2.671640in}{3.503783in}}%
\pgfpathlineto{\pgfqpoint{2.739973in}{3.503783in}}%
\pgfpathlineto{\pgfqpoint{2.783761in}{3.368354in}}%
\pgfusepath{stroke}%
\end{pgfscope}%
\begin{pgfscope}%
\pgfsetrectcap%
\pgfsetmiterjoin%
\pgfsetlinewidth{0.803000pt}%
\definecolor{currentstroke}{rgb}{0.000000,0.000000,0.000000}%
\pgfsetstrokecolor{currentstroke}%
\pgfsetdash{}{0pt}%
\pgfpathmoveto{\pgfqpoint{0.629216in}{1.472354in}}%
\pgfpathlineto{\pgfqpoint{0.629216in}{3.842354in}}%
\pgfusepath{stroke}%
\end{pgfscope}%
\begin{pgfscope}%
\pgfsetrectcap%
\pgfsetmiterjoin%
\pgfsetlinewidth{0.803000pt}%
\definecolor{currentstroke}{rgb}{0.000000,0.000000,0.000000}%
\pgfsetstrokecolor{currentstroke}%
\pgfsetdash{}{0pt}%
\pgfpathmoveto{\pgfqpoint{2.886359in}{1.472354in}}%
\pgfpathlineto{\pgfqpoint{2.886359in}{3.842354in}}%
\pgfusepath{stroke}%
\end{pgfscope}%
\begin{pgfscope}%
\pgfsetrectcap%
\pgfsetmiterjoin%
\pgfsetlinewidth{0.803000pt}%
\definecolor{currentstroke}{rgb}{0.000000,0.000000,0.000000}%
\pgfsetstrokecolor{currentstroke}%
\pgfsetdash{}{0pt}%
\pgfpathmoveto{\pgfqpoint{0.629216in}{1.472354in}}%
\pgfpathlineto{\pgfqpoint{2.886359in}{1.472354in}}%
\pgfusepath{stroke}%
\end{pgfscope}%
\begin{pgfscope}%
\pgfsetrectcap%
\pgfsetmiterjoin%
\pgfsetlinewidth{0.803000pt}%
\definecolor{currentstroke}{rgb}{0.000000,0.000000,0.000000}%
\pgfsetstrokecolor{currentstroke}%
\pgfsetdash{}{0pt}%
\pgfpathmoveto{\pgfqpoint{0.629216in}{3.842354in}}%
\pgfpathlineto{\pgfqpoint{2.886359in}{3.842354in}}%
\pgfusepath{stroke}%
\end{pgfscope}%
\begin{pgfscope}%
\pgfsetrectcap%
\pgfsetroundjoin%
\pgfsetlinewidth{1.505625pt}%
\definecolor{currentstroke}{rgb}{0.121569,0.466667,0.705882}%
\pgfsetstrokecolor{currentstroke}%
\pgfsetdash{}{0pt}%
\pgfpathmoveto{\pgfqpoint{2.155929in}{2.042205in}}%
\pgfpathlineto{\pgfqpoint{2.350374in}{2.042205in}}%
\pgfusepath{stroke}%
\end{pgfscope}%
\begin{pgfscope}%
\definecolor{textcolor}{rgb}{0.000000,0.000000,0.000000}%
\pgfsetstrokecolor{textcolor}%
\pgfsetfillcolor{textcolor}%
\pgftext[x=2.428152in,y=2.008177in,left,base]{\color{textcolor}\rmfamily\fontsize{7.000000}{8.400000}\selectfont 1.0\%}%
\end{pgfscope}%
\begin{pgfscope}%
\pgfsetrectcap%
\pgfsetroundjoin%
\pgfsetlinewidth{1.505625pt}%
\definecolor{currentstroke}{rgb}{1.000000,0.498039,0.054902}%
\pgfsetstrokecolor{currentstroke}%
\pgfsetdash{}{0pt}%
\pgfpathmoveto{\pgfqpoint{2.155929in}{1.899505in}}%
\pgfpathlineto{\pgfqpoint{2.350374in}{1.899505in}}%
\pgfusepath{stroke}%
\end{pgfscope}%
\begin{pgfscope}%
\definecolor{textcolor}{rgb}{0.000000,0.000000,0.000000}%
\pgfsetstrokecolor{textcolor}%
\pgfsetfillcolor{textcolor}%
\pgftext[x=2.428152in,y=1.865477in,left,base]{\color{textcolor}\rmfamily\fontsize{7.000000}{8.400000}\selectfont 0.1\%}%
\end{pgfscope}%
\begin{pgfscope}%
\pgfsetrectcap%
\pgfsetroundjoin%
\pgfsetlinewidth{1.505625pt}%
\definecolor{currentstroke}{rgb}{0.172549,0.627451,0.172549}%
\pgfsetstrokecolor{currentstroke}%
\pgfsetdash{}{0pt}%
\pgfpathmoveto{\pgfqpoint{2.155929in}{1.756805in}}%
\pgfpathlineto{\pgfqpoint{2.350374in}{1.756805in}}%
\pgfusepath{stroke}%
\end{pgfscope}%
\begin{pgfscope}%
\definecolor{textcolor}{rgb}{0.000000,0.000000,0.000000}%
\pgfsetstrokecolor{textcolor}%
\pgfsetfillcolor{textcolor}%
\pgftext[x=2.428152in,y=1.722777in,left,base]{\color{textcolor}\rmfamily\fontsize{7.000000}{8.400000}\selectfont 0.01\%}%
\end{pgfscope}%
\begin{pgfscope}%
\pgfsetrectcap%
\pgfsetroundjoin%
\pgfsetlinewidth{1.505625pt}%
\definecolor{currentstroke}{rgb}{0.839216,0.152941,0.156863}%
\pgfsetstrokecolor{currentstroke}%
\pgfsetdash{}{0pt}%
\pgfpathmoveto{\pgfqpoint{2.155929in}{1.614105in}}%
\pgfpathlineto{\pgfqpoint{2.350374in}{1.614105in}}%
\pgfusepath{stroke}%
\end{pgfscope}%
\begin{pgfscope}%
\definecolor{textcolor}{rgb}{0.000000,0.000000,0.000000}%
\pgfsetstrokecolor{textcolor}%
\pgfsetfillcolor{textcolor}%
\pgftext[x=2.428152in,y=1.580077in,left,base]{\color{textcolor}\rmfamily\fontsize{7.000000}{8.400000}\selectfont 0.001\%}%
\end{pgfscope}%
\begin{pgfscope}%
\pgfsetbuttcap%
\pgfsetmiterjoin%
\definecolor{currentfill}{rgb}{1.000000,1.000000,1.000000}%
\pgfsetfillcolor{currentfill}%
\pgfsetlinewidth{0.000000pt}%
\definecolor{currentstroke}{rgb}{0.000000,0.000000,0.000000}%
\pgfsetstrokecolor{currentstroke}%
\pgfsetstrokeopacity{0.000000}%
\pgfsetdash{}{0pt}%
\pgfpathmoveto{\pgfqpoint{0.629216in}{0.484854in}}%
\pgfpathlineto{\pgfqpoint{2.886359in}{0.484854in}}%
\pgfpathlineto{\pgfqpoint{2.886359in}{1.472354in}}%
\pgfpathlineto{\pgfqpoint{0.629216in}{1.472354in}}%
\pgfpathclose%
\pgfusepath{fill}%
\end{pgfscope}%
\begin{pgfscope}%
\pgfsetbuttcap%
\pgfsetroundjoin%
\definecolor{currentfill}{rgb}{0.000000,0.000000,0.000000}%
\pgfsetfillcolor{currentfill}%
\pgfsetlinewidth{0.803000pt}%
\definecolor{currentstroke}{rgb}{0.000000,0.000000,0.000000}%
\pgfsetstrokecolor{currentstroke}%
\pgfsetdash{}{0pt}%
\pgfsys@defobject{currentmarker}{\pgfqpoint{0.000000in}{-0.048611in}}{\pgfqpoint{0.000000in}{0.000000in}}{%
\pgfpathmoveto{\pgfqpoint{0.000000in}{0.000000in}}%
\pgfpathlineto{\pgfqpoint{0.000000in}{-0.048611in}}%
\pgfusepath{stroke,fill}%
}%
\begin{pgfscope}%
\pgfsys@transformshift{0.731814in}{0.484854in}%
\pgfsys@useobject{currentmarker}{}%
\end{pgfscope}%
\end{pgfscope}%
\begin{pgfscope}%
\definecolor{textcolor}{rgb}{0.000000,0.000000,0.000000}%
\pgfsetstrokecolor{textcolor}%
\pgfsetfillcolor{textcolor}%
\pgftext[x=0.731814in,y=0.387632in,,top]{\color{textcolor}\rmfamily\fontsize{8.000000}{9.600000}\selectfont \(\displaystyle 0.0\)}%
\end{pgfscope}%
\begin{pgfscope}%
\pgfsetbuttcap%
\pgfsetroundjoin%
\definecolor{currentfill}{rgb}{0.000000,0.000000,0.000000}%
\pgfsetfillcolor{currentfill}%
\pgfsetlinewidth{0.803000pt}%
\definecolor{currentstroke}{rgb}{0.000000,0.000000,0.000000}%
\pgfsetstrokecolor{currentstroke}%
\pgfsetdash{}{0pt}%
\pgfsys@defobject{currentmarker}{\pgfqpoint{0.000000in}{-0.048611in}}{\pgfqpoint{0.000000in}{0.000000in}}{%
\pgfpathmoveto{\pgfqpoint{0.000000in}{0.000000in}}%
\pgfpathlineto{\pgfqpoint{0.000000in}{-0.048611in}}%
\pgfusepath{stroke,fill}%
}%
\begin{pgfscope}%
\pgfsys@transformshift{1.415796in}{0.484854in}%
\pgfsys@useobject{currentmarker}{}%
\end{pgfscope}%
\end{pgfscope}%
\begin{pgfscope}%
\definecolor{textcolor}{rgb}{0.000000,0.000000,0.000000}%
\pgfsetstrokecolor{textcolor}%
\pgfsetfillcolor{textcolor}%
\pgftext[x=1.415796in,y=0.387632in,,top]{\color{textcolor}\rmfamily\fontsize{8.000000}{9.600000}\selectfont \(\displaystyle 0.5\)}%
\end{pgfscope}%
\begin{pgfscope}%
\pgfsetbuttcap%
\pgfsetroundjoin%
\definecolor{currentfill}{rgb}{0.000000,0.000000,0.000000}%
\pgfsetfillcolor{currentfill}%
\pgfsetlinewidth{0.803000pt}%
\definecolor{currentstroke}{rgb}{0.000000,0.000000,0.000000}%
\pgfsetstrokecolor{currentstroke}%
\pgfsetdash{}{0pt}%
\pgfsys@defobject{currentmarker}{\pgfqpoint{0.000000in}{-0.048611in}}{\pgfqpoint{0.000000in}{0.000000in}}{%
\pgfpathmoveto{\pgfqpoint{0.000000in}{0.000000in}}%
\pgfpathlineto{\pgfqpoint{0.000000in}{-0.048611in}}%
\pgfusepath{stroke,fill}%
}%
\begin{pgfscope}%
\pgfsys@transformshift{2.099779in}{0.484854in}%
\pgfsys@useobject{currentmarker}{}%
\end{pgfscope}%
\end{pgfscope}%
\begin{pgfscope}%
\definecolor{textcolor}{rgb}{0.000000,0.000000,0.000000}%
\pgfsetstrokecolor{textcolor}%
\pgfsetfillcolor{textcolor}%
\pgftext[x=2.099779in,y=0.387632in,,top]{\color{textcolor}\rmfamily\fontsize{8.000000}{9.600000}\selectfont \(\displaystyle 1.0\)}%
\end{pgfscope}%
\begin{pgfscope}%
\pgfsetbuttcap%
\pgfsetroundjoin%
\definecolor{currentfill}{rgb}{0.000000,0.000000,0.000000}%
\pgfsetfillcolor{currentfill}%
\pgfsetlinewidth{0.803000pt}%
\definecolor{currentstroke}{rgb}{0.000000,0.000000,0.000000}%
\pgfsetstrokecolor{currentstroke}%
\pgfsetdash{}{0pt}%
\pgfsys@defobject{currentmarker}{\pgfqpoint{0.000000in}{-0.048611in}}{\pgfqpoint{0.000000in}{0.000000in}}{%
\pgfpathmoveto{\pgfqpoint{0.000000in}{0.000000in}}%
\pgfpathlineto{\pgfqpoint{0.000000in}{-0.048611in}}%
\pgfusepath{stroke,fill}%
}%
\begin{pgfscope}%
\pgfsys@transformshift{2.783761in}{0.484854in}%
\pgfsys@useobject{currentmarker}{}%
\end{pgfscope}%
\end{pgfscope}%
\begin{pgfscope}%
\definecolor{textcolor}{rgb}{0.000000,0.000000,0.000000}%
\pgfsetstrokecolor{textcolor}%
\pgfsetfillcolor{textcolor}%
\pgftext[x=2.783761in,y=0.387632in,,top]{\color{textcolor}\rmfamily\fontsize{8.000000}{9.600000}\selectfont \(\displaystyle 1.5\)}%
\end{pgfscope}%
\begin{pgfscope}%
\definecolor{textcolor}{rgb}{0.000000,0.000000,0.000000}%
\pgfsetstrokecolor{textcolor}%
\pgfsetfillcolor{textcolor}%
\pgftext[x=1.757788in,y=0.224546in,,top]{\color{textcolor}\rmfamily\fontsize{8.000000}{9.600000}\selectfont Time (\(\displaystyle \times 10^6 \, \mathrm{yr}\))}%
\end{pgfscope}%
\begin{pgfscope}%
\pgfsetbuttcap%
\pgfsetroundjoin%
\definecolor{currentfill}{rgb}{0.000000,0.000000,0.000000}%
\pgfsetfillcolor{currentfill}%
\pgfsetlinewidth{0.803000pt}%
\definecolor{currentstroke}{rgb}{0.000000,0.000000,0.000000}%
\pgfsetstrokecolor{currentstroke}%
\pgfsetdash{}{0pt}%
\pgfsys@defobject{currentmarker}{\pgfqpoint{-0.048611in}{0.000000in}}{\pgfqpoint{0.000000in}{0.000000in}}{%
\pgfpathmoveto{\pgfqpoint{0.000000in}{0.000000in}}%
\pgfpathlineto{\pgfqpoint{-0.048611in}{0.000000in}}%
\pgfusepath{stroke,fill}%
}%
\begin{pgfscope}%
\pgfsys@transformshift{0.629216in}{0.934759in}%
\pgfsys@useobject{currentmarker}{}%
\end{pgfscope}%
\end{pgfscope}%
\begin{pgfscope}%
\definecolor{textcolor}{rgb}{0.000000,0.000000,0.000000}%
\pgfsetstrokecolor{textcolor}%
\pgfsetfillcolor{textcolor}%
\pgftext[x=0.413937in,y=0.892549in,left,base]{\color{textcolor}\rmfamily\fontsize{8.000000}{9.600000}\selectfont \(\displaystyle 50\)}%
\end{pgfscope}%
\begin{pgfscope}%
\pgfsetbuttcap%
\pgfsetroundjoin%
\definecolor{currentfill}{rgb}{0.000000,0.000000,0.000000}%
\pgfsetfillcolor{currentfill}%
\pgfsetlinewidth{0.803000pt}%
\definecolor{currentstroke}{rgb}{0.000000,0.000000,0.000000}%
\pgfsetstrokecolor{currentstroke}%
\pgfsetdash{}{0pt}%
\pgfsys@defobject{currentmarker}{\pgfqpoint{-0.048611in}{0.000000in}}{\pgfqpoint{0.000000in}{0.000000in}}{%
\pgfpathmoveto{\pgfqpoint{0.000000in}{0.000000in}}%
\pgfpathlineto{\pgfqpoint{-0.048611in}{0.000000in}}%
\pgfusepath{stroke,fill}%
}%
\begin{pgfscope}%
\pgfsys@transformshift{0.629216in}{1.427468in}%
\pgfsys@useobject{currentmarker}{}%
\end{pgfscope}%
\end{pgfscope}%
\begin{pgfscope}%
\definecolor{textcolor}{rgb}{0.000000,0.000000,0.000000}%
\pgfsetstrokecolor{textcolor}%
\pgfsetfillcolor{textcolor}%
\pgftext[x=0.354908in,y=1.385259in,left,base]{\color{textcolor}\rmfamily\fontsize{8.000000}{9.600000}\selectfont \(\displaystyle 100\)}%
\end{pgfscope}%
\begin{pgfscope}%
\definecolor{textcolor}{rgb}{0.000000,0.000000,0.000000}%
\pgfsetstrokecolor{textcolor}%
\pgfsetfillcolor{textcolor}%
\pgftext[x=0.299353in,y=0.978604in,,bottom,rotate=90.000000]{\color{textcolor}\rmfamily\fontsize{8.000000}{9.600000}\selectfont Timestep (\(\displaystyle \times 10^3 \, \mathrm{yr}\))}%
\end{pgfscope}%
\begin{pgfscope}%
\pgfpathrectangle{\pgfqpoint{0.629216in}{0.484854in}}{\pgfqpoint{2.257143in}{0.987500in}}%
\pgfusepath{clip}%
\pgfsetrectcap%
\pgfsetroundjoin%
\pgfsetlinewidth{1.505625pt}%
\definecolor{currentstroke}{rgb}{0.121569,0.466667,0.705882}%
\pgfsetstrokecolor{currentstroke}%
\pgfsetdash{}{0pt}%
\pgfpathmoveto{\pgfqpoint{0.731814in}{1.427468in}}%
\pgfpathlineto{\pgfqpoint{0.868610in}{1.427468in}}%
\pgfpathlineto{\pgfqpoint{1.005407in}{1.427468in}}%
\pgfpathlineto{\pgfqpoint{1.142203in}{1.427468in}}%
\pgfpathlineto{\pgfqpoint{1.279000in}{0.940979in}}%
\pgfpathlineto{\pgfqpoint{1.348261in}{0.743727in}}%
\pgfpathlineto{\pgfqpoint{1.390141in}{0.671226in}}%
\pgfpathlineto{\pgfqpoint{1.421955in}{0.630953in}}%
\pgfpathlineto{\pgfqpoint{1.448179in}{0.605176in}}%
\pgfpathlineto{\pgfqpoint{1.470824in}{0.587449in}}%
\pgfpathlineto{\pgfqpoint{1.491009in}{0.574760in}}%
\pgfpathlineto{\pgfqpoint{1.509432in}{0.564234in}}%
\pgfpathlineto{\pgfqpoint{1.526393in}{0.554773in}}%
\pgfpathlineto{\pgfqpoint{1.542042in}{0.547804in}}%
\pgfpathlineto{\pgfqpoint{1.556723in}{0.542668in}}%
\pgfpathlineto{\pgfqpoint{1.570691in}{0.538938in}}%
\pgfpathlineto{\pgfqpoint{1.584141in}{0.536327in}}%
\pgfpathlineto{\pgfqpoint{1.597229in}{0.534640in}}%
\pgfpathlineto{\pgfqpoint{1.610082in}{0.533745in}}%
\pgfpathlineto{\pgfqpoint{1.622811in}{0.533555in}}%
\pgfpathlineto{\pgfqpoint{1.635514in}{0.534009in}}%
\pgfpathlineto{\pgfqpoint{1.648280in}{0.535062in}}%
\pgfpathlineto{\pgfqpoint{1.661192in}{0.534151in}}%
\pgfpathlineto{\pgfqpoint{1.673978in}{0.531698in}}%
\pgfpathlineto{\pgfqpoint{1.686423in}{0.530382in}}%
\pgfpathlineto{\pgfqpoint{1.698685in}{0.529741in}}%
\pgfpathlineto{\pgfqpoint{1.710859in}{0.529939in}}%
\pgfpathlineto{\pgfqpoint{1.723059in}{0.530900in}}%
\pgfpathlineto{\pgfqpoint{1.735394in}{0.532549in}}%
\pgfpathlineto{\pgfqpoint{1.747957in}{0.534831in}}%
\pgfpathlineto{\pgfqpoint{1.760837in}{0.538071in}}%
\pgfpathlineto{\pgfqpoint{1.774167in}{0.542473in}}%
\pgfpathlineto{\pgfqpoint{1.788108in}{0.547462in}}%
\pgfpathlineto{\pgfqpoint{1.802741in}{0.552156in}}%
\pgfpathlineto{\pgfqpoint{1.818026in}{0.544496in}}%
\pgfpathlineto{\pgfqpoint{1.832248in}{0.562571in}}%
\pgfpathlineto{\pgfqpoint{1.848979in}{0.567735in}}%
\pgfpathlineto{\pgfqpoint{1.866427in}{0.569074in}}%
\pgfpathlineto{\pgfqpoint{1.884060in}{0.567348in}}%
\pgfpathlineto{\pgfqpoint{1.901454in}{0.575130in}}%
\pgfpathlineto{\pgfqpoint{1.919929in}{0.539702in}}%
\pgfpathlineto{\pgfqpoint{1.933485in}{0.567161in}}%
\pgfpathlineto{\pgfqpoint{1.950853in}{0.607271in}}%
\pgfpathlineto{\pgfqpoint{1.973789in}{0.629249in}}%
\pgfpathlineto{\pgfqpoint{1.999776in}{0.642838in}}%
\pgfpathlineto{\pgfqpoint{2.027650in}{0.654021in}}%
\pgfpathlineto{\pgfqpoint{2.057076in}{0.666478in}}%
\pgfpathlineto{\pgfqpoint{2.088231in}{0.678813in}}%
\pgfpathlineto{\pgfqpoint{2.121099in}{0.692292in}}%
\pgfpathlineto{\pgfqpoint{2.155838in}{0.707623in}}%
\pgfpathlineto{\pgfqpoint{2.192705in}{0.724725in}}%
\pgfpathlineto{\pgfqpoint{2.231946in}{0.743768in}}%
\pgfpathlineto{\pgfqpoint{2.273831in}{0.763620in}}%
\pgfpathlineto{\pgfqpoint{2.318472in}{0.784917in}}%
\pgfpathlineto{\pgfqpoint{2.366069in}{0.806991in}}%
\pgfpathlineto{\pgfqpoint{2.416730in}{0.831207in}}%
\pgfpathlineto{\pgfqpoint{2.470753in}{0.856738in}}%
\pgfpathlineto{\pgfqpoint{2.528321in}{0.881642in}}%
\pgfpathlineto{\pgfqpoint{2.589345in}{0.905084in}}%
\pgfpathlineto{\pgfqpoint{2.653624in}{0.928969in}}%
\pgfpathlineto{\pgfqpoint{2.721219in}{0.892578in}}%
\pgfusepath{stroke}%
\end{pgfscope}%
\begin{pgfscope}%
\pgfsetrectcap%
\pgfsetmiterjoin%
\pgfsetlinewidth{0.803000pt}%
\definecolor{currentstroke}{rgb}{0.000000,0.000000,0.000000}%
\pgfsetstrokecolor{currentstroke}%
\pgfsetdash{}{0pt}%
\pgfpathmoveto{\pgfqpoint{0.629216in}{0.484854in}}%
\pgfpathlineto{\pgfqpoint{0.629216in}{1.472354in}}%
\pgfusepath{stroke}%
\end{pgfscope}%
\begin{pgfscope}%
\pgfsetrectcap%
\pgfsetmiterjoin%
\pgfsetlinewidth{0.803000pt}%
\definecolor{currentstroke}{rgb}{0.000000,0.000000,0.000000}%
\pgfsetstrokecolor{currentstroke}%
\pgfsetdash{}{0pt}%
\pgfpathmoveto{\pgfqpoint{2.886359in}{0.484854in}}%
\pgfpathlineto{\pgfqpoint{2.886359in}{1.472354in}}%
\pgfusepath{stroke}%
\end{pgfscope}%
\begin{pgfscope}%
\pgfsetrectcap%
\pgfsetmiterjoin%
\pgfsetlinewidth{0.803000pt}%
\definecolor{currentstroke}{rgb}{0.000000,0.000000,0.000000}%
\pgfsetstrokecolor{currentstroke}%
\pgfsetdash{}{0pt}%
\pgfpathmoveto{\pgfqpoint{0.629216in}{0.484854in}}%
\pgfpathlineto{\pgfqpoint{2.886359in}{0.484854in}}%
\pgfusepath{stroke}%
\end{pgfscope}%
\begin{pgfscope}%
\pgfsetrectcap%
\pgfsetmiterjoin%
\pgfsetlinewidth{0.803000pt}%
\definecolor{currentstroke}{rgb}{0.000000,0.000000,0.000000}%
\pgfsetstrokecolor{currentstroke}%
\pgfsetdash{}{0pt}%
\pgfpathmoveto{\pgfqpoint{0.629216in}{1.472354in}}%
\pgfpathlineto{\pgfqpoint{2.886359in}{1.472354in}}%
\pgfusepath{stroke}%
\end{pgfscope}%
\end{pgfpicture}%
\makeatother%
\endgroup%

    \caption{Caption}
    \label{fig:my_label}
\end{figure}
