\begin{figure}
    \centering
    \begin{tabular}{c|c|c}
        \textit{Time} (\si{\year}) & \textit{Particle property plugin} & \textit{Material model plugin} \\
        \hline
        \num{677115} & \parbox[c][75pt]{0.38\textwidth}{
            \includegraphics[
                trim={0 180pt 0 180pt}, 
                clip, 
                width=0.38\textwidth
            ]{figures/closed_box/particle_property/677115}
        } & \parbox[c][75pt]{0.38\textwidth}{
            \includegraphics[
                trim={0 180pt 0 180pt}, 
                clip, 
                width=0.38\textwidth
            ]{figures/closed_box/material_model/677115}
        } \\
        \hline
        \num{800992} & \parbox[c][75pt]{0.38\textwidth}{
            \includegraphics[
                trim={0 180pt 0 180pt}, 
                clip, 
                width=0.38\textwidth
            ]{figures/closed_box/particle_property/800992}
        } & \parbox[c][75pt]{0.38\textwidth}{
            \includegraphics[
                trim={0 180pt 0 180pt}, 
                clip, 
                width=0.38\textwidth
            ]{figures/closed_box/material_model/800992}
        } \\
        \hline
        \num{958796} & \parbox[c][75pt]{0.38\textwidth}{
            \includegraphics[
                trim={0 180pt 0 180pt}, 
                clip, 
                width=0.38\textwidth
            ]{figures/closed_box/particle_property/958796}
        } & \parbox[c][75pt]{0.38\textwidth}{
            \includegraphics[
                trim={0 180pt 0 180pt}, 
                clip, 
                width=0.38\textwidth
            ]{figures/closed_box/material_model/958796}
        } \\
        \hline
        \num{1.0513e+06} & \parbox[c][75pt]{0.38\textwidth}{
            \includegraphics[
                trim={0 180pt 0 180pt}, 
                clip, 
                width=0.38\textwidth
            ]{figures/closed_box/particle_property/1.0513e+06}
        } & \parbox[c][75pt]{0.38\textwidth}{
            \includegraphics[
                trim={0 180pt 0 180pt}, 
                clip, 
                width=0.38\textwidth
            ]{figures/closed_box/material_model/1.0513e+06}
        } \\
        \hline
        \num{1.28082e+06} & \parbox[c][75pt]{0.38\textwidth}{
            \includegraphics[
                trim={0 180pt 0 180pt}, 
                clip, 
                width=0.38\textwidth
            ]{figures/closed_box/particle_property/1.28082e+06}
        } & \parbox[c][75pt]{0.38\textwidth}{
            \includegraphics[
                trim={0 180pt 0 180pt}, 
                clip, 
                width=0.38\textwidth
            ]{figures/closed_box/material_model/1.28082e+06}
        }
    \end{tabular}
    \caption{
        Example runs of the closed box model setup with both plugin types.
        \num{2000} particles were used when using the particle property plugin and none were used in the material model one.
        The temperature is represented by the red to blue colour map and the melt is yellow and white.
        Whilst the simulation was run for 3 million years, only the period of \num{600000} years is shown because it shows the principal behaviour.
        Before this the velocities of the particle property plugin are too low to show any dispersion behaviour, and at later times the plumes disperse and the melt disappears.
    }
    \label{fig:closed_box}
\end{figure}

\begin{figure}
    \centering
    \begin{subfigure}{0.49\textwidth}
        \centering
        %% Creator: Matplotlib, PGF backend
%%
%% To include the figure in your LaTeX document, write
%%   \input{<filename>.pgf}
%%
%% Make sure the required packages are loaded in your preamble
%%   \usepackage{pgf}
%%
%% Figures using additional raster images can only be included by \input if
%% they are in the same directory as the main LaTeX file. For loading figures
%% from other directories you can use the `import` package
%%   \usepackage{import}
%% and then include the figures with
%%   \import{<path to file>}{<filename>.pgf}
%%
%% Matplotlib used the following preamble
%%   \usepackage{fontspec}
%%   \setmainfont{DejaVuSerif.ttf}[Path=/home/connor/.local/lib/python3.8/site-packages/matplotlib/mpl-data/fonts/ttf/]
%%   \setsansfont{DejaVuSans.ttf}[Path=/home/connor/.local/lib/python3.8/site-packages/matplotlib/mpl-data/fonts/ttf/]
%%   \setmonofont{DejaVuSansMono.ttf}[Path=/home/connor/.local/lib/python3.8/site-packages/matplotlib/mpl-data/fonts/ttf/]
%%
\begingroup%
\makeatletter%
\begin{pgfpicture}%
\pgfpathrectangle{\pgfpointorigin}{\pgfqpoint{2.853515in}{2.214229in}}%
\pgfusepath{use as bounding box, clip}%
\begin{pgfscope}%
\pgfsetbuttcap%
\pgfsetmiterjoin%
\definecolor{currentfill}{rgb}{1.000000,1.000000,1.000000}%
\pgfsetfillcolor{currentfill}%
\pgfsetlinewidth{0.000000pt}%
\definecolor{currentstroke}{rgb}{1.000000,1.000000,1.000000}%
\pgfsetstrokecolor{currentstroke}%
\pgfsetdash{}{0pt}%
\pgfpathmoveto{\pgfqpoint{0.000000in}{0.000000in}}%
\pgfpathlineto{\pgfqpoint{2.853515in}{0.000000in}}%
\pgfpathlineto{\pgfqpoint{2.853515in}{2.214229in}}%
\pgfpathlineto{\pgfqpoint{0.000000in}{2.214229in}}%
\pgfpathclose%
\pgfusepath{fill}%
\end{pgfscope}%
\begin{pgfscope}%
\pgfsetbuttcap%
\pgfsetmiterjoin%
\definecolor{currentfill}{rgb}{1.000000,1.000000,1.000000}%
\pgfsetfillcolor{currentfill}%
\pgfsetlinewidth{0.000000pt}%
\definecolor{currentstroke}{rgb}{0.000000,0.000000,0.000000}%
\pgfsetstrokecolor{currentstroke}%
\pgfsetstrokeopacity{0.000000}%
\pgfsetdash{}{0pt}%
\pgfpathmoveto{\pgfqpoint{0.478365in}{0.484854in}}%
\pgfpathlineto{\pgfqpoint{2.664972in}{0.484854in}}%
\pgfpathlineto{\pgfqpoint{2.664972in}{2.114229in}}%
\pgfpathlineto{\pgfqpoint{0.478365in}{2.114229in}}%
\pgfpathclose%
\pgfusepath{fill}%
\end{pgfscope}%
\begin{pgfscope}%
\pgfpathrectangle{\pgfqpoint{0.478365in}{0.484854in}}{\pgfqpoint{2.186607in}{1.629375in}}%
\pgfusepath{clip}%
\pgfsetbuttcap%
\pgfsetroundjoin%
\definecolor{currentfill}{rgb}{0.121569,0.466667,0.705882}%
\pgfsetfillcolor{currentfill}%
\pgfsetlinewidth{0.000000pt}%
\definecolor{currentstroke}{rgb}{0.000000,0.000000,0.000000}%
\pgfsetstrokecolor{currentstroke}%
\pgfsetdash{}{0pt}%
\pgfpathmoveto{\pgfqpoint{-0.979373in}{0.484854in}}%
\pgfpathlineto{\pgfqpoint{-0.979373in}{0.484854in}}%
\pgfpathlineto{\pgfqpoint{-0.933818in}{0.484854in}}%
\pgfpathlineto{\pgfqpoint{-0.897375in}{0.484854in}}%
\pgfpathlineto{\pgfqpoint{-0.860931in}{0.484854in}}%
\pgfpathlineto{\pgfqpoint{-0.824488in}{0.484854in}}%
\pgfpathlineto{\pgfqpoint{-0.788045in}{0.484854in}}%
\pgfpathlineto{\pgfqpoint{-0.751601in}{0.484854in}}%
\pgfpathlineto{\pgfqpoint{-0.715158in}{0.484854in}}%
\pgfpathlineto{\pgfqpoint{-0.678714in}{0.484854in}}%
\pgfpathlineto{\pgfqpoint{-0.642271in}{0.484854in}}%
\pgfpathlineto{\pgfqpoint{-0.605827in}{0.484854in}}%
\pgfpathlineto{\pgfqpoint{-0.569384in}{0.484854in}}%
\pgfpathlineto{\pgfqpoint{-0.532940in}{0.484854in}}%
\pgfpathlineto{\pgfqpoint{-0.496497in}{0.484854in}}%
\pgfpathlineto{\pgfqpoint{-0.460054in}{0.484854in}}%
\pgfpathlineto{\pgfqpoint{-0.423610in}{0.484854in}}%
\pgfpathlineto{\pgfqpoint{-0.387167in}{0.484854in}}%
\pgfpathlineto{\pgfqpoint{-0.350723in}{0.484854in}}%
\pgfpathlineto{\pgfqpoint{-0.314280in}{0.484854in}}%
\pgfpathlineto{\pgfqpoint{-0.277836in}{0.484854in}}%
\pgfpathlineto{\pgfqpoint{-0.241393in}{0.484854in}}%
\pgfpathlineto{\pgfqpoint{-0.204949in}{0.484854in}}%
\pgfpathlineto{\pgfqpoint{-0.168506in}{0.484854in}}%
\pgfpathlineto{\pgfqpoint{-0.132062in}{0.484854in}}%
\pgfpathlineto{\pgfqpoint{-0.095619in}{0.484854in}}%
\pgfpathlineto{\pgfqpoint{-0.059176in}{0.484854in}}%
\pgfpathlineto{\pgfqpoint{-0.022732in}{0.484854in}}%
\pgfpathlineto{\pgfqpoint{0.013711in}{0.484854in}}%
\pgfpathlineto{\pgfqpoint{0.050155in}{0.484854in}}%
\pgfpathlineto{\pgfqpoint{0.086598in}{0.484854in}}%
\pgfpathlineto{\pgfqpoint{0.123042in}{0.484854in}}%
\pgfpathlineto{\pgfqpoint{0.159485in}{0.484854in}}%
\pgfpathlineto{\pgfqpoint{0.195929in}{0.484854in}}%
\pgfpathlineto{\pgfqpoint{0.232372in}{0.484854in}}%
\pgfpathlineto{\pgfqpoint{0.268815in}{0.484854in}}%
\pgfpathlineto{\pgfqpoint{0.305259in}{0.484854in}}%
\pgfpathlineto{\pgfqpoint{0.341702in}{0.484854in}}%
\pgfpathlineto{\pgfqpoint{0.378146in}{0.484854in}}%
\pgfpathlineto{\pgfqpoint{0.414589in}{0.484854in}}%
\pgfpathlineto{\pgfqpoint{0.451033in}{0.484854in}}%
\pgfpathlineto{\pgfqpoint{0.487476in}{0.484854in}}%
\pgfpathlineto{\pgfqpoint{0.523920in}{0.484854in}}%
\pgfpathlineto{\pgfqpoint{0.560363in}{0.484854in}}%
\pgfpathlineto{\pgfqpoint{0.596806in}{0.484854in}}%
\pgfpathlineto{\pgfqpoint{0.633250in}{0.484854in}}%
\pgfpathlineto{\pgfqpoint{0.669693in}{0.484854in}}%
\pgfpathlineto{\pgfqpoint{0.706137in}{0.484854in}}%
\pgfpathlineto{\pgfqpoint{0.742580in}{0.484854in}}%
\pgfpathlineto{\pgfqpoint{0.779024in}{0.484854in}}%
\pgfpathlineto{\pgfqpoint{0.815467in}{0.484854in}}%
\pgfpathlineto{\pgfqpoint{0.851911in}{0.484854in}}%
\pgfpathlineto{\pgfqpoint{0.888354in}{0.484854in}}%
\pgfpathlineto{\pgfqpoint{0.924798in}{0.484854in}}%
\pgfpathlineto{\pgfqpoint{0.961241in}{0.484854in}}%
\pgfpathlineto{\pgfqpoint{0.997684in}{0.484854in}}%
\pgfpathlineto{\pgfqpoint{1.034128in}{0.484854in}}%
\pgfpathlineto{\pgfqpoint{1.070571in}{0.484854in}}%
\pgfpathlineto{\pgfqpoint{1.107015in}{0.484854in}}%
\pgfpathlineto{\pgfqpoint{1.143458in}{0.484854in}}%
\pgfpathlineto{\pgfqpoint{1.179902in}{0.484854in}}%
\pgfpathlineto{\pgfqpoint{1.216345in}{0.484854in}}%
\pgfpathlineto{\pgfqpoint{1.252789in}{0.484854in}}%
\pgfpathlineto{\pgfqpoint{1.289232in}{0.484854in}}%
\pgfpathlineto{\pgfqpoint{1.325675in}{0.484854in}}%
\pgfpathlineto{\pgfqpoint{1.362119in}{0.484854in}}%
\pgfpathlineto{\pgfqpoint{1.398562in}{0.484854in}}%
\pgfpathlineto{\pgfqpoint{1.435006in}{0.484854in}}%
\pgfpathlineto{\pgfqpoint{1.471449in}{0.484854in}}%
\pgfpathlineto{\pgfqpoint{1.507893in}{0.484854in}}%
\pgfpathlineto{\pgfqpoint{1.544336in}{0.484854in}}%
\pgfpathlineto{\pgfqpoint{1.580780in}{0.484854in}}%
\pgfpathlineto{\pgfqpoint{1.617223in}{0.484854in}}%
\pgfpathlineto{\pgfqpoint{1.653666in}{0.484854in}}%
\pgfpathlineto{\pgfqpoint{1.690110in}{0.484854in}}%
\pgfpathlineto{\pgfqpoint{1.726553in}{0.484854in}}%
\pgfpathlineto{\pgfqpoint{1.762997in}{0.484854in}}%
\pgfpathlineto{\pgfqpoint{1.799440in}{0.484854in}}%
\pgfpathlineto{\pgfqpoint{1.835884in}{0.484854in}}%
\pgfpathlineto{\pgfqpoint{1.872327in}{0.484854in}}%
\pgfpathlineto{\pgfqpoint{1.908771in}{0.484854in}}%
\pgfpathlineto{\pgfqpoint{1.945214in}{0.484854in}}%
\pgfpathlineto{\pgfqpoint{1.981658in}{0.484854in}}%
\pgfpathlineto{\pgfqpoint{2.018101in}{0.484854in}}%
\pgfpathlineto{\pgfqpoint{2.054544in}{0.484854in}}%
\pgfpathlineto{\pgfqpoint{2.090988in}{0.484854in}}%
\pgfpathlineto{\pgfqpoint{2.127431in}{0.484854in}}%
\pgfpathlineto{\pgfqpoint{2.163875in}{0.484854in}}%
\pgfpathlineto{\pgfqpoint{2.200318in}{0.484854in}}%
\pgfpathlineto{\pgfqpoint{2.236762in}{0.484854in}}%
\pgfpathlineto{\pgfqpoint{2.273205in}{0.484854in}}%
\pgfpathlineto{\pgfqpoint{2.309649in}{0.484854in}}%
\pgfpathlineto{\pgfqpoint{2.346092in}{0.484854in}}%
\pgfpathlineto{\pgfqpoint{2.382535in}{0.484854in}}%
\pgfpathlineto{\pgfqpoint{2.418979in}{0.484854in}}%
\pgfpathlineto{\pgfqpoint{2.455422in}{0.484854in}}%
\pgfpathlineto{\pgfqpoint{2.491866in}{0.484854in}}%
\pgfpathlineto{\pgfqpoint{2.528309in}{0.484854in}}%
\pgfpathlineto{\pgfqpoint{2.564753in}{0.484854in}}%
\pgfpathlineto{\pgfqpoint{2.601196in}{0.484854in}}%
\pgfpathlineto{\pgfqpoint{2.637640in}{0.484854in}}%
\pgfpathlineto{\pgfqpoint{2.664972in}{0.484854in}}%
\pgfpathlineto{\pgfqpoint{2.664972in}{0.636214in}}%
\pgfpathlineto{\pgfqpoint{2.664972in}{0.636214in}}%
\pgfpathlineto{\pgfqpoint{2.637640in}{0.636153in}}%
\pgfpathlineto{\pgfqpoint{2.601196in}{0.636075in}}%
\pgfpathlineto{\pgfqpoint{2.564753in}{0.636003in}}%
\pgfpathlineto{\pgfqpoint{2.528309in}{0.635925in}}%
\pgfpathlineto{\pgfqpoint{2.491866in}{0.636197in}}%
\pgfpathlineto{\pgfqpoint{2.455422in}{0.636119in}}%
\pgfpathlineto{\pgfqpoint{2.418979in}{0.636045in}}%
\pgfpathlineto{\pgfqpoint{2.382535in}{0.635354in}}%
\pgfpathlineto{\pgfqpoint{2.346092in}{0.635296in}}%
\pgfpathlineto{\pgfqpoint{2.309649in}{0.635181in}}%
\pgfpathlineto{\pgfqpoint{2.273205in}{0.635087in}}%
\pgfpathlineto{\pgfqpoint{2.236762in}{0.634250in}}%
\pgfpathlineto{\pgfqpoint{2.200318in}{0.632029in}}%
\pgfpathlineto{\pgfqpoint{2.163875in}{0.628624in}}%
\pgfpathlineto{\pgfqpoint{2.127431in}{0.627851in}}%
\pgfpathlineto{\pgfqpoint{2.090988in}{0.625985in}}%
\pgfpathlineto{\pgfqpoint{2.054544in}{0.625046in}}%
\pgfpathlineto{\pgfqpoint{2.018101in}{0.623343in}}%
\pgfpathlineto{\pgfqpoint{1.981658in}{0.620584in}}%
\pgfpathlineto{\pgfqpoint{1.945214in}{0.618823in}}%
\pgfpathlineto{\pgfqpoint{1.908771in}{0.616853in}}%
\pgfpathlineto{\pgfqpoint{1.872327in}{0.614922in}}%
\pgfpathlineto{\pgfqpoint{1.835884in}{0.611718in}}%
\pgfpathlineto{\pgfqpoint{1.799440in}{0.609729in}}%
\pgfpathlineto{\pgfqpoint{1.762997in}{0.607249in}}%
\pgfpathlineto{\pgfqpoint{1.726553in}{0.601246in}}%
\pgfpathlineto{\pgfqpoint{1.690110in}{0.599151in}}%
\pgfpathlineto{\pgfqpoint{1.653666in}{0.596882in}}%
\pgfpathlineto{\pgfqpoint{1.617223in}{0.593614in}}%
\pgfpathlineto{\pgfqpoint{1.580780in}{0.591104in}}%
\pgfpathlineto{\pgfqpoint{1.544336in}{0.588385in}}%
\pgfpathlineto{\pgfqpoint{1.507893in}{0.585337in}}%
\pgfpathlineto{\pgfqpoint{1.471449in}{0.580960in}}%
\pgfpathlineto{\pgfqpoint{1.435006in}{0.569059in}}%
\pgfpathlineto{\pgfqpoint{1.398562in}{0.567746in}}%
\pgfpathlineto{\pgfqpoint{1.362119in}{0.567143in}}%
\pgfpathlineto{\pgfqpoint{1.325675in}{0.563374in}}%
\pgfpathlineto{\pgfqpoint{1.289232in}{0.558673in}}%
\pgfpathlineto{\pgfqpoint{1.252789in}{0.553179in}}%
\pgfpathlineto{\pgfqpoint{1.216345in}{0.546783in}}%
\pgfpathlineto{\pgfqpoint{1.179902in}{0.541475in}}%
\pgfpathlineto{\pgfqpoint{1.143458in}{0.534812in}}%
\pgfpathlineto{\pgfqpoint{1.107015in}{0.528068in}}%
\pgfpathlineto{\pgfqpoint{1.070571in}{0.520275in}}%
\pgfpathlineto{\pgfqpoint{1.034128in}{0.511484in}}%
\pgfpathlineto{\pgfqpoint{0.997684in}{0.502707in}}%
\pgfpathlineto{\pgfqpoint{0.961241in}{0.492974in}}%
\pgfpathlineto{\pgfqpoint{0.924798in}{0.484854in}}%
\pgfpathlineto{\pgfqpoint{0.888354in}{0.484854in}}%
\pgfpathlineto{\pgfqpoint{0.851911in}{0.484854in}}%
\pgfpathlineto{\pgfqpoint{0.815467in}{0.484854in}}%
\pgfpathlineto{\pgfqpoint{0.779024in}{0.484854in}}%
\pgfpathlineto{\pgfqpoint{0.742580in}{0.484854in}}%
\pgfpathlineto{\pgfqpoint{0.706137in}{0.484854in}}%
\pgfpathlineto{\pgfqpoint{0.669693in}{0.484854in}}%
\pgfpathlineto{\pgfqpoint{0.633250in}{0.484854in}}%
\pgfpathlineto{\pgfqpoint{0.596806in}{0.484854in}}%
\pgfpathlineto{\pgfqpoint{0.560363in}{0.484854in}}%
\pgfpathlineto{\pgfqpoint{0.523920in}{0.484854in}}%
\pgfpathlineto{\pgfqpoint{0.487476in}{0.484854in}}%
\pgfpathlineto{\pgfqpoint{0.451033in}{0.484854in}}%
\pgfpathlineto{\pgfqpoint{0.414589in}{0.484854in}}%
\pgfpathlineto{\pgfqpoint{0.378146in}{0.484854in}}%
\pgfpathlineto{\pgfqpoint{0.341702in}{0.484854in}}%
\pgfpathlineto{\pgfqpoint{0.305259in}{0.484854in}}%
\pgfpathlineto{\pgfqpoint{0.268815in}{0.484854in}}%
\pgfpathlineto{\pgfqpoint{0.232372in}{0.484854in}}%
\pgfpathlineto{\pgfqpoint{0.195929in}{0.484854in}}%
\pgfpathlineto{\pgfqpoint{0.159485in}{0.484854in}}%
\pgfpathlineto{\pgfqpoint{0.123042in}{0.484854in}}%
\pgfpathlineto{\pgfqpoint{0.086598in}{0.484854in}}%
\pgfpathlineto{\pgfqpoint{0.050155in}{0.484854in}}%
\pgfpathlineto{\pgfqpoint{0.013711in}{0.484854in}}%
\pgfpathlineto{\pgfqpoint{-0.022732in}{0.484854in}}%
\pgfpathlineto{\pgfqpoint{-0.059176in}{0.484854in}}%
\pgfpathlineto{\pgfqpoint{-0.095619in}{0.484854in}}%
\pgfpathlineto{\pgfqpoint{-0.132062in}{0.484854in}}%
\pgfpathlineto{\pgfqpoint{-0.168506in}{0.484854in}}%
\pgfpathlineto{\pgfqpoint{-0.204949in}{0.484854in}}%
\pgfpathlineto{\pgfqpoint{-0.241393in}{0.484854in}}%
\pgfpathlineto{\pgfqpoint{-0.277836in}{0.484854in}}%
\pgfpathlineto{\pgfqpoint{-0.314280in}{0.484854in}}%
\pgfpathlineto{\pgfqpoint{-0.350723in}{0.484854in}}%
\pgfpathlineto{\pgfqpoint{-0.387167in}{0.484854in}}%
\pgfpathlineto{\pgfqpoint{-0.423610in}{0.484854in}}%
\pgfpathlineto{\pgfqpoint{-0.460054in}{0.484854in}}%
\pgfpathlineto{\pgfqpoint{-0.496497in}{0.484854in}}%
\pgfpathlineto{\pgfqpoint{-0.532940in}{0.484854in}}%
\pgfpathlineto{\pgfqpoint{-0.569384in}{0.484854in}}%
\pgfpathlineto{\pgfqpoint{-0.605827in}{0.484854in}}%
\pgfpathlineto{\pgfqpoint{-0.642271in}{0.484854in}}%
\pgfpathlineto{\pgfqpoint{-0.678714in}{0.484854in}}%
\pgfpathlineto{\pgfqpoint{-0.715158in}{0.484854in}}%
\pgfpathlineto{\pgfqpoint{-0.751601in}{0.484854in}}%
\pgfpathlineto{\pgfqpoint{-0.788045in}{0.484854in}}%
\pgfpathlineto{\pgfqpoint{-0.824488in}{0.484854in}}%
\pgfpathlineto{\pgfqpoint{-0.860931in}{0.484854in}}%
\pgfpathlineto{\pgfqpoint{-0.897375in}{0.484854in}}%
\pgfpathlineto{\pgfqpoint{-0.933818in}{0.484854in}}%
\pgfpathlineto{\pgfqpoint{-0.979373in}{0.484854in}}%
\pgfpathclose%
\pgfusepath{fill}%
\end{pgfscope}%
\begin{pgfscope}%
\pgfpathrectangle{\pgfqpoint{0.478365in}{0.484854in}}{\pgfqpoint{2.186607in}{1.629375in}}%
\pgfusepath{clip}%
\pgfsetbuttcap%
\pgfsetroundjoin%
\definecolor{currentfill}{rgb}{1.000000,0.498039,0.054902}%
\pgfsetfillcolor{currentfill}%
\pgfsetlinewidth{0.000000pt}%
\definecolor{currentstroke}{rgb}{0.000000,0.000000,0.000000}%
\pgfsetstrokecolor{currentstroke}%
\pgfsetdash{}{0pt}%
\pgfpathmoveto{\pgfqpoint{-0.979373in}{0.484854in}}%
\pgfpathlineto{\pgfqpoint{-0.979373in}{0.484854in}}%
\pgfpathlineto{\pgfqpoint{-0.933818in}{0.484854in}}%
\pgfpathlineto{\pgfqpoint{-0.897375in}{0.484854in}}%
\pgfpathlineto{\pgfqpoint{-0.860931in}{0.484854in}}%
\pgfpathlineto{\pgfqpoint{-0.824488in}{0.484854in}}%
\pgfpathlineto{\pgfqpoint{-0.788045in}{0.484854in}}%
\pgfpathlineto{\pgfqpoint{-0.751601in}{0.484854in}}%
\pgfpathlineto{\pgfqpoint{-0.715158in}{0.484854in}}%
\pgfpathlineto{\pgfqpoint{-0.678714in}{0.484854in}}%
\pgfpathlineto{\pgfqpoint{-0.642271in}{0.484854in}}%
\pgfpathlineto{\pgfqpoint{-0.605827in}{0.484854in}}%
\pgfpathlineto{\pgfqpoint{-0.569384in}{0.484854in}}%
\pgfpathlineto{\pgfqpoint{-0.532940in}{0.484854in}}%
\pgfpathlineto{\pgfqpoint{-0.496497in}{0.484854in}}%
\pgfpathlineto{\pgfqpoint{-0.460054in}{0.484854in}}%
\pgfpathlineto{\pgfqpoint{-0.423610in}{0.484854in}}%
\pgfpathlineto{\pgfqpoint{-0.387167in}{0.484854in}}%
\pgfpathlineto{\pgfqpoint{-0.350723in}{0.484854in}}%
\pgfpathlineto{\pgfqpoint{-0.314280in}{0.484854in}}%
\pgfpathlineto{\pgfqpoint{-0.277836in}{0.484854in}}%
\pgfpathlineto{\pgfqpoint{-0.241393in}{0.484854in}}%
\pgfpathlineto{\pgfqpoint{-0.204949in}{0.484854in}}%
\pgfpathlineto{\pgfqpoint{-0.168506in}{0.484854in}}%
\pgfpathlineto{\pgfqpoint{-0.132062in}{0.484854in}}%
\pgfpathlineto{\pgfqpoint{-0.095619in}{0.484854in}}%
\pgfpathlineto{\pgfqpoint{-0.059176in}{0.484854in}}%
\pgfpathlineto{\pgfqpoint{-0.022732in}{0.484854in}}%
\pgfpathlineto{\pgfqpoint{0.013711in}{0.484854in}}%
\pgfpathlineto{\pgfqpoint{0.050155in}{0.484854in}}%
\pgfpathlineto{\pgfqpoint{0.086598in}{0.484854in}}%
\pgfpathlineto{\pgfqpoint{0.123042in}{0.484854in}}%
\pgfpathlineto{\pgfqpoint{0.159485in}{0.484854in}}%
\pgfpathlineto{\pgfqpoint{0.195929in}{0.484854in}}%
\pgfpathlineto{\pgfqpoint{0.232372in}{0.484854in}}%
\pgfpathlineto{\pgfqpoint{0.268815in}{0.484854in}}%
\pgfpathlineto{\pgfqpoint{0.305259in}{0.484854in}}%
\pgfpathlineto{\pgfqpoint{0.341702in}{0.484854in}}%
\pgfpathlineto{\pgfqpoint{0.378146in}{0.484854in}}%
\pgfpathlineto{\pgfqpoint{0.414589in}{0.484854in}}%
\pgfpathlineto{\pgfqpoint{0.451033in}{0.484854in}}%
\pgfpathlineto{\pgfqpoint{0.487476in}{0.484854in}}%
\pgfpathlineto{\pgfqpoint{0.523920in}{0.484854in}}%
\pgfpathlineto{\pgfqpoint{0.560363in}{0.484854in}}%
\pgfpathlineto{\pgfqpoint{0.596806in}{0.484854in}}%
\pgfpathlineto{\pgfqpoint{0.633250in}{0.484854in}}%
\pgfpathlineto{\pgfqpoint{0.669693in}{0.484854in}}%
\pgfpathlineto{\pgfqpoint{0.706137in}{0.484854in}}%
\pgfpathlineto{\pgfqpoint{0.742580in}{0.484854in}}%
\pgfpathlineto{\pgfqpoint{0.779024in}{0.484854in}}%
\pgfpathlineto{\pgfqpoint{0.815467in}{0.484854in}}%
\pgfpathlineto{\pgfqpoint{0.851911in}{0.484854in}}%
\pgfpathlineto{\pgfqpoint{0.888354in}{0.484854in}}%
\pgfpathlineto{\pgfqpoint{0.924798in}{0.484854in}}%
\pgfpathlineto{\pgfqpoint{0.961241in}{0.492974in}}%
\pgfpathlineto{\pgfqpoint{0.997684in}{0.502707in}}%
\pgfpathlineto{\pgfqpoint{1.034128in}{0.511484in}}%
\pgfpathlineto{\pgfqpoint{1.070571in}{0.520275in}}%
\pgfpathlineto{\pgfqpoint{1.107015in}{0.528068in}}%
\pgfpathlineto{\pgfqpoint{1.143458in}{0.534812in}}%
\pgfpathlineto{\pgfqpoint{1.179902in}{0.541475in}}%
\pgfpathlineto{\pgfqpoint{1.216345in}{0.546783in}}%
\pgfpathlineto{\pgfqpoint{1.252789in}{0.553179in}}%
\pgfpathlineto{\pgfqpoint{1.289232in}{0.558673in}}%
\pgfpathlineto{\pgfqpoint{1.325675in}{0.563374in}}%
\pgfpathlineto{\pgfqpoint{1.362119in}{0.567143in}}%
\pgfpathlineto{\pgfqpoint{1.398562in}{0.567746in}}%
\pgfpathlineto{\pgfqpoint{1.435006in}{0.569059in}}%
\pgfpathlineto{\pgfqpoint{1.471449in}{0.580960in}}%
\pgfpathlineto{\pgfqpoint{1.507893in}{0.585337in}}%
\pgfpathlineto{\pgfqpoint{1.544336in}{0.588385in}}%
\pgfpathlineto{\pgfqpoint{1.580780in}{0.591104in}}%
\pgfpathlineto{\pgfqpoint{1.617223in}{0.593614in}}%
\pgfpathlineto{\pgfqpoint{1.653666in}{0.596882in}}%
\pgfpathlineto{\pgfqpoint{1.690110in}{0.599151in}}%
\pgfpathlineto{\pgfqpoint{1.726553in}{0.601246in}}%
\pgfpathlineto{\pgfqpoint{1.762997in}{0.607249in}}%
\pgfpathlineto{\pgfqpoint{1.799440in}{0.609729in}}%
\pgfpathlineto{\pgfqpoint{1.835884in}{0.611718in}}%
\pgfpathlineto{\pgfqpoint{1.872327in}{0.614922in}}%
\pgfpathlineto{\pgfqpoint{1.908771in}{0.616853in}}%
\pgfpathlineto{\pgfqpoint{1.945214in}{0.618823in}}%
\pgfpathlineto{\pgfqpoint{1.981658in}{0.620584in}}%
\pgfpathlineto{\pgfqpoint{2.018101in}{0.623343in}}%
\pgfpathlineto{\pgfqpoint{2.054544in}{0.625046in}}%
\pgfpathlineto{\pgfqpoint{2.090988in}{0.625985in}}%
\pgfpathlineto{\pgfqpoint{2.127431in}{0.627851in}}%
\pgfpathlineto{\pgfqpoint{2.163875in}{0.628624in}}%
\pgfpathlineto{\pgfqpoint{2.200318in}{0.632029in}}%
\pgfpathlineto{\pgfqpoint{2.236762in}{0.634250in}}%
\pgfpathlineto{\pgfqpoint{2.273205in}{0.635087in}}%
\pgfpathlineto{\pgfqpoint{2.309649in}{0.635181in}}%
\pgfpathlineto{\pgfqpoint{2.346092in}{0.635296in}}%
\pgfpathlineto{\pgfqpoint{2.382535in}{0.635354in}}%
\pgfpathlineto{\pgfqpoint{2.418979in}{0.636045in}}%
\pgfpathlineto{\pgfqpoint{2.455422in}{0.636119in}}%
\pgfpathlineto{\pgfqpoint{2.491866in}{0.636197in}}%
\pgfpathlineto{\pgfqpoint{2.528309in}{0.635925in}}%
\pgfpathlineto{\pgfqpoint{2.564753in}{0.636003in}}%
\pgfpathlineto{\pgfqpoint{2.601196in}{0.636075in}}%
\pgfpathlineto{\pgfqpoint{2.637640in}{0.636153in}}%
\pgfpathlineto{\pgfqpoint{2.664972in}{0.636214in}}%
\pgfpathlineto{\pgfqpoint{2.664972in}{0.759668in}}%
\pgfpathlineto{\pgfqpoint{2.664972in}{0.759668in}}%
\pgfpathlineto{\pgfqpoint{2.637640in}{0.759498in}}%
\pgfpathlineto{\pgfqpoint{2.601196in}{0.759357in}}%
\pgfpathlineto{\pgfqpoint{2.564753in}{0.758723in}}%
\pgfpathlineto{\pgfqpoint{2.528309in}{0.758581in}}%
\pgfpathlineto{\pgfqpoint{2.491866in}{0.756452in}}%
\pgfpathlineto{\pgfqpoint{2.455422in}{0.756311in}}%
\pgfpathlineto{\pgfqpoint{2.418979in}{0.756178in}}%
\pgfpathlineto{\pgfqpoint{2.382535in}{0.756983in}}%
\pgfpathlineto{\pgfqpoint{2.346092in}{0.755980in}}%
\pgfpathlineto{\pgfqpoint{2.309649in}{0.755891in}}%
\pgfpathlineto{\pgfqpoint{2.273205in}{0.754856in}}%
\pgfpathlineto{\pgfqpoint{2.236762in}{0.752957in}}%
\pgfpathlineto{\pgfqpoint{2.200318in}{0.748000in}}%
\pgfpathlineto{\pgfqpoint{2.163875in}{0.735387in}}%
\pgfpathlineto{\pgfqpoint{2.127431in}{0.734039in}}%
\pgfpathlineto{\pgfqpoint{2.090988in}{0.731667in}}%
\pgfpathlineto{\pgfqpoint{2.054544in}{0.730630in}}%
\pgfpathlineto{\pgfqpoint{2.018101in}{0.726462in}}%
\pgfpathlineto{\pgfqpoint{1.981658in}{0.720942in}}%
\pgfpathlineto{\pgfqpoint{1.945214in}{0.716412in}}%
\pgfpathlineto{\pgfqpoint{1.908771in}{0.712496in}}%
\pgfpathlineto{\pgfqpoint{1.872327in}{0.708828in}}%
\pgfpathlineto{\pgfqpoint{1.835884in}{0.702724in}}%
\pgfpathlineto{\pgfqpoint{1.799440in}{0.698293in}}%
\pgfpathlineto{\pgfqpoint{1.762997in}{0.693311in}}%
\pgfpathlineto{\pgfqpoint{1.726553in}{0.676923in}}%
\pgfpathlineto{\pgfqpoint{1.690110in}{0.674060in}}%
\pgfpathlineto{\pgfqpoint{1.653666in}{0.670888in}}%
\pgfpathlineto{\pgfqpoint{1.617223in}{0.666026in}}%
\pgfpathlineto{\pgfqpoint{1.580780in}{0.662398in}}%
\pgfpathlineto{\pgfqpoint{1.544336in}{0.658392in}}%
\pgfpathlineto{\pgfqpoint{1.507893in}{0.653541in}}%
\pgfpathlineto{\pgfqpoint{1.471449in}{0.645441in}}%
\pgfpathlineto{\pgfqpoint{1.435006in}{0.626052in}}%
\pgfpathlineto{\pgfqpoint{1.398562in}{0.622801in}}%
\pgfpathlineto{\pgfqpoint{1.362119in}{0.622071in}}%
\pgfpathlineto{\pgfqpoint{1.325675in}{0.615269in}}%
\pgfpathlineto{\pgfqpoint{1.289232in}{0.607464in}}%
\pgfpathlineto{\pgfqpoint{1.252789in}{0.598673in}}%
\pgfpathlineto{\pgfqpoint{1.216345in}{0.587621in}}%
\pgfpathlineto{\pgfqpoint{1.179902in}{0.578816in}}%
\pgfpathlineto{\pgfqpoint{1.143458in}{0.567762in}}%
\pgfpathlineto{\pgfqpoint{1.107015in}{0.556572in}}%
\pgfpathlineto{\pgfqpoint{1.070571in}{0.543834in}}%
\pgfpathlineto{\pgfqpoint{1.034128in}{0.529217in}}%
\pgfpathlineto{\pgfqpoint{0.997684in}{0.514692in}}%
\pgfpathlineto{\pgfqpoint{0.961241in}{0.498468in}}%
\pgfpathlineto{\pgfqpoint{0.924798in}{0.484854in}}%
\pgfpathlineto{\pgfqpoint{0.888354in}{0.484854in}}%
\pgfpathlineto{\pgfqpoint{0.851911in}{0.484854in}}%
\pgfpathlineto{\pgfqpoint{0.815467in}{0.484854in}}%
\pgfpathlineto{\pgfqpoint{0.779024in}{0.484854in}}%
\pgfpathlineto{\pgfqpoint{0.742580in}{0.484854in}}%
\pgfpathlineto{\pgfqpoint{0.706137in}{0.484854in}}%
\pgfpathlineto{\pgfqpoint{0.669693in}{0.484854in}}%
\pgfpathlineto{\pgfqpoint{0.633250in}{0.484854in}}%
\pgfpathlineto{\pgfqpoint{0.596806in}{0.484854in}}%
\pgfpathlineto{\pgfqpoint{0.560363in}{0.484854in}}%
\pgfpathlineto{\pgfqpoint{0.523920in}{0.484854in}}%
\pgfpathlineto{\pgfqpoint{0.487476in}{0.484854in}}%
\pgfpathlineto{\pgfqpoint{0.451033in}{0.484854in}}%
\pgfpathlineto{\pgfqpoint{0.414589in}{0.484854in}}%
\pgfpathlineto{\pgfqpoint{0.378146in}{0.484854in}}%
\pgfpathlineto{\pgfqpoint{0.341702in}{0.484854in}}%
\pgfpathlineto{\pgfqpoint{0.305259in}{0.484854in}}%
\pgfpathlineto{\pgfqpoint{0.268815in}{0.484854in}}%
\pgfpathlineto{\pgfqpoint{0.232372in}{0.484854in}}%
\pgfpathlineto{\pgfqpoint{0.195929in}{0.484854in}}%
\pgfpathlineto{\pgfqpoint{0.159485in}{0.484854in}}%
\pgfpathlineto{\pgfqpoint{0.123042in}{0.484854in}}%
\pgfpathlineto{\pgfqpoint{0.086598in}{0.484854in}}%
\pgfpathlineto{\pgfqpoint{0.050155in}{0.484854in}}%
\pgfpathlineto{\pgfqpoint{0.013711in}{0.484854in}}%
\pgfpathlineto{\pgfqpoint{-0.022732in}{0.484854in}}%
\pgfpathlineto{\pgfqpoint{-0.059176in}{0.484854in}}%
\pgfpathlineto{\pgfqpoint{-0.095619in}{0.484854in}}%
\pgfpathlineto{\pgfqpoint{-0.132062in}{0.484854in}}%
\pgfpathlineto{\pgfqpoint{-0.168506in}{0.484854in}}%
\pgfpathlineto{\pgfqpoint{-0.204949in}{0.484854in}}%
\pgfpathlineto{\pgfqpoint{-0.241393in}{0.484854in}}%
\pgfpathlineto{\pgfqpoint{-0.277836in}{0.484854in}}%
\pgfpathlineto{\pgfqpoint{-0.314280in}{0.484854in}}%
\pgfpathlineto{\pgfqpoint{-0.350723in}{0.484854in}}%
\pgfpathlineto{\pgfqpoint{-0.387167in}{0.484854in}}%
\pgfpathlineto{\pgfqpoint{-0.423610in}{0.484854in}}%
\pgfpathlineto{\pgfqpoint{-0.460054in}{0.484854in}}%
\pgfpathlineto{\pgfqpoint{-0.496497in}{0.484854in}}%
\pgfpathlineto{\pgfqpoint{-0.532940in}{0.484854in}}%
\pgfpathlineto{\pgfqpoint{-0.569384in}{0.484854in}}%
\pgfpathlineto{\pgfqpoint{-0.605827in}{0.484854in}}%
\pgfpathlineto{\pgfqpoint{-0.642271in}{0.484854in}}%
\pgfpathlineto{\pgfqpoint{-0.678714in}{0.484854in}}%
\pgfpathlineto{\pgfqpoint{-0.715158in}{0.484854in}}%
\pgfpathlineto{\pgfqpoint{-0.751601in}{0.484854in}}%
\pgfpathlineto{\pgfqpoint{-0.788045in}{0.484854in}}%
\pgfpathlineto{\pgfqpoint{-0.824488in}{0.484854in}}%
\pgfpathlineto{\pgfqpoint{-0.860931in}{0.484854in}}%
\pgfpathlineto{\pgfqpoint{-0.897375in}{0.484854in}}%
\pgfpathlineto{\pgfqpoint{-0.933818in}{0.484854in}}%
\pgfpathlineto{\pgfqpoint{-0.979373in}{0.484854in}}%
\pgfpathclose%
\pgfusepath{fill}%
\end{pgfscope}%
\begin{pgfscope}%
\pgfpathrectangle{\pgfqpoint{0.478365in}{0.484854in}}{\pgfqpoint{2.186607in}{1.629375in}}%
\pgfusepath{clip}%
\pgfsetbuttcap%
\pgfsetroundjoin%
\definecolor{currentfill}{rgb}{0.172549,0.627451,0.172549}%
\pgfsetfillcolor{currentfill}%
\pgfsetlinewidth{0.000000pt}%
\definecolor{currentstroke}{rgb}{0.000000,0.000000,0.000000}%
\pgfsetstrokecolor{currentstroke}%
\pgfsetdash{}{0pt}%
\pgfpathmoveto{\pgfqpoint{-0.979373in}{0.484854in}}%
\pgfpathlineto{\pgfqpoint{-0.979373in}{0.484854in}}%
\pgfpathlineto{\pgfqpoint{-0.933818in}{0.484854in}}%
\pgfpathlineto{\pgfqpoint{-0.897375in}{0.484854in}}%
\pgfpathlineto{\pgfqpoint{-0.860931in}{0.484854in}}%
\pgfpathlineto{\pgfqpoint{-0.824488in}{0.484854in}}%
\pgfpathlineto{\pgfqpoint{-0.788045in}{0.484854in}}%
\pgfpathlineto{\pgfqpoint{-0.751601in}{0.484854in}}%
\pgfpathlineto{\pgfqpoint{-0.715158in}{0.484854in}}%
\pgfpathlineto{\pgfqpoint{-0.678714in}{0.484854in}}%
\pgfpathlineto{\pgfqpoint{-0.642271in}{0.484854in}}%
\pgfpathlineto{\pgfqpoint{-0.605827in}{0.484854in}}%
\pgfpathlineto{\pgfqpoint{-0.569384in}{0.484854in}}%
\pgfpathlineto{\pgfqpoint{-0.532940in}{0.484854in}}%
\pgfpathlineto{\pgfqpoint{-0.496497in}{0.484854in}}%
\pgfpathlineto{\pgfqpoint{-0.460054in}{0.484854in}}%
\pgfpathlineto{\pgfqpoint{-0.423610in}{0.484854in}}%
\pgfpathlineto{\pgfqpoint{-0.387167in}{0.484854in}}%
\pgfpathlineto{\pgfqpoint{-0.350723in}{0.484854in}}%
\pgfpathlineto{\pgfqpoint{-0.314280in}{0.484854in}}%
\pgfpathlineto{\pgfqpoint{-0.277836in}{0.484854in}}%
\pgfpathlineto{\pgfqpoint{-0.241393in}{0.484854in}}%
\pgfpathlineto{\pgfqpoint{-0.204949in}{0.484854in}}%
\pgfpathlineto{\pgfqpoint{-0.168506in}{0.484854in}}%
\pgfpathlineto{\pgfqpoint{-0.132062in}{0.484854in}}%
\pgfpathlineto{\pgfqpoint{-0.095619in}{0.484854in}}%
\pgfpathlineto{\pgfqpoint{-0.059176in}{0.484854in}}%
\pgfpathlineto{\pgfqpoint{-0.022732in}{0.484854in}}%
\pgfpathlineto{\pgfqpoint{0.013711in}{0.484854in}}%
\pgfpathlineto{\pgfqpoint{0.050155in}{0.484854in}}%
\pgfpathlineto{\pgfqpoint{0.086598in}{0.484854in}}%
\pgfpathlineto{\pgfqpoint{0.123042in}{0.484854in}}%
\pgfpathlineto{\pgfqpoint{0.159485in}{0.484854in}}%
\pgfpathlineto{\pgfqpoint{0.195929in}{0.484854in}}%
\pgfpathlineto{\pgfqpoint{0.232372in}{0.484854in}}%
\pgfpathlineto{\pgfqpoint{0.268815in}{0.484854in}}%
\pgfpathlineto{\pgfqpoint{0.305259in}{0.484854in}}%
\pgfpathlineto{\pgfqpoint{0.341702in}{0.484854in}}%
\pgfpathlineto{\pgfqpoint{0.378146in}{0.484854in}}%
\pgfpathlineto{\pgfqpoint{0.414589in}{0.484854in}}%
\pgfpathlineto{\pgfqpoint{0.451033in}{0.484854in}}%
\pgfpathlineto{\pgfqpoint{0.487476in}{0.484854in}}%
\pgfpathlineto{\pgfqpoint{0.523920in}{0.484854in}}%
\pgfpathlineto{\pgfqpoint{0.560363in}{0.484854in}}%
\pgfpathlineto{\pgfqpoint{0.596806in}{0.484854in}}%
\pgfpathlineto{\pgfqpoint{0.633250in}{0.484854in}}%
\pgfpathlineto{\pgfqpoint{0.669693in}{0.484854in}}%
\pgfpathlineto{\pgfqpoint{0.706137in}{0.484854in}}%
\pgfpathlineto{\pgfqpoint{0.742580in}{0.484854in}}%
\pgfpathlineto{\pgfqpoint{0.779024in}{0.484854in}}%
\pgfpathlineto{\pgfqpoint{0.815467in}{0.484854in}}%
\pgfpathlineto{\pgfqpoint{0.851911in}{0.484854in}}%
\pgfpathlineto{\pgfqpoint{0.888354in}{0.484854in}}%
\pgfpathlineto{\pgfqpoint{0.924798in}{0.484854in}}%
\pgfpathlineto{\pgfqpoint{0.961241in}{0.498468in}}%
\pgfpathlineto{\pgfqpoint{0.997684in}{0.514692in}}%
\pgfpathlineto{\pgfqpoint{1.034128in}{0.529217in}}%
\pgfpathlineto{\pgfqpoint{1.070571in}{0.543834in}}%
\pgfpathlineto{\pgfqpoint{1.107015in}{0.556572in}}%
\pgfpathlineto{\pgfqpoint{1.143458in}{0.567762in}}%
\pgfpathlineto{\pgfqpoint{1.179902in}{0.578816in}}%
\pgfpathlineto{\pgfqpoint{1.216345in}{0.587621in}}%
\pgfpathlineto{\pgfqpoint{1.252789in}{0.598673in}}%
\pgfpathlineto{\pgfqpoint{1.289232in}{0.607464in}}%
\pgfpathlineto{\pgfqpoint{1.325675in}{0.615269in}}%
\pgfpathlineto{\pgfqpoint{1.362119in}{0.622071in}}%
\pgfpathlineto{\pgfqpoint{1.398562in}{0.622801in}}%
\pgfpathlineto{\pgfqpoint{1.435006in}{0.626052in}}%
\pgfpathlineto{\pgfqpoint{1.471449in}{0.645441in}}%
\pgfpathlineto{\pgfqpoint{1.507893in}{0.653541in}}%
\pgfpathlineto{\pgfqpoint{1.544336in}{0.658392in}}%
\pgfpathlineto{\pgfqpoint{1.580780in}{0.662398in}}%
\pgfpathlineto{\pgfqpoint{1.617223in}{0.666026in}}%
\pgfpathlineto{\pgfqpoint{1.653666in}{0.670888in}}%
\pgfpathlineto{\pgfqpoint{1.690110in}{0.674060in}}%
\pgfpathlineto{\pgfqpoint{1.726553in}{0.676923in}}%
\pgfpathlineto{\pgfqpoint{1.762997in}{0.693311in}}%
\pgfpathlineto{\pgfqpoint{1.799440in}{0.698293in}}%
\pgfpathlineto{\pgfqpoint{1.835884in}{0.702724in}}%
\pgfpathlineto{\pgfqpoint{1.872327in}{0.708828in}}%
\pgfpathlineto{\pgfqpoint{1.908771in}{0.712496in}}%
\pgfpathlineto{\pgfqpoint{1.945214in}{0.716412in}}%
\pgfpathlineto{\pgfqpoint{1.981658in}{0.720942in}}%
\pgfpathlineto{\pgfqpoint{2.018101in}{0.726462in}}%
\pgfpathlineto{\pgfqpoint{2.054544in}{0.730630in}}%
\pgfpathlineto{\pgfqpoint{2.090988in}{0.731667in}}%
\pgfpathlineto{\pgfqpoint{2.127431in}{0.734039in}}%
\pgfpathlineto{\pgfqpoint{2.163875in}{0.735387in}}%
\pgfpathlineto{\pgfqpoint{2.200318in}{0.748000in}}%
\pgfpathlineto{\pgfqpoint{2.236762in}{0.752957in}}%
\pgfpathlineto{\pgfqpoint{2.273205in}{0.754856in}}%
\pgfpathlineto{\pgfqpoint{2.309649in}{0.755891in}}%
\pgfpathlineto{\pgfqpoint{2.346092in}{0.755980in}}%
\pgfpathlineto{\pgfqpoint{2.382535in}{0.756983in}}%
\pgfpathlineto{\pgfqpoint{2.418979in}{0.756178in}}%
\pgfpathlineto{\pgfqpoint{2.455422in}{0.756311in}}%
\pgfpathlineto{\pgfqpoint{2.491866in}{0.756452in}}%
\pgfpathlineto{\pgfqpoint{2.528309in}{0.758581in}}%
\pgfpathlineto{\pgfqpoint{2.564753in}{0.758723in}}%
\pgfpathlineto{\pgfqpoint{2.601196in}{0.759357in}}%
\pgfpathlineto{\pgfqpoint{2.637640in}{0.759498in}}%
\pgfpathlineto{\pgfqpoint{2.664972in}{0.759668in}}%
\pgfpathlineto{\pgfqpoint{2.664972in}{1.188423in}}%
\pgfpathlineto{\pgfqpoint{2.664972in}{1.188423in}}%
\pgfpathlineto{\pgfqpoint{2.637640in}{1.188059in}}%
\pgfpathlineto{\pgfqpoint{2.601196in}{1.187697in}}%
\pgfpathlineto{\pgfqpoint{2.564753in}{1.186678in}}%
\pgfpathlineto{\pgfqpoint{2.528309in}{1.186316in}}%
\pgfpathlineto{\pgfqpoint{2.491866in}{1.180927in}}%
\pgfpathlineto{\pgfqpoint{2.455422in}{1.180566in}}%
\pgfpathlineto{\pgfqpoint{2.418979in}{1.180226in}}%
\pgfpathlineto{\pgfqpoint{2.382535in}{1.192628in}}%
\pgfpathlineto{\pgfqpoint{2.346092in}{1.191281in}}%
\pgfpathlineto{\pgfqpoint{2.309649in}{1.192313in}}%
\pgfpathlineto{\pgfqpoint{2.273205in}{1.192102in}}%
\pgfpathlineto{\pgfqpoint{2.236762in}{1.183153in}}%
\pgfpathlineto{\pgfqpoint{2.200318in}{1.163742in}}%
\pgfpathlineto{\pgfqpoint{2.163875in}{1.103219in}}%
\pgfpathlineto{\pgfqpoint{2.127431in}{1.102564in}}%
\pgfpathlineto{\pgfqpoint{2.090988in}{1.098021in}}%
\pgfpathlineto{\pgfqpoint{2.054544in}{1.097436in}}%
\pgfpathlineto{\pgfqpoint{2.018101in}{1.083533in}}%
\pgfpathlineto{\pgfqpoint{1.981658in}{1.064777in}}%
\pgfpathlineto{\pgfqpoint{1.945214in}{1.049960in}}%
\pgfpathlineto{\pgfqpoint{1.908771in}{1.035968in}}%
\pgfpathlineto{\pgfqpoint{1.872327in}{1.023099in}}%
\pgfpathlineto{\pgfqpoint{1.835884in}{1.004129in}}%
\pgfpathlineto{\pgfqpoint{1.799440in}{0.990904in}}%
\pgfpathlineto{\pgfqpoint{1.762997in}{0.974944in}}%
\pgfpathlineto{\pgfqpoint{1.726553in}{0.913031in}}%
\pgfpathlineto{\pgfqpoint{1.690110in}{0.907698in}}%
\pgfpathlineto{\pgfqpoint{1.653666in}{0.901053in}}%
\pgfpathlineto{\pgfqpoint{1.617223in}{0.890628in}}%
\pgfpathlineto{\pgfqpoint{1.580780in}{0.883472in}}%
\pgfpathlineto{\pgfqpoint{1.544336in}{0.874883in}}%
\pgfpathlineto{\pgfqpoint{1.507893in}{0.863362in}}%
\pgfpathlineto{\pgfqpoint{1.471449in}{0.842827in}}%
\pgfpathlineto{\pgfqpoint{1.435006in}{0.797467in}}%
\pgfpathlineto{\pgfqpoint{1.398562in}{0.786302in}}%
\pgfpathlineto{\pgfqpoint{1.362119in}{0.784811in}}%
\pgfpathlineto{\pgfqpoint{1.325675in}{0.768622in}}%
\pgfpathlineto{\pgfqpoint{1.289232in}{0.750216in}}%
\pgfpathlineto{\pgfqpoint{1.252789in}{0.729173in}}%
\pgfpathlineto{\pgfqpoint{1.216345in}{0.704172in}}%
\pgfpathlineto{\pgfqpoint{1.179902in}{0.684343in}}%
\pgfpathlineto{\pgfqpoint{1.143458in}{0.659972in}}%
\pgfpathlineto{\pgfqpoint{1.107015in}{0.635365in}}%
\pgfpathlineto{\pgfqpoint{1.070571in}{0.607873in}}%
\pgfpathlineto{\pgfqpoint{1.034128in}{0.576959in}}%
\pgfpathlineto{\pgfqpoint{0.997684in}{0.546348in}}%
\pgfpathlineto{\pgfqpoint{0.961241in}{0.512745in}}%
\pgfpathlineto{\pgfqpoint{0.924798in}{0.484854in}}%
\pgfpathlineto{\pgfqpoint{0.888354in}{0.484854in}}%
\pgfpathlineto{\pgfqpoint{0.851911in}{0.484854in}}%
\pgfpathlineto{\pgfqpoint{0.815467in}{0.484854in}}%
\pgfpathlineto{\pgfqpoint{0.779024in}{0.484854in}}%
\pgfpathlineto{\pgfqpoint{0.742580in}{0.484854in}}%
\pgfpathlineto{\pgfqpoint{0.706137in}{0.484854in}}%
\pgfpathlineto{\pgfqpoint{0.669693in}{0.484854in}}%
\pgfpathlineto{\pgfqpoint{0.633250in}{0.484854in}}%
\pgfpathlineto{\pgfqpoint{0.596806in}{0.484854in}}%
\pgfpathlineto{\pgfqpoint{0.560363in}{0.484854in}}%
\pgfpathlineto{\pgfqpoint{0.523920in}{0.484854in}}%
\pgfpathlineto{\pgfqpoint{0.487476in}{0.484854in}}%
\pgfpathlineto{\pgfqpoint{0.451033in}{0.484854in}}%
\pgfpathlineto{\pgfqpoint{0.414589in}{0.484854in}}%
\pgfpathlineto{\pgfqpoint{0.378146in}{0.484854in}}%
\pgfpathlineto{\pgfqpoint{0.341702in}{0.484854in}}%
\pgfpathlineto{\pgfqpoint{0.305259in}{0.484854in}}%
\pgfpathlineto{\pgfqpoint{0.268815in}{0.484854in}}%
\pgfpathlineto{\pgfqpoint{0.232372in}{0.484854in}}%
\pgfpathlineto{\pgfqpoint{0.195929in}{0.484854in}}%
\pgfpathlineto{\pgfqpoint{0.159485in}{0.484854in}}%
\pgfpathlineto{\pgfqpoint{0.123042in}{0.484854in}}%
\pgfpathlineto{\pgfqpoint{0.086598in}{0.484854in}}%
\pgfpathlineto{\pgfqpoint{0.050155in}{0.484854in}}%
\pgfpathlineto{\pgfqpoint{0.013711in}{0.484854in}}%
\pgfpathlineto{\pgfqpoint{-0.022732in}{0.484854in}}%
\pgfpathlineto{\pgfqpoint{-0.059176in}{0.484854in}}%
\pgfpathlineto{\pgfqpoint{-0.095619in}{0.484854in}}%
\pgfpathlineto{\pgfqpoint{-0.132062in}{0.484854in}}%
\pgfpathlineto{\pgfqpoint{-0.168506in}{0.484854in}}%
\pgfpathlineto{\pgfqpoint{-0.204949in}{0.484854in}}%
\pgfpathlineto{\pgfqpoint{-0.241393in}{0.484854in}}%
\pgfpathlineto{\pgfqpoint{-0.277836in}{0.484854in}}%
\pgfpathlineto{\pgfqpoint{-0.314280in}{0.484854in}}%
\pgfpathlineto{\pgfqpoint{-0.350723in}{0.484854in}}%
\pgfpathlineto{\pgfqpoint{-0.387167in}{0.484854in}}%
\pgfpathlineto{\pgfqpoint{-0.423610in}{0.484854in}}%
\pgfpathlineto{\pgfqpoint{-0.460054in}{0.484854in}}%
\pgfpathlineto{\pgfqpoint{-0.496497in}{0.484854in}}%
\pgfpathlineto{\pgfqpoint{-0.532940in}{0.484854in}}%
\pgfpathlineto{\pgfqpoint{-0.569384in}{0.484854in}}%
\pgfpathlineto{\pgfqpoint{-0.605827in}{0.484854in}}%
\pgfpathlineto{\pgfqpoint{-0.642271in}{0.484854in}}%
\pgfpathlineto{\pgfqpoint{-0.678714in}{0.484854in}}%
\pgfpathlineto{\pgfqpoint{-0.715158in}{0.484854in}}%
\pgfpathlineto{\pgfqpoint{-0.751601in}{0.484854in}}%
\pgfpathlineto{\pgfqpoint{-0.788045in}{0.484854in}}%
\pgfpathlineto{\pgfqpoint{-0.824488in}{0.484854in}}%
\pgfpathlineto{\pgfqpoint{-0.860931in}{0.484854in}}%
\pgfpathlineto{\pgfqpoint{-0.897375in}{0.484854in}}%
\pgfpathlineto{\pgfqpoint{-0.933818in}{0.484854in}}%
\pgfpathlineto{\pgfqpoint{-0.979373in}{0.484854in}}%
\pgfpathclose%
\pgfusepath{fill}%
\end{pgfscope}%
\begin{pgfscope}%
\pgfpathrectangle{\pgfqpoint{0.478365in}{0.484854in}}{\pgfqpoint{2.186607in}{1.629375in}}%
\pgfusepath{clip}%
\pgfsetbuttcap%
\pgfsetroundjoin%
\definecolor{currentfill}{rgb}{0.839216,0.152941,0.156863}%
\pgfsetfillcolor{currentfill}%
\pgfsetlinewidth{0.000000pt}%
\definecolor{currentstroke}{rgb}{0.000000,0.000000,0.000000}%
\pgfsetstrokecolor{currentstroke}%
\pgfsetdash{}{0pt}%
\pgfpathmoveto{\pgfqpoint{-0.979373in}{0.484854in}}%
\pgfpathlineto{\pgfqpoint{-0.979373in}{0.484854in}}%
\pgfpathlineto{\pgfqpoint{-0.933818in}{0.484854in}}%
\pgfpathlineto{\pgfqpoint{-0.897375in}{0.484854in}}%
\pgfpathlineto{\pgfqpoint{-0.860931in}{0.484854in}}%
\pgfpathlineto{\pgfqpoint{-0.824488in}{0.484854in}}%
\pgfpathlineto{\pgfqpoint{-0.788045in}{0.484854in}}%
\pgfpathlineto{\pgfqpoint{-0.751601in}{0.484854in}}%
\pgfpathlineto{\pgfqpoint{-0.715158in}{0.484854in}}%
\pgfpathlineto{\pgfqpoint{-0.678714in}{0.484854in}}%
\pgfpathlineto{\pgfqpoint{-0.642271in}{0.484854in}}%
\pgfpathlineto{\pgfqpoint{-0.605827in}{0.484854in}}%
\pgfpathlineto{\pgfqpoint{-0.569384in}{0.484854in}}%
\pgfpathlineto{\pgfqpoint{-0.532940in}{0.484854in}}%
\pgfpathlineto{\pgfqpoint{-0.496497in}{0.484854in}}%
\pgfpathlineto{\pgfqpoint{-0.460054in}{0.484854in}}%
\pgfpathlineto{\pgfqpoint{-0.423610in}{0.484854in}}%
\pgfpathlineto{\pgfqpoint{-0.387167in}{0.484854in}}%
\pgfpathlineto{\pgfqpoint{-0.350723in}{0.484854in}}%
\pgfpathlineto{\pgfqpoint{-0.314280in}{0.484854in}}%
\pgfpathlineto{\pgfqpoint{-0.277836in}{0.484854in}}%
\pgfpathlineto{\pgfqpoint{-0.241393in}{0.484854in}}%
\pgfpathlineto{\pgfqpoint{-0.204949in}{0.484854in}}%
\pgfpathlineto{\pgfqpoint{-0.168506in}{0.484854in}}%
\pgfpathlineto{\pgfqpoint{-0.132062in}{0.484854in}}%
\pgfpathlineto{\pgfqpoint{-0.095619in}{0.484854in}}%
\pgfpathlineto{\pgfqpoint{-0.059176in}{0.484854in}}%
\pgfpathlineto{\pgfqpoint{-0.022732in}{0.484854in}}%
\pgfpathlineto{\pgfqpoint{0.013711in}{0.484854in}}%
\pgfpathlineto{\pgfqpoint{0.050155in}{0.484854in}}%
\pgfpathlineto{\pgfqpoint{0.086598in}{0.484854in}}%
\pgfpathlineto{\pgfqpoint{0.123042in}{0.484854in}}%
\pgfpathlineto{\pgfqpoint{0.159485in}{0.484854in}}%
\pgfpathlineto{\pgfqpoint{0.195929in}{0.484854in}}%
\pgfpathlineto{\pgfqpoint{0.232372in}{0.484854in}}%
\pgfpathlineto{\pgfqpoint{0.268815in}{0.484854in}}%
\pgfpathlineto{\pgfqpoint{0.305259in}{0.484854in}}%
\pgfpathlineto{\pgfqpoint{0.341702in}{0.484854in}}%
\pgfpathlineto{\pgfqpoint{0.378146in}{0.484854in}}%
\pgfpathlineto{\pgfqpoint{0.414589in}{0.484854in}}%
\pgfpathlineto{\pgfqpoint{0.451033in}{0.484854in}}%
\pgfpathlineto{\pgfqpoint{0.487476in}{0.484854in}}%
\pgfpathlineto{\pgfqpoint{0.523920in}{0.484854in}}%
\pgfpathlineto{\pgfqpoint{0.560363in}{0.484854in}}%
\pgfpathlineto{\pgfqpoint{0.596806in}{0.484854in}}%
\pgfpathlineto{\pgfqpoint{0.633250in}{0.484854in}}%
\pgfpathlineto{\pgfqpoint{0.669693in}{0.484854in}}%
\pgfpathlineto{\pgfqpoint{0.706137in}{0.484854in}}%
\pgfpathlineto{\pgfqpoint{0.742580in}{0.484854in}}%
\pgfpathlineto{\pgfqpoint{0.779024in}{0.484854in}}%
\pgfpathlineto{\pgfqpoint{0.815467in}{0.484854in}}%
\pgfpathlineto{\pgfqpoint{0.851911in}{0.484854in}}%
\pgfpathlineto{\pgfqpoint{0.888354in}{0.484854in}}%
\pgfpathlineto{\pgfqpoint{0.924798in}{0.484854in}}%
\pgfpathlineto{\pgfqpoint{0.961241in}{0.512745in}}%
\pgfpathlineto{\pgfqpoint{0.997684in}{0.546348in}}%
\pgfpathlineto{\pgfqpoint{1.034128in}{0.576959in}}%
\pgfpathlineto{\pgfqpoint{1.070571in}{0.607873in}}%
\pgfpathlineto{\pgfqpoint{1.107015in}{0.635365in}}%
\pgfpathlineto{\pgfqpoint{1.143458in}{0.659972in}}%
\pgfpathlineto{\pgfqpoint{1.179902in}{0.684343in}}%
\pgfpathlineto{\pgfqpoint{1.216345in}{0.704172in}}%
\pgfpathlineto{\pgfqpoint{1.252789in}{0.729173in}}%
\pgfpathlineto{\pgfqpoint{1.289232in}{0.750216in}}%
\pgfpathlineto{\pgfqpoint{1.325675in}{0.768622in}}%
\pgfpathlineto{\pgfqpoint{1.362119in}{0.784811in}}%
\pgfpathlineto{\pgfqpoint{1.398562in}{0.786302in}}%
\pgfpathlineto{\pgfqpoint{1.435006in}{0.797467in}}%
\pgfpathlineto{\pgfqpoint{1.471449in}{0.842827in}}%
\pgfpathlineto{\pgfqpoint{1.507893in}{0.863362in}}%
\pgfpathlineto{\pgfqpoint{1.544336in}{0.874883in}}%
\pgfpathlineto{\pgfqpoint{1.580780in}{0.883472in}}%
\pgfpathlineto{\pgfqpoint{1.617223in}{0.890628in}}%
\pgfpathlineto{\pgfqpoint{1.653666in}{0.901053in}}%
\pgfpathlineto{\pgfqpoint{1.690110in}{0.907698in}}%
\pgfpathlineto{\pgfqpoint{1.726553in}{0.913031in}}%
\pgfpathlineto{\pgfqpoint{1.762997in}{0.974944in}}%
\pgfpathlineto{\pgfqpoint{1.799440in}{0.990904in}}%
\pgfpathlineto{\pgfqpoint{1.835884in}{1.004129in}}%
\pgfpathlineto{\pgfqpoint{1.872327in}{1.023099in}}%
\pgfpathlineto{\pgfqpoint{1.908771in}{1.035968in}}%
\pgfpathlineto{\pgfqpoint{1.945214in}{1.049960in}}%
\pgfpathlineto{\pgfqpoint{1.981658in}{1.064777in}}%
\pgfpathlineto{\pgfqpoint{2.018101in}{1.083533in}}%
\pgfpathlineto{\pgfqpoint{2.054544in}{1.097436in}}%
\pgfpathlineto{\pgfqpoint{2.090988in}{1.098021in}}%
\pgfpathlineto{\pgfqpoint{2.127431in}{1.102564in}}%
\pgfpathlineto{\pgfqpoint{2.163875in}{1.103219in}}%
\pgfpathlineto{\pgfqpoint{2.200318in}{1.163742in}}%
\pgfpathlineto{\pgfqpoint{2.236762in}{1.183153in}}%
\pgfpathlineto{\pgfqpoint{2.273205in}{1.192102in}}%
\pgfpathlineto{\pgfqpoint{2.309649in}{1.192313in}}%
\pgfpathlineto{\pgfqpoint{2.346092in}{1.191281in}}%
\pgfpathlineto{\pgfqpoint{2.382535in}{1.192628in}}%
\pgfpathlineto{\pgfqpoint{2.418979in}{1.180226in}}%
\pgfpathlineto{\pgfqpoint{2.455422in}{1.180566in}}%
\pgfpathlineto{\pgfqpoint{2.491866in}{1.180927in}}%
\pgfpathlineto{\pgfqpoint{2.528309in}{1.186316in}}%
\pgfpathlineto{\pgfqpoint{2.564753in}{1.186678in}}%
\pgfpathlineto{\pgfqpoint{2.601196in}{1.187697in}}%
\pgfpathlineto{\pgfqpoint{2.637640in}{1.188059in}}%
\pgfpathlineto{\pgfqpoint{2.664972in}{1.188423in}}%
\pgfpathlineto{\pgfqpoint{2.664972in}{2.045181in}}%
\pgfpathlineto{\pgfqpoint{2.664972in}{2.045181in}}%
\pgfpathlineto{\pgfqpoint{2.637640in}{2.044426in}}%
\pgfpathlineto{\pgfqpoint{2.601196in}{2.043623in}}%
\pgfpathlineto{\pgfqpoint{2.564753in}{2.041875in}}%
\pgfpathlineto{\pgfqpoint{2.528309in}{2.041073in}}%
\pgfpathlineto{\pgfqpoint{2.491866in}{2.033233in}}%
\pgfpathlineto{\pgfqpoint{2.455422in}{2.032430in}}%
\pgfpathlineto{\pgfqpoint{2.418979in}{2.031674in}}%
\pgfpathlineto{\pgfqpoint{2.382535in}{2.048535in}}%
\pgfpathlineto{\pgfqpoint{2.346092in}{2.046315in}}%
\pgfpathlineto{\pgfqpoint{2.309649in}{2.047448in}}%
\pgfpathlineto{\pgfqpoint{2.273205in}{2.046834in}}%
\pgfpathlineto{\pgfqpoint{2.236762in}{2.018923in}}%
\pgfpathlineto{\pgfqpoint{2.200318in}{1.965980in}}%
\pgfpathlineto{\pgfqpoint{2.163875in}{1.812629in}}%
\pgfpathlineto{\pgfqpoint{2.127431in}{1.806820in}}%
\pgfpathlineto{\pgfqpoint{2.090988in}{1.791330in}}%
\pgfpathlineto{\pgfqpoint{2.054544in}{1.784387in}}%
\pgfpathlineto{\pgfqpoint{2.018101in}{1.746982in}}%
\pgfpathlineto{\pgfqpoint{1.981658in}{1.698998in}}%
\pgfpathlineto{\pgfqpoint{1.945214in}{1.661121in}}%
\pgfpathlineto{\pgfqpoint{1.908771in}{1.624803in}}%
\pgfpathlineto{\pgfqpoint{1.872327in}{1.592026in}}%
\pgfpathlineto{\pgfqpoint{1.835884in}{1.545601in}}%
\pgfpathlineto{\pgfqpoint{1.799440in}{1.513958in}}%
\pgfpathlineto{\pgfqpoint{1.762997in}{1.475656in}}%
\pgfpathlineto{\pgfqpoint{1.726553in}{1.337891in}}%
\pgfpathlineto{\pgfqpoint{1.690110in}{1.323723in}}%
\pgfpathlineto{\pgfqpoint{1.653666in}{1.307948in}}%
\pgfpathlineto{\pgfqpoint{1.617223in}{1.284807in}}%
\pgfpathlineto{\pgfqpoint{1.580780in}{1.267427in}}%
\pgfpathlineto{\pgfqpoint{1.544336in}{1.248205in}}%
\pgfpathlineto{\pgfqpoint{1.507893in}{1.222607in}}%
\pgfpathlineto{\pgfqpoint{1.471449in}{1.179629in}}%
\pgfpathlineto{\pgfqpoint{1.435006in}{1.088856in}}%
\pgfpathlineto{\pgfqpoint{1.398562in}{1.064156in}}%
\pgfpathlineto{\pgfqpoint{1.362119in}{1.058725in}}%
\pgfpathlineto{\pgfqpoint{1.325675in}{1.025287in}}%
\pgfpathlineto{\pgfqpoint{1.289232in}{0.988119in}}%
\pgfpathlineto{\pgfqpoint{1.252789in}{0.946270in}}%
\pgfpathlineto{\pgfqpoint{1.216345in}{0.897563in}}%
\pgfpathlineto{\pgfqpoint{1.179902in}{0.858666in}}%
\pgfpathlineto{\pgfqpoint{1.143458in}{0.811622in}}%
\pgfpathlineto{\pgfqpoint{1.107015in}{0.764805in}}%
\pgfpathlineto{\pgfqpoint{1.070571in}{0.712641in}}%
\pgfpathlineto{\pgfqpoint{1.034128in}{0.654805in}}%
\pgfpathlineto{\pgfqpoint{0.997684in}{0.597923in}}%
\pgfpathlineto{\pgfqpoint{0.961241in}{0.535908in}}%
\pgfpathlineto{\pgfqpoint{0.924798in}{0.484854in}}%
\pgfpathlineto{\pgfqpoint{0.888354in}{0.484854in}}%
\pgfpathlineto{\pgfqpoint{0.851911in}{0.484854in}}%
\pgfpathlineto{\pgfqpoint{0.815467in}{0.484854in}}%
\pgfpathlineto{\pgfqpoint{0.779024in}{0.484854in}}%
\pgfpathlineto{\pgfqpoint{0.742580in}{0.484854in}}%
\pgfpathlineto{\pgfqpoint{0.706137in}{0.484854in}}%
\pgfpathlineto{\pgfqpoint{0.669693in}{0.484854in}}%
\pgfpathlineto{\pgfqpoint{0.633250in}{0.484854in}}%
\pgfpathlineto{\pgfqpoint{0.596806in}{0.484854in}}%
\pgfpathlineto{\pgfqpoint{0.560363in}{0.484854in}}%
\pgfpathlineto{\pgfqpoint{0.523920in}{0.484854in}}%
\pgfpathlineto{\pgfqpoint{0.487476in}{0.484854in}}%
\pgfpathlineto{\pgfqpoint{0.451033in}{0.484854in}}%
\pgfpathlineto{\pgfqpoint{0.414589in}{0.484854in}}%
\pgfpathlineto{\pgfqpoint{0.378146in}{0.484854in}}%
\pgfpathlineto{\pgfqpoint{0.341702in}{0.484854in}}%
\pgfpathlineto{\pgfqpoint{0.305259in}{0.484854in}}%
\pgfpathlineto{\pgfqpoint{0.268815in}{0.484854in}}%
\pgfpathlineto{\pgfqpoint{0.232372in}{0.484854in}}%
\pgfpathlineto{\pgfqpoint{0.195929in}{0.484854in}}%
\pgfpathlineto{\pgfqpoint{0.159485in}{0.484854in}}%
\pgfpathlineto{\pgfqpoint{0.123042in}{0.484854in}}%
\pgfpathlineto{\pgfqpoint{0.086598in}{0.484854in}}%
\pgfpathlineto{\pgfqpoint{0.050155in}{0.484854in}}%
\pgfpathlineto{\pgfqpoint{0.013711in}{0.484854in}}%
\pgfpathlineto{\pgfqpoint{-0.022732in}{0.484854in}}%
\pgfpathlineto{\pgfqpoint{-0.059176in}{0.484854in}}%
\pgfpathlineto{\pgfqpoint{-0.095619in}{0.484854in}}%
\pgfpathlineto{\pgfqpoint{-0.132062in}{0.484854in}}%
\pgfpathlineto{\pgfqpoint{-0.168506in}{0.484854in}}%
\pgfpathlineto{\pgfqpoint{-0.204949in}{0.484854in}}%
\pgfpathlineto{\pgfqpoint{-0.241393in}{0.484854in}}%
\pgfpathlineto{\pgfqpoint{-0.277836in}{0.484854in}}%
\pgfpathlineto{\pgfqpoint{-0.314280in}{0.484854in}}%
\pgfpathlineto{\pgfqpoint{-0.350723in}{0.484854in}}%
\pgfpathlineto{\pgfqpoint{-0.387167in}{0.484854in}}%
\pgfpathlineto{\pgfqpoint{-0.423610in}{0.484854in}}%
\pgfpathlineto{\pgfqpoint{-0.460054in}{0.484854in}}%
\pgfpathlineto{\pgfqpoint{-0.496497in}{0.484854in}}%
\pgfpathlineto{\pgfqpoint{-0.532940in}{0.484854in}}%
\pgfpathlineto{\pgfqpoint{-0.569384in}{0.484854in}}%
\pgfpathlineto{\pgfqpoint{-0.605827in}{0.484854in}}%
\pgfpathlineto{\pgfqpoint{-0.642271in}{0.484854in}}%
\pgfpathlineto{\pgfqpoint{-0.678714in}{0.484854in}}%
\pgfpathlineto{\pgfqpoint{-0.715158in}{0.484854in}}%
\pgfpathlineto{\pgfqpoint{-0.751601in}{0.484854in}}%
\pgfpathlineto{\pgfqpoint{-0.788045in}{0.484854in}}%
\pgfpathlineto{\pgfqpoint{-0.824488in}{0.484854in}}%
\pgfpathlineto{\pgfqpoint{-0.860931in}{0.484854in}}%
\pgfpathlineto{\pgfqpoint{-0.897375in}{0.484854in}}%
\pgfpathlineto{\pgfqpoint{-0.933818in}{0.484854in}}%
\pgfpathlineto{\pgfqpoint{-0.979373in}{0.484854in}}%
\pgfpathclose%
\pgfusepath{fill}%
\end{pgfscope}%
\begin{pgfscope}%
\pgfsetbuttcap%
\pgfsetroundjoin%
\definecolor{currentfill}{rgb}{0.000000,0.000000,0.000000}%
\pgfsetfillcolor{currentfill}%
\pgfsetlinewidth{0.803000pt}%
\definecolor{currentstroke}{rgb}{0.000000,0.000000,0.000000}%
\pgfsetstrokecolor{currentstroke}%
\pgfsetdash{}{0pt}%
\pgfsys@defobject{currentmarker}{\pgfqpoint{0.000000in}{-0.048611in}}{\pgfqpoint{0.000000in}{0.000000in}}{%
\pgfpathmoveto{\pgfqpoint{0.000000in}{0.000000in}}%
\pgfpathlineto{\pgfqpoint{0.000000in}{-0.048611in}}%
\pgfusepath{stroke,fill}%
}%
\begin{pgfscope}%
\pgfsys@transformshift{0.478365in}{0.484854in}%
\pgfsys@useobject{currentmarker}{}%
\end{pgfscope}%
\end{pgfscope}%
\begin{pgfscope}%
\definecolor{textcolor}{rgb}{0.000000,0.000000,0.000000}%
\pgfsetstrokecolor{textcolor}%
\pgfsetfillcolor{textcolor}%
\pgftext[x=0.478365in,y=0.387632in,,top]{\color{textcolor}\rmfamily\fontsize{8.000000}{9.600000}\selectfont \(\displaystyle 40\)}%
\end{pgfscope}%
\begin{pgfscope}%
\pgfsetbuttcap%
\pgfsetroundjoin%
\definecolor{currentfill}{rgb}{0.000000,0.000000,0.000000}%
\pgfsetfillcolor{currentfill}%
\pgfsetlinewidth{0.803000pt}%
\definecolor{currentstroke}{rgb}{0.000000,0.000000,0.000000}%
\pgfsetstrokecolor{currentstroke}%
\pgfsetdash{}{0pt}%
\pgfsys@defobject{currentmarker}{\pgfqpoint{0.000000in}{-0.048611in}}{\pgfqpoint{0.000000in}{0.000000in}}{%
\pgfpathmoveto{\pgfqpoint{0.000000in}{0.000000in}}%
\pgfpathlineto{\pgfqpoint{0.000000in}{-0.048611in}}%
\pgfusepath{stroke,fill}%
}%
\begin{pgfscope}%
\pgfsys@transformshift{0.842800in}{0.484854in}%
\pgfsys@useobject{currentmarker}{}%
\end{pgfscope}%
\end{pgfscope}%
\begin{pgfscope}%
\definecolor{textcolor}{rgb}{0.000000,0.000000,0.000000}%
\pgfsetstrokecolor{textcolor}%
\pgfsetfillcolor{textcolor}%
\pgftext[x=0.842800in,y=0.387632in,,top]{\color{textcolor}\rmfamily\fontsize{8.000000}{9.600000}\selectfont \(\displaystyle 50\)}%
\end{pgfscope}%
\begin{pgfscope}%
\pgfsetbuttcap%
\pgfsetroundjoin%
\definecolor{currentfill}{rgb}{0.000000,0.000000,0.000000}%
\pgfsetfillcolor{currentfill}%
\pgfsetlinewidth{0.803000pt}%
\definecolor{currentstroke}{rgb}{0.000000,0.000000,0.000000}%
\pgfsetstrokecolor{currentstroke}%
\pgfsetdash{}{0pt}%
\pgfsys@defobject{currentmarker}{\pgfqpoint{0.000000in}{-0.048611in}}{\pgfqpoint{0.000000in}{0.000000in}}{%
\pgfpathmoveto{\pgfqpoint{0.000000in}{0.000000in}}%
\pgfpathlineto{\pgfqpoint{0.000000in}{-0.048611in}}%
\pgfusepath{stroke,fill}%
}%
\begin{pgfscope}%
\pgfsys@transformshift{1.207234in}{0.484854in}%
\pgfsys@useobject{currentmarker}{}%
\end{pgfscope}%
\end{pgfscope}%
\begin{pgfscope}%
\definecolor{textcolor}{rgb}{0.000000,0.000000,0.000000}%
\pgfsetstrokecolor{textcolor}%
\pgfsetfillcolor{textcolor}%
\pgftext[x=1.207234in,y=0.387632in,,top]{\color{textcolor}\rmfamily\fontsize{8.000000}{9.600000}\selectfont \(\displaystyle 60\)}%
\end{pgfscope}%
\begin{pgfscope}%
\pgfsetbuttcap%
\pgfsetroundjoin%
\definecolor{currentfill}{rgb}{0.000000,0.000000,0.000000}%
\pgfsetfillcolor{currentfill}%
\pgfsetlinewidth{0.803000pt}%
\definecolor{currentstroke}{rgb}{0.000000,0.000000,0.000000}%
\pgfsetstrokecolor{currentstroke}%
\pgfsetdash{}{0pt}%
\pgfsys@defobject{currentmarker}{\pgfqpoint{0.000000in}{-0.048611in}}{\pgfqpoint{0.000000in}{0.000000in}}{%
\pgfpathmoveto{\pgfqpoint{0.000000in}{0.000000in}}%
\pgfpathlineto{\pgfqpoint{0.000000in}{-0.048611in}}%
\pgfusepath{stroke,fill}%
}%
\begin{pgfscope}%
\pgfsys@transformshift{1.571669in}{0.484854in}%
\pgfsys@useobject{currentmarker}{}%
\end{pgfscope}%
\end{pgfscope}%
\begin{pgfscope}%
\definecolor{textcolor}{rgb}{0.000000,0.000000,0.000000}%
\pgfsetstrokecolor{textcolor}%
\pgfsetfillcolor{textcolor}%
\pgftext[x=1.571669in,y=0.387632in,,top]{\color{textcolor}\rmfamily\fontsize{8.000000}{9.600000}\selectfont \(\displaystyle 70\)}%
\end{pgfscope}%
\begin{pgfscope}%
\pgfsetbuttcap%
\pgfsetroundjoin%
\definecolor{currentfill}{rgb}{0.000000,0.000000,0.000000}%
\pgfsetfillcolor{currentfill}%
\pgfsetlinewidth{0.803000pt}%
\definecolor{currentstroke}{rgb}{0.000000,0.000000,0.000000}%
\pgfsetstrokecolor{currentstroke}%
\pgfsetdash{}{0pt}%
\pgfsys@defobject{currentmarker}{\pgfqpoint{0.000000in}{-0.048611in}}{\pgfqpoint{0.000000in}{0.000000in}}{%
\pgfpathmoveto{\pgfqpoint{0.000000in}{0.000000in}}%
\pgfpathlineto{\pgfqpoint{0.000000in}{-0.048611in}}%
\pgfusepath{stroke,fill}%
}%
\begin{pgfscope}%
\pgfsys@transformshift{1.936103in}{0.484854in}%
\pgfsys@useobject{currentmarker}{}%
\end{pgfscope}%
\end{pgfscope}%
\begin{pgfscope}%
\definecolor{textcolor}{rgb}{0.000000,0.000000,0.000000}%
\pgfsetstrokecolor{textcolor}%
\pgfsetfillcolor{textcolor}%
\pgftext[x=1.936103in,y=0.387632in,,top]{\color{textcolor}\rmfamily\fontsize{8.000000}{9.600000}\selectfont \(\displaystyle 80\)}%
\end{pgfscope}%
\begin{pgfscope}%
\pgfsetbuttcap%
\pgfsetroundjoin%
\definecolor{currentfill}{rgb}{0.000000,0.000000,0.000000}%
\pgfsetfillcolor{currentfill}%
\pgfsetlinewidth{0.803000pt}%
\definecolor{currentstroke}{rgb}{0.000000,0.000000,0.000000}%
\pgfsetstrokecolor{currentstroke}%
\pgfsetdash{}{0pt}%
\pgfsys@defobject{currentmarker}{\pgfqpoint{0.000000in}{-0.048611in}}{\pgfqpoint{0.000000in}{0.000000in}}{%
\pgfpathmoveto{\pgfqpoint{0.000000in}{0.000000in}}%
\pgfpathlineto{\pgfqpoint{0.000000in}{-0.048611in}}%
\pgfusepath{stroke,fill}%
}%
\begin{pgfscope}%
\pgfsys@transformshift{2.300538in}{0.484854in}%
\pgfsys@useobject{currentmarker}{}%
\end{pgfscope}%
\end{pgfscope}%
\begin{pgfscope}%
\definecolor{textcolor}{rgb}{0.000000,0.000000,0.000000}%
\pgfsetstrokecolor{textcolor}%
\pgfsetfillcolor{textcolor}%
\pgftext[x=2.300538in,y=0.387632in,,top]{\color{textcolor}\rmfamily\fontsize{8.000000}{9.600000}\selectfont \(\displaystyle 90\)}%
\end{pgfscope}%
\begin{pgfscope}%
\pgfsetbuttcap%
\pgfsetroundjoin%
\definecolor{currentfill}{rgb}{0.000000,0.000000,0.000000}%
\pgfsetfillcolor{currentfill}%
\pgfsetlinewidth{0.803000pt}%
\definecolor{currentstroke}{rgb}{0.000000,0.000000,0.000000}%
\pgfsetstrokecolor{currentstroke}%
\pgfsetdash{}{0pt}%
\pgfsys@defobject{currentmarker}{\pgfqpoint{0.000000in}{-0.048611in}}{\pgfqpoint{0.000000in}{0.000000in}}{%
\pgfpathmoveto{\pgfqpoint{0.000000in}{0.000000in}}%
\pgfpathlineto{\pgfqpoint{0.000000in}{-0.048611in}}%
\pgfusepath{stroke,fill}%
}%
\begin{pgfscope}%
\pgfsys@transformshift{2.664972in}{0.484854in}%
\pgfsys@useobject{currentmarker}{}%
\end{pgfscope}%
\end{pgfscope}%
\begin{pgfscope}%
\definecolor{textcolor}{rgb}{0.000000,0.000000,0.000000}%
\pgfsetstrokecolor{textcolor}%
\pgfsetfillcolor{textcolor}%
\pgftext[x=2.664972in,y=0.387632in,,top]{\color{textcolor}\rmfamily\fontsize{8.000000}{9.600000}\selectfont \(\displaystyle 100\)}%
\end{pgfscope}%
\begin{pgfscope}%
\definecolor{textcolor}{rgb}{0.000000,0.000000,0.000000}%
\pgfsetstrokecolor{textcolor}%
\pgfsetfillcolor{textcolor}%
\pgftext[x=1.571669in,y=0.224546in,,top]{\color{textcolor}\rmfamily\fontsize{8.000000}{9.600000}\selectfont Time (\(\displaystyle \times 10^3 \, \mathrm{yr}\))}%
\end{pgfscope}%
\begin{pgfscope}%
\pgfsetbuttcap%
\pgfsetroundjoin%
\definecolor{currentfill}{rgb}{0.000000,0.000000,0.000000}%
\pgfsetfillcolor{currentfill}%
\pgfsetlinewidth{0.803000pt}%
\definecolor{currentstroke}{rgb}{0.000000,0.000000,0.000000}%
\pgfsetstrokecolor{currentstroke}%
\pgfsetdash{}{0pt}%
\pgfsys@defobject{currentmarker}{\pgfqpoint{-0.048611in}{0.000000in}}{\pgfqpoint{0.000000in}{0.000000in}}{%
\pgfpathmoveto{\pgfqpoint{0.000000in}{0.000000in}}%
\pgfpathlineto{\pgfqpoint{-0.048611in}{0.000000in}}%
\pgfusepath{stroke,fill}%
}%
\begin{pgfscope}%
\pgfsys@transformshift{0.478365in}{0.484854in}%
\pgfsys@useobject{currentmarker}{}%
\end{pgfscope}%
\end{pgfscope}%
\begin{pgfscope}%
\definecolor{textcolor}{rgb}{0.000000,0.000000,0.000000}%
\pgfsetstrokecolor{textcolor}%
\pgfsetfillcolor{textcolor}%
\pgftext[x=0.322114in,y=0.442645in,left,base]{\color{textcolor}\rmfamily\fontsize{8.000000}{9.600000}\selectfont \(\displaystyle 0\)}%
\end{pgfscope}%
\begin{pgfscope}%
\pgfsetbuttcap%
\pgfsetroundjoin%
\definecolor{currentfill}{rgb}{0.000000,0.000000,0.000000}%
\pgfsetfillcolor{currentfill}%
\pgfsetlinewidth{0.803000pt}%
\definecolor{currentstroke}{rgb}{0.000000,0.000000,0.000000}%
\pgfsetstrokecolor{currentstroke}%
\pgfsetdash{}{0pt}%
\pgfsys@defobject{currentmarker}{\pgfqpoint{-0.048611in}{0.000000in}}{\pgfqpoint{0.000000in}{0.000000in}}{%
\pgfpathmoveto{\pgfqpoint{0.000000in}{0.000000in}}%
\pgfpathlineto{\pgfqpoint{-0.048611in}{0.000000in}}%
\pgfusepath{stroke,fill}%
}%
\begin{pgfscope}%
\pgfsys@transformshift{0.478365in}{0.720996in}%
\pgfsys@useobject{currentmarker}{}%
\end{pgfscope}%
\end{pgfscope}%
\begin{pgfscope}%
\definecolor{textcolor}{rgb}{0.000000,0.000000,0.000000}%
\pgfsetstrokecolor{textcolor}%
\pgfsetfillcolor{textcolor}%
\pgftext[x=0.322114in,y=0.678786in,left,base]{\color{textcolor}\rmfamily\fontsize{8.000000}{9.600000}\selectfont \(\displaystyle 5\)}%
\end{pgfscope}%
\begin{pgfscope}%
\pgfsetbuttcap%
\pgfsetroundjoin%
\definecolor{currentfill}{rgb}{0.000000,0.000000,0.000000}%
\pgfsetfillcolor{currentfill}%
\pgfsetlinewidth{0.803000pt}%
\definecolor{currentstroke}{rgb}{0.000000,0.000000,0.000000}%
\pgfsetstrokecolor{currentstroke}%
\pgfsetdash{}{0pt}%
\pgfsys@defobject{currentmarker}{\pgfqpoint{-0.048611in}{0.000000in}}{\pgfqpoint{0.000000in}{0.000000in}}{%
\pgfpathmoveto{\pgfqpoint{0.000000in}{0.000000in}}%
\pgfpathlineto{\pgfqpoint{-0.048611in}{0.000000in}}%
\pgfusepath{stroke,fill}%
}%
\begin{pgfscope}%
\pgfsys@transformshift{0.478365in}{0.957137in}%
\pgfsys@useobject{currentmarker}{}%
\end{pgfscope}%
\end{pgfscope}%
\begin{pgfscope}%
\definecolor{textcolor}{rgb}{0.000000,0.000000,0.000000}%
\pgfsetstrokecolor{textcolor}%
\pgfsetfillcolor{textcolor}%
\pgftext[x=0.263086in,y=0.914928in,left,base]{\color{textcolor}\rmfamily\fontsize{8.000000}{9.600000}\selectfont \(\displaystyle 10\)}%
\end{pgfscope}%
\begin{pgfscope}%
\pgfsetbuttcap%
\pgfsetroundjoin%
\definecolor{currentfill}{rgb}{0.000000,0.000000,0.000000}%
\pgfsetfillcolor{currentfill}%
\pgfsetlinewidth{0.803000pt}%
\definecolor{currentstroke}{rgb}{0.000000,0.000000,0.000000}%
\pgfsetstrokecolor{currentstroke}%
\pgfsetdash{}{0pt}%
\pgfsys@defobject{currentmarker}{\pgfqpoint{-0.048611in}{0.000000in}}{\pgfqpoint{0.000000in}{0.000000in}}{%
\pgfpathmoveto{\pgfqpoint{0.000000in}{0.000000in}}%
\pgfpathlineto{\pgfqpoint{-0.048611in}{0.000000in}}%
\pgfusepath{stroke,fill}%
}%
\begin{pgfscope}%
\pgfsys@transformshift{0.478365in}{1.193278in}%
\pgfsys@useobject{currentmarker}{}%
\end{pgfscope}%
\end{pgfscope}%
\begin{pgfscope}%
\definecolor{textcolor}{rgb}{0.000000,0.000000,0.000000}%
\pgfsetstrokecolor{textcolor}%
\pgfsetfillcolor{textcolor}%
\pgftext[x=0.263086in,y=1.151069in,left,base]{\color{textcolor}\rmfamily\fontsize{8.000000}{9.600000}\selectfont \(\displaystyle 15\)}%
\end{pgfscope}%
\begin{pgfscope}%
\pgfsetbuttcap%
\pgfsetroundjoin%
\definecolor{currentfill}{rgb}{0.000000,0.000000,0.000000}%
\pgfsetfillcolor{currentfill}%
\pgfsetlinewidth{0.803000pt}%
\definecolor{currentstroke}{rgb}{0.000000,0.000000,0.000000}%
\pgfsetstrokecolor{currentstroke}%
\pgfsetdash{}{0pt}%
\pgfsys@defobject{currentmarker}{\pgfqpoint{-0.048611in}{0.000000in}}{\pgfqpoint{0.000000in}{0.000000in}}{%
\pgfpathmoveto{\pgfqpoint{0.000000in}{0.000000in}}%
\pgfpathlineto{\pgfqpoint{-0.048611in}{0.000000in}}%
\pgfusepath{stroke,fill}%
}%
\begin{pgfscope}%
\pgfsys@transformshift{0.478365in}{1.429419in}%
\pgfsys@useobject{currentmarker}{}%
\end{pgfscope}%
\end{pgfscope}%
\begin{pgfscope}%
\definecolor{textcolor}{rgb}{0.000000,0.000000,0.000000}%
\pgfsetstrokecolor{textcolor}%
\pgfsetfillcolor{textcolor}%
\pgftext[x=0.263086in,y=1.387210in,left,base]{\color{textcolor}\rmfamily\fontsize{8.000000}{9.600000}\selectfont \(\displaystyle 20\)}%
\end{pgfscope}%
\begin{pgfscope}%
\pgfsetbuttcap%
\pgfsetroundjoin%
\definecolor{currentfill}{rgb}{0.000000,0.000000,0.000000}%
\pgfsetfillcolor{currentfill}%
\pgfsetlinewidth{0.803000pt}%
\definecolor{currentstroke}{rgb}{0.000000,0.000000,0.000000}%
\pgfsetstrokecolor{currentstroke}%
\pgfsetdash{}{0pt}%
\pgfsys@defobject{currentmarker}{\pgfqpoint{-0.048611in}{0.000000in}}{\pgfqpoint{0.000000in}{0.000000in}}{%
\pgfpathmoveto{\pgfqpoint{0.000000in}{0.000000in}}%
\pgfpathlineto{\pgfqpoint{-0.048611in}{0.000000in}}%
\pgfusepath{stroke,fill}%
}%
\begin{pgfscope}%
\pgfsys@transformshift{0.478365in}{1.665561in}%
\pgfsys@useobject{currentmarker}{}%
\end{pgfscope}%
\end{pgfscope}%
\begin{pgfscope}%
\definecolor{textcolor}{rgb}{0.000000,0.000000,0.000000}%
\pgfsetstrokecolor{textcolor}%
\pgfsetfillcolor{textcolor}%
\pgftext[x=0.263086in,y=1.623351in,left,base]{\color{textcolor}\rmfamily\fontsize{8.000000}{9.600000}\selectfont \(\displaystyle 25\)}%
\end{pgfscope}%
\begin{pgfscope}%
\pgfsetbuttcap%
\pgfsetroundjoin%
\definecolor{currentfill}{rgb}{0.000000,0.000000,0.000000}%
\pgfsetfillcolor{currentfill}%
\pgfsetlinewidth{0.803000pt}%
\definecolor{currentstroke}{rgb}{0.000000,0.000000,0.000000}%
\pgfsetstrokecolor{currentstroke}%
\pgfsetdash{}{0pt}%
\pgfsys@defobject{currentmarker}{\pgfqpoint{-0.048611in}{0.000000in}}{\pgfqpoint{0.000000in}{0.000000in}}{%
\pgfpathmoveto{\pgfqpoint{0.000000in}{0.000000in}}%
\pgfpathlineto{\pgfqpoint{-0.048611in}{0.000000in}}%
\pgfusepath{stroke,fill}%
}%
\begin{pgfscope}%
\pgfsys@transformshift{0.478365in}{1.901702in}%
\pgfsys@useobject{currentmarker}{}%
\end{pgfscope}%
\end{pgfscope}%
\begin{pgfscope}%
\definecolor{textcolor}{rgb}{0.000000,0.000000,0.000000}%
\pgfsetstrokecolor{textcolor}%
\pgfsetfillcolor{textcolor}%
\pgftext[x=0.263086in,y=1.859493in,left,base]{\color{textcolor}\rmfamily\fontsize{8.000000}{9.600000}\selectfont \(\displaystyle 30\)}%
\end{pgfscope}%
\begin{pgfscope}%
\definecolor{textcolor}{rgb}{0.000000,0.000000,0.000000}%
\pgfsetstrokecolor{textcolor}%
\pgfsetfillcolor{textcolor}%
\pgftext[x=0.207530in,y=1.299542in,,bottom,rotate=90.000000]{\color{textcolor}\rmfamily\fontsize{8.000000}{9.600000}\selectfont Melt volume (\%)}%
\end{pgfscope}%
\begin{pgfscope}%
\pgfpathrectangle{\pgfqpoint{0.478365in}{0.484854in}}{\pgfqpoint{2.186607in}{1.629375in}}%
\pgfusepath{clip}%
\pgfsetrectcap%
\pgfsetroundjoin%
\pgfsetlinewidth{1.505625pt}%
\definecolor{currentstroke}{rgb}{0.000000,0.000000,0.000000}%
\pgfsetstrokecolor{currentstroke}%
\pgfsetdash{}{0pt}%
\pgfpathmoveto{\pgfqpoint{0.473365in}{0.484854in}}%
\pgfpathlineto{\pgfqpoint{0.487476in}{0.484854in}}%
\pgfpathlineto{\pgfqpoint{0.523920in}{0.484854in}}%
\pgfpathlineto{\pgfqpoint{0.560363in}{0.484854in}}%
\pgfpathlineto{\pgfqpoint{0.596806in}{0.484854in}}%
\pgfpathlineto{\pgfqpoint{0.633250in}{0.484854in}}%
\pgfpathlineto{\pgfqpoint{0.669693in}{0.484854in}}%
\pgfpathlineto{\pgfqpoint{0.706137in}{0.484854in}}%
\pgfpathlineto{\pgfqpoint{0.742580in}{0.484854in}}%
\pgfpathlineto{\pgfqpoint{0.779024in}{0.484854in}}%
\pgfpathlineto{\pgfqpoint{0.815467in}{0.484854in}}%
\pgfpathlineto{\pgfqpoint{0.851911in}{0.484854in}}%
\pgfpathlineto{\pgfqpoint{0.888354in}{0.484854in}}%
\pgfpathlineto{\pgfqpoint{0.924798in}{0.484854in}}%
\pgfpathlineto{\pgfqpoint{0.961241in}{0.535908in}}%
\pgfpathlineto{\pgfqpoint{0.997684in}{0.597923in}}%
\pgfpathlineto{\pgfqpoint{1.034128in}{0.654805in}}%
\pgfpathlineto{\pgfqpoint{1.070571in}{0.712641in}}%
\pgfpathlineto{\pgfqpoint{1.107015in}{0.764805in}}%
\pgfpathlineto{\pgfqpoint{1.143458in}{0.811622in}}%
\pgfpathlineto{\pgfqpoint{1.179902in}{0.858666in}}%
\pgfpathlineto{\pgfqpoint{1.216345in}{0.897563in}}%
\pgfpathlineto{\pgfqpoint{1.252789in}{0.946270in}}%
\pgfpathlineto{\pgfqpoint{1.289232in}{0.988119in}}%
\pgfpathlineto{\pgfqpoint{1.325675in}{1.025287in}}%
\pgfpathlineto{\pgfqpoint{1.362119in}{1.058725in}}%
\pgfpathlineto{\pgfqpoint{1.398562in}{1.064156in}}%
\pgfpathlineto{\pgfqpoint{1.435006in}{1.088856in}}%
\pgfpathlineto{\pgfqpoint{1.471449in}{1.179629in}}%
\pgfpathlineto{\pgfqpoint{1.507893in}{1.222607in}}%
\pgfpathlineto{\pgfqpoint{1.544336in}{1.248205in}}%
\pgfpathlineto{\pgfqpoint{1.580780in}{1.267427in}}%
\pgfpathlineto{\pgfqpoint{1.617223in}{1.284807in}}%
\pgfpathlineto{\pgfqpoint{1.653666in}{1.307948in}}%
\pgfpathlineto{\pgfqpoint{1.690110in}{1.323723in}}%
\pgfpathlineto{\pgfqpoint{1.726553in}{1.337891in}}%
\pgfpathlineto{\pgfqpoint{1.762997in}{1.475656in}}%
\pgfpathlineto{\pgfqpoint{1.799440in}{1.513958in}}%
\pgfpathlineto{\pgfqpoint{1.835884in}{1.545601in}}%
\pgfpathlineto{\pgfqpoint{1.872327in}{1.592026in}}%
\pgfpathlineto{\pgfqpoint{1.908771in}{1.624803in}}%
\pgfpathlineto{\pgfqpoint{1.945214in}{1.661121in}}%
\pgfpathlineto{\pgfqpoint{1.981658in}{1.698998in}}%
\pgfpathlineto{\pgfqpoint{2.018101in}{1.746982in}}%
\pgfpathlineto{\pgfqpoint{2.054544in}{1.784387in}}%
\pgfpathlineto{\pgfqpoint{2.090988in}{1.791330in}}%
\pgfpathlineto{\pgfqpoint{2.127431in}{1.806820in}}%
\pgfpathlineto{\pgfqpoint{2.163875in}{1.812629in}}%
\pgfpathlineto{\pgfqpoint{2.200318in}{1.965980in}}%
\pgfpathlineto{\pgfqpoint{2.236762in}{2.018923in}}%
\pgfpathlineto{\pgfqpoint{2.273205in}{2.046834in}}%
\pgfpathlineto{\pgfqpoint{2.309649in}{2.047448in}}%
\pgfpathlineto{\pgfqpoint{2.346092in}{2.046315in}}%
\pgfpathlineto{\pgfqpoint{2.382535in}{2.048535in}}%
\pgfpathlineto{\pgfqpoint{2.418979in}{2.031674in}}%
\pgfpathlineto{\pgfqpoint{2.455422in}{2.032430in}}%
\pgfpathlineto{\pgfqpoint{2.491866in}{2.033233in}}%
\pgfpathlineto{\pgfqpoint{2.528309in}{2.041073in}}%
\pgfpathlineto{\pgfqpoint{2.564753in}{2.041875in}}%
\pgfpathlineto{\pgfqpoint{2.601196in}{2.043623in}}%
\pgfpathlineto{\pgfqpoint{2.637640in}{2.044426in}}%
\pgfpathlineto{\pgfqpoint{2.664972in}{2.045181in}}%
\pgfusepath{stroke}%
\end{pgfscope}%
\begin{pgfscope}%
\pgfsetrectcap%
\pgfsetmiterjoin%
\pgfsetlinewidth{0.803000pt}%
\definecolor{currentstroke}{rgb}{0.000000,0.000000,0.000000}%
\pgfsetstrokecolor{currentstroke}%
\pgfsetdash{}{0pt}%
\pgfpathmoveto{\pgfqpoint{0.478365in}{0.484854in}}%
\pgfpathlineto{\pgfqpoint{0.478365in}{2.114229in}}%
\pgfusepath{stroke}%
\end{pgfscope}%
\begin{pgfscope}%
\pgfsetrectcap%
\pgfsetmiterjoin%
\pgfsetlinewidth{0.803000pt}%
\definecolor{currentstroke}{rgb}{0.000000,0.000000,0.000000}%
\pgfsetstrokecolor{currentstroke}%
\pgfsetdash{}{0pt}%
\pgfpathmoveto{\pgfqpoint{2.664972in}{0.484854in}}%
\pgfpathlineto{\pgfqpoint{2.664972in}{2.114229in}}%
\pgfusepath{stroke}%
\end{pgfscope}%
\begin{pgfscope}%
\pgfsetrectcap%
\pgfsetmiterjoin%
\pgfsetlinewidth{0.803000pt}%
\definecolor{currentstroke}{rgb}{0.000000,0.000000,0.000000}%
\pgfsetstrokecolor{currentstroke}%
\pgfsetdash{}{0pt}%
\pgfpathmoveto{\pgfqpoint{0.478365in}{0.484854in}}%
\pgfpathlineto{\pgfqpoint{2.664972in}{0.484854in}}%
\pgfusepath{stroke}%
\end{pgfscope}%
\begin{pgfscope}%
\pgfsetrectcap%
\pgfsetmiterjoin%
\pgfsetlinewidth{0.803000pt}%
\definecolor{currentstroke}{rgb}{0.000000,0.000000,0.000000}%
\pgfsetstrokecolor{currentstroke}%
\pgfsetdash{}{0pt}%
\pgfpathmoveto{\pgfqpoint{0.478365in}{2.114229in}}%
\pgfpathlineto{\pgfqpoint{2.664972in}{2.114229in}}%
\pgfusepath{stroke}%
\end{pgfscope}%
\begin{pgfscope}%
\pgfsetbuttcap%
\pgfsetmiterjoin%
\definecolor{currentfill}{rgb}{0.121569,0.466667,0.705882}%
\pgfsetfillcolor{currentfill}%
\pgfsetlinewidth{0.000000pt}%
\definecolor{currentstroke}{rgb}{0.000000,0.000000,0.000000}%
\pgfsetstrokecolor{currentstroke}%
\pgfsetstrokeopacity{0.000000}%
\pgfsetdash{}{0pt}%
\pgfpathmoveto{\pgfqpoint{0.565865in}{1.952863in}}%
\pgfpathlineto{\pgfqpoint{0.760310in}{1.952863in}}%
\pgfpathlineto{\pgfqpoint{0.760310in}{2.020919in}}%
\pgfpathlineto{\pgfqpoint{0.565865in}{2.020919in}}%
\pgfpathclose%
\pgfusepath{fill}%
\end{pgfscope}%
\begin{pgfscope}%
\definecolor{textcolor}{rgb}{0.000000,0.000000,0.000000}%
\pgfsetstrokecolor{textcolor}%
\pgfsetfillcolor{textcolor}%
\pgftext[x=0.838088in,y=1.952863in,left,base]{\color{textcolor}\rmfamily\fontsize{7.000000}{8.400000}\selectfont CaO}%
\end{pgfscope}%
\begin{pgfscope}%
\pgfsetbuttcap%
\pgfsetmiterjoin%
\definecolor{currentfill}{rgb}{1.000000,0.498039,0.054902}%
\pgfsetfillcolor{currentfill}%
\pgfsetlinewidth{0.000000pt}%
\definecolor{currentstroke}{rgb}{0.000000,0.000000,0.000000}%
\pgfsetstrokecolor{currentstroke}%
\pgfsetstrokeopacity{0.000000}%
\pgfsetdash{}{0pt}%
\pgfpathmoveto{\pgfqpoint{0.565865in}{1.810163in}}%
\pgfpathlineto{\pgfqpoint{0.760310in}{1.810163in}}%
\pgfpathlineto{\pgfqpoint{0.760310in}{1.878219in}}%
\pgfpathlineto{\pgfqpoint{0.565865in}{1.878219in}}%
\pgfpathclose%
\pgfusepath{fill}%
\end{pgfscope}%
\begin{pgfscope}%
\definecolor{textcolor}{rgb}{0.000000,0.000000,0.000000}%
\pgfsetstrokecolor{textcolor}%
\pgfsetfillcolor{textcolor}%
\pgftext[x=0.838088in,y=1.810163in,left,base]{\color{textcolor}\rmfamily\fontsize{7.000000}{8.400000}\selectfont FeO}%
\end{pgfscope}%
\begin{pgfscope}%
\pgfsetbuttcap%
\pgfsetmiterjoin%
\definecolor{currentfill}{rgb}{0.172549,0.627451,0.172549}%
\pgfsetfillcolor{currentfill}%
\pgfsetlinewidth{0.000000pt}%
\definecolor{currentstroke}{rgb}{0.000000,0.000000,0.000000}%
\pgfsetstrokecolor{currentstroke}%
\pgfsetstrokeopacity{0.000000}%
\pgfsetdash{}{0pt}%
\pgfpathmoveto{\pgfqpoint{0.565865in}{1.667463in}}%
\pgfpathlineto{\pgfqpoint{0.760310in}{1.667463in}}%
\pgfpathlineto{\pgfqpoint{0.760310in}{1.735519in}}%
\pgfpathlineto{\pgfqpoint{0.565865in}{1.735519in}}%
\pgfpathclose%
\pgfusepath{fill}%
\end{pgfscope}%
\begin{pgfscope}%
\definecolor{textcolor}{rgb}{0.000000,0.000000,0.000000}%
\pgfsetstrokecolor{textcolor}%
\pgfsetfillcolor{textcolor}%
\pgftext[x=0.838088in,y=1.667463in,left,base]{\color{textcolor}\rmfamily\fontsize{7.000000}{8.400000}\selectfont MgO}%
\end{pgfscope}%
\begin{pgfscope}%
\pgfsetbuttcap%
\pgfsetmiterjoin%
\definecolor{currentfill}{rgb}{0.839216,0.152941,0.156863}%
\pgfsetfillcolor{currentfill}%
\pgfsetlinewidth{0.000000pt}%
\definecolor{currentstroke}{rgb}{0.000000,0.000000,0.000000}%
\pgfsetstrokecolor{currentstroke}%
\pgfsetstrokeopacity{0.000000}%
\pgfsetdash{}{0pt}%
\pgfpathmoveto{\pgfqpoint{0.565865in}{1.523386in}}%
\pgfpathlineto{\pgfqpoint{0.760310in}{1.523386in}}%
\pgfpathlineto{\pgfqpoint{0.760310in}{1.591442in}}%
\pgfpathlineto{\pgfqpoint{0.565865in}{1.591442in}}%
\pgfpathclose%
\pgfusepath{fill}%
\end{pgfscope}%
\begin{pgfscope}%
\definecolor{textcolor}{rgb}{0.000000,0.000000,0.000000}%
\pgfsetstrokecolor{textcolor}%
\pgfsetfillcolor{textcolor}%
\pgftext[x=0.838088in,y=1.523386in,left,base]{\color{textcolor}\rmfamily\fontsize{7.000000}{8.400000}\selectfont SiO2}%
\end{pgfscope}%
\end{pgfpicture}%
\makeatother%
\endgroup%

        \caption{Particle property plugin with batch melting.}
        \label{fig:decompression_event_particle_plugin_batch}
    \end{subfigure}
    \hfill
    \begin{subfigure}{0.49\textwidth}
        \centering
        %% Creator: Matplotlib, PGF backend
%%
%% To include the figure in your LaTeX document, write
%%   \input{<filename>.pgf}
%%
%% Make sure the required packages are loaded in your preamble
%%   \usepackage{pgf}
%%
%% Figures using additional raster images can only be included by \input if
%% they are in the same directory as the main LaTeX file. For loading figures
%% from other directories you can use the `import` package
%%   \usepackage{import}
%% and then include the figures with
%%   \import{<path to file>}{<filename>.pgf}
%%
%% Matplotlib used the following preamble
%%   \usepackage{fontspec}
%%   \setmainfont{DejaVuSerif.ttf}[Path=/home/connor/.local/lib/python3.8/site-packages/matplotlib/mpl-data/fonts/ttf/]
%%   \setsansfont{DejaVuSans.ttf}[Path=/home/connor/.local/lib/python3.8/site-packages/matplotlib/mpl-data/fonts/ttf/]
%%   \setmonofont{DejaVuSansMono.ttf}[Path=/home/connor/.local/lib/python3.8/site-packages/matplotlib/mpl-data/fonts/ttf/]
%%
\begingroup%
\makeatletter%
\begin{pgfpicture}%
\pgfpathrectangle{\pgfpointorigin}{\pgfqpoint{2.794486in}{2.214229in}}%
\pgfusepath{use as bounding box, clip}%
\begin{pgfscope}%
\pgfsetbuttcap%
\pgfsetmiterjoin%
\definecolor{currentfill}{rgb}{1.000000,1.000000,1.000000}%
\pgfsetfillcolor{currentfill}%
\pgfsetlinewidth{0.000000pt}%
\definecolor{currentstroke}{rgb}{1.000000,1.000000,1.000000}%
\pgfsetstrokecolor{currentstroke}%
\pgfsetdash{}{0pt}%
\pgfpathmoveto{\pgfqpoint{0.000000in}{0.000000in}}%
\pgfpathlineto{\pgfqpoint{2.794486in}{0.000000in}}%
\pgfpathlineto{\pgfqpoint{2.794486in}{2.214229in}}%
\pgfpathlineto{\pgfqpoint{0.000000in}{2.214229in}}%
\pgfpathclose%
\pgfusepath{fill}%
\end{pgfscope}%
\begin{pgfscope}%
\pgfsetbuttcap%
\pgfsetmiterjoin%
\definecolor{currentfill}{rgb}{1.000000,1.000000,1.000000}%
\pgfsetfillcolor{currentfill}%
\pgfsetlinewidth{0.000000pt}%
\definecolor{currentstroke}{rgb}{0.000000,0.000000,0.000000}%
\pgfsetstrokecolor{currentstroke}%
\pgfsetstrokeopacity{0.000000}%
\pgfsetdash{}{0pt}%
\pgfpathmoveto{\pgfqpoint{0.419337in}{0.484854in}}%
\pgfpathlineto{\pgfqpoint{2.605944in}{0.484854in}}%
\pgfpathlineto{\pgfqpoint{2.605944in}{2.114229in}}%
\pgfpathlineto{\pgfqpoint{0.419337in}{2.114229in}}%
\pgfpathclose%
\pgfusepath{fill}%
\end{pgfscope}%
\begin{pgfscope}%
\pgfpathrectangle{\pgfqpoint{0.419337in}{0.484854in}}{\pgfqpoint{2.186607in}{1.629375in}}%
\pgfusepath{clip}%
\pgfsetbuttcap%
\pgfsetroundjoin%
\definecolor{currentfill}{rgb}{0.121569,0.466667,0.705882}%
\pgfsetfillcolor{currentfill}%
\pgfsetlinewidth{0.000000pt}%
\definecolor{currentstroke}{rgb}{0.000000,0.000000,0.000000}%
\pgfsetstrokecolor{currentstroke}%
\pgfsetdash{}{0pt}%
\pgfpathmoveto{\pgfqpoint{-1.038401in}{0.484854in}}%
\pgfpathlineto{\pgfqpoint{-1.038401in}{0.484854in}}%
\pgfpathlineto{\pgfqpoint{-0.992847in}{0.484854in}}%
\pgfpathlineto{\pgfqpoint{-0.956404in}{0.484854in}}%
\pgfpathlineto{\pgfqpoint{-0.919960in}{0.484854in}}%
\pgfpathlineto{\pgfqpoint{-0.883517in}{0.484854in}}%
\pgfpathlineto{\pgfqpoint{-0.847073in}{0.484854in}}%
\pgfpathlineto{\pgfqpoint{-0.810630in}{0.484854in}}%
\pgfpathlineto{\pgfqpoint{-0.774186in}{0.484854in}}%
\pgfpathlineto{\pgfqpoint{-0.737743in}{0.484854in}}%
\pgfpathlineto{\pgfqpoint{-0.701299in}{0.484854in}}%
\pgfpathlineto{\pgfqpoint{-0.664856in}{0.484854in}}%
\pgfpathlineto{\pgfqpoint{-0.628412in}{0.484854in}}%
\pgfpathlineto{\pgfqpoint{-0.591969in}{0.484854in}}%
\pgfpathlineto{\pgfqpoint{-0.555526in}{0.484854in}}%
\pgfpathlineto{\pgfqpoint{-0.519082in}{0.484854in}}%
\pgfpathlineto{\pgfqpoint{-0.482639in}{0.484854in}}%
\pgfpathlineto{\pgfqpoint{-0.446195in}{0.484854in}}%
\pgfpathlineto{\pgfqpoint{-0.409752in}{0.484854in}}%
\pgfpathlineto{\pgfqpoint{-0.373308in}{0.484854in}}%
\pgfpathlineto{\pgfqpoint{-0.336865in}{0.484854in}}%
\pgfpathlineto{\pgfqpoint{-0.300421in}{0.484854in}}%
\pgfpathlineto{\pgfqpoint{-0.263978in}{0.484854in}}%
\pgfpathlineto{\pgfqpoint{-0.227535in}{0.484854in}}%
\pgfpathlineto{\pgfqpoint{-0.191091in}{0.484854in}}%
\pgfpathlineto{\pgfqpoint{-0.154648in}{0.484854in}}%
\pgfpathlineto{\pgfqpoint{-0.118204in}{0.484854in}}%
\pgfpathlineto{\pgfqpoint{-0.081761in}{0.484854in}}%
\pgfpathlineto{\pgfqpoint{-0.045317in}{0.484854in}}%
\pgfpathlineto{\pgfqpoint{-0.008874in}{0.484854in}}%
\pgfpathlineto{\pgfqpoint{0.027570in}{0.484854in}}%
\pgfpathlineto{\pgfqpoint{0.064013in}{0.484854in}}%
\pgfpathlineto{\pgfqpoint{0.100456in}{0.484854in}}%
\pgfpathlineto{\pgfqpoint{0.136900in}{0.484854in}}%
\pgfpathlineto{\pgfqpoint{0.173343in}{0.484854in}}%
\pgfpathlineto{\pgfqpoint{0.209787in}{0.484854in}}%
\pgfpathlineto{\pgfqpoint{0.246230in}{0.484854in}}%
\pgfpathlineto{\pgfqpoint{0.282674in}{0.484854in}}%
\pgfpathlineto{\pgfqpoint{0.319117in}{0.484854in}}%
\pgfpathlineto{\pgfqpoint{0.355561in}{0.484854in}}%
\pgfpathlineto{\pgfqpoint{0.392004in}{0.484854in}}%
\pgfpathlineto{\pgfqpoint{0.428448in}{0.484854in}}%
\pgfpathlineto{\pgfqpoint{0.464891in}{0.484854in}}%
\pgfpathlineto{\pgfqpoint{0.501334in}{0.484854in}}%
\pgfpathlineto{\pgfqpoint{0.537778in}{0.484854in}}%
\pgfpathlineto{\pgfqpoint{0.574221in}{0.484854in}}%
\pgfpathlineto{\pgfqpoint{0.610665in}{0.484854in}}%
\pgfpathlineto{\pgfqpoint{0.647108in}{0.484854in}}%
\pgfpathlineto{\pgfqpoint{0.683552in}{0.484854in}}%
\pgfpathlineto{\pgfqpoint{0.719995in}{0.484854in}}%
\pgfpathlineto{\pgfqpoint{0.756439in}{0.484854in}}%
\pgfpathlineto{\pgfqpoint{0.792882in}{0.484854in}}%
\pgfpathlineto{\pgfqpoint{0.829325in}{0.484854in}}%
\pgfpathlineto{\pgfqpoint{0.865769in}{0.484854in}}%
\pgfpathlineto{\pgfqpoint{0.902212in}{0.484854in}}%
\pgfpathlineto{\pgfqpoint{0.938656in}{0.484854in}}%
\pgfpathlineto{\pgfqpoint{0.975099in}{0.484854in}}%
\pgfpathlineto{\pgfqpoint{1.011543in}{0.484854in}}%
\pgfpathlineto{\pgfqpoint{1.047986in}{0.484854in}}%
\pgfpathlineto{\pgfqpoint{1.084430in}{0.484854in}}%
\pgfpathlineto{\pgfqpoint{1.120873in}{0.484854in}}%
\pgfpathlineto{\pgfqpoint{1.157316in}{0.484854in}}%
\pgfpathlineto{\pgfqpoint{1.193760in}{0.484854in}}%
\pgfpathlineto{\pgfqpoint{1.230203in}{0.484854in}}%
\pgfpathlineto{\pgfqpoint{1.266647in}{0.484854in}}%
\pgfpathlineto{\pgfqpoint{1.303090in}{0.484854in}}%
\pgfpathlineto{\pgfqpoint{1.339534in}{0.484854in}}%
\pgfpathlineto{\pgfqpoint{1.375977in}{0.484854in}}%
\pgfpathlineto{\pgfqpoint{1.412421in}{0.484854in}}%
\pgfpathlineto{\pgfqpoint{1.448864in}{0.484854in}}%
\pgfpathlineto{\pgfqpoint{1.485308in}{0.484854in}}%
\pgfpathlineto{\pgfqpoint{1.521751in}{0.484854in}}%
\pgfpathlineto{\pgfqpoint{1.558194in}{0.484854in}}%
\pgfpathlineto{\pgfqpoint{1.594638in}{0.484854in}}%
\pgfpathlineto{\pgfqpoint{1.631081in}{0.484854in}}%
\pgfpathlineto{\pgfqpoint{1.667525in}{0.484854in}}%
\pgfpathlineto{\pgfqpoint{1.703968in}{0.484854in}}%
\pgfpathlineto{\pgfqpoint{1.740412in}{0.484854in}}%
\pgfpathlineto{\pgfqpoint{1.776855in}{0.484854in}}%
\pgfpathlineto{\pgfqpoint{1.813299in}{0.484854in}}%
\pgfpathlineto{\pgfqpoint{1.849742in}{0.484854in}}%
\pgfpathlineto{\pgfqpoint{1.886185in}{0.484854in}}%
\pgfpathlineto{\pgfqpoint{1.922629in}{0.484854in}}%
\pgfpathlineto{\pgfqpoint{1.959072in}{0.484854in}}%
\pgfpathlineto{\pgfqpoint{1.995516in}{0.484854in}}%
\pgfpathlineto{\pgfqpoint{2.031959in}{0.484854in}}%
\pgfpathlineto{\pgfqpoint{2.068403in}{0.484854in}}%
\pgfpathlineto{\pgfqpoint{2.104846in}{0.484854in}}%
\pgfpathlineto{\pgfqpoint{2.141290in}{0.484854in}}%
\pgfpathlineto{\pgfqpoint{2.177733in}{0.484854in}}%
\pgfpathlineto{\pgfqpoint{2.214177in}{0.484854in}}%
\pgfpathlineto{\pgfqpoint{2.250620in}{0.484854in}}%
\pgfpathlineto{\pgfqpoint{2.287063in}{0.484854in}}%
\pgfpathlineto{\pgfqpoint{2.323507in}{0.484854in}}%
\pgfpathlineto{\pgfqpoint{2.359950in}{0.484854in}}%
\pgfpathlineto{\pgfqpoint{2.396394in}{0.484854in}}%
\pgfpathlineto{\pgfqpoint{2.432837in}{0.484854in}}%
\pgfpathlineto{\pgfqpoint{2.469281in}{0.484854in}}%
\pgfpathlineto{\pgfqpoint{2.505724in}{0.484854in}}%
\pgfpathlineto{\pgfqpoint{2.542168in}{0.484854in}}%
\pgfpathlineto{\pgfqpoint{2.578611in}{0.484854in}}%
\pgfpathlineto{\pgfqpoint{2.605944in}{0.484854in}}%
\pgfpathlineto{\pgfqpoint{2.605944in}{0.692867in}}%
\pgfpathlineto{\pgfqpoint{2.605944in}{0.692867in}}%
\pgfpathlineto{\pgfqpoint{2.578611in}{0.692867in}}%
\pgfpathlineto{\pgfqpoint{2.542168in}{0.692867in}}%
\pgfpathlineto{\pgfqpoint{2.505724in}{0.692867in}}%
\pgfpathlineto{\pgfqpoint{2.469281in}{0.692867in}}%
\pgfpathlineto{\pgfqpoint{2.432837in}{0.692867in}}%
\pgfpathlineto{\pgfqpoint{2.396394in}{0.692867in}}%
\pgfpathlineto{\pgfqpoint{2.359950in}{0.692867in}}%
\pgfpathlineto{\pgfqpoint{2.323507in}{0.692867in}}%
\pgfpathlineto{\pgfqpoint{2.287063in}{0.680252in}}%
\pgfpathlineto{\pgfqpoint{2.250620in}{0.680252in}}%
\pgfpathlineto{\pgfqpoint{2.214177in}{0.680252in}}%
\pgfpathlineto{\pgfqpoint{2.177733in}{0.680252in}}%
\pgfpathlineto{\pgfqpoint{2.141290in}{0.680252in}}%
\pgfpathlineto{\pgfqpoint{2.104846in}{0.680252in}}%
\pgfpathlineto{\pgfqpoint{2.068403in}{0.680252in}}%
\pgfpathlineto{\pgfqpoint{2.031959in}{0.680252in}}%
\pgfpathlineto{\pgfqpoint{1.995516in}{0.662383in}}%
\pgfpathlineto{\pgfqpoint{1.959072in}{0.662383in}}%
\pgfpathlineto{\pgfqpoint{1.922629in}{0.662383in}}%
\pgfpathlineto{\pgfqpoint{1.886185in}{0.662383in}}%
\pgfpathlineto{\pgfqpoint{1.849742in}{0.662383in}}%
\pgfpathlineto{\pgfqpoint{1.813299in}{0.662383in}}%
\pgfpathlineto{\pgfqpoint{1.776855in}{0.662383in}}%
\pgfpathlineto{\pgfqpoint{1.740412in}{0.641061in}}%
\pgfpathlineto{\pgfqpoint{1.703968in}{0.641061in}}%
\pgfpathlineto{\pgfqpoint{1.667525in}{0.641061in}}%
\pgfpathlineto{\pgfqpoint{1.631081in}{0.641061in}}%
\pgfpathlineto{\pgfqpoint{1.594638in}{0.641061in}}%
\pgfpathlineto{\pgfqpoint{1.558194in}{0.641061in}}%
\pgfpathlineto{\pgfqpoint{1.521751in}{0.615460in}}%
\pgfpathlineto{\pgfqpoint{1.485308in}{0.615460in}}%
\pgfpathlineto{\pgfqpoint{1.448864in}{0.615460in}}%
\pgfpathlineto{\pgfqpoint{1.412421in}{0.615460in}}%
\pgfpathlineto{\pgfqpoint{1.375977in}{0.615460in}}%
\pgfpathlineto{\pgfqpoint{1.339534in}{0.615460in}}%
\pgfpathlineto{\pgfqpoint{1.303090in}{0.586724in}}%
\pgfpathlineto{\pgfqpoint{1.266647in}{0.586724in}}%
\pgfpathlineto{\pgfqpoint{1.230203in}{0.586724in}}%
\pgfpathlineto{\pgfqpoint{1.193760in}{0.586724in}}%
\pgfpathlineto{\pgfqpoint{1.157316in}{0.554258in}}%
\pgfpathlineto{\pgfqpoint{1.120873in}{0.554258in}}%
\pgfpathlineto{\pgfqpoint{1.084430in}{0.554258in}}%
\pgfpathlineto{\pgfqpoint{1.047986in}{0.520709in}}%
\pgfpathlineto{\pgfqpoint{1.011543in}{0.520709in}}%
\pgfpathlineto{\pgfqpoint{0.975099in}{0.520709in}}%
\pgfpathlineto{\pgfqpoint{0.938656in}{0.484854in}}%
\pgfpathlineto{\pgfqpoint{0.902212in}{0.484854in}}%
\pgfpathlineto{\pgfqpoint{0.865769in}{0.484854in}}%
\pgfpathlineto{\pgfqpoint{0.829325in}{0.484854in}}%
\pgfpathlineto{\pgfqpoint{0.792882in}{0.484854in}}%
\pgfpathlineto{\pgfqpoint{0.756439in}{0.484854in}}%
\pgfpathlineto{\pgfqpoint{0.719995in}{0.484854in}}%
\pgfpathlineto{\pgfqpoint{0.683552in}{0.484854in}}%
\pgfpathlineto{\pgfqpoint{0.647108in}{0.484854in}}%
\pgfpathlineto{\pgfqpoint{0.610665in}{0.484854in}}%
\pgfpathlineto{\pgfqpoint{0.574221in}{0.484854in}}%
\pgfpathlineto{\pgfqpoint{0.537778in}{0.484854in}}%
\pgfpathlineto{\pgfqpoint{0.501334in}{0.484854in}}%
\pgfpathlineto{\pgfqpoint{0.464891in}{0.484854in}}%
\pgfpathlineto{\pgfqpoint{0.428448in}{0.484854in}}%
\pgfpathlineto{\pgfqpoint{0.392004in}{0.484854in}}%
\pgfpathlineto{\pgfqpoint{0.355561in}{0.484854in}}%
\pgfpathlineto{\pgfqpoint{0.319117in}{0.484854in}}%
\pgfpathlineto{\pgfqpoint{0.282674in}{0.484854in}}%
\pgfpathlineto{\pgfqpoint{0.246230in}{0.484854in}}%
\pgfpathlineto{\pgfqpoint{0.209787in}{0.484854in}}%
\pgfpathlineto{\pgfqpoint{0.173343in}{0.484854in}}%
\pgfpathlineto{\pgfqpoint{0.136900in}{0.484854in}}%
\pgfpathlineto{\pgfqpoint{0.100456in}{0.484854in}}%
\pgfpathlineto{\pgfqpoint{0.064013in}{0.484854in}}%
\pgfpathlineto{\pgfqpoint{0.027570in}{0.484854in}}%
\pgfpathlineto{\pgfqpoint{-0.008874in}{0.484854in}}%
\pgfpathlineto{\pgfqpoint{-0.045317in}{0.484854in}}%
\pgfpathlineto{\pgfqpoint{-0.081761in}{0.484854in}}%
\pgfpathlineto{\pgfqpoint{-0.118204in}{0.484854in}}%
\pgfpathlineto{\pgfqpoint{-0.154648in}{0.484854in}}%
\pgfpathlineto{\pgfqpoint{-0.191091in}{0.484854in}}%
\pgfpathlineto{\pgfqpoint{-0.227535in}{0.484854in}}%
\pgfpathlineto{\pgfqpoint{-0.263978in}{0.484854in}}%
\pgfpathlineto{\pgfqpoint{-0.300421in}{0.484854in}}%
\pgfpathlineto{\pgfqpoint{-0.336865in}{0.484854in}}%
\pgfpathlineto{\pgfqpoint{-0.373308in}{0.484854in}}%
\pgfpathlineto{\pgfqpoint{-0.409752in}{0.484854in}}%
\pgfpathlineto{\pgfqpoint{-0.446195in}{0.484854in}}%
\pgfpathlineto{\pgfqpoint{-0.482639in}{0.484854in}}%
\pgfpathlineto{\pgfqpoint{-0.519082in}{0.484854in}}%
\pgfpathlineto{\pgfqpoint{-0.555526in}{0.484854in}}%
\pgfpathlineto{\pgfqpoint{-0.591969in}{0.484854in}}%
\pgfpathlineto{\pgfqpoint{-0.628412in}{0.484854in}}%
\pgfpathlineto{\pgfqpoint{-0.664856in}{0.484854in}}%
\pgfpathlineto{\pgfqpoint{-0.701299in}{0.484854in}}%
\pgfpathlineto{\pgfqpoint{-0.737743in}{0.484854in}}%
\pgfpathlineto{\pgfqpoint{-0.774186in}{0.484854in}}%
\pgfpathlineto{\pgfqpoint{-0.810630in}{0.484854in}}%
\pgfpathlineto{\pgfqpoint{-0.847073in}{0.484854in}}%
\pgfpathlineto{\pgfqpoint{-0.883517in}{0.484854in}}%
\pgfpathlineto{\pgfqpoint{-0.919960in}{0.484854in}}%
\pgfpathlineto{\pgfqpoint{-0.956404in}{0.484854in}}%
\pgfpathlineto{\pgfqpoint{-0.992847in}{0.484854in}}%
\pgfpathlineto{\pgfqpoint{-1.038401in}{0.484854in}}%
\pgfpathclose%
\pgfusepath{fill}%
\end{pgfscope}%
\begin{pgfscope}%
\pgfpathrectangle{\pgfqpoint{0.419337in}{0.484854in}}{\pgfqpoint{2.186607in}{1.629375in}}%
\pgfusepath{clip}%
\pgfsetbuttcap%
\pgfsetroundjoin%
\definecolor{currentfill}{rgb}{1.000000,0.498039,0.054902}%
\pgfsetfillcolor{currentfill}%
\pgfsetlinewidth{0.000000pt}%
\definecolor{currentstroke}{rgb}{0.000000,0.000000,0.000000}%
\pgfsetstrokecolor{currentstroke}%
\pgfsetdash{}{0pt}%
\pgfpathmoveto{\pgfqpoint{-1.038401in}{0.484854in}}%
\pgfpathlineto{\pgfqpoint{-1.038401in}{0.484854in}}%
\pgfpathlineto{\pgfqpoint{-0.992847in}{0.484854in}}%
\pgfpathlineto{\pgfqpoint{-0.956404in}{0.484854in}}%
\pgfpathlineto{\pgfqpoint{-0.919960in}{0.484854in}}%
\pgfpathlineto{\pgfqpoint{-0.883517in}{0.484854in}}%
\pgfpathlineto{\pgfqpoint{-0.847073in}{0.484854in}}%
\pgfpathlineto{\pgfqpoint{-0.810630in}{0.484854in}}%
\pgfpathlineto{\pgfqpoint{-0.774186in}{0.484854in}}%
\pgfpathlineto{\pgfqpoint{-0.737743in}{0.484854in}}%
\pgfpathlineto{\pgfqpoint{-0.701299in}{0.484854in}}%
\pgfpathlineto{\pgfqpoint{-0.664856in}{0.484854in}}%
\pgfpathlineto{\pgfqpoint{-0.628412in}{0.484854in}}%
\pgfpathlineto{\pgfqpoint{-0.591969in}{0.484854in}}%
\pgfpathlineto{\pgfqpoint{-0.555526in}{0.484854in}}%
\pgfpathlineto{\pgfqpoint{-0.519082in}{0.484854in}}%
\pgfpathlineto{\pgfqpoint{-0.482639in}{0.484854in}}%
\pgfpathlineto{\pgfqpoint{-0.446195in}{0.484854in}}%
\pgfpathlineto{\pgfqpoint{-0.409752in}{0.484854in}}%
\pgfpathlineto{\pgfqpoint{-0.373308in}{0.484854in}}%
\pgfpathlineto{\pgfqpoint{-0.336865in}{0.484854in}}%
\pgfpathlineto{\pgfqpoint{-0.300421in}{0.484854in}}%
\pgfpathlineto{\pgfqpoint{-0.263978in}{0.484854in}}%
\pgfpathlineto{\pgfqpoint{-0.227535in}{0.484854in}}%
\pgfpathlineto{\pgfqpoint{-0.191091in}{0.484854in}}%
\pgfpathlineto{\pgfqpoint{-0.154648in}{0.484854in}}%
\pgfpathlineto{\pgfqpoint{-0.118204in}{0.484854in}}%
\pgfpathlineto{\pgfqpoint{-0.081761in}{0.484854in}}%
\pgfpathlineto{\pgfqpoint{-0.045317in}{0.484854in}}%
\pgfpathlineto{\pgfqpoint{-0.008874in}{0.484854in}}%
\pgfpathlineto{\pgfqpoint{0.027570in}{0.484854in}}%
\pgfpathlineto{\pgfqpoint{0.064013in}{0.484854in}}%
\pgfpathlineto{\pgfqpoint{0.100456in}{0.484854in}}%
\pgfpathlineto{\pgfqpoint{0.136900in}{0.484854in}}%
\pgfpathlineto{\pgfqpoint{0.173343in}{0.484854in}}%
\pgfpathlineto{\pgfqpoint{0.209787in}{0.484854in}}%
\pgfpathlineto{\pgfqpoint{0.246230in}{0.484854in}}%
\pgfpathlineto{\pgfqpoint{0.282674in}{0.484854in}}%
\pgfpathlineto{\pgfqpoint{0.319117in}{0.484854in}}%
\pgfpathlineto{\pgfqpoint{0.355561in}{0.484854in}}%
\pgfpathlineto{\pgfqpoint{0.392004in}{0.484854in}}%
\pgfpathlineto{\pgfqpoint{0.428448in}{0.484854in}}%
\pgfpathlineto{\pgfqpoint{0.464891in}{0.484854in}}%
\pgfpathlineto{\pgfqpoint{0.501334in}{0.484854in}}%
\pgfpathlineto{\pgfqpoint{0.537778in}{0.484854in}}%
\pgfpathlineto{\pgfqpoint{0.574221in}{0.484854in}}%
\pgfpathlineto{\pgfqpoint{0.610665in}{0.484854in}}%
\pgfpathlineto{\pgfqpoint{0.647108in}{0.484854in}}%
\pgfpathlineto{\pgfqpoint{0.683552in}{0.484854in}}%
\pgfpathlineto{\pgfqpoint{0.719995in}{0.484854in}}%
\pgfpathlineto{\pgfqpoint{0.756439in}{0.484854in}}%
\pgfpathlineto{\pgfqpoint{0.792882in}{0.484854in}}%
\pgfpathlineto{\pgfqpoint{0.829325in}{0.484854in}}%
\pgfpathlineto{\pgfqpoint{0.865769in}{0.484854in}}%
\pgfpathlineto{\pgfqpoint{0.902212in}{0.484854in}}%
\pgfpathlineto{\pgfqpoint{0.938656in}{0.484854in}}%
\pgfpathlineto{\pgfqpoint{0.975099in}{0.520709in}}%
\pgfpathlineto{\pgfqpoint{1.011543in}{0.520709in}}%
\pgfpathlineto{\pgfqpoint{1.047986in}{0.520709in}}%
\pgfpathlineto{\pgfqpoint{1.084430in}{0.554258in}}%
\pgfpathlineto{\pgfqpoint{1.120873in}{0.554258in}}%
\pgfpathlineto{\pgfqpoint{1.157316in}{0.554258in}}%
\pgfpathlineto{\pgfqpoint{1.193760in}{0.586724in}}%
\pgfpathlineto{\pgfqpoint{1.230203in}{0.586724in}}%
\pgfpathlineto{\pgfqpoint{1.266647in}{0.586724in}}%
\pgfpathlineto{\pgfqpoint{1.303090in}{0.586724in}}%
\pgfpathlineto{\pgfqpoint{1.339534in}{0.615460in}}%
\pgfpathlineto{\pgfqpoint{1.375977in}{0.615460in}}%
\pgfpathlineto{\pgfqpoint{1.412421in}{0.615460in}}%
\pgfpathlineto{\pgfqpoint{1.448864in}{0.615460in}}%
\pgfpathlineto{\pgfqpoint{1.485308in}{0.615460in}}%
\pgfpathlineto{\pgfqpoint{1.521751in}{0.615460in}}%
\pgfpathlineto{\pgfqpoint{1.558194in}{0.641061in}}%
\pgfpathlineto{\pgfqpoint{1.594638in}{0.641061in}}%
\pgfpathlineto{\pgfqpoint{1.631081in}{0.641061in}}%
\pgfpathlineto{\pgfqpoint{1.667525in}{0.641061in}}%
\pgfpathlineto{\pgfqpoint{1.703968in}{0.641061in}}%
\pgfpathlineto{\pgfqpoint{1.740412in}{0.641061in}}%
\pgfpathlineto{\pgfqpoint{1.776855in}{0.662383in}}%
\pgfpathlineto{\pgfqpoint{1.813299in}{0.662383in}}%
\pgfpathlineto{\pgfqpoint{1.849742in}{0.662383in}}%
\pgfpathlineto{\pgfqpoint{1.886185in}{0.662383in}}%
\pgfpathlineto{\pgfqpoint{1.922629in}{0.662383in}}%
\pgfpathlineto{\pgfqpoint{1.959072in}{0.662383in}}%
\pgfpathlineto{\pgfqpoint{1.995516in}{0.662383in}}%
\pgfpathlineto{\pgfqpoint{2.031959in}{0.680252in}}%
\pgfpathlineto{\pgfqpoint{2.068403in}{0.680252in}}%
\pgfpathlineto{\pgfqpoint{2.104846in}{0.680252in}}%
\pgfpathlineto{\pgfqpoint{2.141290in}{0.680252in}}%
\pgfpathlineto{\pgfqpoint{2.177733in}{0.680252in}}%
\pgfpathlineto{\pgfqpoint{2.214177in}{0.680252in}}%
\pgfpathlineto{\pgfqpoint{2.250620in}{0.680252in}}%
\pgfpathlineto{\pgfqpoint{2.287063in}{0.680252in}}%
\pgfpathlineto{\pgfqpoint{2.323507in}{0.692867in}}%
\pgfpathlineto{\pgfqpoint{2.359950in}{0.692867in}}%
\pgfpathlineto{\pgfqpoint{2.396394in}{0.692867in}}%
\pgfpathlineto{\pgfqpoint{2.432837in}{0.692867in}}%
\pgfpathlineto{\pgfqpoint{2.469281in}{0.692867in}}%
\pgfpathlineto{\pgfqpoint{2.505724in}{0.692867in}}%
\pgfpathlineto{\pgfqpoint{2.542168in}{0.692867in}}%
\pgfpathlineto{\pgfqpoint{2.578611in}{0.692867in}}%
\pgfpathlineto{\pgfqpoint{2.605944in}{0.692867in}}%
\pgfpathlineto{\pgfqpoint{2.605944in}{0.838464in}}%
\pgfpathlineto{\pgfqpoint{2.605944in}{0.838464in}}%
\pgfpathlineto{\pgfqpoint{2.578611in}{0.838464in}}%
\pgfpathlineto{\pgfqpoint{2.542168in}{0.838464in}}%
\pgfpathlineto{\pgfqpoint{2.505724in}{0.838464in}}%
\pgfpathlineto{\pgfqpoint{2.469281in}{0.838464in}}%
\pgfpathlineto{\pgfqpoint{2.432837in}{0.838464in}}%
\pgfpathlineto{\pgfqpoint{2.396394in}{0.838464in}}%
\pgfpathlineto{\pgfqpoint{2.359950in}{0.838464in}}%
\pgfpathlineto{\pgfqpoint{2.323507in}{0.838464in}}%
\pgfpathlineto{\pgfqpoint{2.287063in}{0.813908in}}%
\pgfpathlineto{\pgfqpoint{2.250620in}{0.813908in}}%
\pgfpathlineto{\pgfqpoint{2.214177in}{0.813908in}}%
\pgfpathlineto{\pgfqpoint{2.177733in}{0.813908in}}%
\pgfpathlineto{\pgfqpoint{2.141290in}{0.813908in}}%
\pgfpathlineto{\pgfqpoint{2.104846in}{0.813908in}}%
\pgfpathlineto{\pgfqpoint{2.068403in}{0.813908in}}%
\pgfpathlineto{\pgfqpoint{2.031959in}{0.813908in}}%
\pgfpathlineto{\pgfqpoint{1.995516in}{0.781598in}}%
\pgfpathlineto{\pgfqpoint{1.959072in}{0.781598in}}%
\pgfpathlineto{\pgfqpoint{1.922629in}{0.781598in}}%
\pgfpathlineto{\pgfqpoint{1.886185in}{0.781598in}}%
\pgfpathlineto{\pgfqpoint{1.849742in}{0.781598in}}%
\pgfpathlineto{\pgfqpoint{1.813299in}{0.781598in}}%
\pgfpathlineto{\pgfqpoint{1.776855in}{0.781598in}}%
\pgfpathlineto{\pgfqpoint{1.740412in}{0.744659in}}%
\pgfpathlineto{\pgfqpoint{1.703968in}{0.744659in}}%
\pgfpathlineto{\pgfqpoint{1.667525in}{0.744659in}}%
\pgfpathlineto{\pgfqpoint{1.631081in}{0.744659in}}%
\pgfpathlineto{\pgfqpoint{1.594638in}{0.744659in}}%
\pgfpathlineto{\pgfqpoint{1.558194in}{0.744659in}}%
\pgfpathlineto{\pgfqpoint{1.521751in}{0.701874in}}%
\pgfpathlineto{\pgfqpoint{1.485308in}{0.701874in}}%
\pgfpathlineto{\pgfqpoint{1.448864in}{0.701874in}}%
\pgfpathlineto{\pgfqpoint{1.412421in}{0.701874in}}%
\pgfpathlineto{\pgfqpoint{1.375977in}{0.701874in}}%
\pgfpathlineto{\pgfqpoint{1.339534in}{0.701874in}}%
\pgfpathlineto{\pgfqpoint{1.303090in}{0.654119in}}%
\pgfpathlineto{\pgfqpoint{1.266647in}{0.654119in}}%
\pgfpathlineto{\pgfqpoint{1.230203in}{0.654119in}}%
\pgfpathlineto{\pgfqpoint{1.193760in}{0.654119in}}%
\pgfpathlineto{\pgfqpoint{1.157316in}{0.600453in}}%
\pgfpathlineto{\pgfqpoint{1.120873in}{0.600453in}}%
\pgfpathlineto{\pgfqpoint{1.084430in}{0.600453in}}%
\pgfpathlineto{\pgfqpoint{1.047986in}{0.544723in}}%
\pgfpathlineto{\pgfqpoint{1.011543in}{0.544723in}}%
\pgfpathlineto{\pgfqpoint{0.975099in}{0.544723in}}%
\pgfpathlineto{\pgfqpoint{0.938656in}{0.484854in}}%
\pgfpathlineto{\pgfqpoint{0.902212in}{0.484854in}}%
\pgfpathlineto{\pgfqpoint{0.865769in}{0.484854in}}%
\pgfpathlineto{\pgfqpoint{0.829325in}{0.484854in}}%
\pgfpathlineto{\pgfqpoint{0.792882in}{0.484854in}}%
\pgfpathlineto{\pgfqpoint{0.756439in}{0.484854in}}%
\pgfpathlineto{\pgfqpoint{0.719995in}{0.484854in}}%
\pgfpathlineto{\pgfqpoint{0.683552in}{0.484854in}}%
\pgfpathlineto{\pgfqpoint{0.647108in}{0.484854in}}%
\pgfpathlineto{\pgfqpoint{0.610665in}{0.484854in}}%
\pgfpathlineto{\pgfqpoint{0.574221in}{0.484854in}}%
\pgfpathlineto{\pgfqpoint{0.537778in}{0.484854in}}%
\pgfpathlineto{\pgfqpoint{0.501334in}{0.484854in}}%
\pgfpathlineto{\pgfqpoint{0.464891in}{0.484854in}}%
\pgfpathlineto{\pgfqpoint{0.428448in}{0.484854in}}%
\pgfpathlineto{\pgfqpoint{0.392004in}{0.484854in}}%
\pgfpathlineto{\pgfqpoint{0.355561in}{0.484854in}}%
\pgfpathlineto{\pgfqpoint{0.319117in}{0.484854in}}%
\pgfpathlineto{\pgfqpoint{0.282674in}{0.484854in}}%
\pgfpathlineto{\pgfqpoint{0.246230in}{0.484854in}}%
\pgfpathlineto{\pgfqpoint{0.209787in}{0.484854in}}%
\pgfpathlineto{\pgfqpoint{0.173343in}{0.484854in}}%
\pgfpathlineto{\pgfqpoint{0.136900in}{0.484854in}}%
\pgfpathlineto{\pgfqpoint{0.100456in}{0.484854in}}%
\pgfpathlineto{\pgfqpoint{0.064013in}{0.484854in}}%
\pgfpathlineto{\pgfqpoint{0.027570in}{0.484854in}}%
\pgfpathlineto{\pgfqpoint{-0.008874in}{0.484854in}}%
\pgfpathlineto{\pgfqpoint{-0.045317in}{0.484854in}}%
\pgfpathlineto{\pgfqpoint{-0.081761in}{0.484854in}}%
\pgfpathlineto{\pgfqpoint{-0.118204in}{0.484854in}}%
\pgfpathlineto{\pgfqpoint{-0.154648in}{0.484854in}}%
\pgfpathlineto{\pgfqpoint{-0.191091in}{0.484854in}}%
\pgfpathlineto{\pgfqpoint{-0.227535in}{0.484854in}}%
\pgfpathlineto{\pgfqpoint{-0.263978in}{0.484854in}}%
\pgfpathlineto{\pgfqpoint{-0.300421in}{0.484854in}}%
\pgfpathlineto{\pgfqpoint{-0.336865in}{0.484854in}}%
\pgfpathlineto{\pgfqpoint{-0.373308in}{0.484854in}}%
\pgfpathlineto{\pgfqpoint{-0.409752in}{0.484854in}}%
\pgfpathlineto{\pgfqpoint{-0.446195in}{0.484854in}}%
\pgfpathlineto{\pgfqpoint{-0.482639in}{0.484854in}}%
\pgfpathlineto{\pgfqpoint{-0.519082in}{0.484854in}}%
\pgfpathlineto{\pgfqpoint{-0.555526in}{0.484854in}}%
\pgfpathlineto{\pgfqpoint{-0.591969in}{0.484854in}}%
\pgfpathlineto{\pgfqpoint{-0.628412in}{0.484854in}}%
\pgfpathlineto{\pgfqpoint{-0.664856in}{0.484854in}}%
\pgfpathlineto{\pgfqpoint{-0.701299in}{0.484854in}}%
\pgfpathlineto{\pgfqpoint{-0.737743in}{0.484854in}}%
\pgfpathlineto{\pgfqpoint{-0.774186in}{0.484854in}}%
\pgfpathlineto{\pgfqpoint{-0.810630in}{0.484854in}}%
\pgfpathlineto{\pgfqpoint{-0.847073in}{0.484854in}}%
\pgfpathlineto{\pgfqpoint{-0.883517in}{0.484854in}}%
\pgfpathlineto{\pgfqpoint{-0.919960in}{0.484854in}}%
\pgfpathlineto{\pgfqpoint{-0.956404in}{0.484854in}}%
\pgfpathlineto{\pgfqpoint{-0.992847in}{0.484854in}}%
\pgfpathlineto{\pgfqpoint{-1.038401in}{0.484854in}}%
\pgfpathclose%
\pgfusepath{fill}%
\end{pgfscope}%
\begin{pgfscope}%
\pgfpathrectangle{\pgfqpoint{0.419337in}{0.484854in}}{\pgfqpoint{2.186607in}{1.629375in}}%
\pgfusepath{clip}%
\pgfsetbuttcap%
\pgfsetroundjoin%
\definecolor{currentfill}{rgb}{0.172549,0.627451,0.172549}%
\pgfsetfillcolor{currentfill}%
\pgfsetlinewidth{0.000000pt}%
\definecolor{currentstroke}{rgb}{0.000000,0.000000,0.000000}%
\pgfsetstrokecolor{currentstroke}%
\pgfsetdash{}{0pt}%
\pgfpathmoveto{\pgfqpoint{-1.038401in}{0.484854in}}%
\pgfpathlineto{\pgfqpoint{-1.038401in}{0.484854in}}%
\pgfpathlineto{\pgfqpoint{-0.992847in}{0.484854in}}%
\pgfpathlineto{\pgfqpoint{-0.956404in}{0.484854in}}%
\pgfpathlineto{\pgfqpoint{-0.919960in}{0.484854in}}%
\pgfpathlineto{\pgfqpoint{-0.883517in}{0.484854in}}%
\pgfpathlineto{\pgfqpoint{-0.847073in}{0.484854in}}%
\pgfpathlineto{\pgfqpoint{-0.810630in}{0.484854in}}%
\pgfpathlineto{\pgfqpoint{-0.774186in}{0.484854in}}%
\pgfpathlineto{\pgfqpoint{-0.737743in}{0.484854in}}%
\pgfpathlineto{\pgfqpoint{-0.701299in}{0.484854in}}%
\pgfpathlineto{\pgfqpoint{-0.664856in}{0.484854in}}%
\pgfpathlineto{\pgfqpoint{-0.628412in}{0.484854in}}%
\pgfpathlineto{\pgfqpoint{-0.591969in}{0.484854in}}%
\pgfpathlineto{\pgfqpoint{-0.555526in}{0.484854in}}%
\pgfpathlineto{\pgfqpoint{-0.519082in}{0.484854in}}%
\pgfpathlineto{\pgfqpoint{-0.482639in}{0.484854in}}%
\pgfpathlineto{\pgfqpoint{-0.446195in}{0.484854in}}%
\pgfpathlineto{\pgfqpoint{-0.409752in}{0.484854in}}%
\pgfpathlineto{\pgfqpoint{-0.373308in}{0.484854in}}%
\pgfpathlineto{\pgfqpoint{-0.336865in}{0.484854in}}%
\pgfpathlineto{\pgfqpoint{-0.300421in}{0.484854in}}%
\pgfpathlineto{\pgfqpoint{-0.263978in}{0.484854in}}%
\pgfpathlineto{\pgfqpoint{-0.227535in}{0.484854in}}%
\pgfpathlineto{\pgfqpoint{-0.191091in}{0.484854in}}%
\pgfpathlineto{\pgfqpoint{-0.154648in}{0.484854in}}%
\pgfpathlineto{\pgfqpoint{-0.118204in}{0.484854in}}%
\pgfpathlineto{\pgfqpoint{-0.081761in}{0.484854in}}%
\pgfpathlineto{\pgfqpoint{-0.045317in}{0.484854in}}%
\pgfpathlineto{\pgfqpoint{-0.008874in}{0.484854in}}%
\pgfpathlineto{\pgfqpoint{0.027570in}{0.484854in}}%
\pgfpathlineto{\pgfqpoint{0.064013in}{0.484854in}}%
\pgfpathlineto{\pgfqpoint{0.100456in}{0.484854in}}%
\pgfpathlineto{\pgfqpoint{0.136900in}{0.484854in}}%
\pgfpathlineto{\pgfqpoint{0.173343in}{0.484854in}}%
\pgfpathlineto{\pgfqpoint{0.209787in}{0.484854in}}%
\pgfpathlineto{\pgfqpoint{0.246230in}{0.484854in}}%
\pgfpathlineto{\pgfqpoint{0.282674in}{0.484854in}}%
\pgfpathlineto{\pgfqpoint{0.319117in}{0.484854in}}%
\pgfpathlineto{\pgfqpoint{0.355561in}{0.484854in}}%
\pgfpathlineto{\pgfqpoint{0.392004in}{0.484854in}}%
\pgfpathlineto{\pgfqpoint{0.428448in}{0.484854in}}%
\pgfpathlineto{\pgfqpoint{0.464891in}{0.484854in}}%
\pgfpathlineto{\pgfqpoint{0.501334in}{0.484854in}}%
\pgfpathlineto{\pgfqpoint{0.537778in}{0.484854in}}%
\pgfpathlineto{\pgfqpoint{0.574221in}{0.484854in}}%
\pgfpathlineto{\pgfqpoint{0.610665in}{0.484854in}}%
\pgfpathlineto{\pgfqpoint{0.647108in}{0.484854in}}%
\pgfpathlineto{\pgfqpoint{0.683552in}{0.484854in}}%
\pgfpathlineto{\pgfqpoint{0.719995in}{0.484854in}}%
\pgfpathlineto{\pgfqpoint{0.756439in}{0.484854in}}%
\pgfpathlineto{\pgfqpoint{0.792882in}{0.484854in}}%
\pgfpathlineto{\pgfqpoint{0.829325in}{0.484854in}}%
\pgfpathlineto{\pgfqpoint{0.865769in}{0.484854in}}%
\pgfpathlineto{\pgfqpoint{0.902212in}{0.484854in}}%
\pgfpathlineto{\pgfqpoint{0.938656in}{0.484854in}}%
\pgfpathlineto{\pgfqpoint{0.975099in}{0.544723in}}%
\pgfpathlineto{\pgfqpoint{1.011543in}{0.544723in}}%
\pgfpathlineto{\pgfqpoint{1.047986in}{0.544723in}}%
\pgfpathlineto{\pgfqpoint{1.084430in}{0.600453in}}%
\pgfpathlineto{\pgfqpoint{1.120873in}{0.600453in}}%
\pgfpathlineto{\pgfqpoint{1.157316in}{0.600453in}}%
\pgfpathlineto{\pgfqpoint{1.193760in}{0.654119in}}%
\pgfpathlineto{\pgfqpoint{1.230203in}{0.654119in}}%
\pgfpathlineto{\pgfqpoint{1.266647in}{0.654119in}}%
\pgfpathlineto{\pgfqpoint{1.303090in}{0.654119in}}%
\pgfpathlineto{\pgfqpoint{1.339534in}{0.701874in}}%
\pgfpathlineto{\pgfqpoint{1.375977in}{0.701874in}}%
\pgfpathlineto{\pgfqpoint{1.412421in}{0.701874in}}%
\pgfpathlineto{\pgfqpoint{1.448864in}{0.701874in}}%
\pgfpathlineto{\pgfqpoint{1.485308in}{0.701874in}}%
\pgfpathlineto{\pgfqpoint{1.521751in}{0.701874in}}%
\pgfpathlineto{\pgfqpoint{1.558194in}{0.744659in}}%
\pgfpathlineto{\pgfqpoint{1.594638in}{0.744659in}}%
\pgfpathlineto{\pgfqpoint{1.631081in}{0.744659in}}%
\pgfpathlineto{\pgfqpoint{1.667525in}{0.744659in}}%
\pgfpathlineto{\pgfqpoint{1.703968in}{0.744659in}}%
\pgfpathlineto{\pgfqpoint{1.740412in}{0.744659in}}%
\pgfpathlineto{\pgfqpoint{1.776855in}{0.781598in}}%
\pgfpathlineto{\pgfqpoint{1.813299in}{0.781598in}}%
\pgfpathlineto{\pgfqpoint{1.849742in}{0.781598in}}%
\pgfpathlineto{\pgfqpoint{1.886185in}{0.781598in}}%
\pgfpathlineto{\pgfqpoint{1.922629in}{0.781598in}}%
\pgfpathlineto{\pgfqpoint{1.959072in}{0.781598in}}%
\pgfpathlineto{\pgfqpoint{1.995516in}{0.781598in}}%
\pgfpathlineto{\pgfqpoint{2.031959in}{0.813908in}}%
\pgfpathlineto{\pgfqpoint{2.068403in}{0.813908in}}%
\pgfpathlineto{\pgfqpoint{2.104846in}{0.813908in}}%
\pgfpathlineto{\pgfqpoint{2.141290in}{0.813908in}}%
\pgfpathlineto{\pgfqpoint{2.177733in}{0.813908in}}%
\pgfpathlineto{\pgfqpoint{2.214177in}{0.813908in}}%
\pgfpathlineto{\pgfqpoint{2.250620in}{0.813908in}}%
\pgfpathlineto{\pgfqpoint{2.287063in}{0.813908in}}%
\pgfpathlineto{\pgfqpoint{2.323507in}{0.838464in}}%
\pgfpathlineto{\pgfqpoint{2.359950in}{0.838464in}}%
\pgfpathlineto{\pgfqpoint{2.396394in}{0.838464in}}%
\pgfpathlineto{\pgfqpoint{2.432837in}{0.838464in}}%
\pgfpathlineto{\pgfqpoint{2.469281in}{0.838464in}}%
\pgfpathlineto{\pgfqpoint{2.505724in}{0.838464in}}%
\pgfpathlineto{\pgfqpoint{2.542168in}{0.838464in}}%
\pgfpathlineto{\pgfqpoint{2.578611in}{0.838464in}}%
\pgfpathlineto{\pgfqpoint{2.605944in}{0.838464in}}%
\pgfpathlineto{\pgfqpoint{2.605944in}{1.277046in}}%
\pgfpathlineto{\pgfqpoint{2.605944in}{1.277046in}}%
\pgfpathlineto{\pgfqpoint{2.578611in}{1.277046in}}%
\pgfpathlineto{\pgfqpoint{2.542168in}{1.277046in}}%
\pgfpathlineto{\pgfqpoint{2.505724in}{1.277046in}}%
\pgfpathlineto{\pgfqpoint{2.469281in}{1.277046in}}%
\pgfpathlineto{\pgfqpoint{2.432837in}{1.277046in}}%
\pgfpathlineto{\pgfqpoint{2.396394in}{1.277046in}}%
\pgfpathlineto{\pgfqpoint{2.359950in}{1.277046in}}%
\pgfpathlineto{\pgfqpoint{2.323507in}{1.277046in}}%
\pgfpathlineto{\pgfqpoint{2.287063in}{1.210388in}}%
\pgfpathlineto{\pgfqpoint{2.250620in}{1.210388in}}%
\pgfpathlineto{\pgfqpoint{2.214177in}{1.210388in}}%
\pgfpathlineto{\pgfqpoint{2.177733in}{1.210388in}}%
\pgfpathlineto{\pgfqpoint{2.141290in}{1.210388in}}%
\pgfpathlineto{\pgfqpoint{2.104846in}{1.210388in}}%
\pgfpathlineto{\pgfqpoint{2.068403in}{1.210388in}}%
\pgfpathlineto{\pgfqpoint{2.031959in}{1.210388in}}%
\pgfpathlineto{\pgfqpoint{1.995516in}{1.128973in}}%
\pgfpathlineto{\pgfqpoint{1.959072in}{1.128973in}}%
\pgfpathlineto{\pgfqpoint{1.922629in}{1.128973in}}%
\pgfpathlineto{\pgfqpoint{1.886185in}{1.128973in}}%
\pgfpathlineto{\pgfqpoint{1.849742in}{1.128973in}}%
\pgfpathlineto{\pgfqpoint{1.813299in}{1.128973in}}%
\pgfpathlineto{\pgfqpoint{1.776855in}{1.128973in}}%
\pgfpathlineto{\pgfqpoint{1.740412in}{1.040822in}}%
\pgfpathlineto{\pgfqpoint{1.703968in}{1.040822in}}%
\pgfpathlineto{\pgfqpoint{1.667525in}{1.040822in}}%
\pgfpathlineto{\pgfqpoint{1.631081in}{1.040822in}}%
\pgfpathlineto{\pgfqpoint{1.594638in}{1.040822in}}%
\pgfpathlineto{\pgfqpoint{1.558194in}{1.040822in}}%
\pgfpathlineto{\pgfqpoint{1.521751in}{0.944939in}}%
\pgfpathlineto{\pgfqpoint{1.485308in}{0.944939in}}%
\pgfpathlineto{\pgfqpoint{1.448864in}{0.944939in}}%
\pgfpathlineto{\pgfqpoint{1.412421in}{0.944939in}}%
\pgfpathlineto{\pgfqpoint{1.375977in}{0.944939in}}%
\pgfpathlineto{\pgfqpoint{1.339534in}{0.944939in}}%
\pgfpathlineto{\pgfqpoint{1.303090in}{0.840292in}}%
\pgfpathlineto{\pgfqpoint{1.266647in}{0.840292in}}%
\pgfpathlineto{\pgfqpoint{1.230203in}{0.840292in}}%
\pgfpathlineto{\pgfqpoint{1.193760in}{0.840292in}}%
\pgfpathlineto{\pgfqpoint{1.157316in}{0.725971in}}%
\pgfpathlineto{\pgfqpoint{1.120873in}{0.725971in}}%
\pgfpathlineto{\pgfqpoint{1.084430in}{0.725971in}}%
\pgfpathlineto{\pgfqpoint{1.047986in}{0.608760in}}%
\pgfpathlineto{\pgfqpoint{1.011543in}{0.608760in}}%
\pgfpathlineto{\pgfqpoint{0.975099in}{0.608760in}}%
\pgfpathlineto{\pgfqpoint{0.938656in}{0.484854in}}%
\pgfpathlineto{\pgfqpoint{0.902212in}{0.484854in}}%
\pgfpathlineto{\pgfqpoint{0.865769in}{0.484854in}}%
\pgfpathlineto{\pgfqpoint{0.829325in}{0.484854in}}%
\pgfpathlineto{\pgfqpoint{0.792882in}{0.484854in}}%
\pgfpathlineto{\pgfqpoint{0.756439in}{0.484854in}}%
\pgfpathlineto{\pgfqpoint{0.719995in}{0.484854in}}%
\pgfpathlineto{\pgfqpoint{0.683552in}{0.484854in}}%
\pgfpathlineto{\pgfqpoint{0.647108in}{0.484854in}}%
\pgfpathlineto{\pgfqpoint{0.610665in}{0.484854in}}%
\pgfpathlineto{\pgfqpoint{0.574221in}{0.484854in}}%
\pgfpathlineto{\pgfqpoint{0.537778in}{0.484854in}}%
\pgfpathlineto{\pgfqpoint{0.501334in}{0.484854in}}%
\pgfpathlineto{\pgfqpoint{0.464891in}{0.484854in}}%
\pgfpathlineto{\pgfqpoint{0.428448in}{0.484854in}}%
\pgfpathlineto{\pgfqpoint{0.392004in}{0.484854in}}%
\pgfpathlineto{\pgfqpoint{0.355561in}{0.484854in}}%
\pgfpathlineto{\pgfqpoint{0.319117in}{0.484854in}}%
\pgfpathlineto{\pgfqpoint{0.282674in}{0.484854in}}%
\pgfpathlineto{\pgfqpoint{0.246230in}{0.484854in}}%
\pgfpathlineto{\pgfqpoint{0.209787in}{0.484854in}}%
\pgfpathlineto{\pgfqpoint{0.173343in}{0.484854in}}%
\pgfpathlineto{\pgfqpoint{0.136900in}{0.484854in}}%
\pgfpathlineto{\pgfqpoint{0.100456in}{0.484854in}}%
\pgfpathlineto{\pgfqpoint{0.064013in}{0.484854in}}%
\pgfpathlineto{\pgfqpoint{0.027570in}{0.484854in}}%
\pgfpathlineto{\pgfqpoint{-0.008874in}{0.484854in}}%
\pgfpathlineto{\pgfqpoint{-0.045317in}{0.484854in}}%
\pgfpathlineto{\pgfqpoint{-0.081761in}{0.484854in}}%
\pgfpathlineto{\pgfqpoint{-0.118204in}{0.484854in}}%
\pgfpathlineto{\pgfqpoint{-0.154648in}{0.484854in}}%
\pgfpathlineto{\pgfqpoint{-0.191091in}{0.484854in}}%
\pgfpathlineto{\pgfqpoint{-0.227535in}{0.484854in}}%
\pgfpathlineto{\pgfqpoint{-0.263978in}{0.484854in}}%
\pgfpathlineto{\pgfqpoint{-0.300421in}{0.484854in}}%
\pgfpathlineto{\pgfqpoint{-0.336865in}{0.484854in}}%
\pgfpathlineto{\pgfqpoint{-0.373308in}{0.484854in}}%
\pgfpathlineto{\pgfqpoint{-0.409752in}{0.484854in}}%
\pgfpathlineto{\pgfqpoint{-0.446195in}{0.484854in}}%
\pgfpathlineto{\pgfqpoint{-0.482639in}{0.484854in}}%
\pgfpathlineto{\pgfqpoint{-0.519082in}{0.484854in}}%
\pgfpathlineto{\pgfqpoint{-0.555526in}{0.484854in}}%
\pgfpathlineto{\pgfqpoint{-0.591969in}{0.484854in}}%
\pgfpathlineto{\pgfqpoint{-0.628412in}{0.484854in}}%
\pgfpathlineto{\pgfqpoint{-0.664856in}{0.484854in}}%
\pgfpathlineto{\pgfqpoint{-0.701299in}{0.484854in}}%
\pgfpathlineto{\pgfqpoint{-0.737743in}{0.484854in}}%
\pgfpathlineto{\pgfqpoint{-0.774186in}{0.484854in}}%
\pgfpathlineto{\pgfqpoint{-0.810630in}{0.484854in}}%
\pgfpathlineto{\pgfqpoint{-0.847073in}{0.484854in}}%
\pgfpathlineto{\pgfqpoint{-0.883517in}{0.484854in}}%
\pgfpathlineto{\pgfqpoint{-0.919960in}{0.484854in}}%
\pgfpathlineto{\pgfqpoint{-0.956404in}{0.484854in}}%
\pgfpathlineto{\pgfqpoint{-0.992847in}{0.484854in}}%
\pgfpathlineto{\pgfqpoint{-1.038401in}{0.484854in}}%
\pgfpathclose%
\pgfusepath{fill}%
\end{pgfscope}%
\begin{pgfscope}%
\pgfpathrectangle{\pgfqpoint{0.419337in}{0.484854in}}{\pgfqpoint{2.186607in}{1.629375in}}%
\pgfusepath{clip}%
\pgfsetbuttcap%
\pgfsetroundjoin%
\definecolor{currentfill}{rgb}{0.839216,0.152941,0.156863}%
\pgfsetfillcolor{currentfill}%
\pgfsetlinewidth{0.000000pt}%
\definecolor{currentstroke}{rgb}{0.000000,0.000000,0.000000}%
\pgfsetstrokecolor{currentstroke}%
\pgfsetdash{}{0pt}%
\pgfpathmoveto{\pgfqpoint{-1.038401in}{0.484854in}}%
\pgfpathlineto{\pgfqpoint{-1.038401in}{0.484854in}}%
\pgfpathlineto{\pgfqpoint{-0.992847in}{0.484854in}}%
\pgfpathlineto{\pgfqpoint{-0.956404in}{0.484854in}}%
\pgfpathlineto{\pgfqpoint{-0.919960in}{0.484854in}}%
\pgfpathlineto{\pgfqpoint{-0.883517in}{0.484854in}}%
\pgfpathlineto{\pgfqpoint{-0.847073in}{0.484854in}}%
\pgfpathlineto{\pgfqpoint{-0.810630in}{0.484854in}}%
\pgfpathlineto{\pgfqpoint{-0.774186in}{0.484854in}}%
\pgfpathlineto{\pgfqpoint{-0.737743in}{0.484854in}}%
\pgfpathlineto{\pgfqpoint{-0.701299in}{0.484854in}}%
\pgfpathlineto{\pgfqpoint{-0.664856in}{0.484854in}}%
\pgfpathlineto{\pgfqpoint{-0.628412in}{0.484854in}}%
\pgfpathlineto{\pgfqpoint{-0.591969in}{0.484854in}}%
\pgfpathlineto{\pgfqpoint{-0.555526in}{0.484854in}}%
\pgfpathlineto{\pgfqpoint{-0.519082in}{0.484854in}}%
\pgfpathlineto{\pgfqpoint{-0.482639in}{0.484854in}}%
\pgfpathlineto{\pgfqpoint{-0.446195in}{0.484854in}}%
\pgfpathlineto{\pgfqpoint{-0.409752in}{0.484854in}}%
\pgfpathlineto{\pgfqpoint{-0.373308in}{0.484854in}}%
\pgfpathlineto{\pgfqpoint{-0.336865in}{0.484854in}}%
\pgfpathlineto{\pgfqpoint{-0.300421in}{0.484854in}}%
\pgfpathlineto{\pgfqpoint{-0.263978in}{0.484854in}}%
\pgfpathlineto{\pgfqpoint{-0.227535in}{0.484854in}}%
\pgfpathlineto{\pgfqpoint{-0.191091in}{0.484854in}}%
\pgfpathlineto{\pgfqpoint{-0.154648in}{0.484854in}}%
\pgfpathlineto{\pgfqpoint{-0.118204in}{0.484854in}}%
\pgfpathlineto{\pgfqpoint{-0.081761in}{0.484854in}}%
\pgfpathlineto{\pgfqpoint{-0.045317in}{0.484854in}}%
\pgfpathlineto{\pgfqpoint{-0.008874in}{0.484854in}}%
\pgfpathlineto{\pgfqpoint{0.027570in}{0.484854in}}%
\pgfpathlineto{\pgfqpoint{0.064013in}{0.484854in}}%
\pgfpathlineto{\pgfqpoint{0.100456in}{0.484854in}}%
\pgfpathlineto{\pgfqpoint{0.136900in}{0.484854in}}%
\pgfpathlineto{\pgfqpoint{0.173343in}{0.484854in}}%
\pgfpathlineto{\pgfqpoint{0.209787in}{0.484854in}}%
\pgfpathlineto{\pgfqpoint{0.246230in}{0.484854in}}%
\pgfpathlineto{\pgfqpoint{0.282674in}{0.484854in}}%
\pgfpathlineto{\pgfqpoint{0.319117in}{0.484854in}}%
\pgfpathlineto{\pgfqpoint{0.355561in}{0.484854in}}%
\pgfpathlineto{\pgfqpoint{0.392004in}{0.484854in}}%
\pgfpathlineto{\pgfqpoint{0.428448in}{0.484854in}}%
\pgfpathlineto{\pgfqpoint{0.464891in}{0.484854in}}%
\pgfpathlineto{\pgfqpoint{0.501334in}{0.484854in}}%
\pgfpathlineto{\pgfqpoint{0.537778in}{0.484854in}}%
\pgfpathlineto{\pgfqpoint{0.574221in}{0.484854in}}%
\pgfpathlineto{\pgfqpoint{0.610665in}{0.484854in}}%
\pgfpathlineto{\pgfqpoint{0.647108in}{0.484854in}}%
\pgfpathlineto{\pgfqpoint{0.683552in}{0.484854in}}%
\pgfpathlineto{\pgfqpoint{0.719995in}{0.484854in}}%
\pgfpathlineto{\pgfqpoint{0.756439in}{0.484854in}}%
\pgfpathlineto{\pgfqpoint{0.792882in}{0.484854in}}%
\pgfpathlineto{\pgfqpoint{0.829325in}{0.484854in}}%
\pgfpathlineto{\pgfqpoint{0.865769in}{0.484854in}}%
\pgfpathlineto{\pgfqpoint{0.902212in}{0.484854in}}%
\pgfpathlineto{\pgfqpoint{0.938656in}{0.484854in}}%
\pgfpathlineto{\pgfqpoint{0.975099in}{0.608760in}}%
\pgfpathlineto{\pgfqpoint{1.011543in}{0.608760in}}%
\pgfpathlineto{\pgfqpoint{1.047986in}{0.608760in}}%
\pgfpathlineto{\pgfqpoint{1.084430in}{0.725971in}}%
\pgfpathlineto{\pgfqpoint{1.120873in}{0.725971in}}%
\pgfpathlineto{\pgfqpoint{1.157316in}{0.725971in}}%
\pgfpathlineto{\pgfqpoint{1.193760in}{0.840292in}}%
\pgfpathlineto{\pgfqpoint{1.230203in}{0.840292in}}%
\pgfpathlineto{\pgfqpoint{1.266647in}{0.840292in}}%
\pgfpathlineto{\pgfqpoint{1.303090in}{0.840292in}}%
\pgfpathlineto{\pgfqpoint{1.339534in}{0.944939in}}%
\pgfpathlineto{\pgfqpoint{1.375977in}{0.944939in}}%
\pgfpathlineto{\pgfqpoint{1.412421in}{0.944939in}}%
\pgfpathlineto{\pgfqpoint{1.448864in}{0.944939in}}%
\pgfpathlineto{\pgfqpoint{1.485308in}{0.944939in}}%
\pgfpathlineto{\pgfqpoint{1.521751in}{0.944939in}}%
\pgfpathlineto{\pgfqpoint{1.558194in}{1.040822in}}%
\pgfpathlineto{\pgfqpoint{1.594638in}{1.040822in}}%
\pgfpathlineto{\pgfqpoint{1.631081in}{1.040822in}}%
\pgfpathlineto{\pgfqpoint{1.667525in}{1.040822in}}%
\pgfpathlineto{\pgfqpoint{1.703968in}{1.040822in}}%
\pgfpathlineto{\pgfqpoint{1.740412in}{1.040822in}}%
\pgfpathlineto{\pgfqpoint{1.776855in}{1.128973in}}%
\pgfpathlineto{\pgfqpoint{1.813299in}{1.128973in}}%
\pgfpathlineto{\pgfqpoint{1.849742in}{1.128973in}}%
\pgfpathlineto{\pgfqpoint{1.886185in}{1.128973in}}%
\pgfpathlineto{\pgfqpoint{1.922629in}{1.128973in}}%
\pgfpathlineto{\pgfqpoint{1.959072in}{1.128973in}}%
\pgfpathlineto{\pgfqpoint{1.995516in}{1.128973in}}%
\pgfpathlineto{\pgfqpoint{2.031959in}{1.210388in}}%
\pgfpathlineto{\pgfqpoint{2.068403in}{1.210388in}}%
\pgfpathlineto{\pgfqpoint{2.104846in}{1.210388in}}%
\pgfpathlineto{\pgfqpoint{2.141290in}{1.210388in}}%
\pgfpathlineto{\pgfqpoint{2.177733in}{1.210388in}}%
\pgfpathlineto{\pgfqpoint{2.214177in}{1.210388in}}%
\pgfpathlineto{\pgfqpoint{2.250620in}{1.210388in}}%
\pgfpathlineto{\pgfqpoint{2.287063in}{1.210388in}}%
\pgfpathlineto{\pgfqpoint{2.323507in}{1.277046in}}%
\pgfpathlineto{\pgfqpoint{2.359950in}{1.277046in}}%
\pgfpathlineto{\pgfqpoint{2.396394in}{1.277046in}}%
\pgfpathlineto{\pgfqpoint{2.432837in}{1.277046in}}%
\pgfpathlineto{\pgfqpoint{2.469281in}{1.277046in}}%
\pgfpathlineto{\pgfqpoint{2.505724in}{1.277046in}}%
\pgfpathlineto{\pgfqpoint{2.542168in}{1.277046in}}%
\pgfpathlineto{\pgfqpoint{2.578611in}{1.277046in}}%
\pgfpathlineto{\pgfqpoint{2.605944in}{1.277046in}}%
\pgfpathlineto{\pgfqpoint{2.605944in}{2.043736in}}%
\pgfpathlineto{\pgfqpoint{2.605944in}{2.043736in}}%
\pgfpathlineto{\pgfqpoint{2.578611in}{2.043736in}}%
\pgfpathlineto{\pgfqpoint{2.542168in}{2.043736in}}%
\pgfpathlineto{\pgfqpoint{2.505724in}{2.043736in}}%
\pgfpathlineto{\pgfqpoint{2.469281in}{2.043736in}}%
\pgfpathlineto{\pgfqpoint{2.432837in}{2.043736in}}%
\pgfpathlineto{\pgfqpoint{2.396394in}{2.043736in}}%
\pgfpathlineto{\pgfqpoint{2.359950in}{2.043736in}}%
\pgfpathlineto{\pgfqpoint{2.323507in}{2.043736in}}%
\pgfpathlineto{\pgfqpoint{2.287063in}{1.891558in}}%
\pgfpathlineto{\pgfqpoint{2.250620in}{1.891558in}}%
\pgfpathlineto{\pgfqpoint{2.214177in}{1.891558in}}%
\pgfpathlineto{\pgfqpoint{2.177733in}{1.891558in}}%
\pgfpathlineto{\pgfqpoint{2.141290in}{1.891558in}}%
\pgfpathlineto{\pgfqpoint{2.104846in}{1.891558in}}%
\pgfpathlineto{\pgfqpoint{2.068403in}{1.891558in}}%
\pgfpathlineto{\pgfqpoint{2.031959in}{1.891558in}}%
\pgfpathlineto{\pgfqpoint{1.995516in}{1.717092in}}%
\pgfpathlineto{\pgfqpoint{1.959072in}{1.717092in}}%
\pgfpathlineto{\pgfqpoint{1.922629in}{1.717092in}}%
\pgfpathlineto{\pgfqpoint{1.886185in}{1.717092in}}%
\pgfpathlineto{\pgfqpoint{1.849742in}{1.717092in}}%
\pgfpathlineto{\pgfqpoint{1.813299in}{1.717092in}}%
\pgfpathlineto{\pgfqpoint{1.776855in}{1.717092in}}%
\pgfpathlineto{\pgfqpoint{1.740412in}{1.536816in}}%
\pgfpathlineto{\pgfqpoint{1.703968in}{1.536816in}}%
\pgfpathlineto{\pgfqpoint{1.667525in}{1.536816in}}%
\pgfpathlineto{\pgfqpoint{1.631081in}{1.536816in}}%
\pgfpathlineto{\pgfqpoint{1.594638in}{1.536816in}}%
\pgfpathlineto{\pgfqpoint{1.558194in}{1.536816in}}%
\pgfpathlineto{\pgfqpoint{1.521751in}{1.347993in}}%
\pgfpathlineto{\pgfqpoint{1.485308in}{1.347993in}}%
\pgfpathlineto{\pgfqpoint{1.448864in}{1.347993in}}%
\pgfpathlineto{\pgfqpoint{1.412421in}{1.347993in}}%
\pgfpathlineto{\pgfqpoint{1.375977in}{1.347993in}}%
\pgfpathlineto{\pgfqpoint{1.339534in}{1.347993in}}%
\pgfpathlineto{\pgfqpoint{1.303090in}{1.146903in}}%
\pgfpathlineto{\pgfqpoint{1.266647in}{1.146903in}}%
\pgfpathlineto{\pgfqpoint{1.230203in}{1.146903in}}%
\pgfpathlineto{\pgfqpoint{1.193760in}{1.146903in}}%
\pgfpathlineto{\pgfqpoint{1.157316in}{0.931426in}}%
\pgfpathlineto{\pgfqpoint{1.120873in}{0.931426in}}%
\pgfpathlineto{\pgfqpoint{1.084430in}{0.931426in}}%
\pgfpathlineto{\pgfqpoint{1.047986in}{0.713320in}}%
\pgfpathlineto{\pgfqpoint{1.011543in}{0.713320in}}%
\pgfpathlineto{\pgfqpoint{0.975099in}{0.713320in}}%
\pgfpathlineto{\pgfqpoint{0.938656in}{0.484854in}}%
\pgfpathlineto{\pgfqpoint{0.902212in}{0.484854in}}%
\pgfpathlineto{\pgfqpoint{0.865769in}{0.484854in}}%
\pgfpathlineto{\pgfqpoint{0.829325in}{0.484854in}}%
\pgfpathlineto{\pgfqpoint{0.792882in}{0.484854in}}%
\pgfpathlineto{\pgfqpoint{0.756439in}{0.484854in}}%
\pgfpathlineto{\pgfqpoint{0.719995in}{0.484854in}}%
\pgfpathlineto{\pgfqpoint{0.683552in}{0.484854in}}%
\pgfpathlineto{\pgfqpoint{0.647108in}{0.484854in}}%
\pgfpathlineto{\pgfqpoint{0.610665in}{0.484854in}}%
\pgfpathlineto{\pgfqpoint{0.574221in}{0.484854in}}%
\pgfpathlineto{\pgfqpoint{0.537778in}{0.484854in}}%
\pgfpathlineto{\pgfqpoint{0.501334in}{0.484854in}}%
\pgfpathlineto{\pgfqpoint{0.464891in}{0.484854in}}%
\pgfpathlineto{\pgfqpoint{0.428448in}{0.484854in}}%
\pgfpathlineto{\pgfqpoint{0.392004in}{0.484854in}}%
\pgfpathlineto{\pgfqpoint{0.355561in}{0.484854in}}%
\pgfpathlineto{\pgfqpoint{0.319117in}{0.484854in}}%
\pgfpathlineto{\pgfqpoint{0.282674in}{0.484854in}}%
\pgfpathlineto{\pgfqpoint{0.246230in}{0.484854in}}%
\pgfpathlineto{\pgfqpoint{0.209787in}{0.484854in}}%
\pgfpathlineto{\pgfqpoint{0.173343in}{0.484854in}}%
\pgfpathlineto{\pgfqpoint{0.136900in}{0.484854in}}%
\pgfpathlineto{\pgfqpoint{0.100456in}{0.484854in}}%
\pgfpathlineto{\pgfqpoint{0.064013in}{0.484854in}}%
\pgfpathlineto{\pgfqpoint{0.027570in}{0.484854in}}%
\pgfpathlineto{\pgfqpoint{-0.008874in}{0.484854in}}%
\pgfpathlineto{\pgfqpoint{-0.045317in}{0.484854in}}%
\pgfpathlineto{\pgfqpoint{-0.081761in}{0.484854in}}%
\pgfpathlineto{\pgfqpoint{-0.118204in}{0.484854in}}%
\pgfpathlineto{\pgfqpoint{-0.154648in}{0.484854in}}%
\pgfpathlineto{\pgfqpoint{-0.191091in}{0.484854in}}%
\pgfpathlineto{\pgfqpoint{-0.227535in}{0.484854in}}%
\pgfpathlineto{\pgfqpoint{-0.263978in}{0.484854in}}%
\pgfpathlineto{\pgfqpoint{-0.300421in}{0.484854in}}%
\pgfpathlineto{\pgfqpoint{-0.336865in}{0.484854in}}%
\pgfpathlineto{\pgfqpoint{-0.373308in}{0.484854in}}%
\pgfpathlineto{\pgfqpoint{-0.409752in}{0.484854in}}%
\pgfpathlineto{\pgfqpoint{-0.446195in}{0.484854in}}%
\pgfpathlineto{\pgfqpoint{-0.482639in}{0.484854in}}%
\pgfpathlineto{\pgfqpoint{-0.519082in}{0.484854in}}%
\pgfpathlineto{\pgfqpoint{-0.555526in}{0.484854in}}%
\pgfpathlineto{\pgfqpoint{-0.591969in}{0.484854in}}%
\pgfpathlineto{\pgfqpoint{-0.628412in}{0.484854in}}%
\pgfpathlineto{\pgfqpoint{-0.664856in}{0.484854in}}%
\pgfpathlineto{\pgfqpoint{-0.701299in}{0.484854in}}%
\pgfpathlineto{\pgfqpoint{-0.737743in}{0.484854in}}%
\pgfpathlineto{\pgfqpoint{-0.774186in}{0.484854in}}%
\pgfpathlineto{\pgfqpoint{-0.810630in}{0.484854in}}%
\pgfpathlineto{\pgfqpoint{-0.847073in}{0.484854in}}%
\pgfpathlineto{\pgfqpoint{-0.883517in}{0.484854in}}%
\pgfpathlineto{\pgfqpoint{-0.919960in}{0.484854in}}%
\pgfpathlineto{\pgfqpoint{-0.956404in}{0.484854in}}%
\pgfpathlineto{\pgfqpoint{-0.992847in}{0.484854in}}%
\pgfpathlineto{\pgfqpoint{-1.038401in}{0.484854in}}%
\pgfpathclose%
\pgfusepath{fill}%
\end{pgfscope}%
\begin{pgfscope}%
\pgfsetbuttcap%
\pgfsetroundjoin%
\definecolor{currentfill}{rgb}{0.000000,0.000000,0.000000}%
\pgfsetfillcolor{currentfill}%
\pgfsetlinewidth{0.803000pt}%
\definecolor{currentstroke}{rgb}{0.000000,0.000000,0.000000}%
\pgfsetstrokecolor{currentstroke}%
\pgfsetdash{}{0pt}%
\pgfsys@defobject{currentmarker}{\pgfqpoint{0.000000in}{-0.048611in}}{\pgfqpoint{0.000000in}{0.000000in}}{%
\pgfpathmoveto{\pgfqpoint{0.000000in}{0.000000in}}%
\pgfpathlineto{\pgfqpoint{0.000000in}{-0.048611in}}%
\pgfusepath{stroke,fill}%
}%
\begin{pgfscope}%
\pgfsys@transformshift{0.419337in}{0.484854in}%
\pgfsys@useobject{currentmarker}{}%
\end{pgfscope}%
\end{pgfscope}%
\begin{pgfscope}%
\definecolor{textcolor}{rgb}{0.000000,0.000000,0.000000}%
\pgfsetstrokecolor{textcolor}%
\pgfsetfillcolor{textcolor}%
\pgftext[x=0.419337in,y=0.387632in,,top]{\color{textcolor}\rmfamily\fontsize{8.000000}{9.600000}\selectfont \(\displaystyle 40\)}%
\end{pgfscope}%
\begin{pgfscope}%
\pgfsetbuttcap%
\pgfsetroundjoin%
\definecolor{currentfill}{rgb}{0.000000,0.000000,0.000000}%
\pgfsetfillcolor{currentfill}%
\pgfsetlinewidth{0.803000pt}%
\definecolor{currentstroke}{rgb}{0.000000,0.000000,0.000000}%
\pgfsetstrokecolor{currentstroke}%
\pgfsetdash{}{0pt}%
\pgfsys@defobject{currentmarker}{\pgfqpoint{0.000000in}{-0.048611in}}{\pgfqpoint{0.000000in}{0.000000in}}{%
\pgfpathmoveto{\pgfqpoint{0.000000in}{0.000000in}}%
\pgfpathlineto{\pgfqpoint{0.000000in}{-0.048611in}}%
\pgfusepath{stroke,fill}%
}%
\begin{pgfscope}%
\pgfsys@transformshift{0.783771in}{0.484854in}%
\pgfsys@useobject{currentmarker}{}%
\end{pgfscope}%
\end{pgfscope}%
\begin{pgfscope}%
\definecolor{textcolor}{rgb}{0.000000,0.000000,0.000000}%
\pgfsetstrokecolor{textcolor}%
\pgfsetfillcolor{textcolor}%
\pgftext[x=0.783771in,y=0.387632in,,top]{\color{textcolor}\rmfamily\fontsize{8.000000}{9.600000}\selectfont \(\displaystyle 50\)}%
\end{pgfscope}%
\begin{pgfscope}%
\pgfsetbuttcap%
\pgfsetroundjoin%
\definecolor{currentfill}{rgb}{0.000000,0.000000,0.000000}%
\pgfsetfillcolor{currentfill}%
\pgfsetlinewidth{0.803000pt}%
\definecolor{currentstroke}{rgb}{0.000000,0.000000,0.000000}%
\pgfsetstrokecolor{currentstroke}%
\pgfsetdash{}{0pt}%
\pgfsys@defobject{currentmarker}{\pgfqpoint{0.000000in}{-0.048611in}}{\pgfqpoint{0.000000in}{0.000000in}}{%
\pgfpathmoveto{\pgfqpoint{0.000000in}{0.000000in}}%
\pgfpathlineto{\pgfqpoint{0.000000in}{-0.048611in}}%
\pgfusepath{stroke,fill}%
}%
\begin{pgfscope}%
\pgfsys@transformshift{1.148206in}{0.484854in}%
\pgfsys@useobject{currentmarker}{}%
\end{pgfscope}%
\end{pgfscope}%
\begin{pgfscope}%
\definecolor{textcolor}{rgb}{0.000000,0.000000,0.000000}%
\pgfsetstrokecolor{textcolor}%
\pgfsetfillcolor{textcolor}%
\pgftext[x=1.148206in,y=0.387632in,,top]{\color{textcolor}\rmfamily\fontsize{8.000000}{9.600000}\selectfont \(\displaystyle 60\)}%
\end{pgfscope}%
\begin{pgfscope}%
\pgfsetbuttcap%
\pgfsetroundjoin%
\definecolor{currentfill}{rgb}{0.000000,0.000000,0.000000}%
\pgfsetfillcolor{currentfill}%
\pgfsetlinewidth{0.803000pt}%
\definecolor{currentstroke}{rgb}{0.000000,0.000000,0.000000}%
\pgfsetstrokecolor{currentstroke}%
\pgfsetdash{}{0pt}%
\pgfsys@defobject{currentmarker}{\pgfqpoint{0.000000in}{-0.048611in}}{\pgfqpoint{0.000000in}{0.000000in}}{%
\pgfpathmoveto{\pgfqpoint{0.000000in}{0.000000in}}%
\pgfpathlineto{\pgfqpoint{0.000000in}{-0.048611in}}%
\pgfusepath{stroke,fill}%
}%
\begin{pgfscope}%
\pgfsys@transformshift{1.512640in}{0.484854in}%
\pgfsys@useobject{currentmarker}{}%
\end{pgfscope}%
\end{pgfscope}%
\begin{pgfscope}%
\definecolor{textcolor}{rgb}{0.000000,0.000000,0.000000}%
\pgfsetstrokecolor{textcolor}%
\pgfsetfillcolor{textcolor}%
\pgftext[x=1.512640in,y=0.387632in,,top]{\color{textcolor}\rmfamily\fontsize{8.000000}{9.600000}\selectfont \(\displaystyle 70\)}%
\end{pgfscope}%
\begin{pgfscope}%
\pgfsetbuttcap%
\pgfsetroundjoin%
\definecolor{currentfill}{rgb}{0.000000,0.000000,0.000000}%
\pgfsetfillcolor{currentfill}%
\pgfsetlinewidth{0.803000pt}%
\definecolor{currentstroke}{rgb}{0.000000,0.000000,0.000000}%
\pgfsetstrokecolor{currentstroke}%
\pgfsetdash{}{0pt}%
\pgfsys@defobject{currentmarker}{\pgfqpoint{0.000000in}{-0.048611in}}{\pgfqpoint{0.000000in}{0.000000in}}{%
\pgfpathmoveto{\pgfqpoint{0.000000in}{0.000000in}}%
\pgfpathlineto{\pgfqpoint{0.000000in}{-0.048611in}}%
\pgfusepath{stroke,fill}%
}%
\begin{pgfscope}%
\pgfsys@transformshift{1.877075in}{0.484854in}%
\pgfsys@useobject{currentmarker}{}%
\end{pgfscope}%
\end{pgfscope}%
\begin{pgfscope}%
\definecolor{textcolor}{rgb}{0.000000,0.000000,0.000000}%
\pgfsetstrokecolor{textcolor}%
\pgfsetfillcolor{textcolor}%
\pgftext[x=1.877075in,y=0.387632in,,top]{\color{textcolor}\rmfamily\fontsize{8.000000}{9.600000}\selectfont \(\displaystyle 80\)}%
\end{pgfscope}%
\begin{pgfscope}%
\pgfsetbuttcap%
\pgfsetroundjoin%
\definecolor{currentfill}{rgb}{0.000000,0.000000,0.000000}%
\pgfsetfillcolor{currentfill}%
\pgfsetlinewidth{0.803000pt}%
\definecolor{currentstroke}{rgb}{0.000000,0.000000,0.000000}%
\pgfsetstrokecolor{currentstroke}%
\pgfsetdash{}{0pt}%
\pgfsys@defobject{currentmarker}{\pgfqpoint{0.000000in}{-0.048611in}}{\pgfqpoint{0.000000in}{0.000000in}}{%
\pgfpathmoveto{\pgfqpoint{0.000000in}{0.000000in}}%
\pgfpathlineto{\pgfqpoint{0.000000in}{-0.048611in}}%
\pgfusepath{stroke,fill}%
}%
\begin{pgfscope}%
\pgfsys@transformshift{2.241509in}{0.484854in}%
\pgfsys@useobject{currentmarker}{}%
\end{pgfscope}%
\end{pgfscope}%
\begin{pgfscope}%
\definecolor{textcolor}{rgb}{0.000000,0.000000,0.000000}%
\pgfsetstrokecolor{textcolor}%
\pgfsetfillcolor{textcolor}%
\pgftext[x=2.241509in,y=0.387632in,,top]{\color{textcolor}\rmfamily\fontsize{8.000000}{9.600000}\selectfont \(\displaystyle 90\)}%
\end{pgfscope}%
\begin{pgfscope}%
\pgfsetbuttcap%
\pgfsetroundjoin%
\definecolor{currentfill}{rgb}{0.000000,0.000000,0.000000}%
\pgfsetfillcolor{currentfill}%
\pgfsetlinewidth{0.803000pt}%
\definecolor{currentstroke}{rgb}{0.000000,0.000000,0.000000}%
\pgfsetstrokecolor{currentstroke}%
\pgfsetdash{}{0pt}%
\pgfsys@defobject{currentmarker}{\pgfqpoint{0.000000in}{-0.048611in}}{\pgfqpoint{0.000000in}{0.000000in}}{%
\pgfpathmoveto{\pgfqpoint{0.000000in}{0.000000in}}%
\pgfpathlineto{\pgfqpoint{0.000000in}{-0.048611in}}%
\pgfusepath{stroke,fill}%
}%
\begin{pgfscope}%
\pgfsys@transformshift{2.605944in}{0.484854in}%
\pgfsys@useobject{currentmarker}{}%
\end{pgfscope}%
\end{pgfscope}%
\begin{pgfscope}%
\definecolor{textcolor}{rgb}{0.000000,0.000000,0.000000}%
\pgfsetstrokecolor{textcolor}%
\pgfsetfillcolor{textcolor}%
\pgftext[x=2.605944in,y=0.387632in,,top]{\color{textcolor}\rmfamily\fontsize{8.000000}{9.600000}\selectfont \(\displaystyle 100\)}%
\end{pgfscope}%
\begin{pgfscope}%
\definecolor{textcolor}{rgb}{0.000000,0.000000,0.000000}%
\pgfsetstrokecolor{textcolor}%
\pgfsetfillcolor{textcolor}%
\pgftext[x=1.512640in,y=0.224546in,,top]{\color{textcolor}\rmfamily\fontsize{8.000000}{9.600000}\selectfont Time (\(\displaystyle \times 10^3 \, \mathrm{yr}\))}%
\end{pgfscope}%
\begin{pgfscope}%
\pgfsetbuttcap%
\pgfsetroundjoin%
\definecolor{currentfill}{rgb}{0.000000,0.000000,0.000000}%
\pgfsetfillcolor{currentfill}%
\pgfsetlinewidth{0.803000pt}%
\definecolor{currentstroke}{rgb}{0.000000,0.000000,0.000000}%
\pgfsetstrokecolor{currentstroke}%
\pgfsetdash{}{0pt}%
\pgfsys@defobject{currentmarker}{\pgfqpoint{-0.048611in}{0.000000in}}{\pgfqpoint{0.000000in}{0.000000in}}{%
\pgfpathmoveto{\pgfqpoint{0.000000in}{0.000000in}}%
\pgfpathlineto{\pgfqpoint{-0.048611in}{0.000000in}}%
\pgfusepath{stroke,fill}%
}%
\begin{pgfscope}%
\pgfsys@transformshift{0.419337in}{0.484854in}%
\pgfsys@useobject{currentmarker}{}%
\end{pgfscope}%
\end{pgfscope}%
\begin{pgfscope}%
\definecolor{textcolor}{rgb}{0.000000,0.000000,0.000000}%
\pgfsetstrokecolor{textcolor}%
\pgfsetfillcolor{textcolor}%
\pgftext[x=0.263086in,y=0.442645in,left,base]{\color{textcolor}\rmfamily\fontsize{8.000000}{9.600000}\selectfont \(\displaystyle 0\)}%
\end{pgfscope}%
\begin{pgfscope}%
\pgfsetbuttcap%
\pgfsetroundjoin%
\definecolor{currentfill}{rgb}{0.000000,0.000000,0.000000}%
\pgfsetfillcolor{currentfill}%
\pgfsetlinewidth{0.803000pt}%
\definecolor{currentstroke}{rgb}{0.000000,0.000000,0.000000}%
\pgfsetstrokecolor{currentstroke}%
\pgfsetdash{}{0pt}%
\pgfsys@defobject{currentmarker}{\pgfqpoint{-0.048611in}{0.000000in}}{\pgfqpoint{0.000000in}{0.000000in}}{%
\pgfpathmoveto{\pgfqpoint{0.000000in}{0.000000in}}%
\pgfpathlineto{\pgfqpoint{-0.048611in}{0.000000in}}%
\pgfusepath{stroke,fill}%
}%
\begin{pgfscope}%
\pgfsys@transformshift{0.419337in}{0.792284in}%
\pgfsys@useobject{currentmarker}{}%
\end{pgfscope}%
\end{pgfscope}%
\begin{pgfscope}%
\definecolor{textcolor}{rgb}{0.000000,0.000000,0.000000}%
\pgfsetstrokecolor{textcolor}%
\pgfsetfillcolor{textcolor}%
\pgftext[x=0.263086in,y=0.750074in,left,base]{\color{textcolor}\rmfamily\fontsize{8.000000}{9.600000}\selectfont \(\displaystyle 1\)}%
\end{pgfscope}%
\begin{pgfscope}%
\pgfsetbuttcap%
\pgfsetroundjoin%
\definecolor{currentfill}{rgb}{0.000000,0.000000,0.000000}%
\pgfsetfillcolor{currentfill}%
\pgfsetlinewidth{0.803000pt}%
\definecolor{currentstroke}{rgb}{0.000000,0.000000,0.000000}%
\pgfsetstrokecolor{currentstroke}%
\pgfsetdash{}{0pt}%
\pgfsys@defobject{currentmarker}{\pgfqpoint{-0.048611in}{0.000000in}}{\pgfqpoint{0.000000in}{0.000000in}}{%
\pgfpathmoveto{\pgfqpoint{0.000000in}{0.000000in}}%
\pgfpathlineto{\pgfqpoint{-0.048611in}{0.000000in}}%
\pgfusepath{stroke,fill}%
}%
\begin{pgfscope}%
\pgfsys@transformshift{0.419337in}{1.099713in}%
\pgfsys@useobject{currentmarker}{}%
\end{pgfscope}%
\end{pgfscope}%
\begin{pgfscope}%
\definecolor{textcolor}{rgb}{0.000000,0.000000,0.000000}%
\pgfsetstrokecolor{textcolor}%
\pgfsetfillcolor{textcolor}%
\pgftext[x=0.263086in,y=1.057503in,left,base]{\color{textcolor}\rmfamily\fontsize{8.000000}{9.600000}\selectfont \(\displaystyle 2\)}%
\end{pgfscope}%
\begin{pgfscope}%
\pgfsetbuttcap%
\pgfsetroundjoin%
\definecolor{currentfill}{rgb}{0.000000,0.000000,0.000000}%
\pgfsetfillcolor{currentfill}%
\pgfsetlinewidth{0.803000pt}%
\definecolor{currentstroke}{rgb}{0.000000,0.000000,0.000000}%
\pgfsetstrokecolor{currentstroke}%
\pgfsetdash{}{0pt}%
\pgfsys@defobject{currentmarker}{\pgfqpoint{-0.048611in}{0.000000in}}{\pgfqpoint{0.000000in}{0.000000in}}{%
\pgfpathmoveto{\pgfqpoint{0.000000in}{0.000000in}}%
\pgfpathlineto{\pgfqpoint{-0.048611in}{0.000000in}}%
\pgfusepath{stroke,fill}%
}%
\begin{pgfscope}%
\pgfsys@transformshift{0.419337in}{1.407142in}%
\pgfsys@useobject{currentmarker}{}%
\end{pgfscope}%
\end{pgfscope}%
\begin{pgfscope}%
\definecolor{textcolor}{rgb}{0.000000,0.000000,0.000000}%
\pgfsetstrokecolor{textcolor}%
\pgfsetfillcolor{textcolor}%
\pgftext[x=0.263086in,y=1.364933in,left,base]{\color{textcolor}\rmfamily\fontsize{8.000000}{9.600000}\selectfont \(\displaystyle 3\)}%
\end{pgfscope}%
\begin{pgfscope}%
\pgfsetbuttcap%
\pgfsetroundjoin%
\definecolor{currentfill}{rgb}{0.000000,0.000000,0.000000}%
\pgfsetfillcolor{currentfill}%
\pgfsetlinewidth{0.803000pt}%
\definecolor{currentstroke}{rgb}{0.000000,0.000000,0.000000}%
\pgfsetstrokecolor{currentstroke}%
\pgfsetdash{}{0pt}%
\pgfsys@defobject{currentmarker}{\pgfqpoint{-0.048611in}{0.000000in}}{\pgfqpoint{0.000000in}{0.000000in}}{%
\pgfpathmoveto{\pgfqpoint{0.000000in}{0.000000in}}%
\pgfpathlineto{\pgfqpoint{-0.048611in}{0.000000in}}%
\pgfusepath{stroke,fill}%
}%
\begin{pgfscope}%
\pgfsys@transformshift{0.419337in}{1.714571in}%
\pgfsys@useobject{currentmarker}{}%
\end{pgfscope}%
\end{pgfscope}%
\begin{pgfscope}%
\definecolor{textcolor}{rgb}{0.000000,0.000000,0.000000}%
\pgfsetstrokecolor{textcolor}%
\pgfsetfillcolor{textcolor}%
\pgftext[x=0.263086in,y=1.672362in,left,base]{\color{textcolor}\rmfamily\fontsize{8.000000}{9.600000}\selectfont \(\displaystyle 4\)}%
\end{pgfscope}%
\begin{pgfscope}%
\pgfsetbuttcap%
\pgfsetroundjoin%
\definecolor{currentfill}{rgb}{0.000000,0.000000,0.000000}%
\pgfsetfillcolor{currentfill}%
\pgfsetlinewidth{0.803000pt}%
\definecolor{currentstroke}{rgb}{0.000000,0.000000,0.000000}%
\pgfsetstrokecolor{currentstroke}%
\pgfsetdash{}{0pt}%
\pgfsys@defobject{currentmarker}{\pgfqpoint{-0.048611in}{0.000000in}}{\pgfqpoint{0.000000in}{0.000000in}}{%
\pgfpathmoveto{\pgfqpoint{0.000000in}{0.000000in}}%
\pgfpathlineto{\pgfqpoint{-0.048611in}{0.000000in}}%
\pgfusepath{stroke,fill}%
}%
\begin{pgfscope}%
\pgfsys@transformshift{0.419337in}{2.022000in}%
\pgfsys@useobject{currentmarker}{}%
\end{pgfscope}%
\end{pgfscope}%
\begin{pgfscope}%
\definecolor{textcolor}{rgb}{0.000000,0.000000,0.000000}%
\pgfsetstrokecolor{textcolor}%
\pgfsetfillcolor{textcolor}%
\pgftext[x=0.263086in,y=1.979791in,left,base]{\color{textcolor}\rmfamily\fontsize{8.000000}{9.600000}\selectfont \(\displaystyle 5\)}%
\end{pgfscope}%
\begin{pgfscope}%
\definecolor{textcolor}{rgb}{0.000000,0.000000,0.000000}%
\pgfsetstrokecolor{textcolor}%
\pgfsetfillcolor{textcolor}%
\pgftext[x=0.207530in,y=1.299542in,,bottom,rotate=90.000000]{\color{textcolor}\rmfamily\fontsize{8.000000}{9.600000}\selectfont Melt volume (\%)}%
\end{pgfscope}%
\begin{pgfscope}%
\pgfpathrectangle{\pgfqpoint{0.419337in}{0.484854in}}{\pgfqpoint{2.186607in}{1.629375in}}%
\pgfusepath{clip}%
\pgfsetrectcap%
\pgfsetroundjoin%
\pgfsetlinewidth{1.505625pt}%
\definecolor{currentstroke}{rgb}{0.000000,0.000000,0.000000}%
\pgfsetstrokecolor{currentstroke}%
\pgfsetdash{}{0pt}%
\pgfpathmoveto{\pgfqpoint{0.414337in}{0.484854in}}%
\pgfpathlineto{\pgfqpoint{0.428448in}{0.484854in}}%
\pgfpathlineto{\pgfqpoint{0.464891in}{0.484854in}}%
\pgfpathlineto{\pgfqpoint{0.501334in}{0.484854in}}%
\pgfpathlineto{\pgfqpoint{0.537778in}{0.484854in}}%
\pgfpathlineto{\pgfqpoint{0.574221in}{0.484854in}}%
\pgfpathlineto{\pgfqpoint{0.610665in}{0.484854in}}%
\pgfpathlineto{\pgfqpoint{0.647108in}{0.484854in}}%
\pgfpathlineto{\pgfqpoint{0.683552in}{0.484854in}}%
\pgfpathlineto{\pgfqpoint{0.719995in}{0.484854in}}%
\pgfpathlineto{\pgfqpoint{0.756439in}{0.484854in}}%
\pgfpathlineto{\pgfqpoint{0.792882in}{0.484854in}}%
\pgfpathlineto{\pgfqpoint{0.829325in}{0.484854in}}%
\pgfpathlineto{\pgfqpoint{0.865769in}{0.484854in}}%
\pgfpathlineto{\pgfqpoint{0.902212in}{0.484854in}}%
\pgfpathlineto{\pgfqpoint{0.938656in}{0.484854in}}%
\pgfpathlineto{\pgfqpoint{0.975099in}{0.713320in}}%
\pgfpathlineto{\pgfqpoint{1.011543in}{0.713320in}}%
\pgfpathlineto{\pgfqpoint{1.047986in}{0.713320in}}%
\pgfpathlineto{\pgfqpoint{1.084430in}{0.931426in}}%
\pgfpathlineto{\pgfqpoint{1.120873in}{0.931426in}}%
\pgfpathlineto{\pgfqpoint{1.157316in}{0.931426in}}%
\pgfpathlineto{\pgfqpoint{1.193760in}{1.146903in}}%
\pgfpathlineto{\pgfqpoint{1.230203in}{1.146903in}}%
\pgfpathlineto{\pgfqpoint{1.266647in}{1.146903in}}%
\pgfpathlineto{\pgfqpoint{1.303090in}{1.146903in}}%
\pgfpathlineto{\pgfqpoint{1.339534in}{1.347993in}}%
\pgfpathlineto{\pgfqpoint{1.375977in}{1.347993in}}%
\pgfpathlineto{\pgfqpoint{1.412421in}{1.347993in}}%
\pgfpathlineto{\pgfqpoint{1.448864in}{1.347993in}}%
\pgfpathlineto{\pgfqpoint{1.485308in}{1.347993in}}%
\pgfpathlineto{\pgfqpoint{1.521751in}{1.347993in}}%
\pgfpathlineto{\pgfqpoint{1.558194in}{1.536816in}}%
\pgfpathlineto{\pgfqpoint{1.594638in}{1.536816in}}%
\pgfpathlineto{\pgfqpoint{1.631081in}{1.536816in}}%
\pgfpathlineto{\pgfqpoint{1.667525in}{1.536816in}}%
\pgfpathlineto{\pgfqpoint{1.703968in}{1.536816in}}%
\pgfpathlineto{\pgfqpoint{1.740412in}{1.536816in}}%
\pgfpathlineto{\pgfqpoint{1.776855in}{1.717092in}}%
\pgfpathlineto{\pgfqpoint{1.813299in}{1.717092in}}%
\pgfpathlineto{\pgfqpoint{1.849742in}{1.717092in}}%
\pgfpathlineto{\pgfqpoint{1.886185in}{1.717092in}}%
\pgfpathlineto{\pgfqpoint{1.922629in}{1.717092in}}%
\pgfpathlineto{\pgfqpoint{1.959072in}{1.717092in}}%
\pgfpathlineto{\pgfqpoint{1.995516in}{1.717092in}}%
\pgfpathlineto{\pgfqpoint{2.031959in}{1.891558in}}%
\pgfpathlineto{\pgfqpoint{2.068403in}{1.891558in}}%
\pgfpathlineto{\pgfqpoint{2.104846in}{1.891558in}}%
\pgfpathlineto{\pgfqpoint{2.141290in}{1.891558in}}%
\pgfpathlineto{\pgfqpoint{2.177733in}{1.891558in}}%
\pgfpathlineto{\pgfqpoint{2.214177in}{1.891558in}}%
\pgfpathlineto{\pgfqpoint{2.250620in}{1.891558in}}%
\pgfpathlineto{\pgfqpoint{2.287063in}{1.891558in}}%
\pgfpathlineto{\pgfqpoint{2.323507in}{2.043736in}}%
\pgfpathlineto{\pgfqpoint{2.359950in}{2.043736in}}%
\pgfpathlineto{\pgfqpoint{2.396394in}{2.043736in}}%
\pgfpathlineto{\pgfqpoint{2.432837in}{2.043736in}}%
\pgfpathlineto{\pgfqpoint{2.469281in}{2.043736in}}%
\pgfpathlineto{\pgfqpoint{2.505724in}{2.043736in}}%
\pgfpathlineto{\pgfqpoint{2.542168in}{2.043736in}}%
\pgfpathlineto{\pgfqpoint{2.578611in}{2.043736in}}%
\pgfpathlineto{\pgfqpoint{2.605944in}{2.043736in}}%
\pgfusepath{stroke}%
\end{pgfscope}%
\begin{pgfscope}%
\pgfsetrectcap%
\pgfsetmiterjoin%
\pgfsetlinewidth{0.803000pt}%
\definecolor{currentstroke}{rgb}{0.000000,0.000000,0.000000}%
\pgfsetstrokecolor{currentstroke}%
\pgfsetdash{}{0pt}%
\pgfpathmoveto{\pgfqpoint{0.419337in}{0.484854in}}%
\pgfpathlineto{\pgfqpoint{0.419337in}{2.114229in}}%
\pgfusepath{stroke}%
\end{pgfscope}%
\begin{pgfscope}%
\pgfsetrectcap%
\pgfsetmiterjoin%
\pgfsetlinewidth{0.803000pt}%
\definecolor{currentstroke}{rgb}{0.000000,0.000000,0.000000}%
\pgfsetstrokecolor{currentstroke}%
\pgfsetdash{}{0pt}%
\pgfpathmoveto{\pgfqpoint{2.605944in}{0.484854in}}%
\pgfpathlineto{\pgfqpoint{2.605944in}{2.114229in}}%
\pgfusepath{stroke}%
\end{pgfscope}%
\begin{pgfscope}%
\pgfsetrectcap%
\pgfsetmiterjoin%
\pgfsetlinewidth{0.803000pt}%
\definecolor{currentstroke}{rgb}{0.000000,0.000000,0.000000}%
\pgfsetstrokecolor{currentstroke}%
\pgfsetdash{}{0pt}%
\pgfpathmoveto{\pgfqpoint{0.419337in}{0.484854in}}%
\pgfpathlineto{\pgfqpoint{2.605944in}{0.484854in}}%
\pgfusepath{stroke}%
\end{pgfscope}%
\begin{pgfscope}%
\pgfsetrectcap%
\pgfsetmiterjoin%
\pgfsetlinewidth{0.803000pt}%
\definecolor{currentstroke}{rgb}{0.000000,0.000000,0.000000}%
\pgfsetstrokecolor{currentstroke}%
\pgfsetdash{}{0pt}%
\pgfpathmoveto{\pgfqpoint{0.419337in}{2.114229in}}%
\pgfpathlineto{\pgfqpoint{2.605944in}{2.114229in}}%
\pgfusepath{stroke}%
\end{pgfscope}%
\begin{pgfscope}%
\pgfsetbuttcap%
\pgfsetmiterjoin%
\definecolor{currentfill}{rgb}{0.121569,0.466667,0.705882}%
\pgfsetfillcolor{currentfill}%
\pgfsetlinewidth{0.000000pt}%
\definecolor{currentstroke}{rgb}{0.000000,0.000000,0.000000}%
\pgfsetstrokecolor{currentstroke}%
\pgfsetstrokeopacity{0.000000}%
\pgfsetdash{}{0pt}%
\pgfpathmoveto{\pgfqpoint{0.506837in}{1.952863in}}%
\pgfpathlineto{\pgfqpoint{0.701281in}{1.952863in}}%
\pgfpathlineto{\pgfqpoint{0.701281in}{2.020919in}}%
\pgfpathlineto{\pgfqpoint{0.506837in}{2.020919in}}%
\pgfpathclose%
\pgfusepath{fill}%
\end{pgfscope}%
\begin{pgfscope}%
\definecolor{textcolor}{rgb}{0.000000,0.000000,0.000000}%
\pgfsetstrokecolor{textcolor}%
\pgfsetfillcolor{textcolor}%
\pgftext[x=0.779059in,y=1.952863in,left,base]{\color{textcolor}\rmfamily\fontsize{7.000000}{8.400000}\selectfont CaO}%
\end{pgfscope}%
\begin{pgfscope}%
\pgfsetbuttcap%
\pgfsetmiterjoin%
\definecolor{currentfill}{rgb}{1.000000,0.498039,0.054902}%
\pgfsetfillcolor{currentfill}%
\pgfsetlinewidth{0.000000pt}%
\definecolor{currentstroke}{rgb}{0.000000,0.000000,0.000000}%
\pgfsetstrokecolor{currentstroke}%
\pgfsetstrokeopacity{0.000000}%
\pgfsetdash{}{0pt}%
\pgfpathmoveto{\pgfqpoint{0.506837in}{1.810163in}}%
\pgfpathlineto{\pgfqpoint{0.701281in}{1.810163in}}%
\pgfpathlineto{\pgfqpoint{0.701281in}{1.878219in}}%
\pgfpathlineto{\pgfqpoint{0.506837in}{1.878219in}}%
\pgfpathclose%
\pgfusepath{fill}%
\end{pgfscope}%
\begin{pgfscope}%
\definecolor{textcolor}{rgb}{0.000000,0.000000,0.000000}%
\pgfsetstrokecolor{textcolor}%
\pgfsetfillcolor{textcolor}%
\pgftext[x=0.779059in,y=1.810163in,left,base]{\color{textcolor}\rmfamily\fontsize{7.000000}{8.400000}\selectfont FeO}%
\end{pgfscope}%
\begin{pgfscope}%
\pgfsetbuttcap%
\pgfsetmiterjoin%
\definecolor{currentfill}{rgb}{0.172549,0.627451,0.172549}%
\pgfsetfillcolor{currentfill}%
\pgfsetlinewidth{0.000000pt}%
\definecolor{currentstroke}{rgb}{0.000000,0.000000,0.000000}%
\pgfsetstrokecolor{currentstroke}%
\pgfsetstrokeopacity{0.000000}%
\pgfsetdash{}{0pt}%
\pgfpathmoveto{\pgfqpoint{0.506837in}{1.667463in}}%
\pgfpathlineto{\pgfqpoint{0.701281in}{1.667463in}}%
\pgfpathlineto{\pgfqpoint{0.701281in}{1.735519in}}%
\pgfpathlineto{\pgfqpoint{0.506837in}{1.735519in}}%
\pgfpathclose%
\pgfusepath{fill}%
\end{pgfscope}%
\begin{pgfscope}%
\definecolor{textcolor}{rgb}{0.000000,0.000000,0.000000}%
\pgfsetstrokecolor{textcolor}%
\pgfsetfillcolor{textcolor}%
\pgftext[x=0.779059in,y=1.667463in,left,base]{\color{textcolor}\rmfamily\fontsize{7.000000}{8.400000}\selectfont MgO}%
\end{pgfscope}%
\begin{pgfscope}%
\pgfsetbuttcap%
\pgfsetmiterjoin%
\definecolor{currentfill}{rgb}{0.839216,0.152941,0.156863}%
\pgfsetfillcolor{currentfill}%
\pgfsetlinewidth{0.000000pt}%
\definecolor{currentstroke}{rgb}{0.000000,0.000000,0.000000}%
\pgfsetstrokecolor{currentstroke}%
\pgfsetstrokeopacity{0.000000}%
\pgfsetdash{}{0pt}%
\pgfpathmoveto{\pgfqpoint{0.506837in}{1.523386in}}%
\pgfpathlineto{\pgfqpoint{0.701281in}{1.523386in}}%
\pgfpathlineto{\pgfqpoint{0.701281in}{1.591442in}}%
\pgfpathlineto{\pgfqpoint{0.506837in}{1.591442in}}%
\pgfpathclose%
\pgfusepath{fill}%
\end{pgfscope}%
\begin{pgfscope}%
\definecolor{textcolor}{rgb}{0.000000,0.000000,0.000000}%
\pgfsetstrokecolor{textcolor}%
\pgfsetfillcolor{textcolor}%
\pgftext[x=0.779059in,y=1.523386in,left,base]{\color{textcolor}\rmfamily\fontsize{7.000000}{8.400000}\selectfont SiO2}%
\end{pgfscope}%
\end{pgfpicture}%
\makeatother%
\endgroup%

        \caption{
            Particle property plugin with fractional melting.
        }
        \label{fig:decompression_event_particle_plugin_frac}
    \end{subfigure}
    %
    \begin{subfigure}{0.49\textwidth}
        \centering
        %% Creator: Matplotlib, PGF backend
%%
%% To include the figure in your LaTeX document, write
%%   \input{<filename>.pgf}
%%
%% Make sure the required packages are loaded in your preamble
%%   \usepackage{pgf}
%%
%% Figures using additional raster images can only be included by \input if
%% they are in the same directory as the main LaTeX file. For loading figures
%% from other directories you can use the `import` package
%%   \usepackage{import}
%% and then include the figures with
%%   \import{<path to file>}{<filename>.pgf}
%%
%% Matplotlib used the following preamble
%%   \usepackage{fontspec}
%%   \setmainfont{DejaVuSerif.ttf}[Path=/home/connor/.local/lib/python3.8/site-packages/matplotlib/mpl-data/fonts/ttf/]
%%   \setsansfont{DejaVuSans.ttf}[Path=/home/connor/.local/lib/python3.8/site-packages/matplotlib/mpl-data/fonts/ttf/]
%%   \setmonofont{DejaVuSansMono.ttf}[Path=/home/connor/.local/lib/python3.8/site-packages/matplotlib/mpl-data/fonts/ttf/]
%%
\begingroup%
\makeatletter%
\begin{pgfpicture}%
\pgfpathrectangle{\pgfpointorigin}{\pgfqpoint{2.853515in}{2.214229in}}%
\pgfusepath{use as bounding box, clip}%
\begin{pgfscope}%
\pgfsetbuttcap%
\pgfsetmiterjoin%
\definecolor{currentfill}{rgb}{1.000000,1.000000,1.000000}%
\pgfsetfillcolor{currentfill}%
\pgfsetlinewidth{0.000000pt}%
\definecolor{currentstroke}{rgb}{1.000000,1.000000,1.000000}%
\pgfsetstrokecolor{currentstroke}%
\pgfsetdash{}{0pt}%
\pgfpathmoveto{\pgfqpoint{0.000000in}{0.000000in}}%
\pgfpathlineto{\pgfqpoint{2.853515in}{0.000000in}}%
\pgfpathlineto{\pgfqpoint{2.853515in}{2.214229in}}%
\pgfpathlineto{\pgfqpoint{0.000000in}{2.214229in}}%
\pgfpathclose%
\pgfusepath{fill}%
\end{pgfscope}%
\begin{pgfscope}%
\pgfsetbuttcap%
\pgfsetmiterjoin%
\definecolor{currentfill}{rgb}{1.000000,1.000000,1.000000}%
\pgfsetfillcolor{currentfill}%
\pgfsetlinewidth{0.000000pt}%
\definecolor{currentstroke}{rgb}{0.000000,0.000000,0.000000}%
\pgfsetstrokecolor{currentstroke}%
\pgfsetstrokeopacity{0.000000}%
\pgfsetdash{}{0pt}%
\pgfpathmoveto{\pgfqpoint{0.478365in}{0.484854in}}%
\pgfpathlineto{\pgfqpoint{2.664972in}{0.484854in}}%
\pgfpathlineto{\pgfqpoint{2.664972in}{2.114229in}}%
\pgfpathlineto{\pgfqpoint{0.478365in}{2.114229in}}%
\pgfpathclose%
\pgfusepath{fill}%
\end{pgfscope}%
\begin{pgfscope}%
\pgfpathrectangle{\pgfqpoint{0.478365in}{0.484854in}}{\pgfqpoint{2.186607in}{1.629375in}}%
\pgfusepath{clip}%
\pgfsetbuttcap%
\pgfsetroundjoin%
\definecolor{currentfill}{rgb}{0.121569,0.466667,0.705882}%
\pgfsetfillcolor{currentfill}%
\pgfsetlinewidth{0.000000pt}%
\definecolor{currentstroke}{rgb}{0.000000,0.000000,0.000000}%
\pgfsetstrokecolor{currentstroke}%
\pgfsetdash{}{0pt}%
\pgfpathmoveto{\pgfqpoint{-0.979373in}{0.484854in}}%
\pgfpathlineto{\pgfqpoint{-0.979373in}{0.484854in}}%
\pgfpathlineto{\pgfqpoint{-0.933818in}{0.484854in}}%
\pgfpathlineto{\pgfqpoint{-0.897375in}{0.484854in}}%
\pgfpathlineto{\pgfqpoint{-0.860931in}{0.484854in}}%
\pgfpathlineto{\pgfqpoint{-0.824488in}{0.484854in}}%
\pgfpathlineto{\pgfqpoint{-0.788045in}{0.484854in}}%
\pgfpathlineto{\pgfqpoint{-0.751605in}{0.484854in}}%
\pgfpathlineto{\pgfqpoint{-0.715161in}{0.484854in}}%
\pgfpathlineto{\pgfqpoint{-0.678718in}{0.484854in}}%
\pgfpathlineto{\pgfqpoint{-0.642274in}{0.484854in}}%
\pgfpathlineto{\pgfqpoint{-0.605827in}{0.484854in}}%
\pgfpathlineto{\pgfqpoint{-0.569384in}{0.484854in}}%
\pgfpathlineto{\pgfqpoint{-0.532940in}{0.484854in}}%
\pgfpathlineto{\pgfqpoint{-0.496497in}{0.484854in}}%
\pgfpathlineto{\pgfqpoint{-0.460054in}{0.484854in}}%
\pgfpathlineto{\pgfqpoint{-0.423610in}{0.484854in}}%
\pgfpathlineto{\pgfqpoint{-0.387167in}{0.484854in}}%
\pgfpathlineto{\pgfqpoint{-0.350723in}{0.484854in}}%
\pgfpathlineto{\pgfqpoint{-0.314280in}{0.484854in}}%
\pgfpathlineto{\pgfqpoint{-0.277836in}{0.484854in}}%
\pgfpathlineto{\pgfqpoint{-0.241393in}{0.484854in}}%
\pgfpathlineto{\pgfqpoint{-0.204949in}{0.484854in}}%
\pgfpathlineto{\pgfqpoint{-0.168506in}{0.484854in}}%
\pgfpathlineto{\pgfqpoint{-0.132062in}{0.484854in}}%
\pgfpathlineto{\pgfqpoint{-0.095619in}{0.484854in}}%
\pgfpathlineto{\pgfqpoint{-0.059176in}{0.484854in}}%
\pgfpathlineto{\pgfqpoint{-0.022732in}{0.484854in}}%
\pgfpathlineto{\pgfqpoint{0.013711in}{0.484854in}}%
\pgfpathlineto{\pgfqpoint{0.050155in}{0.484854in}}%
\pgfpathlineto{\pgfqpoint{0.086598in}{0.484854in}}%
\pgfpathlineto{\pgfqpoint{0.123042in}{0.484854in}}%
\pgfpathlineto{\pgfqpoint{0.159485in}{0.484854in}}%
\pgfpathlineto{\pgfqpoint{0.195929in}{0.484854in}}%
\pgfpathlineto{\pgfqpoint{0.232372in}{0.484854in}}%
\pgfpathlineto{\pgfqpoint{0.268815in}{0.484854in}}%
\pgfpathlineto{\pgfqpoint{0.305259in}{0.484854in}}%
\pgfpathlineto{\pgfqpoint{0.341702in}{0.484854in}}%
\pgfpathlineto{\pgfqpoint{0.378146in}{0.484854in}}%
\pgfpathlineto{\pgfqpoint{0.414589in}{0.484854in}}%
\pgfpathlineto{\pgfqpoint{0.451033in}{0.484854in}}%
\pgfpathlineto{\pgfqpoint{0.487476in}{0.484854in}}%
\pgfpathlineto{\pgfqpoint{0.523920in}{0.484854in}}%
\pgfpathlineto{\pgfqpoint{0.560363in}{0.484854in}}%
\pgfpathlineto{\pgfqpoint{0.596806in}{0.484854in}}%
\pgfpathlineto{\pgfqpoint{0.633250in}{0.484854in}}%
\pgfpathlineto{\pgfqpoint{0.669693in}{0.484854in}}%
\pgfpathlineto{\pgfqpoint{0.706137in}{0.484854in}}%
\pgfpathlineto{\pgfqpoint{0.742580in}{0.484854in}}%
\pgfpathlineto{\pgfqpoint{0.779024in}{0.484854in}}%
\pgfpathlineto{\pgfqpoint{0.815467in}{0.484854in}}%
\pgfpathlineto{\pgfqpoint{0.851911in}{0.484854in}}%
\pgfpathlineto{\pgfqpoint{0.888026in}{0.484854in}}%
\pgfpathlineto{\pgfqpoint{0.922939in}{0.484854in}}%
\pgfpathlineto{\pgfqpoint{0.955629in}{0.484854in}}%
\pgfpathlineto{\pgfqpoint{0.993348in}{0.484854in}}%
\pgfpathlineto{\pgfqpoint{1.028078in}{0.484854in}}%
\pgfpathlineto{\pgfqpoint{1.066271in}{0.484854in}}%
\pgfpathlineto{\pgfqpoint{1.101475in}{0.484854in}}%
\pgfpathlineto{\pgfqpoint{1.139522in}{0.484854in}}%
\pgfpathlineto{\pgfqpoint{1.175273in}{0.484854in}}%
\pgfpathlineto{\pgfqpoint{1.209348in}{0.484854in}}%
\pgfpathlineto{\pgfqpoint{1.246630in}{0.484854in}}%
\pgfpathlineto{\pgfqpoint{1.282235in}{0.484854in}}%
\pgfpathlineto{\pgfqpoint{1.316893in}{0.484854in}}%
\pgfpathlineto{\pgfqpoint{1.355268in}{0.484854in}}%
\pgfpathlineto{\pgfqpoint{1.393242in}{0.484854in}}%
\pgfpathlineto{\pgfqpoint{1.426879in}{0.484854in}}%
\pgfpathlineto{\pgfqpoint{1.463833in}{0.484854in}}%
\pgfpathlineto{\pgfqpoint{1.502098in}{0.484854in}}%
\pgfpathlineto{\pgfqpoint{1.535335in}{0.484854in}}%
\pgfpathlineto{\pgfqpoint{1.571815in}{0.484854in}}%
\pgfpathlineto{\pgfqpoint{1.608149in}{0.484854in}}%
\pgfpathlineto{\pgfqpoint{1.647763in}{0.484854in}}%
\pgfpathlineto{\pgfqpoint{1.683404in}{0.484854in}}%
\pgfpathlineto{\pgfqpoint{1.718390in}{0.484854in}}%
\pgfpathlineto{\pgfqpoint{1.756401in}{0.484854in}}%
\pgfpathlineto{\pgfqpoint{1.790402in}{0.484854in}}%
\pgfpathlineto{\pgfqpoint{1.827356in}{0.484854in}}%
\pgfpathlineto{\pgfqpoint{1.866059in}{0.484854in}}%
\pgfpathlineto{\pgfqpoint{1.900206in}{0.484854in}}%
\pgfpathlineto{\pgfqpoint{1.936723in}{0.484854in}}%
\pgfpathlineto{\pgfqpoint{1.975389in}{0.484854in}}%
\pgfpathlineto{\pgfqpoint{2.010047in}{0.484854in}}%
\pgfpathlineto{\pgfqpoint{2.046673in}{0.484854in}}%
\pgfpathlineto{\pgfqpoint{2.082788in}{0.484854in}}%
\pgfpathlineto{\pgfqpoint{2.118357in}{0.484854in}}%
\pgfpathlineto{\pgfqpoint{2.156367in}{0.484854in}}%
\pgfpathlineto{\pgfqpoint{2.193831in}{0.484854in}}%
\pgfpathlineto{\pgfqpoint{2.228161in}{0.484854in}}%
\pgfpathlineto{\pgfqpoint{2.264823in}{0.484854in}}%
\pgfpathlineto{\pgfqpoint{2.301121in}{0.484854in}}%
\pgfpathlineto{\pgfqpoint{2.337637in}{0.484854in}}%
\pgfpathlineto{\pgfqpoint{2.374372in}{0.484854in}}%
\pgfpathlineto{\pgfqpoint{2.410196in}{0.484854in}}%
\pgfpathlineto{\pgfqpoint{2.447223in}{0.484854in}}%
\pgfpathlineto{\pgfqpoint{2.483156in}{0.484854in}}%
\pgfpathlineto{\pgfqpoint{2.520073in}{0.484854in}}%
\pgfpathlineto{\pgfqpoint{2.556917in}{0.484854in}}%
\pgfpathlineto{\pgfqpoint{2.594235in}{0.484854in}}%
\pgfpathlineto{\pgfqpoint{2.630606in}{0.484854in}}%
\pgfpathlineto{\pgfqpoint{2.664972in}{0.484854in}}%
\pgfpathlineto{\pgfqpoint{2.664972in}{0.654454in}}%
\pgfpathlineto{\pgfqpoint{2.664972in}{0.654454in}}%
\pgfpathlineto{\pgfqpoint{2.630606in}{0.652531in}}%
\pgfpathlineto{\pgfqpoint{2.594235in}{0.656614in}}%
\pgfpathlineto{\pgfqpoint{2.556917in}{0.654091in}}%
\pgfpathlineto{\pgfqpoint{2.520073in}{0.649189in}}%
\pgfpathlineto{\pgfqpoint{2.483156in}{0.657130in}}%
\pgfpathlineto{\pgfqpoint{2.447223in}{0.654781in}}%
\pgfpathlineto{\pgfqpoint{2.410196in}{0.658329in}}%
\pgfpathlineto{\pgfqpoint{2.374372in}{0.667190in}}%
\pgfpathlineto{\pgfqpoint{2.337637in}{0.655454in}}%
\pgfpathlineto{\pgfqpoint{2.301121in}{0.654694in}}%
\pgfpathlineto{\pgfqpoint{2.264823in}{0.651097in}}%
\pgfpathlineto{\pgfqpoint{2.228161in}{0.658314in}}%
\pgfpathlineto{\pgfqpoint{2.193831in}{0.653140in}}%
\pgfpathlineto{\pgfqpoint{2.156367in}{0.654939in}}%
\pgfpathlineto{\pgfqpoint{2.118357in}{0.651309in}}%
\pgfpathlineto{\pgfqpoint{2.082788in}{0.655472in}}%
\pgfpathlineto{\pgfqpoint{2.046673in}{0.650923in}}%
\pgfpathlineto{\pgfqpoint{2.010047in}{0.649104in}}%
\pgfpathlineto{\pgfqpoint{1.975389in}{0.647252in}}%
\pgfpathlineto{\pgfqpoint{1.936723in}{0.646587in}}%
\pgfpathlineto{\pgfqpoint{1.900206in}{0.644996in}}%
\pgfpathlineto{\pgfqpoint{1.866059in}{0.642198in}}%
\pgfpathlineto{\pgfqpoint{1.827356in}{0.647158in}}%
\pgfpathlineto{\pgfqpoint{1.790402in}{0.633775in}}%
\pgfpathlineto{\pgfqpoint{1.756401in}{0.632389in}}%
\pgfpathlineto{\pgfqpoint{1.718390in}{0.636421in}}%
\pgfpathlineto{\pgfqpoint{1.683404in}{0.632006in}}%
\pgfpathlineto{\pgfqpoint{1.647763in}{0.628526in}}%
\pgfpathlineto{\pgfqpoint{1.608149in}{0.630648in}}%
\pgfpathlineto{\pgfqpoint{1.571815in}{0.622178in}}%
\pgfpathlineto{\pgfqpoint{1.535335in}{0.630821in}}%
\pgfpathlineto{\pgfqpoint{1.502098in}{0.651001in}}%
\pgfpathlineto{\pgfqpoint{1.463833in}{0.610914in}}%
\pgfpathlineto{\pgfqpoint{1.426879in}{0.601164in}}%
\pgfpathlineto{\pgfqpoint{1.393242in}{0.601918in}}%
\pgfpathlineto{\pgfqpoint{1.355268in}{0.606637in}}%
\pgfpathlineto{\pgfqpoint{1.316893in}{0.609421in}}%
\pgfpathlineto{\pgfqpoint{1.282235in}{0.608034in}}%
\pgfpathlineto{\pgfqpoint{1.246630in}{0.606498in}}%
\pgfpathlineto{\pgfqpoint{1.209348in}{0.605452in}}%
\pgfpathlineto{\pgfqpoint{1.175273in}{0.609209in}}%
\pgfpathlineto{\pgfqpoint{1.139522in}{0.603392in}}%
\pgfpathlineto{\pgfqpoint{1.101475in}{0.584486in}}%
\pgfpathlineto{\pgfqpoint{1.066271in}{0.559987in}}%
\pgfpathlineto{\pgfqpoint{1.028078in}{0.535511in}}%
\pgfpathlineto{\pgfqpoint{0.993348in}{0.510848in}}%
\pgfpathlineto{\pgfqpoint{0.955629in}{0.493915in}}%
\pgfpathlineto{\pgfqpoint{0.922939in}{0.486970in}}%
\pgfpathlineto{\pgfqpoint{0.888026in}{0.484854in}}%
\pgfpathlineto{\pgfqpoint{0.851911in}{0.484859in}}%
\pgfpathlineto{\pgfqpoint{0.815467in}{0.484854in}}%
\pgfpathlineto{\pgfqpoint{0.779024in}{0.484854in}}%
\pgfpathlineto{\pgfqpoint{0.742580in}{0.484854in}}%
\pgfpathlineto{\pgfqpoint{0.706137in}{0.484854in}}%
\pgfpathlineto{\pgfqpoint{0.669693in}{0.484854in}}%
\pgfpathlineto{\pgfqpoint{0.633250in}{0.484854in}}%
\pgfpathlineto{\pgfqpoint{0.596806in}{0.484854in}}%
\pgfpathlineto{\pgfqpoint{0.560363in}{0.484854in}}%
\pgfpathlineto{\pgfqpoint{0.523920in}{0.484854in}}%
\pgfpathlineto{\pgfqpoint{0.487476in}{0.484854in}}%
\pgfpathlineto{\pgfqpoint{0.451033in}{0.484854in}}%
\pgfpathlineto{\pgfqpoint{0.414589in}{0.484854in}}%
\pgfpathlineto{\pgfqpoint{0.378146in}{0.484854in}}%
\pgfpathlineto{\pgfqpoint{0.341702in}{0.484854in}}%
\pgfpathlineto{\pgfqpoint{0.305259in}{0.484854in}}%
\pgfpathlineto{\pgfqpoint{0.268815in}{0.484854in}}%
\pgfpathlineto{\pgfqpoint{0.232372in}{0.484854in}}%
\pgfpathlineto{\pgfqpoint{0.195929in}{0.484854in}}%
\pgfpathlineto{\pgfqpoint{0.159485in}{0.484854in}}%
\pgfpathlineto{\pgfqpoint{0.123042in}{0.484854in}}%
\pgfpathlineto{\pgfqpoint{0.086598in}{0.484854in}}%
\pgfpathlineto{\pgfqpoint{0.050155in}{0.484854in}}%
\pgfpathlineto{\pgfqpoint{0.013711in}{0.484854in}}%
\pgfpathlineto{\pgfqpoint{-0.022732in}{0.484854in}}%
\pgfpathlineto{\pgfqpoint{-0.059176in}{0.484854in}}%
\pgfpathlineto{\pgfqpoint{-0.095619in}{0.484854in}}%
\pgfpathlineto{\pgfqpoint{-0.132062in}{0.484854in}}%
\pgfpathlineto{\pgfqpoint{-0.168506in}{0.484854in}}%
\pgfpathlineto{\pgfqpoint{-0.204949in}{0.484854in}}%
\pgfpathlineto{\pgfqpoint{-0.241393in}{0.484854in}}%
\pgfpathlineto{\pgfqpoint{-0.277836in}{0.484854in}}%
\pgfpathlineto{\pgfqpoint{-0.314280in}{0.484854in}}%
\pgfpathlineto{\pgfqpoint{-0.350723in}{0.484854in}}%
\pgfpathlineto{\pgfqpoint{-0.387167in}{0.484854in}}%
\pgfpathlineto{\pgfqpoint{-0.423610in}{0.484854in}}%
\pgfpathlineto{\pgfqpoint{-0.460054in}{0.484854in}}%
\pgfpathlineto{\pgfqpoint{-0.496497in}{0.484854in}}%
\pgfpathlineto{\pgfqpoint{-0.532940in}{0.484854in}}%
\pgfpathlineto{\pgfqpoint{-0.569384in}{0.484854in}}%
\pgfpathlineto{\pgfqpoint{-0.605827in}{0.484854in}}%
\pgfpathlineto{\pgfqpoint{-0.642274in}{0.484854in}}%
\pgfpathlineto{\pgfqpoint{-0.678718in}{0.484854in}}%
\pgfpathlineto{\pgfqpoint{-0.715161in}{0.484854in}}%
\pgfpathlineto{\pgfqpoint{-0.751605in}{0.484854in}}%
\pgfpathlineto{\pgfqpoint{-0.788045in}{0.484854in}}%
\pgfpathlineto{\pgfqpoint{-0.824488in}{0.484854in}}%
\pgfpathlineto{\pgfqpoint{-0.860931in}{0.484854in}}%
\pgfpathlineto{\pgfqpoint{-0.897375in}{0.484854in}}%
\pgfpathlineto{\pgfqpoint{-0.933818in}{0.484854in}}%
\pgfpathlineto{\pgfqpoint{-0.979373in}{0.484854in}}%
\pgfpathclose%
\pgfusepath{fill}%
\end{pgfscope}%
\begin{pgfscope}%
\pgfpathrectangle{\pgfqpoint{0.478365in}{0.484854in}}{\pgfqpoint{2.186607in}{1.629375in}}%
\pgfusepath{clip}%
\pgfsetbuttcap%
\pgfsetroundjoin%
\definecolor{currentfill}{rgb}{1.000000,0.498039,0.054902}%
\pgfsetfillcolor{currentfill}%
\pgfsetlinewidth{0.000000pt}%
\definecolor{currentstroke}{rgb}{0.000000,0.000000,0.000000}%
\pgfsetstrokecolor{currentstroke}%
\pgfsetdash{}{0pt}%
\pgfpathmoveto{\pgfqpoint{-0.979373in}{0.484854in}}%
\pgfpathlineto{\pgfqpoint{-0.979373in}{0.484854in}}%
\pgfpathlineto{\pgfqpoint{-0.933818in}{0.484854in}}%
\pgfpathlineto{\pgfqpoint{-0.897375in}{0.484854in}}%
\pgfpathlineto{\pgfqpoint{-0.860931in}{0.484854in}}%
\pgfpathlineto{\pgfqpoint{-0.824488in}{0.484854in}}%
\pgfpathlineto{\pgfqpoint{-0.788045in}{0.484854in}}%
\pgfpathlineto{\pgfqpoint{-0.751605in}{0.484854in}}%
\pgfpathlineto{\pgfqpoint{-0.715161in}{0.484854in}}%
\pgfpathlineto{\pgfqpoint{-0.678718in}{0.484854in}}%
\pgfpathlineto{\pgfqpoint{-0.642274in}{0.484854in}}%
\pgfpathlineto{\pgfqpoint{-0.605827in}{0.484854in}}%
\pgfpathlineto{\pgfqpoint{-0.569384in}{0.484854in}}%
\pgfpathlineto{\pgfqpoint{-0.532940in}{0.484854in}}%
\pgfpathlineto{\pgfqpoint{-0.496497in}{0.484854in}}%
\pgfpathlineto{\pgfqpoint{-0.460054in}{0.484854in}}%
\pgfpathlineto{\pgfqpoint{-0.423610in}{0.484854in}}%
\pgfpathlineto{\pgfqpoint{-0.387167in}{0.484854in}}%
\pgfpathlineto{\pgfqpoint{-0.350723in}{0.484854in}}%
\pgfpathlineto{\pgfqpoint{-0.314280in}{0.484854in}}%
\pgfpathlineto{\pgfqpoint{-0.277836in}{0.484854in}}%
\pgfpathlineto{\pgfqpoint{-0.241393in}{0.484854in}}%
\pgfpathlineto{\pgfqpoint{-0.204949in}{0.484854in}}%
\pgfpathlineto{\pgfqpoint{-0.168506in}{0.484854in}}%
\pgfpathlineto{\pgfqpoint{-0.132062in}{0.484854in}}%
\pgfpathlineto{\pgfqpoint{-0.095619in}{0.484854in}}%
\pgfpathlineto{\pgfqpoint{-0.059176in}{0.484854in}}%
\pgfpathlineto{\pgfqpoint{-0.022732in}{0.484854in}}%
\pgfpathlineto{\pgfqpoint{0.013711in}{0.484854in}}%
\pgfpathlineto{\pgfqpoint{0.050155in}{0.484854in}}%
\pgfpathlineto{\pgfqpoint{0.086598in}{0.484854in}}%
\pgfpathlineto{\pgfqpoint{0.123042in}{0.484854in}}%
\pgfpathlineto{\pgfqpoint{0.159485in}{0.484854in}}%
\pgfpathlineto{\pgfqpoint{0.195929in}{0.484854in}}%
\pgfpathlineto{\pgfqpoint{0.232372in}{0.484854in}}%
\pgfpathlineto{\pgfqpoint{0.268815in}{0.484854in}}%
\pgfpathlineto{\pgfqpoint{0.305259in}{0.484854in}}%
\pgfpathlineto{\pgfqpoint{0.341702in}{0.484854in}}%
\pgfpathlineto{\pgfqpoint{0.378146in}{0.484854in}}%
\pgfpathlineto{\pgfqpoint{0.414589in}{0.484854in}}%
\pgfpathlineto{\pgfqpoint{0.451033in}{0.484854in}}%
\pgfpathlineto{\pgfqpoint{0.487476in}{0.484854in}}%
\pgfpathlineto{\pgfqpoint{0.523920in}{0.484854in}}%
\pgfpathlineto{\pgfqpoint{0.560363in}{0.484854in}}%
\pgfpathlineto{\pgfqpoint{0.596806in}{0.484854in}}%
\pgfpathlineto{\pgfqpoint{0.633250in}{0.484854in}}%
\pgfpathlineto{\pgfqpoint{0.669693in}{0.484854in}}%
\pgfpathlineto{\pgfqpoint{0.706137in}{0.484854in}}%
\pgfpathlineto{\pgfqpoint{0.742580in}{0.484854in}}%
\pgfpathlineto{\pgfqpoint{0.779024in}{0.484854in}}%
\pgfpathlineto{\pgfqpoint{0.815467in}{0.484854in}}%
\pgfpathlineto{\pgfqpoint{0.851911in}{0.484859in}}%
\pgfpathlineto{\pgfqpoint{0.888026in}{0.484854in}}%
\pgfpathlineto{\pgfqpoint{0.922939in}{0.486970in}}%
\pgfpathlineto{\pgfqpoint{0.955629in}{0.493915in}}%
\pgfpathlineto{\pgfqpoint{0.993348in}{0.510848in}}%
\pgfpathlineto{\pgfqpoint{1.028078in}{0.535511in}}%
\pgfpathlineto{\pgfqpoint{1.066271in}{0.559987in}}%
\pgfpathlineto{\pgfqpoint{1.101475in}{0.584486in}}%
\pgfpathlineto{\pgfqpoint{1.139522in}{0.603392in}}%
\pgfpathlineto{\pgfqpoint{1.175273in}{0.609209in}}%
\pgfpathlineto{\pgfqpoint{1.209348in}{0.605452in}}%
\pgfpathlineto{\pgfqpoint{1.246630in}{0.606498in}}%
\pgfpathlineto{\pgfqpoint{1.282235in}{0.608034in}}%
\pgfpathlineto{\pgfqpoint{1.316893in}{0.609421in}}%
\pgfpathlineto{\pgfqpoint{1.355268in}{0.606637in}}%
\pgfpathlineto{\pgfqpoint{1.393242in}{0.601918in}}%
\pgfpathlineto{\pgfqpoint{1.426879in}{0.601164in}}%
\pgfpathlineto{\pgfqpoint{1.463833in}{0.610914in}}%
\pgfpathlineto{\pgfqpoint{1.502098in}{0.651001in}}%
\pgfpathlineto{\pgfqpoint{1.535335in}{0.630821in}}%
\pgfpathlineto{\pgfqpoint{1.571815in}{0.622178in}}%
\pgfpathlineto{\pgfqpoint{1.608149in}{0.630648in}}%
\pgfpathlineto{\pgfqpoint{1.647763in}{0.628526in}}%
\pgfpathlineto{\pgfqpoint{1.683404in}{0.632006in}}%
\pgfpathlineto{\pgfqpoint{1.718390in}{0.636421in}}%
\pgfpathlineto{\pgfqpoint{1.756401in}{0.632389in}}%
\pgfpathlineto{\pgfqpoint{1.790402in}{0.633775in}}%
\pgfpathlineto{\pgfqpoint{1.827356in}{0.647158in}}%
\pgfpathlineto{\pgfqpoint{1.866059in}{0.642198in}}%
\pgfpathlineto{\pgfqpoint{1.900206in}{0.644996in}}%
\pgfpathlineto{\pgfqpoint{1.936723in}{0.646587in}}%
\pgfpathlineto{\pgfqpoint{1.975389in}{0.647252in}}%
\pgfpathlineto{\pgfqpoint{2.010047in}{0.649104in}}%
\pgfpathlineto{\pgfqpoint{2.046673in}{0.650923in}}%
\pgfpathlineto{\pgfqpoint{2.082788in}{0.655472in}}%
\pgfpathlineto{\pgfqpoint{2.118357in}{0.651309in}}%
\pgfpathlineto{\pgfqpoint{2.156367in}{0.654939in}}%
\pgfpathlineto{\pgfqpoint{2.193831in}{0.653140in}}%
\pgfpathlineto{\pgfqpoint{2.228161in}{0.658314in}}%
\pgfpathlineto{\pgfqpoint{2.264823in}{0.651097in}}%
\pgfpathlineto{\pgfqpoint{2.301121in}{0.654694in}}%
\pgfpathlineto{\pgfqpoint{2.337637in}{0.655454in}}%
\pgfpathlineto{\pgfqpoint{2.374372in}{0.667190in}}%
\pgfpathlineto{\pgfqpoint{2.410196in}{0.658329in}}%
\pgfpathlineto{\pgfqpoint{2.447223in}{0.654781in}}%
\pgfpathlineto{\pgfqpoint{2.483156in}{0.657130in}}%
\pgfpathlineto{\pgfqpoint{2.520073in}{0.649189in}}%
\pgfpathlineto{\pgfqpoint{2.556917in}{0.654091in}}%
\pgfpathlineto{\pgfqpoint{2.594235in}{0.656614in}}%
\pgfpathlineto{\pgfqpoint{2.630606in}{0.652531in}}%
\pgfpathlineto{\pgfqpoint{2.664972in}{0.654454in}}%
\pgfpathlineto{\pgfqpoint{2.664972in}{0.776915in}}%
\pgfpathlineto{\pgfqpoint{2.664972in}{0.776915in}}%
\pgfpathlineto{\pgfqpoint{2.630606in}{0.775748in}}%
\pgfpathlineto{\pgfqpoint{2.594235in}{0.782318in}}%
\pgfpathlineto{\pgfqpoint{2.556917in}{0.775403in}}%
\pgfpathlineto{\pgfqpoint{2.520073in}{0.762953in}}%
\pgfpathlineto{\pgfqpoint{2.483156in}{0.779940in}}%
\pgfpathlineto{\pgfqpoint{2.447223in}{0.775824in}}%
\pgfpathlineto{\pgfqpoint{2.410196in}{0.780113in}}%
\pgfpathlineto{\pgfqpoint{2.374372in}{0.796121in}}%
\pgfpathlineto{\pgfqpoint{2.337637in}{0.778902in}}%
\pgfpathlineto{\pgfqpoint{2.301121in}{0.778752in}}%
\pgfpathlineto{\pgfqpoint{2.264823in}{0.773996in}}%
\pgfpathlineto{\pgfqpoint{2.228161in}{0.785001in}}%
\pgfpathlineto{\pgfqpoint{2.193831in}{0.777269in}}%
\pgfpathlineto{\pgfqpoint{2.156367in}{0.778445in}}%
\pgfpathlineto{\pgfqpoint{2.118357in}{0.773783in}}%
\pgfpathlineto{\pgfqpoint{2.082788in}{0.779070in}}%
\pgfpathlineto{\pgfqpoint{2.046673in}{0.771895in}}%
\pgfpathlineto{\pgfqpoint{2.010047in}{0.767482in}}%
\pgfpathlineto{\pgfqpoint{1.975389in}{0.765118in}}%
\pgfpathlineto{\pgfqpoint{1.936723in}{0.763041in}}%
\pgfpathlineto{\pgfqpoint{1.900206in}{0.760024in}}%
\pgfpathlineto{\pgfqpoint{1.866059in}{0.755011in}}%
\pgfpathlineto{\pgfqpoint{1.827356in}{0.761274in}}%
\pgfpathlineto{\pgfqpoint{1.790402in}{0.731277in}}%
\pgfpathlineto{\pgfqpoint{1.756401in}{0.729480in}}%
\pgfpathlineto{\pgfqpoint{1.718390in}{0.736591in}}%
\pgfpathlineto{\pgfqpoint{1.683404in}{0.730056in}}%
\pgfpathlineto{\pgfqpoint{1.647763in}{0.725012in}}%
\pgfpathlineto{\pgfqpoint{1.608149in}{0.728986in}}%
\pgfpathlineto{\pgfqpoint{1.571815in}{0.715912in}}%
\pgfpathlineto{\pgfqpoint{1.535335in}{0.730531in}}%
\pgfpathlineto{\pgfqpoint{1.502098in}{0.764152in}}%
\pgfpathlineto{\pgfqpoint{1.463833in}{0.698136in}}%
\pgfpathlineto{\pgfqpoint{1.426879in}{0.680398in}}%
\pgfpathlineto{\pgfqpoint{1.393242in}{0.681112in}}%
\pgfpathlineto{\pgfqpoint{1.355268in}{0.688245in}}%
\pgfpathlineto{\pgfqpoint{1.316893in}{0.692273in}}%
\pgfpathlineto{\pgfqpoint{1.282235in}{0.689495in}}%
\pgfpathlineto{\pgfqpoint{1.246630in}{0.686586in}}%
\pgfpathlineto{\pgfqpoint{1.209348in}{0.685273in}}%
\pgfpathlineto{\pgfqpoint{1.175273in}{0.691685in}}%
\pgfpathlineto{\pgfqpoint{1.139522in}{0.682034in}}%
\pgfpathlineto{\pgfqpoint{1.101475in}{0.650808in}}%
\pgfpathlineto{\pgfqpoint{1.066271in}{0.610133in}}%
\pgfpathlineto{\pgfqpoint{1.028078in}{0.569412in}}%
\pgfpathlineto{\pgfqpoint{0.993348in}{0.528367in}}%
\pgfpathlineto{\pgfqpoint{0.955629in}{0.500043in}}%
\pgfpathlineto{\pgfqpoint{0.922939in}{0.488408in}}%
\pgfpathlineto{\pgfqpoint{0.888026in}{0.484854in}}%
\pgfpathlineto{\pgfqpoint{0.851911in}{0.484862in}}%
\pgfpathlineto{\pgfqpoint{0.815467in}{0.484854in}}%
\pgfpathlineto{\pgfqpoint{0.779024in}{0.484854in}}%
\pgfpathlineto{\pgfqpoint{0.742580in}{0.484854in}}%
\pgfpathlineto{\pgfqpoint{0.706137in}{0.484854in}}%
\pgfpathlineto{\pgfqpoint{0.669693in}{0.484854in}}%
\pgfpathlineto{\pgfqpoint{0.633250in}{0.484854in}}%
\pgfpathlineto{\pgfqpoint{0.596806in}{0.484854in}}%
\pgfpathlineto{\pgfqpoint{0.560363in}{0.484854in}}%
\pgfpathlineto{\pgfqpoint{0.523920in}{0.484854in}}%
\pgfpathlineto{\pgfqpoint{0.487476in}{0.484854in}}%
\pgfpathlineto{\pgfqpoint{0.451033in}{0.484854in}}%
\pgfpathlineto{\pgfqpoint{0.414589in}{0.484854in}}%
\pgfpathlineto{\pgfqpoint{0.378146in}{0.484854in}}%
\pgfpathlineto{\pgfqpoint{0.341702in}{0.484854in}}%
\pgfpathlineto{\pgfqpoint{0.305259in}{0.484854in}}%
\pgfpathlineto{\pgfqpoint{0.268815in}{0.484854in}}%
\pgfpathlineto{\pgfqpoint{0.232372in}{0.484854in}}%
\pgfpathlineto{\pgfqpoint{0.195929in}{0.484854in}}%
\pgfpathlineto{\pgfqpoint{0.159485in}{0.484854in}}%
\pgfpathlineto{\pgfqpoint{0.123042in}{0.484854in}}%
\pgfpathlineto{\pgfqpoint{0.086598in}{0.484854in}}%
\pgfpathlineto{\pgfqpoint{0.050155in}{0.484854in}}%
\pgfpathlineto{\pgfqpoint{0.013711in}{0.484854in}}%
\pgfpathlineto{\pgfqpoint{-0.022732in}{0.484854in}}%
\pgfpathlineto{\pgfqpoint{-0.059176in}{0.484854in}}%
\pgfpathlineto{\pgfqpoint{-0.095619in}{0.484854in}}%
\pgfpathlineto{\pgfqpoint{-0.132062in}{0.484854in}}%
\pgfpathlineto{\pgfqpoint{-0.168506in}{0.484854in}}%
\pgfpathlineto{\pgfqpoint{-0.204949in}{0.484854in}}%
\pgfpathlineto{\pgfqpoint{-0.241393in}{0.484854in}}%
\pgfpathlineto{\pgfqpoint{-0.277836in}{0.484854in}}%
\pgfpathlineto{\pgfqpoint{-0.314280in}{0.484854in}}%
\pgfpathlineto{\pgfqpoint{-0.350723in}{0.484854in}}%
\pgfpathlineto{\pgfqpoint{-0.387167in}{0.484854in}}%
\pgfpathlineto{\pgfqpoint{-0.423610in}{0.484854in}}%
\pgfpathlineto{\pgfqpoint{-0.460054in}{0.484854in}}%
\pgfpathlineto{\pgfqpoint{-0.496497in}{0.484854in}}%
\pgfpathlineto{\pgfqpoint{-0.532940in}{0.484854in}}%
\pgfpathlineto{\pgfqpoint{-0.569384in}{0.484854in}}%
\pgfpathlineto{\pgfqpoint{-0.605827in}{0.484854in}}%
\pgfpathlineto{\pgfqpoint{-0.642274in}{0.484854in}}%
\pgfpathlineto{\pgfqpoint{-0.678718in}{0.484854in}}%
\pgfpathlineto{\pgfqpoint{-0.715161in}{0.484854in}}%
\pgfpathlineto{\pgfqpoint{-0.751605in}{0.484854in}}%
\pgfpathlineto{\pgfqpoint{-0.788045in}{0.484854in}}%
\pgfpathlineto{\pgfqpoint{-0.824488in}{0.484854in}}%
\pgfpathlineto{\pgfqpoint{-0.860931in}{0.484854in}}%
\pgfpathlineto{\pgfqpoint{-0.897375in}{0.484854in}}%
\pgfpathlineto{\pgfqpoint{-0.933818in}{0.484854in}}%
\pgfpathlineto{\pgfqpoint{-0.979373in}{0.484854in}}%
\pgfpathclose%
\pgfusepath{fill}%
\end{pgfscope}%
\begin{pgfscope}%
\pgfpathrectangle{\pgfqpoint{0.478365in}{0.484854in}}{\pgfqpoint{2.186607in}{1.629375in}}%
\pgfusepath{clip}%
\pgfsetbuttcap%
\pgfsetroundjoin%
\definecolor{currentfill}{rgb}{0.172549,0.627451,0.172549}%
\pgfsetfillcolor{currentfill}%
\pgfsetlinewidth{0.000000pt}%
\definecolor{currentstroke}{rgb}{0.000000,0.000000,0.000000}%
\pgfsetstrokecolor{currentstroke}%
\pgfsetdash{}{0pt}%
\pgfpathmoveto{\pgfqpoint{-0.979373in}{0.484854in}}%
\pgfpathlineto{\pgfqpoint{-0.979373in}{0.484854in}}%
\pgfpathlineto{\pgfqpoint{-0.933818in}{0.484854in}}%
\pgfpathlineto{\pgfqpoint{-0.897375in}{0.484854in}}%
\pgfpathlineto{\pgfqpoint{-0.860931in}{0.484854in}}%
\pgfpathlineto{\pgfqpoint{-0.824488in}{0.484854in}}%
\pgfpathlineto{\pgfqpoint{-0.788045in}{0.484854in}}%
\pgfpathlineto{\pgfqpoint{-0.751605in}{0.484854in}}%
\pgfpathlineto{\pgfqpoint{-0.715161in}{0.484854in}}%
\pgfpathlineto{\pgfqpoint{-0.678718in}{0.484854in}}%
\pgfpathlineto{\pgfqpoint{-0.642274in}{0.484854in}}%
\pgfpathlineto{\pgfqpoint{-0.605827in}{0.484854in}}%
\pgfpathlineto{\pgfqpoint{-0.569384in}{0.484854in}}%
\pgfpathlineto{\pgfqpoint{-0.532940in}{0.484854in}}%
\pgfpathlineto{\pgfqpoint{-0.496497in}{0.484854in}}%
\pgfpathlineto{\pgfqpoint{-0.460054in}{0.484854in}}%
\pgfpathlineto{\pgfqpoint{-0.423610in}{0.484854in}}%
\pgfpathlineto{\pgfqpoint{-0.387167in}{0.484854in}}%
\pgfpathlineto{\pgfqpoint{-0.350723in}{0.484854in}}%
\pgfpathlineto{\pgfqpoint{-0.314280in}{0.484854in}}%
\pgfpathlineto{\pgfqpoint{-0.277836in}{0.484854in}}%
\pgfpathlineto{\pgfqpoint{-0.241393in}{0.484854in}}%
\pgfpathlineto{\pgfqpoint{-0.204949in}{0.484854in}}%
\pgfpathlineto{\pgfqpoint{-0.168506in}{0.484854in}}%
\pgfpathlineto{\pgfqpoint{-0.132062in}{0.484854in}}%
\pgfpathlineto{\pgfqpoint{-0.095619in}{0.484854in}}%
\pgfpathlineto{\pgfqpoint{-0.059176in}{0.484854in}}%
\pgfpathlineto{\pgfqpoint{-0.022732in}{0.484854in}}%
\pgfpathlineto{\pgfqpoint{0.013711in}{0.484854in}}%
\pgfpathlineto{\pgfqpoint{0.050155in}{0.484854in}}%
\pgfpathlineto{\pgfqpoint{0.086598in}{0.484854in}}%
\pgfpathlineto{\pgfqpoint{0.123042in}{0.484854in}}%
\pgfpathlineto{\pgfqpoint{0.159485in}{0.484854in}}%
\pgfpathlineto{\pgfqpoint{0.195929in}{0.484854in}}%
\pgfpathlineto{\pgfqpoint{0.232372in}{0.484854in}}%
\pgfpathlineto{\pgfqpoint{0.268815in}{0.484854in}}%
\pgfpathlineto{\pgfqpoint{0.305259in}{0.484854in}}%
\pgfpathlineto{\pgfqpoint{0.341702in}{0.484854in}}%
\pgfpathlineto{\pgfqpoint{0.378146in}{0.484854in}}%
\pgfpathlineto{\pgfqpoint{0.414589in}{0.484854in}}%
\pgfpathlineto{\pgfqpoint{0.451033in}{0.484854in}}%
\pgfpathlineto{\pgfqpoint{0.487476in}{0.484854in}}%
\pgfpathlineto{\pgfqpoint{0.523920in}{0.484854in}}%
\pgfpathlineto{\pgfqpoint{0.560363in}{0.484854in}}%
\pgfpathlineto{\pgfqpoint{0.596806in}{0.484854in}}%
\pgfpathlineto{\pgfqpoint{0.633250in}{0.484854in}}%
\pgfpathlineto{\pgfqpoint{0.669693in}{0.484854in}}%
\pgfpathlineto{\pgfqpoint{0.706137in}{0.484854in}}%
\pgfpathlineto{\pgfqpoint{0.742580in}{0.484854in}}%
\pgfpathlineto{\pgfqpoint{0.779024in}{0.484854in}}%
\pgfpathlineto{\pgfqpoint{0.815467in}{0.484854in}}%
\pgfpathlineto{\pgfqpoint{0.851911in}{0.484862in}}%
\pgfpathlineto{\pgfqpoint{0.888026in}{0.484854in}}%
\pgfpathlineto{\pgfqpoint{0.922939in}{0.488408in}}%
\pgfpathlineto{\pgfqpoint{0.955629in}{0.500043in}}%
\pgfpathlineto{\pgfqpoint{0.993348in}{0.528367in}}%
\pgfpathlineto{\pgfqpoint{1.028078in}{0.569412in}}%
\pgfpathlineto{\pgfqpoint{1.066271in}{0.610133in}}%
\pgfpathlineto{\pgfqpoint{1.101475in}{0.650808in}}%
\pgfpathlineto{\pgfqpoint{1.139522in}{0.682034in}}%
\pgfpathlineto{\pgfqpoint{1.175273in}{0.691685in}}%
\pgfpathlineto{\pgfqpoint{1.209348in}{0.685273in}}%
\pgfpathlineto{\pgfqpoint{1.246630in}{0.686586in}}%
\pgfpathlineto{\pgfqpoint{1.282235in}{0.689495in}}%
\pgfpathlineto{\pgfqpoint{1.316893in}{0.692273in}}%
\pgfpathlineto{\pgfqpoint{1.355268in}{0.688245in}}%
\pgfpathlineto{\pgfqpoint{1.393242in}{0.681112in}}%
\pgfpathlineto{\pgfqpoint{1.426879in}{0.680398in}}%
\pgfpathlineto{\pgfqpoint{1.463833in}{0.698136in}}%
\pgfpathlineto{\pgfqpoint{1.502098in}{0.764152in}}%
\pgfpathlineto{\pgfqpoint{1.535335in}{0.730531in}}%
\pgfpathlineto{\pgfqpoint{1.571815in}{0.715912in}}%
\pgfpathlineto{\pgfqpoint{1.608149in}{0.728986in}}%
\pgfpathlineto{\pgfqpoint{1.647763in}{0.725012in}}%
\pgfpathlineto{\pgfqpoint{1.683404in}{0.730056in}}%
\pgfpathlineto{\pgfqpoint{1.718390in}{0.736591in}}%
\pgfpathlineto{\pgfqpoint{1.756401in}{0.729480in}}%
\pgfpathlineto{\pgfqpoint{1.790402in}{0.731277in}}%
\pgfpathlineto{\pgfqpoint{1.827356in}{0.761274in}}%
\pgfpathlineto{\pgfqpoint{1.866059in}{0.755011in}}%
\pgfpathlineto{\pgfqpoint{1.900206in}{0.760024in}}%
\pgfpathlineto{\pgfqpoint{1.936723in}{0.763041in}}%
\pgfpathlineto{\pgfqpoint{1.975389in}{0.765118in}}%
\pgfpathlineto{\pgfqpoint{2.010047in}{0.767482in}}%
\pgfpathlineto{\pgfqpoint{2.046673in}{0.771895in}}%
\pgfpathlineto{\pgfqpoint{2.082788in}{0.779070in}}%
\pgfpathlineto{\pgfqpoint{2.118357in}{0.773783in}}%
\pgfpathlineto{\pgfqpoint{2.156367in}{0.778445in}}%
\pgfpathlineto{\pgfqpoint{2.193831in}{0.777269in}}%
\pgfpathlineto{\pgfqpoint{2.228161in}{0.785001in}}%
\pgfpathlineto{\pgfqpoint{2.264823in}{0.773996in}}%
\pgfpathlineto{\pgfqpoint{2.301121in}{0.778752in}}%
\pgfpathlineto{\pgfqpoint{2.337637in}{0.778902in}}%
\pgfpathlineto{\pgfqpoint{2.374372in}{0.796121in}}%
\pgfpathlineto{\pgfqpoint{2.410196in}{0.780113in}}%
\pgfpathlineto{\pgfqpoint{2.447223in}{0.775824in}}%
\pgfpathlineto{\pgfqpoint{2.483156in}{0.779940in}}%
\pgfpathlineto{\pgfqpoint{2.520073in}{0.762953in}}%
\pgfpathlineto{\pgfqpoint{2.556917in}{0.775403in}}%
\pgfpathlineto{\pgfqpoint{2.594235in}{0.782318in}}%
\pgfpathlineto{\pgfqpoint{2.630606in}{0.775748in}}%
\pgfpathlineto{\pgfqpoint{2.664972in}{0.776915in}}%
\pgfpathlineto{\pgfqpoint{2.664972in}{1.187273in}}%
\pgfpathlineto{\pgfqpoint{2.664972in}{1.187273in}}%
\pgfpathlineto{\pgfqpoint{2.630606in}{1.189803in}}%
\pgfpathlineto{\pgfqpoint{2.594235in}{1.205108in}}%
\pgfpathlineto{\pgfqpoint{2.556917in}{1.184343in}}%
\pgfpathlineto{\pgfqpoint{2.520073in}{1.149015in}}%
\pgfpathlineto{\pgfqpoint{2.483156in}{1.200139in}}%
\pgfpathlineto{\pgfqpoint{2.447223in}{1.192557in}}%
\pgfpathlineto{\pgfqpoint{2.410196in}{1.203026in}}%
\pgfpathlineto{\pgfqpoint{2.374372in}{1.252590in}}%
\pgfpathlineto{\pgfqpoint{2.337637in}{1.225324in}}%
\pgfpathlineto{\pgfqpoint{2.301121in}{1.223070in}}%
\pgfpathlineto{\pgfqpoint{2.264823in}{1.211963in}}%
\pgfpathlineto{\pgfqpoint{2.228161in}{1.232964in}}%
\pgfpathlineto{\pgfqpoint{2.193831in}{1.213488in}}%
\pgfpathlineto{\pgfqpoint{2.156367in}{1.207066in}}%
\pgfpathlineto{\pgfqpoint{2.118357in}{1.197005in}}%
\pgfpathlineto{\pgfqpoint{2.082788in}{1.200829in}}%
\pgfpathlineto{\pgfqpoint{2.046673in}{1.185280in}}%
\pgfpathlineto{\pgfqpoint{2.010047in}{1.167784in}}%
\pgfpathlineto{\pgfqpoint{1.975389in}{1.161127in}}%
\pgfpathlineto{\pgfqpoint{1.936723in}{1.150593in}}%
\pgfpathlineto{\pgfqpoint{1.900206in}{1.139176in}}%
\pgfpathlineto{\pgfqpoint{1.866059in}{1.124426in}}%
\pgfpathlineto{\pgfqpoint{1.827356in}{1.130631in}}%
\pgfpathlineto{\pgfqpoint{1.790402in}{1.035750in}}%
\pgfpathlineto{\pgfqpoint{1.756401in}{1.032886in}}%
\pgfpathlineto{\pgfqpoint{1.718390in}{1.048483in}}%
\pgfpathlineto{\pgfqpoint{1.683404in}{1.034989in}}%
\pgfpathlineto{\pgfqpoint{1.647763in}{1.024390in}}%
\pgfpathlineto{\pgfqpoint{1.608149in}{1.032822in}}%
\pgfpathlineto{\pgfqpoint{1.571815in}{1.005234in}}%
\pgfpathlineto{\pgfqpoint{1.535335in}{1.037822in}}%
\pgfpathlineto{\pgfqpoint{1.502098in}{1.108934in}}%
\pgfpathlineto{\pgfqpoint{1.463833in}{0.959369in}}%
\pgfpathlineto{\pgfqpoint{1.426879in}{0.914709in}}%
\pgfpathlineto{\pgfqpoint{1.393242in}{0.914898in}}%
\pgfpathlineto{\pgfqpoint{1.355268in}{0.928484in}}%
\pgfpathlineto{\pgfqpoint{1.316893in}{0.935655in}}%
\pgfpathlineto{\pgfqpoint{1.282235in}{0.927346in}}%
\pgfpathlineto{\pgfqpoint{1.246630in}{0.916700in}}%
\pgfpathlineto{\pgfqpoint{1.209348in}{0.912014in}}%
\pgfpathlineto{\pgfqpoint{1.175273in}{0.923799in}}%
\pgfpathlineto{\pgfqpoint{1.139522in}{0.900996in}}%
\pgfpathlineto{\pgfqpoint{1.101475in}{0.833340in}}%
\pgfpathlineto{\pgfqpoint{1.066271in}{0.745990in}}%
\pgfpathlineto{\pgfqpoint{1.028078in}{0.659920in}}%
\pgfpathlineto{\pgfqpoint{0.993348in}{0.574527in}}%
\pgfpathlineto{\pgfqpoint{0.955629in}{0.516085in}}%
\pgfpathlineto{\pgfqpoint{0.922939in}{0.492133in}}%
\pgfpathlineto{\pgfqpoint{0.888026in}{0.484854in}}%
\pgfpathlineto{\pgfqpoint{0.851911in}{0.484871in}}%
\pgfpathlineto{\pgfqpoint{0.815467in}{0.484854in}}%
\pgfpathlineto{\pgfqpoint{0.779024in}{0.484854in}}%
\pgfpathlineto{\pgfqpoint{0.742580in}{0.484854in}}%
\pgfpathlineto{\pgfqpoint{0.706137in}{0.484854in}}%
\pgfpathlineto{\pgfqpoint{0.669693in}{0.484854in}}%
\pgfpathlineto{\pgfqpoint{0.633250in}{0.484854in}}%
\pgfpathlineto{\pgfqpoint{0.596806in}{0.484854in}}%
\pgfpathlineto{\pgfqpoint{0.560363in}{0.484854in}}%
\pgfpathlineto{\pgfqpoint{0.523920in}{0.484854in}}%
\pgfpathlineto{\pgfqpoint{0.487476in}{0.484854in}}%
\pgfpathlineto{\pgfqpoint{0.451033in}{0.484854in}}%
\pgfpathlineto{\pgfqpoint{0.414589in}{0.484854in}}%
\pgfpathlineto{\pgfqpoint{0.378146in}{0.484854in}}%
\pgfpathlineto{\pgfqpoint{0.341702in}{0.484854in}}%
\pgfpathlineto{\pgfqpoint{0.305259in}{0.484854in}}%
\pgfpathlineto{\pgfqpoint{0.268815in}{0.484854in}}%
\pgfpathlineto{\pgfqpoint{0.232372in}{0.484854in}}%
\pgfpathlineto{\pgfqpoint{0.195929in}{0.484854in}}%
\pgfpathlineto{\pgfqpoint{0.159485in}{0.484854in}}%
\pgfpathlineto{\pgfqpoint{0.123042in}{0.484854in}}%
\pgfpathlineto{\pgfqpoint{0.086598in}{0.484854in}}%
\pgfpathlineto{\pgfqpoint{0.050155in}{0.484854in}}%
\pgfpathlineto{\pgfqpoint{0.013711in}{0.484854in}}%
\pgfpathlineto{\pgfqpoint{-0.022732in}{0.484854in}}%
\pgfpathlineto{\pgfqpoint{-0.059176in}{0.484854in}}%
\pgfpathlineto{\pgfqpoint{-0.095619in}{0.484854in}}%
\pgfpathlineto{\pgfqpoint{-0.132062in}{0.484854in}}%
\pgfpathlineto{\pgfqpoint{-0.168506in}{0.484854in}}%
\pgfpathlineto{\pgfqpoint{-0.204949in}{0.484854in}}%
\pgfpathlineto{\pgfqpoint{-0.241393in}{0.484854in}}%
\pgfpathlineto{\pgfqpoint{-0.277836in}{0.484854in}}%
\pgfpathlineto{\pgfqpoint{-0.314280in}{0.484854in}}%
\pgfpathlineto{\pgfqpoint{-0.350723in}{0.484854in}}%
\pgfpathlineto{\pgfqpoint{-0.387167in}{0.484854in}}%
\pgfpathlineto{\pgfqpoint{-0.423610in}{0.484854in}}%
\pgfpathlineto{\pgfqpoint{-0.460054in}{0.484854in}}%
\pgfpathlineto{\pgfqpoint{-0.496497in}{0.484854in}}%
\pgfpathlineto{\pgfqpoint{-0.532940in}{0.484854in}}%
\pgfpathlineto{\pgfqpoint{-0.569384in}{0.484854in}}%
\pgfpathlineto{\pgfqpoint{-0.605827in}{0.484854in}}%
\pgfpathlineto{\pgfqpoint{-0.642274in}{0.484854in}}%
\pgfpathlineto{\pgfqpoint{-0.678718in}{0.484854in}}%
\pgfpathlineto{\pgfqpoint{-0.715161in}{0.484854in}}%
\pgfpathlineto{\pgfqpoint{-0.751605in}{0.484854in}}%
\pgfpathlineto{\pgfqpoint{-0.788045in}{0.484854in}}%
\pgfpathlineto{\pgfqpoint{-0.824488in}{0.484854in}}%
\pgfpathlineto{\pgfqpoint{-0.860931in}{0.484854in}}%
\pgfpathlineto{\pgfqpoint{-0.897375in}{0.484854in}}%
\pgfpathlineto{\pgfqpoint{-0.933818in}{0.484854in}}%
\pgfpathlineto{\pgfqpoint{-0.979373in}{0.484854in}}%
\pgfpathclose%
\pgfusepath{fill}%
\end{pgfscope}%
\begin{pgfscope}%
\pgfpathrectangle{\pgfqpoint{0.478365in}{0.484854in}}{\pgfqpoint{2.186607in}{1.629375in}}%
\pgfusepath{clip}%
\pgfsetbuttcap%
\pgfsetroundjoin%
\definecolor{currentfill}{rgb}{0.839216,0.152941,0.156863}%
\pgfsetfillcolor{currentfill}%
\pgfsetlinewidth{0.000000pt}%
\definecolor{currentstroke}{rgb}{0.000000,0.000000,0.000000}%
\pgfsetstrokecolor{currentstroke}%
\pgfsetdash{}{0pt}%
\pgfpathmoveto{\pgfqpoint{-0.979373in}{0.484854in}}%
\pgfpathlineto{\pgfqpoint{-0.979373in}{0.484854in}}%
\pgfpathlineto{\pgfqpoint{-0.933818in}{0.484854in}}%
\pgfpathlineto{\pgfqpoint{-0.897375in}{0.484854in}}%
\pgfpathlineto{\pgfqpoint{-0.860931in}{0.484854in}}%
\pgfpathlineto{\pgfqpoint{-0.824488in}{0.484854in}}%
\pgfpathlineto{\pgfqpoint{-0.788045in}{0.484854in}}%
\pgfpathlineto{\pgfqpoint{-0.751605in}{0.484854in}}%
\pgfpathlineto{\pgfqpoint{-0.715161in}{0.484854in}}%
\pgfpathlineto{\pgfqpoint{-0.678718in}{0.484854in}}%
\pgfpathlineto{\pgfqpoint{-0.642274in}{0.484854in}}%
\pgfpathlineto{\pgfqpoint{-0.605827in}{0.484854in}}%
\pgfpathlineto{\pgfqpoint{-0.569384in}{0.484854in}}%
\pgfpathlineto{\pgfqpoint{-0.532940in}{0.484854in}}%
\pgfpathlineto{\pgfqpoint{-0.496497in}{0.484854in}}%
\pgfpathlineto{\pgfqpoint{-0.460054in}{0.484854in}}%
\pgfpathlineto{\pgfqpoint{-0.423610in}{0.484854in}}%
\pgfpathlineto{\pgfqpoint{-0.387167in}{0.484854in}}%
\pgfpathlineto{\pgfqpoint{-0.350723in}{0.484854in}}%
\pgfpathlineto{\pgfqpoint{-0.314280in}{0.484854in}}%
\pgfpathlineto{\pgfqpoint{-0.277836in}{0.484854in}}%
\pgfpathlineto{\pgfqpoint{-0.241393in}{0.484854in}}%
\pgfpathlineto{\pgfqpoint{-0.204949in}{0.484854in}}%
\pgfpathlineto{\pgfqpoint{-0.168506in}{0.484854in}}%
\pgfpathlineto{\pgfqpoint{-0.132062in}{0.484854in}}%
\pgfpathlineto{\pgfqpoint{-0.095619in}{0.484854in}}%
\pgfpathlineto{\pgfqpoint{-0.059176in}{0.484854in}}%
\pgfpathlineto{\pgfqpoint{-0.022732in}{0.484854in}}%
\pgfpathlineto{\pgfqpoint{0.013711in}{0.484854in}}%
\pgfpathlineto{\pgfqpoint{0.050155in}{0.484854in}}%
\pgfpathlineto{\pgfqpoint{0.086598in}{0.484854in}}%
\pgfpathlineto{\pgfqpoint{0.123042in}{0.484854in}}%
\pgfpathlineto{\pgfqpoint{0.159485in}{0.484854in}}%
\pgfpathlineto{\pgfqpoint{0.195929in}{0.484854in}}%
\pgfpathlineto{\pgfqpoint{0.232372in}{0.484854in}}%
\pgfpathlineto{\pgfqpoint{0.268815in}{0.484854in}}%
\pgfpathlineto{\pgfqpoint{0.305259in}{0.484854in}}%
\pgfpathlineto{\pgfqpoint{0.341702in}{0.484854in}}%
\pgfpathlineto{\pgfqpoint{0.378146in}{0.484854in}}%
\pgfpathlineto{\pgfqpoint{0.414589in}{0.484854in}}%
\pgfpathlineto{\pgfqpoint{0.451033in}{0.484854in}}%
\pgfpathlineto{\pgfqpoint{0.487476in}{0.484854in}}%
\pgfpathlineto{\pgfqpoint{0.523920in}{0.484854in}}%
\pgfpathlineto{\pgfqpoint{0.560363in}{0.484854in}}%
\pgfpathlineto{\pgfqpoint{0.596806in}{0.484854in}}%
\pgfpathlineto{\pgfqpoint{0.633250in}{0.484854in}}%
\pgfpathlineto{\pgfqpoint{0.669693in}{0.484854in}}%
\pgfpathlineto{\pgfqpoint{0.706137in}{0.484854in}}%
\pgfpathlineto{\pgfqpoint{0.742580in}{0.484854in}}%
\pgfpathlineto{\pgfqpoint{0.779024in}{0.484854in}}%
\pgfpathlineto{\pgfqpoint{0.815467in}{0.484854in}}%
\pgfpathlineto{\pgfqpoint{0.851911in}{0.484871in}}%
\pgfpathlineto{\pgfqpoint{0.888026in}{0.484854in}}%
\pgfpathlineto{\pgfqpoint{0.922939in}{0.492133in}}%
\pgfpathlineto{\pgfqpoint{0.955629in}{0.516085in}}%
\pgfpathlineto{\pgfqpoint{0.993348in}{0.574527in}}%
\pgfpathlineto{\pgfqpoint{1.028078in}{0.659920in}}%
\pgfpathlineto{\pgfqpoint{1.066271in}{0.745990in}}%
\pgfpathlineto{\pgfqpoint{1.101475in}{0.833340in}}%
\pgfpathlineto{\pgfqpoint{1.139522in}{0.900996in}}%
\pgfpathlineto{\pgfqpoint{1.175273in}{0.923799in}}%
\pgfpathlineto{\pgfqpoint{1.209348in}{0.912014in}}%
\pgfpathlineto{\pgfqpoint{1.246630in}{0.916700in}}%
\pgfpathlineto{\pgfqpoint{1.282235in}{0.927346in}}%
\pgfpathlineto{\pgfqpoint{1.316893in}{0.935655in}}%
\pgfpathlineto{\pgfqpoint{1.355268in}{0.928484in}}%
\pgfpathlineto{\pgfqpoint{1.393242in}{0.914898in}}%
\pgfpathlineto{\pgfqpoint{1.426879in}{0.914709in}}%
\pgfpathlineto{\pgfqpoint{1.463833in}{0.959369in}}%
\pgfpathlineto{\pgfqpoint{1.502098in}{1.108934in}}%
\pgfpathlineto{\pgfqpoint{1.535335in}{1.037822in}}%
\pgfpathlineto{\pgfqpoint{1.571815in}{1.005234in}}%
\pgfpathlineto{\pgfqpoint{1.608149in}{1.032822in}}%
\pgfpathlineto{\pgfqpoint{1.647763in}{1.024390in}}%
\pgfpathlineto{\pgfqpoint{1.683404in}{1.034989in}}%
\pgfpathlineto{\pgfqpoint{1.718390in}{1.048483in}}%
\pgfpathlineto{\pgfqpoint{1.756401in}{1.032886in}}%
\pgfpathlineto{\pgfqpoint{1.790402in}{1.035750in}}%
\pgfpathlineto{\pgfqpoint{1.827356in}{1.130631in}}%
\pgfpathlineto{\pgfqpoint{1.866059in}{1.124426in}}%
\pgfpathlineto{\pgfqpoint{1.900206in}{1.139176in}}%
\pgfpathlineto{\pgfqpoint{1.936723in}{1.150593in}}%
\pgfpathlineto{\pgfqpoint{1.975389in}{1.161127in}}%
\pgfpathlineto{\pgfqpoint{2.010047in}{1.167784in}}%
\pgfpathlineto{\pgfqpoint{2.046673in}{1.185280in}}%
\pgfpathlineto{\pgfqpoint{2.082788in}{1.200829in}}%
\pgfpathlineto{\pgfqpoint{2.118357in}{1.197005in}}%
\pgfpathlineto{\pgfqpoint{2.156367in}{1.207066in}}%
\pgfpathlineto{\pgfqpoint{2.193831in}{1.213488in}}%
\pgfpathlineto{\pgfqpoint{2.228161in}{1.232964in}}%
\pgfpathlineto{\pgfqpoint{2.264823in}{1.211963in}}%
\pgfpathlineto{\pgfqpoint{2.301121in}{1.223070in}}%
\pgfpathlineto{\pgfqpoint{2.337637in}{1.225324in}}%
\pgfpathlineto{\pgfqpoint{2.374372in}{1.252590in}}%
\pgfpathlineto{\pgfqpoint{2.410196in}{1.203026in}}%
\pgfpathlineto{\pgfqpoint{2.447223in}{1.192557in}}%
\pgfpathlineto{\pgfqpoint{2.483156in}{1.200139in}}%
\pgfpathlineto{\pgfqpoint{2.520073in}{1.149015in}}%
\pgfpathlineto{\pgfqpoint{2.556917in}{1.184343in}}%
\pgfpathlineto{\pgfqpoint{2.594235in}{1.205108in}}%
\pgfpathlineto{\pgfqpoint{2.630606in}{1.189803in}}%
\pgfpathlineto{\pgfqpoint{2.664972in}{1.187273in}}%
\pgfpathlineto{\pgfqpoint{2.664972in}{1.926251in}}%
\pgfpathlineto{\pgfqpoint{2.664972in}{1.926251in}}%
\pgfpathlineto{\pgfqpoint{2.630606in}{1.954550in}}%
\pgfpathlineto{\pgfqpoint{2.594235in}{1.963984in}}%
\pgfpathlineto{\pgfqpoint{2.556917in}{1.921534in}}%
\pgfpathlineto{\pgfqpoint{2.520073in}{1.845640in}}%
\pgfpathlineto{\pgfqpoint{2.483156in}{1.959696in}}%
\pgfpathlineto{\pgfqpoint{2.447223in}{1.943488in}}%
\pgfpathlineto{\pgfqpoint{2.410196in}{1.963040in}}%
\pgfpathlineto{\pgfqpoint{2.374372in}{2.083014in}}%
\pgfpathlineto{\pgfqpoint{2.337637in}{2.042794in}}%
\pgfpathlineto{\pgfqpoint{2.301121in}{2.045538in}}%
\pgfpathlineto{\pgfqpoint{2.264823in}{2.014923in}}%
\pgfpathlineto{\pgfqpoint{2.228161in}{2.056858in}}%
\pgfpathlineto{\pgfqpoint{2.193831in}{2.007891in}}%
\pgfpathlineto{\pgfqpoint{2.156367in}{1.991940in}}%
\pgfpathlineto{\pgfqpoint{2.118357in}{1.962440in}}%
\pgfpathlineto{\pgfqpoint{2.082788in}{1.967328in}}%
\pgfpathlineto{\pgfqpoint{2.046673in}{1.927709in}}%
\pgfpathlineto{\pgfqpoint{2.010047in}{1.886717in}}%
\pgfpathlineto{\pgfqpoint{1.975389in}{1.867593in}}%
\pgfpathlineto{\pgfqpoint{1.936723in}{1.840494in}}%
\pgfpathlineto{\pgfqpoint{1.900206in}{1.812109in}}%
\pgfpathlineto{\pgfqpoint{1.866059in}{1.777034in}}%
\pgfpathlineto{\pgfqpoint{1.827356in}{1.783037in}}%
\pgfpathlineto{\pgfqpoint{1.790402in}{1.578079in}}%
\pgfpathlineto{\pgfqpoint{1.756401in}{1.569418in}}%
\pgfpathlineto{\pgfqpoint{1.718390in}{1.597289in}}%
\pgfpathlineto{\pgfqpoint{1.683404in}{1.567960in}}%
\pgfpathlineto{\pgfqpoint{1.647763in}{1.543948in}}%
\pgfpathlineto{\pgfqpoint{1.608149in}{1.557240in}}%
\pgfpathlineto{\pgfqpoint{1.571815in}{1.500298in}}%
\pgfpathlineto{\pgfqpoint{1.535335in}{1.560842in}}%
\pgfpathlineto{\pgfqpoint{1.502098in}{1.694022in}}%
\pgfpathlineto{\pgfqpoint{1.463833in}{1.401335in}}%
\pgfpathlineto{\pgfqpoint{1.426879in}{1.310261in}}%
\pgfpathlineto{\pgfqpoint{1.393242in}{1.307860in}}%
\pgfpathlineto{\pgfqpoint{1.355268in}{1.330834in}}%
\pgfpathlineto{\pgfqpoint{1.316893in}{1.340705in}}%
\pgfpathlineto{\pgfqpoint{1.282235in}{1.321924in}}%
\pgfpathlineto{\pgfqpoint{1.246630in}{1.298110in}}%
\pgfpathlineto{\pgfqpoint{1.209348in}{1.286490in}}%
\pgfpathlineto{\pgfqpoint{1.175273in}{1.305991in}}%
\pgfpathlineto{\pgfqpoint{1.139522in}{1.260265in}}%
\pgfpathlineto{\pgfqpoint{1.101475in}{1.132008in}}%
\pgfpathlineto{\pgfqpoint{1.066271in}{0.967835in}}%
\pgfpathlineto{\pgfqpoint{1.028078in}{0.807325in}}%
\pgfpathlineto{\pgfqpoint{0.993348in}{0.649507in}}%
\pgfpathlineto{\pgfqpoint{0.955629in}{0.542001in}}%
\pgfpathlineto{\pgfqpoint{0.922939in}{0.498132in}}%
\pgfpathlineto{\pgfqpoint{0.888026in}{0.484854in}}%
\pgfpathlineto{\pgfqpoint{0.851911in}{0.484885in}}%
\pgfpathlineto{\pgfqpoint{0.815467in}{0.484854in}}%
\pgfpathlineto{\pgfqpoint{0.779024in}{0.484854in}}%
\pgfpathlineto{\pgfqpoint{0.742580in}{0.484854in}}%
\pgfpathlineto{\pgfqpoint{0.706137in}{0.484854in}}%
\pgfpathlineto{\pgfqpoint{0.669693in}{0.484854in}}%
\pgfpathlineto{\pgfqpoint{0.633250in}{0.484854in}}%
\pgfpathlineto{\pgfqpoint{0.596806in}{0.484854in}}%
\pgfpathlineto{\pgfqpoint{0.560363in}{0.484854in}}%
\pgfpathlineto{\pgfqpoint{0.523920in}{0.484854in}}%
\pgfpathlineto{\pgfqpoint{0.487476in}{0.484854in}}%
\pgfpathlineto{\pgfqpoint{0.451033in}{0.484854in}}%
\pgfpathlineto{\pgfqpoint{0.414589in}{0.484854in}}%
\pgfpathlineto{\pgfqpoint{0.378146in}{0.484854in}}%
\pgfpathlineto{\pgfqpoint{0.341702in}{0.484854in}}%
\pgfpathlineto{\pgfqpoint{0.305259in}{0.484854in}}%
\pgfpathlineto{\pgfqpoint{0.268815in}{0.484854in}}%
\pgfpathlineto{\pgfqpoint{0.232372in}{0.484854in}}%
\pgfpathlineto{\pgfqpoint{0.195929in}{0.484854in}}%
\pgfpathlineto{\pgfqpoint{0.159485in}{0.484854in}}%
\pgfpathlineto{\pgfqpoint{0.123042in}{0.484854in}}%
\pgfpathlineto{\pgfqpoint{0.086598in}{0.484854in}}%
\pgfpathlineto{\pgfqpoint{0.050155in}{0.484854in}}%
\pgfpathlineto{\pgfqpoint{0.013711in}{0.484854in}}%
\pgfpathlineto{\pgfqpoint{-0.022732in}{0.484854in}}%
\pgfpathlineto{\pgfqpoint{-0.059176in}{0.484854in}}%
\pgfpathlineto{\pgfqpoint{-0.095619in}{0.484854in}}%
\pgfpathlineto{\pgfqpoint{-0.132062in}{0.484854in}}%
\pgfpathlineto{\pgfqpoint{-0.168506in}{0.484854in}}%
\pgfpathlineto{\pgfqpoint{-0.204949in}{0.484854in}}%
\pgfpathlineto{\pgfqpoint{-0.241393in}{0.484854in}}%
\pgfpathlineto{\pgfqpoint{-0.277836in}{0.484854in}}%
\pgfpathlineto{\pgfqpoint{-0.314280in}{0.484854in}}%
\pgfpathlineto{\pgfqpoint{-0.350723in}{0.484854in}}%
\pgfpathlineto{\pgfqpoint{-0.387167in}{0.484854in}}%
\pgfpathlineto{\pgfqpoint{-0.423610in}{0.484854in}}%
\pgfpathlineto{\pgfqpoint{-0.460054in}{0.484854in}}%
\pgfpathlineto{\pgfqpoint{-0.496497in}{0.484854in}}%
\pgfpathlineto{\pgfqpoint{-0.532940in}{0.484854in}}%
\pgfpathlineto{\pgfqpoint{-0.569384in}{0.484854in}}%
\pgfpathlineto{\pgfqpoint{-0.605827in}{0.484854in}}%
\pgfpathlineto{\pgfqpoint{-0.642274in}{0.484854in}}%
\pgfpathlineto{\pgfqpoint{-0.678718in}{0.484854in}}%
\pgfpathlineto{\pgfqpoint{-0.715161in}{0.484854in}}%
\pgfpathlineto{\pgfqpoint{-0.751605in}{0.484854in}}%
\pgfpathlineto{\pgfqpoint{-0.788045in}{0.484854in}}%
\pgfpathlineto{\pgfqpoint{-0.824488in}{0.484854in}}%
\pgfpathlineto{\pgfqpoint{-0.860931in}{0.484854in}}%
\pgfpathlineto{\pgfqpoint{-0.897375in}{0.484854in}}%
\pgfpathlineto{\pgfqpoint{-0.933818in}{0.484854in}}%
\pgfpathlineto{\pgfqpoint{-0.979373in}{0.484854in}}%
\pgfpathclose%
\pgfusepath{fill}%
\end{pgfscope}%
\begin{pgfscope}%
\pgfsetbuttcap%
\pgfsetroundjoin%
\definecolor{currentfill}{rgb}{0.000000,0.000000,0.000000}%
\pgfsetfillcolor{currentfill}%
\pgfsetlinewidth{0.803000pt}%
\definecolor{currentstroke}{rgb}{0.000000,0.000000,0.000000}%
\pgfsetstrokecolor{currentstroke}%
\pgfsetdash{}{0pt}%
\pgfsys@defobject{currentmarker}{\pgfqpoint{0.000000in}{-0.048611in}}{\pgfqpoint{0.000000in}{0.000000in}}{%
\pgfpathmoveto{\pgfqpoint{0.000000in}{0.000000in}}%
\pgfpathlineto{\pgfqpoint{0.000000in}{-0.048611in}}%
\pgfusepath{stroke,fill}%
}%
\begin{pgfscope}%
\pgfsys@transformshift{0.478365in}{0.484854in}%
\pgfsys@useobject{currentmarker}{}%
\end{pgfscope}%
\end{pgfscope}%
\begin{pgfscope}%
\definecolor{textcolor}{rgb}{0.000000,0.000000,0.000000}%
\pgfsetstrokecolor{textcolor}%
\pgfsetfillcolor{textcolor}%
\pgftext[x=0.478365in,y=0.387632in,,top]{\color{textcolor}\rmfamily\fontsize{8.000000}{9.600000}\selectfont \(\displaystyle 40\)}%
\end{pgfscope}%
\begin{pgfscope}%
\pgfsetbuttcap%
\pgfsetroundjoin%
\definecolor{currentfill}{rgb}{0.000000,0.000000,0.000000}%
\pgfsetfillcolor{currentfill}%
\pgfsetlinewidth{0.803000pt}%
\definecolor{currentstroke}{rgb}{0.000000,0.000000,0.000000}%
\pgfsetstrokecolor{currentstroke}%
\pgfsetdash{}{0pt}%
\pgfsys@defobject{currentmarker}{\pgfqpoint{0.000000in}{-0.048611in}}{\pgfqpoint{0.000000in}{0.000000in}}{%
\pgfpathmoveto{\pgfqpoint{0.000000in}{0.000000in}}%
\pgfpathlineto{\pgfqpoint{0.000000in}{-0.048611in}}%
\pgfusepath{stroke,fill}%
}%
\begin{pgfscope}%
\pgfsys@transformshift{0.842800in}{0.484854in}%
\pgfsys@useobject{currentmarker}{}%
\end{pgfscope}%
\end{pgfscope}%
\begin{pgfscope}%
\definecolor{textcolor}{rgb}{0.000000,0.000000,0.000000}%
\pgfsetstrokecolor{textcolor}%
\pgfsetfillcolor{textcolor}%
\pgftext[x=0.842800in,y=0.387632in,,top]{\color{textcolor}\rmfamily\fontsize{8.000000}{9.600000}\selectfont \(\displaystyle 50\)}%
\end{pgfscope}%
\begin{pgfscope}%
\pgfsetbuttcap%
\pgfsetroundjoin%
\definecolor{currentfill}{rgb}{0.000000,0.000000,0.000000}%
\pgfsetfillcolor{currentfill}%
\pgfsetlinewidth{0.803000pt}%
\definecolor{currentstroke}{rgb}{0.000000,0.000000,0.000000}%
\pgfsetstrokecolor{currentstroke}%
\pgfsetdash{}{0pt}%
\pgfsys@defobject{currentmarker}{\pgfqpoint{0.000000in}{-0.048611in}}{\pgfqpoint{0.000000in}{0.000000in}}{%
\pgfpathmoveto{\pgfqpoint{0.000000in}{0.000000in}}%
\pgfpathlineto{\pgfqpoint{0.000000in}{-0.048611in}}%
\pgfusepath{stroke,fill}%
}%
\begin{pgfscope}%
\pgfsys@transformshift{1.207234in}{0.484854in}%
\pgfsys@useobject{currentmarker}{}%
\end{pgfscope}%
\end{pgfscope}%
\begin{pgfscope}%
\definecolor{textcolor}{rgb}{0.000000,0.000000,0.000000}%
\pgfsetstrokecolor{textcolor}%
\pgfsetfillcolor{textcolor}%
\pgftext[x=1.207234in,y=0.387632in,,top]{\color{textcolor}\rmfamily\fontsize{8.000000}{9.600000}\selectfont \(\displaystyle 60\)}%
\end{pgfscope}%
\begin{pgfscope}%
\pgfsetbuttcap%
\pgfsetroundjoin%
\definecolor{currentfill}{rgb}{0.000000,0.000000,0.000000}%
\pgfsetfillcolor{currentfill}%
\pgfsetlinewidth{0.803000pt}%
\definecolor{currentstroke}{rgb}{0.000000,0.000000,0.000000}%
\pgfsetstrokecolor{currentstroke}%
\pgfsetdash{}{0pt}%
\pgfsys@defobject{currentmarker}{\pgfqpoint{0.000000in}{-0.048611in}}{\pgfqpoint{0.000000in}{0.000000in}}{%
\pgfpathmoveto{\pgfqpoint{0.000000in}{0.000000in}}%
\pgfpathlineto{\pgfqpoint{0.000000in}{-0.048611in}}%
\pgfusepath{stroke,fill}%
}%
\begin{pgfscope}%
\pgfsys@transformshift{1.571669in}{0.484854in}%
\pgfsys@useobject{currentmarker}{}%
\end{pgfscope}%
\end{pgfscope}%
\begin{pgfscope}%
\definecolor{textcolor}{rgb}{0.000000,0.000000,0.000000}%
\pgfsetstrokecolor{textcolor}%
\pgfsetfillcolor{textcolor}%
\pgftext[x=1.571669in,y=0.387632in,,top]{\color{textcolor}\rmfamily\fontsize{8.000000}{9.600000}\selectfont \(\displaystyle 70\)}%
\end{pgfscope}%
\begin{pgfscope}%
\pgfsetbuttcap%
\pgfsetroundjoin%
\definecolor{currentfill}{rgb}{0.000000,0.000000,0.000000}%
\pgfsetfillcolor{currentfill}%
\pgfsetlinewidth{0.803000pt}%
\definecolor{currentstroke}{rgb}{0.000000,0.000000,0.000000}%
\pgfsetstrokecolor{currentstroke}%
\pgfsetdash{}{0pt}%
\pgfsys@defobject{currentmarker}{\pgfqpoint{0.000000in}{-0.048611in}}{\pgfqpoint{0.000000in}{0.000000in}}{%
\pgfpathmoveto{\pgfqpoint{0.000000in}{0.000000in}}%
\pgfpathlineto{\pgfqpoint{0.000000in}{-0.048611in}}%
\pgfusepath{stroke,fill}%
}%
\begin{pgfscope}%
\pgfsys@transformshift{1.936103in}{0.484854in}%
\pgfsys@useobject{currentmarker}{}%
\end{pgfscope}%
\end{pgfscope}%
\begin{pgfscope}%
\definecolor{textcolor}{rgb}{0.000000,0.000000,0.000000}%
\pgfsetstrokecolor{textcolor}%
\pgfsetfillcolor{textcolor}%
\pgftext[x=1.936103in,y=0.387632in,,top]{\color{textcolor}\rmfamily\fontsize{8.000000}{9.600000}\selectfont \(\displaystyle 80\)}%
\end{pgfscope}%
\begin{pgfscope}%
\pgfsetbuttcap%
\pgfsetroundjoin%
\definecolor{currentfill}{rgb}{0.000000,0.000000,0.000000}%
\pgfsetfillcolor{currentfill}%
\pgfsetlinewidth{0.803000pt}%
\definecolor{currentstroke}{rgb}{0.000000,0.000000,0.000000}%
\pgfsetstrokecolor{currentstroke}%
\pgfsetdash{}{0pt}%
\pgfsys@defobject{currentmarker}{\pgfqpoint{0.000000in}{-0.048611in}}{\pgfqpoint{0.000000in}{0.000000in}}{%
\pgfpathmoveto{\pgfqpoint{0.000000in}{0.000000in}}%
\pgfpathlineto{\pgfqpoint{0.000000in}{-0.048611in}}%
\pgfusepath{stroke,fill}%
}%
\begin{pgfscope}%
\pgfsys@transformshift{2.300538in}{0.484854in}%
\pgfsys@useobject{currentmarker}{}%
\end{pgfscope}%
\end{pgfscope}%
\begin{pgfscope}%
\definecolor{textcolor}{rgb}{0.000000,0.000000,0.000000}%
\pgfsetstrokecolor{textcolor}%
\pgfsetfillcolor{textcolor}%
\pgftext[x=2.300538in,y=0.387632in,,top]{\color{textcolor}\rmfamily\fontsize{8.000000}{9.600000}\selectfont \(\displaystyle 90\)}%
\end{pgfscope}%
\begin{pgfscope}%
\pgfsetbuttcap%
\pgfsetroundjoin%
\definecolor{currentfill}{rgb}{0.000000,0.000000,0.000000}%
\pgfsetfillcolor{currentfill}%
\pgfsetlinewidth{0.803000pt}%
\definecolor{currentstroke}{rgb}{0.000000,0.000000,0.000000}%
\pgfsetstrokecolor{currentstroke}%
\pgfsetdash{}{0pt}%
\pgfsys@defobject{currentmarker}{\pgfqpoint{0.000000in}{-0.048611in}}{\pgfqpoint{0.000000in}{0.000000in}}{%
\pgfpathmoveto{\pgfqpoint{0.000000in}{0.000000in}}%
\pgfpathlineto{\pgfqpoint{0.000000in}{-0.048611in}}%
\pgfusepath{stroke,fill}%
}%
\begin{pgfscope}%
\pgfsys@transformshift{2.664972in}{0.484854in}%
\pgfsys@useobject{currentmarker}{}%
\end{pgfscope}%
\end{pgfscope}%
\begin{pgfscope}%
\definecolor{textcolor}{rgb}{0.000000,0.000000,0.000000}%
\pgfsetstrokecolor{textcolor}%
\pgfsetfillcolor{textcolor}%
\pgftext[x=2.664972in,y=0.387632in,,top]{\color{textcolor}\rmfamily\fontsize{8.000000}{9.600000}\selectfont \(\displaystyle 100\)}%
\end{pgfscope}%
\begin{pgfscope}%
\definecolor{textcolor}{rgb}{0.000000,0.000000,0.000000}%
\pgfsetstrokecolor{textcolor}%
\pgfsetfillcolor{textcolor}%
\pgftext[x=1.571669in,y=0.224546in,,top]{\color{textcolor}\rmfamily\fontsize{8.000000}{9.600000}\selectfont Time (\(\displaystyle \times 10^3 \, \mathrm{yr}\))}%
\end{pgfscope}%
\begin{pgfscope}%
\pgfsetbuttcap%
\pgfsetroundjoin%
\definecolor{currentfill}{rgb}{0.000000,0.000000,0.000000}%
\pgfsetfillcolor{currentfill}%
\pgfsetlinewidth{0.803000pt}%
\definecolor{currentstroke}{rgb}{0.000000,0.000000,0.000000}%
\pgfsetstrokecolor{currentstroke}%
\pgfsetdash{}{0pt}%
\pgfsys@defobject{currentmarker}{\pgfqpoint{-0.048611in}{0.000000in}}{\pgfqpoint{0.000000in}{0.000000in}}{%
\pgfpathmoveto{\pgfqpoint{0.000000in}{0.000000in}}%
\pgfpathlineto{\pgfqpoint{-0.048611in}{0.000000in}}%
\pgfusepath{stroke,fill}%
}%
\begin{pgfscope}%
\pgfsys@transformshift{0.478365in}{0.484854in}%
\pgfsys@useobject{currentmarker}{}%
\end{pgfscope}%
\end{pgfscope}%
\begin{pgfscope}%
\definecolor{textcolor}{rgb}{0.000000,0.000000,0.000000}%
\pgfsetstrokecolor{textcolor}%
\pgfsetfillcolor{textcolor}%
\pgftext[x=0.322114in,y=0.442645in,left,base]{\color{textcolor}\rmfamily\fontsize{8.000000}{9.600000}\selectfont \(\displaystyle 0\)}%
\end{pgfscope}%
\begin{pgfscope}%
\pgfsetbuttcap%
\pgfsetroundjoin%
\definecolor{currentfill}{rgb}{0.000000,0.000000,0.000000}%
\pgfsetfillcolor{currentfill}%
\pgfsetlinewidth{0.803000pt}%
\definecolor{currentstroke}{rgb}{0.000000,0.000000,0.000000}%
\pgfsetstrokecolor{currentstroke}%
\pgfsetdash{}{0pt}%
\pgfsys@defobject{currentmarker}{\pgfqpoint{-0.048611in}{0.000000in}}{\pgfqpoint{0.000000in}{0.000000in}}{%
\pgfpathmoveto{\pgfqpoint{0.000000in}{0.000000in}}%
\pgfpathlineto{\pgfqpoint{-0.048611in}{0.000000in}}%
\pgfusepath{stroke,fill}%
}%
\begin{pgfscope}%
\pgfsys@transformshift{0.478365in}{0.913637in}%
\pgfsys@useobject{currentmarker}{}%
\end{pgfscope}%
\end{pgfscope}%
\begin{pgfscope}%
\definecolor{textcolor}{rgb}{0.000000,0.000000,0.000000}%
\pgfsetstrokecolor{textcolor}%
\pgfsetfillcolor{textcolor}%
\pgftext[x=0.322114in,y=0.871428in,left,base]{\color{textcolor}\rmfamily\fontsize{8.000000}{9.600000}\selectfont \(\displaystyle 5\)}%
\end{pgfscope}%
\begin{pgfscope}%
\pgfsetbuttcap%
\pgfsetroundjoin%
\definecolor{currentfill}{rgb}{0.000000,0.000000,0.000000}%
\pgfsetfillcolor{currentfill}%
\pgfsetlinewidth{0.803000pt}%
\definecolor{currentstroke}{rgb}{0.000000,0.000000,0.000000}%
\pgfsetstrokecolor{currentstroke}%
\pgfsetdash{}{0pt}%
\pgfsys@defobject{currentmarker}{\pgfqpoint{-0.048611in}{0.000000in}}{\pgfqpoint{0.000000in}{0.000000in}}{%
\pgfpathmoveto{\pgfqpoint{0.000000in}{0.000000in}}%
\pgfpathlineto{\pgfqpoint{-0.048611in}{0.000000in}}%
\pgfusepath{stroke,fill}%
}%
\begin{pgfscope}%
\pgfsys@transformshift{0.478365in}{1.342420in}%
\pgfsys@useobject{currentmarker}{}%
\end{pgfscope}%
\end{pgfscope}%
\begin{pgfscope}%
\definecolor{textcolor}{rgb}{0.000000,0.000000,0.000000}%
\pgfsetstrokecolor{textcolor}%
\pgfsetfillcolor{textcolor}%
\pgftext[x=0.263086in,y=1.300211in,left,base]{\color{textcolor}\rmfamily\fontsize{8.000000}{9.600000}\selectfont \(\displaystyle 10\)}%
\end{pgfscope}%
\begin{pgfscope}%
\pgfsetbuttcap%
\pgfsetroundjoin%
\definecolor{currentfill}{rgb}{0.000000,0.000000,0.000000}%
\pgfsetfillcolor{currentfill}%
\pgfsetlinewidth{0.803000pt}%
\definecolor{currentstroke}{rgb}{0.000000,0.000000,0.000000}%
\pgfsetstrokecolor{currentstroke}%
\pgfsetdash{}{0pt}%
\pgfsys@defobject{currentmarker}{\pgfqpoint{-0.048611in}{0.000000in}}{\pgfqpoint{0.000000in}{0.000000in}}{%
\pgfpathmoveto{\pgfqpoint{0.000000in}{0.000000in}}%
\pgfpathlineto{\pgfqpoint{-0.048611in}{0.000000in}}%
\pgfusepath{stroke,fill}%
}%
\begin{pgfscope}%
\pgfsys@transformshift{0.478365in}{1.771203in}%
\pgfsys@useobject{currentmarker}{}%
\end{pgfscope}%
\end{pgfscope}%
\begin{pgfscope}%
\definecolor{textcolor}{rgb}{0.000000,0.000000,0.000000}%
\pgfsetstrokecolor{textcolor}%
\pgfsetfillcolor{textcolor}%
\pgftext[x=0.263086in,y=1.728994in,left,base]{\color{textcolor}\rmfamily\fontsize{8.000000}{9.600000}\selectfont \(\displaystyle 15\)}%
\end{pgfscope}%
\begin{pgfscope}%
\definecolor{textcolor}{rgb}{0.000000,0.000000,0.000000}%
\pgfsetstrokecolor{textcolor}%
\pgfsetfillcolor{textcolor}%
\pgftext[x=0.207530in,y=1.299542in,,bottom,rotate=90.000000]{\color{textcolor}\rmfamily\fontsize{8.000000}{9.600000}\selectfont Melt volume (\%)}%
\end{pgfscope}%
\begin{pgfscope}%
\pgfpathrectangle{\pgfqpoint{0.478365in}{0.484854in}}{\pgfqpoint{2.186607in}{1.629375in}}%
\pgfusepath{clip}%
\pgfsetrectcap%
\pgfsetroundjoin%
\pgfsetlinewidth{1.505625pt}%
\definecolor{currentstroke}{rgb}{0.000000,0.000000,0.000000}%
\pgfsetstrokecolor{currentstroke}%
\pgfsetdash{}{0pt}%
\pgfpathmoveto{\pgfqpoint{0.473365in}{0.484854in}}%
\pgfpathlineto{\pgfqpoint{0.487476in}{0.484854in}}%
\pgfpathlineto{\pgfqpoint{0.523920in}{0.484854in}}%
\pgfpathlineto{\pgfqpoint{0.560363in}{0.484854in}}%
\pgfpathlineto{\pgfqpoint{0.596806in}{0.484854in}}%
\pgfpathlineto{\pgfqpoint{0.633250in}{0.484854in}}%
\pgfpathlineto{\pgfqpoint{0.669693in}{0.484854in}}%
\pgfpathlineto{\pgfqpoint{0.706137in}{0.484854in}}%
\pgfpathlineto{\pgfqpoint{0.742580in}{0.484854in}}%
\pgfpathlineto{\pgfqpoint{0.779024in}{0.484854in}}%
\pgfpathlineto{\pgfqpoint{0.815467in}{0.484854in}}%
\pgfpathlineto{\pgfqpoint{0.851911in}{0.484885in}}%
\pgfpathlineto{\pgfqpoint{0.888026in}{0.484326in}}%
\pgfpathlineto{\pgfqpoint{0.922939in}{0.498132in}}%
\pgfpathlineto{\pgfqpoint{0.955629in}{0.542001in}}%
\pgfpathlineto{\pgfqpoint{0.993348in}{0.649507in}}%
\pgfpathlineto{\pgfqpoint{1.028078in}{0.807325in}}%
\pgfpathlineto{\pgfqpoint{1.066271in}{0.967835in}}%
\pgfpathlineto{\pgfqpoint{1.101475in}{1.132008in}}%
\pgfpathlineto{\pgfqpoint{1.139522in}{1.260265in}}%
\pgfpathlineto{\pgfqpoint{1.175273in}{1.305991in}}%
\pgfpathlineto{\pgfqpoint{1.209348in}{1.286490in}}%
\pgfpathlineto{\pgfqpoint{1.246630in}{1.298110in}}%
\pgfpathlineto{\pgfqpoint{1.282235in}{1.321924in}}%
\pgfpathlineto{\pgfqpoint{1.316893in}{1.340705in}}%
\pgfpathlineto{\pgfqpoint{1.355268in}{1.330834in}}%
\pgfpathlineto{\pgfqpoint{1.393242in}{1.307860in}}%
\pgfpathlineto{\pgfqpoint{1.426879in}{1.310261in}}%
\pgfpathlineto{\pgfqpoint{1.463833in}{1.401335in}}%
\pgfpathlineto{\pgfqpoint{1.502098in}{1.694022in}}%
\pgfpathlineto{\pgfqpoint{1.535335in}{1.560842in}}%
\pgfpathlineto{\pgfqpoint{1.571815in}{1.500298in}}%
\pgfpathlineto{\pgfqpoint{1.608149in}{1.557240in}}%
\pgfpathlineto{\pgfqpoint{1.647763in}{1.543948in}}%
\pgfpathlineto{\pgfqpoint{1.683404in}{1.567960in}}%
\pgfpathlineto{\pgfqpoint{1.718390in}{1.597289in}}%
\pgfpathlineto{\pgfqpoint{1.756401in}{1.569418in}}%
\pgfpathlineto{\pgfqpoint{1.790402in}{1.578079in}}%
\pgfpathlineto{\pgfqpoint{1.827356in}{1.783037in}}%
\pgfpathlineto{\pgfqpoint{1.866059in}{1.777034in}}%
\pgfpathlineto{\pgfqpoint{1.900206in}{1.812109in}}%
\pgfpathlineto{\pgfqpoint{1.936723in}{1.840494in}}%
\pgfpathlineto{\pgfqpoint{1.975389in}{1.867593in}}%
\pgfpathlineto{\pgfqpoint{2.010047in}{1.886717in}}%
\pgfpathlineto{\pgfqpoint{2.046673in}{1.927709in}}%
\pgfpathlineto{\pgfqpoint{2.082788in}{1.967328in}}%
\pgfpathlineto{\pgfqpoint{2.118357in}{1.962440in}}%
\pgfpathlineto{\pgfqpoint{2.156367in}{1.991940in}}%
\pgfpathlineto{\pgfqpoint{2.193831in}{2.007891in}}%
\pgfpathlineto{\pgfqpoint{2.228161in}{2.056858in}}%
\pgfpathlineto{\pgfqpoint{2.264823in}{2.014923in}}%
\pgfpathlineto{\pgfqpoint{2.301121in}{2.045538in}}%
\pgfpathlineto{\pgfqpoint{2.337637in}{2.042794in}}%
\pgfpathlineto{\pgfqpoint{2.374372in}{2.083014in}}%
\pgfpathlineto{\pgfqpoint{2.410196in}{1.963040in}}%
\pgfpathlineto{\pgfqpoint{2.447223in}{1.943488in}}%
\pgfpathlineto{\pgfqpoint{2.483156in}{1.959696in}}%
\pgfpathlineto{\pgfqpoint{2.520073in}{1.845640in}}%
\pgfpathlineto{\pgfqpoint{2.556917in}{1.921534in}}%
\pgfpathlineto{\pgfqpoint{2.594235in}{1.963984in}}%
\pgfpathlineto{\pgfqpoint{2.630606in}{1.954550in}}%
\pgfpathlineto{\pgfqpoint{2.664972in}{1.926251in}}%
\pgfusepath{stroke}%
\end{pgfscope}%
\begin{pgfscope}%
\pgfsetrectcap%
\pgfsetmiterjoin%
\pgfsetlinewidth{0.803000pt}%
\definecolor{currentstroke}{rgb}{0.000000,0.000000,0.000000}%
\pgfsetstrokecolor{currentstroke}%
\pgfsetdash{}{0pt}%
\pgfpathmoveto{\pgfqpoint{0.478365in}{0.484854in}}%
\pgfpathlineto{\pgfqpoint{0.478365in}{2.114229in}}%
\pgfusepath{stroke}%
\end{pgfscope}%
\begin{pgfscope}%
\pgfsetrectcap%
\pgfsetmiterjoin%
\pgfsetlinewidth{0.803000pt}%
\definecolor{currentstroke}{rgb}{0.000000,0.000000,0.000000}%
\pgfsetstrokecolor{currentstroke}%
\pgfsetdash{}{0pt}%
\pgfpathmoveto{\pgfqpoint{2.664972in}{0.484854in}}%
\pgfpathlineto{\pgfqpoint{2.664972in}{2.114229in}}%
\pgfusepath{stroke}%
\end{pgfscope}%
\begin{pgfscope}%
\pgfsetrectcap%
\pgfsetmiterjoin%
\pgfsetlinewidth{0.803000pt}%
\definecolor{currentstroke}{rgb}{0.000000,0.000000,0.000000}%
\pgfsetstrokecolor{currentstroke}%
\pgfsetdash{}{0pt}%
\pgfpathmoveto{\pgfqpoint{0.478365in}{0.484854in}}%
\pgfpathlineto{\pgfqpoint{2.664972in}{0.484854in}}%
\pgfusepath{stroke}%
\end{pgfscope}%
\begin{pgfscope}%
\pgfsetrectcap%
\pgfsetmiterjoin%
\pgfsetlinewidth{0.803000pt}%
\definecolor{currentstroke}{rgb}{0.000000,0.000000,0.000000}%
\pgfsetstrokecolor{currentstroke}%
\pgfsetdash{}{0pt}%
\pgfpathmoveto{\pgfqpoint{0.478365in}{2.114229in}}%
\pgfpathlineto{\pgfqpoint{2.664972in}{2.114229in}}%
\pgfusepath{stroke}%
\end{pgfscope}%
\begin{pgfscope}%
\pgfsetbuttcap%
\pgfsetmiterjoin%
\definecolor{currentfill}{rgb}{0.121569,0.466667,0.705882}%
\pgfsetfillcolor{currentfill}%
\pgfsetlinewidth{0.000000pt}%
\definecolor{currentstroke}{rgb}{0.000000,0.000000,0.000000}%
\pgfsetstrokecolor{currentstroke}%
\pgfsetstrokeopacity{0.000000}%
\pgfsetdash{}{0pt}%
\pgfpathmoveto{\pgfqpoint{0.565865in}{1.952863in}}%
\pgfpathlineto{\pgfqpoint{0.760310in}{1.952863in}}%
\pgfpathlineto{\pgfqpoint{0.760310in}{2.020919in}}%
\pgfpathlineto{\pgfqpoint{0.565865in}{2.020919in}}%
\pgfpathclose%
\pgfusepath{fill}%
\end{pgfscope}%
\begin{pgfscope}%
\definecolor{textcolor}{rgb}{0.000000,0.000000,0.000000}%
\pgfsetstrokecolor{textcolor}%
\pgfsetfillcolor{textcolor}%
\pgftext[x=0.838088in,y=1.952863in,left,base]{\color{textcolor}\rmfamily\fontsize{7.000000}{8.400000}\selectfont CaO}%
\end{pgfscope}%
\begin{pgfscope}%
\pgfsetbuttcap%
\pgfsetmiterjoin%
\definecolor{currentfill}{rgb}{1.000000,0.498039,0.054902}%
\pgfsetfillcolor{currentfill}%
\pgfsetlinewidth{0.000000pt}%
\definecolor{currentstroke}{rgb}{0.000000,0.000000,0.000000}%
\pgfsetstrokecolor{currentstroke}%
\pgfsetstrokeopacity{0.000000}%
\pgfsetdash{}{0pt}%
\pgfpathmoveto{\pgfqpoint{0.565865in}{1.810163in}}%
\pgfpathlineto{\pgfqpoint{0.760310in}{1.810163in}}%
\pgfpathlineto{\pgfqpoint{0.760310in}{1.878219in}}%
\pgfpathlineto{\pgfqpoint{0.565865in}{1.878219in}}%
\pgfpathclose%
\pgfusepath{fill}%
\end{pgfscope}%
\begin{pgfscope}%
\definecolor{textcolor}{rgb}{0.000000,0.000000,0.000000}%
\pgfsetstrokecolor{textcolor}%
\pgfsetfillcolor{textcolor}%
\pgftext[x=0.838088in,y=1.810163in,left,base]{\color{textcolor}\rmfamily\fontsize{7.000000}{8.400000}\selectfont FeO}%
\end{pgfscope}%
\begin{pgfscope}%
\pgfsetbuttcap%
\pgfsetmiterjoin%
\definecolor{currentfill}{rgb}{0.172549,0.627451,0.172549}%
\pgfsetfillcolor{currentfill}%
\pgfsetlinewidth{0.000000pt}%
\definecolor{currentstroke}{rgb}{0.000000,0.000000,0.000000}%
\pgfsetstrokecolor{currentstroke}%
\pgfsetstrokeopacity{0.000000}%
\pgfsetdash{}{0pt}%
\pgfpathmoveto{\pgfqpoint{0.565865in}{1.667463in}}%
\pgfpathlineto{\pgfqpoint{0.760310in}{1.667463in}}%
\pgfpathlineto{\pgfqpoint{0.760310in}{1.735519in}}%
\pgfpathlineto{\pgfqpoint{0.565865in}{1.735519in}}%
\pgfpathclose%
\pgfusepath{fill}%
\end{pgfscope}%
\begin{pgfscope}%
\definecolor{textcolor}{rgb}{0.000000,0.000000,0.000000}%
\pgfsetstrokecolor{textcolor}%
\pgfsetfillcolor{textcolor}%
\pgftext[x=0.838088in,y=1.667463in,left,base]{\color{textcolor}\rmfamily\fontsize{7.000000}{8.400000}\selectfont MgO}%
\end{pgfscope}%
\begin{pgfscope}%
\pgfsetbuttcap%
\pgfsetmiterjoin%
\definecolor{currentfill}{rgb}{0.839216,0.152941,0.156863}%
\pgfsetfillcolor{currentfill}%
\pgfsetlinewidth{0.000000pt}%
\definecolor{currentstroke}{rgb}{0.000000,0.000000,0.000000}%
\pgfsetstrokecolor{currentstroke}%
\pgfsetstrokeopacity{0.000000}%
\pgfsetdash{}{0pt}%
\pgfpathmoveto{\pgfqpoint{0.565865in}{1.523386in}}%
\pgfpathlineto{\pgfqpoint{0.760310in}{1.523386in}}%
\pgfpathlineto{\pgfqpoint{0.760310in}{1.591442in}}%
\pgfpathlineto{\pgfqpoint{0.565865in}{1.591442in}}%
\pgfpathclose%
\pgfusepath{fill}%
\end{pgfscope}%
\begin{pgfscope}%
\definecolor{textcolor}{rgb}{0.000000,0.000000,0.000000}%
\pgfsetstrokecolor{textcolor}%
\pgfsetfillcolor{textcolor}%
\pgftext[x=0.838088in,y=1.523386in,left,base]{\color{textcolor}\rmfamily\fontsize{7.000000}{8.400000}\selectfont SiO2}%
\end{pgfscope}%
\end{pgfpicture}%
\makeatother%
\endgroup%

        \caption{Material model plugin.}
        \label{fig:decompression_event_material_model}
    \end{subfigure}
    \caption{
        Average melt amount and compositional information recorded by the particles during the decompression event model setup for the different plugin types.
        The first \num{40000} years are omitted because no melt is present.
        Note that subfigure~(\subref{fig:decompression_event_particle_plugin_frac}) the extracted melt and the melt amount are recorded instead of melt volume because the melt volume is unknown for the extracted melt.
    }
    \label{fig:decompression_event}
\end{figure}

\begin{table}
    \centering
    \begin{tabular}{c|c|c}
        \textit{Parameter} & 
            \textit{Closed box} & 
            \textit{Decompression event} \\
        \hline
        Height (\si{\km}) & 120 & 120 \\
        Width (\si{\km}) & 300 & 1 \\
        \hline
        Surface pressure (\si{\Pa}) & 0 & 0 \\
        \hline
        Surface temperature (\si{\kelvin}) & 293 & 293 \\
        Bottom temperature (\si{\kelvin}) & 1873 & 1873 \\
        Temperature profile & Linear & Linear \\
        Temperature boundary conditions & Fixed at base & Fixed at base \\
        Temperature perturbations at base & Present & Not present \\
        \hline
        Boundary velocities & Tangential & \SI{1}{\m\per\year} upwards \\
        \hline
        Reference bulk viscosity (\si{\Pa\s}) & \num{1e18} & \num{1e18} \\
        Reference shear viscosity (\si{\Pa\s}) & \num{1e18} & \num{1e18} \\
        Thermal viscosity exponent & 4 & 4 \\
        Thermal bulk viscosity exponent & 4 & 4 \\
        \hline
        Perple\_X data file & Simplified KLB-1 & Simplified KLB-1 \\
    \end{tabular}
    \caption{
        Parameter values used in the two model setups.
        A detailed description of the meaning of the different parameters may be found in the ASPECT manual.
        More information regarding the Perple\_X data file may be found in Section~\ref{sec:discussion}\ref{sec:discussion_perplexdata}.
    }
    \label{tab:model_params}
\end{table}

In the following, the functionality and performance of the libraries will be demonstrated using two model setups run in ASPECT (Table~\ref{tab:model_params}).
The scripts necessary to reproduce the results, including the plotting ones, are available at \url{http://github.com/cward97/miscada-report}.

\vspace{5mm}

The first model setup to be investigated was a generic scenario  (henceforth referred to as the ``closed box" scenario) in which an enclosed amount of material advects inside a box.
Temperature fluctuations at the base of the model, specified in the parameter file, create convection currents transporting hot plumes of mantle material upwards.
As they rise towards the surface, the reduction in pressure causes partial melting.
Some screenshots of the setup running with both plugin types are shown in Figure~\ref{fig:closed_box}.
They show melt forming at the head of the two mantle plumes where the temperature is greatest and also disappearing as the plume disperses and the temperature drops.
At the end of the simulation, the temperature field is mostly mixed and the hot material has almost entirely risen above the colder material that was initially at the surface.

When implemented using the particle property plugin, particles were added to the simulation tracking the melt volume.
Since batch melting was used, the chemical composition that matched the bulk composition in the Perple\_X data file stayed fixed throughout.
An initial mesh refinement level of 5 was used and adaptive mesh refinement was disabled.

For the material model plugin, no particles were required and instead the melt volume could be plotted directly from the porosity compositional field.
The composition initially also matched that found in the Perple\_X data file, but was able to change over time as the melt and residue advected separately.
An initial mesh refinement level of 4 was used, but adaptive mesh refinement was included in order to better resolve the areas where melt is present.

\vspace{5mm}

The second scenario is a less general model aimed at modelling individual ``decompression events" during which a single hot column of material rises and begins to melt.
The amount of melt is tracked by 11 particles, initially placed at the base of the model, that move upwards at the same velocity as the temperature field (specified as \SI{1}{\meter\per\year}) and keep track of the melt amount and composition that forms as the pressure reduces. 
The simulation ran for \num{100000} years as this was enough time for the particles, moving at \SI{1}{\meter\per\year} in a column \SI{120}{\km} high, to approach the surface without being removed.

An advantage to this model setup is that it is a more intuitive way to study the impact of fractional melting.
The closed box setup is designed to run continuously, and so frequently removing the melt from the simulation will quickly make the results unrealistic because the extracted melt is never reintroduced.
In contrast, the decompression event setup has a fixed endpoint so the fact that the melt is not reintroduced is not an issue.

The results are displayed in Figure~\ref{fig:decompression_event}.
In all cases, as the particles migrate upwards and the pressure decreases, the material begins to partially melt leading to an increase in the melt volume.
The chemical composition of the melt is also shown and appears to be similar between the various models tested.

Interestingly, only about half as much melt is formed by the material model plugin compared to the particle property one.
One possible explanation for this effect would be depletion.
Since in the material model the melt advects separately to the residue (and the particles advect with the residue), the bulk composition changes over time, taking away the more volatile components and reducing its melt productivity.

\subsection{Performance analysis}

\begin{figure}
    \centering
    \begin{subfigure}{\textwidth}
        \centering
        \begin{tikzpicture}
            \pie[radius=2, text=pin] {
                93/ Update particle properties,
                7/ Other
            }
        \end{tikzpicture}
        \caption{Particle property plugin.}
        \label{fig:runtime_pie_particle_property}
        \vspace{5mm}
    \end{subfigure}
    %
    \begin{subfigure}{\textwidth}
        \centering
        \begin{tikzpicture}
            \pie[radius=2, text=pin] {
                27/ Solve Stokes system,
                22/ Build Stokes preconditioner,
                20/ Assemble temperature system,
                19/ Assemble Stokes system,
                12/ Other
            }
        \end{tikzpicture}
        \caption{Particle property plugin control.}
        \label{fig:runtime_pie_particle_property_ctrl}
        \vspace{5mm}
    \end{subfigure}
    %
    \begin{subfigure}{\textwidth}
        \centering
        \begin{tikzpicture}
            \pie[radius=2, text=pin] {
                46/ Assemble composition system,
                28/ Solve composition reactions,
                8/ Solve Stokes system,
                18/ Other
            }
        \end{tikzpicture}
        \caption{Material model plugin.}
        \label{fig:runtime_pie_material_model}
    \end{subfigure}
    %
    \caption{
        A comparison of the breakdown of runtime spent in the different elements of ASPECT between the particle property and material model approaches to tracking the composition.
    }
    \label{fig:runtime_pie}
\end{figure}

Having proved the feasibility of the code to make physically sensible results, the next stage is to measure the performance of the two plugins.
In order to do this, the closed box model setup was used with varying parameters.
This model was selected instead of the decompression event model because it was simpler, since it would not need to deal with particles or material being removed from the simulation.

A control setup was used for the particle property plugin, in which the same number of particles was tracked by the simulation but the specific Perple\_X properties were not tracked.
Unfortunately, no control scenario could be created for the material model plugin.
This was because the parametrised melt fraction equation in \texttt{melt simple} produced a lot more melt than observed in Perple\_X for the same initial conditions.
This resulted in much more adaptive mesh refinement and also the solvers took a lot longer to converge to the correct value making any sort of run time comparison impossible.

\vspace{5mm}

In the following analyses, only a single run was used rather than the average of multiple runs.
This approach was chosen because this report only aims to identify general trends in the data and not make precise performance measurements.
In all cases though, care was taken to ``hot-start" every calculation whereby a dummy run was done before taking any measurements to try and mitigate library load times and other similar effects.

\subsubsection{Run time}

An accurate comparison of the run times for the particle property and material model plugins is difficult to achieve because they work in totally different ways.
That notwithstanding, it is certainly the case that the particle property plugin runs more quickly than the material model plugin.
Running both plugins with the same input files reveals that the latter plugin takes roughly twice as long to run as the former.
Breakdowns of the various parts of the runtime are displayed in Figure~\ref{fig:runtime_pie}.

\subsubsection{Scaling}

\begin{figure}
    \centering
    \begin{subfigure}{0.49\textwidth}
        \centering
        %% Creator: Matplotlib, PGF backend
%%
%% To include the figure in your LaTeX document, write
%%   \input{<filename>.pgf}
%%
%% Make sure the required packages are loaded in your preamble
%%   \usepackage{pgf}
%%
%% Figures using additional raster images can only be included by \input if
%% they are in the same directory as the main LaTeX file. For loading figures
%% from other directories you can use the `import` package
%%   \usepackage{import}
%% and then include the figures with
%%   \import{<path to file>}{<filename>.pgf}
%%
%% Matplotlib used the following preamble
%%   \usepackage{fontspec}
%%   \setmainfont{DejaVuSerif.ttf}[Path=/home/connor/.local/lib/python3.8/site-packages/matplotlib/mpl-data/fonts/ttf/]
%%   \setsansfont{DejaVuSans.ttf}[Path=/home/connor/.local/lib/python3.8/site-packages/matplotlib/mpl-data/fonts/ttf/]
%%   \setmonofont{DejaVuSansMono.ttf}[Path=/home/connor/.local/lib/python3.8/site-packages/matplotlib/mpl-data/fonts/ttf/]
%%
\begingroup%
\makeatletter%
\begin{pgfpicture}%
\pgfpathrectangle{\pgfpointorigin}{\pgfqpoint{2.764972in}{2.197213in}}%
\pgfusepath{use as bounding box, clip}%
\begin{pgfscope}%
\pgfsetbuttcap%
\pgfsetmiterjoin%
\definecolor{currentfill}{rgb}{1.000000,1.000000,1.000000}%
\pgfsetfillcolor{currentfill}%
\pgfsetlinewidth{0.000000pt}%
\definecolor{currentstroke}{rgb}{1.000000,1.000000,1.000000}%
\pgfsetstrokecolor{currentstroke}%
\pgfsetdash{}{0pt}%
\pgfpathmoveto{\pgfqpoint{0.000000in}{0.000000in}}%
\pgfpathlineto{\pgfqpoint{2.764972in}{0.000000in}}%
\pgfpathlineto{\pgfqpoint{2.764972in}{2.197213in}}%
\pgfpathlineto{\pgfqpoint{0.000000in}{2.197213in}}%
\pgfpathclose%
\pgfusepath{fill}%
\end{pgfscope}%
\begin{pgfscope}%
\pgfsetbuttcap%
\pgfsetmiterjoin%
\definecolor{currentfill}{rgb}{1.000000,1.000000,1.000000}%
\pgfsetfillcolor{currentfill}%
\pgfsetlinewidth{0.000000pt}%
\definecolor{currentstroke}{rgb}{0.000000,0.000000,0.000000}%
\pgfsetstrokecolor{currentstroke}%
\pgfsetstrokeopacity{0.000000}%
\pgfsetdash{}{0pt}%
\pgfpathmoveto{\pgfqpoint{0.478365in}{0.467838in}}%
\pgfpathlineto{\pgfqpoint{2.664972in}{0.467838in}}%
\pgfpathlineto{\pgfqpoint{2.664972in}{2.097213in}}%
\pgfpathlineto{\pgfqpoint{0.478365in}{2.097213in}}%
\pgfpathclose%
\pgfusepath{fill}%
\end{pgfscope}%
\begin{pgfscope}%
\pgfsetbuttcap%
\pgfsetroundjoin%
\definecolor{currentfill}{rgb}{0.000000,0.000000,0.000000}%
\pgfsetfillcolor{currentfill}%
\pgfsetlinewidth{0.803000pt}%
\definecolor{currentstroke}{rgb}{0.000000,0.000000,0.000000}%
\pgfsetstrokecolor{currentstroke}%
\pgfsetdash{}{0pt}%
\pgfsys@defobject{currentmarker}{\pgfqpoint{0.000000in}{-0.048611in}}{\pgfqpoint{0.000000in}{0.000000in}}{%
\pgfpathmoveto{\pgfqpoint{0.000000in}{0.000000in}}%
\pgfpathlineto{\pgfqpoint{0.000000in}{-0.048611in}}%
\pgfusepath{stroke,fill}%
}%
\begin{pgfscope}%
\pgfsys@transformshift{0.577757in}{0.467838in}%
\pgfsys@useobject{currentmarker}{}%
\end{pgfscope}%
\end{pgfscope}%
\begin{pgfscope}%
\definecolor{textcolor}{rgb}{0.000000,0.000000,0.000000}%
\pgfsetstrokecolor{textcolor}%
\pgfsetfillcolor{textcolor}%
\pgftext[x=0.577757in,y=0.370616in,,top]{\color{textcolor}\rmfamily\fontsize{8.000000}{9.600000}\selectfont \(\displaystyle 1\)}%
\end{pgfscope}%
\begin{pgfscope}%
\pgfsetbuttcap%
\pgfsetroundjoin%
\definecolor{currentfill}{rgb}{0.000000,0.000000,0.000000}%
\pgfsetfillcolor{currentfill}%
\pgfsetlinewidth{0.803000pt}%
\definecolor{currentstroke}{rgb}{0.000000,0.000000,0.000000}%
\pgfsetstrokecolor{currentstroke}%
\pgfsetdash{}{0pt}%
\pgfsys@defobject{currentmarker}{\pgfqpoint{0.000000in}{-0.048611in}}{\pgfqpoint{0.000000in}{0.000000in}}{%
\pgfpathmoveto{\pgfqpoint{0.000000in}{0.000000in}}%
\pgfpathlineto{\pgfqpoint{0.000000in}{-0.048611in}}%
\pgfusepath{stroke,fill}%
}%
\begin{pgfscope}%
\pgfsys@transformshift{0.758468in}{0.467838in}%
\pgfsys@useobject{currentmarker}{}%
\end{pgfscope}%
\end{pgfscope}%
\begin{pgfscope}%
\definecolor{textcolor}{rgb}{0.000000,0.000000,0.000000}%
\pgfsetstrokecolor{textcolor}%
\pgfsetfillcolor{textcolor}%
\pgftext[x=0.758468in,y=0.370616in,,top]{\color{textcolor}\rmfamily\fontsize{8.000000}{9.600000}\selectfont \(\displaystyle 2\)}%
\end{pgfscope}%
\begin{pgfscope}%
\pgfsetbuttcap%
\pgfsetroundjoin%
\definecolor{currentfill}{rgb}{0.000000,0.000000,0.000000}%
\pgfsetfillcolor{currentfill}%
\pgfsetlinewidth{0.803000pt}%
\definecolor{currentstroke}{rgb}{0.000000,0.000000,0.000000}%
\pgfsetstrokecolor{currentstroke}%
\pgfsetdash{}{0pt}%
\pgfsys@defobject{currentmarker}{\pgfqpoint{0.000000in}{-0.048611in}}{\pgfqpoint{0.000000in}{0.000000in}}{%
\pgfpathmoveto{\pgfqpoint{0.000000in}{0.000000in}}%
\pgfpathlineto{\pgfqpoint{0.000000in}{-0.048611in}}%
\pgfusepath{stroke,fill}%
}%
\begin{pgfscope}%
\pgfsys@transformshift{0.939179in}{0.467838in}%
\pgfsys@useobject{currentmarker}{}%
\end{pgfscope}%
\end{pgfscope}%
\begin{pgfscope}%
\definecolor{textcolor}{rgb}{0.000000,0.000000,0.000000}%
\pgfsetstrokecolor{textcolor}%
\pgfsetfillcolor{textcolor}%
\pgftext[x=0.939179in,y=0.370616in,,top]{\color{textcolor}\rmfamily\fontsize{8.000000}{9.600000}\selectfont \(\displaystyle 3\)}%
\end{pgfscope}%
\begin{pgfscope}%
\pgfsetbuttcap%
\pgfsetroundjoin%
\definecolor{currentfill}{rgb}{0.000000,0.000000,0.000000}%
\pgfsetfillcolor{currentfill}%
\pgfsetlinewidth{0.803000pt}%
\definecolor{currentstroke}{rgb}{0.000000,0.000000,0.000000}%
\pgfsetstrokecolor{currentstroke}%
\pgfsetdash{}{0pt}%
\pgfsys@defobject{currentmarker}{\pgfqpoint{0.000000in}{-0.048611in}}{\pgfqpoint{0.000000in}{0.000000in}}{%
\pgfpathmoveto{\pgfqpoint{0.000000in}{0.000000in}}%
\pgfpathlineto{\pgfqpoint{0.000000in}{-0.048611in}}%
\pgfusepath{stroke,fill}%
}%
\begin{pgfscope}%
\pgfsys@transformshift{1.119890in}{0.467838in}%
\pgfsys@useobject{currentmarker}{}%
\end{pgfscope}%
\end{pgfscope}%
\begin{pgfscope}%
\definecolor{textcolor}{rgb}{0.000000,0.000000,0.000000}%
\pgfsetstrokecolor{textcolor}%
\pgfsetfillcolor{textcolor}%
\pgftext[x=1.119890in,y=0.370616in,,top]{\color{textcolor}\rmfamily\fontsize{8.000000}{9.600000}\selectfont \(\displaystyle 4\)}%
\end{pgfscope}%
\begin{pgfscope}%
\pgfsetbuttcap%
\pgfsetroundjoin%
\definecolor{currentfill}{rgb}{0.000000,0.000000,0.000000}%
\pgfsetfillcolor{currentfill}%
\pgfsetlinewidth{0.803000pt}%
\definecolor{currentstroke}{rgb}{0.000000,0.000000,0.000000}%
\pgfsetstrokecolor{currentstroke}%
\pgfsetdash{}{0pt}%
\pgfsys@defobject{currentmarker}{\pgfqpoint{0.000000in}{-0.048611in}}{\pgfqpoint{0.000000in}{0.000000in}}{%
\pgfpathmoveto{\pgfqpoint{0.000000in}{0.000000in}}%
\pgfpathlineto{\pgfqpoint{0.000000in}{-0.048611in}}%
\pgfusepath{stroke,fill}%
}%
\begin{pgfscope}%
\pgfsys@transformshift{1.300602in}{0.467838in}%
\pgfsys@useobject{currentmarker}{}%
\end{pgfscope}%
\end{pgfscope}%
\begin{pgfscope}%
\definecolor{textcolor}{rgb}{0.000000,0.000000,0.000000}%
\pgfsetstrokecolor{textcolor}%
\pgfsetfillcolor{textcolor}%
\pgftext[x=1.300602in,y=0.370616in,,top]{\color{textcolor}\rmfamily\fontsize{8.000000}{9.600000}\selectfont \(\displaystyle 5\)}%
\end{pgfscope}%
\begin{pgfscope}%
\pgfsetbuttcap%
\pgfsetroundjoin%
\definecolor{currentfill}{rgb}{0.000000,0.000000,0.000000}%
\pgfsetfillcolor{currentfill}%
\pgfsetlinewidth{0.803000pt}%
\definecolor{currentstroke}{rgb}{0.000000,0.000000,0.000000}%
\pgfsetstrokecolor{currentstroke}%
\pgfsetdash{}{0pt}%
\pgfsys@defobject{currentmarker}{\pgfqpoint{0.000000in}{-0.048611in}}{\pgfqpoint{0.000000in}{0.000000in}}{%
\pgfpathmoveto{\pgfqpoint{0.000000in}{0.000000in}}%
\pgfpathlineto{\pgfqpoint{0.000000in}{-0.048611in}}%
\pgfusepath{stroke,fill}%
}%
\begin{pgfscope}%
\pgfsys@transformshift{1.481313in}{0.467838in}%
\pgfsys@useobject{currentmarker}{}%
\end{pgfscope}%
\end{pgfscope}%
\begin{pgfscope}%
\definecolor{textcolor}{rgb}{0.000000,0.000000,0.000000}%
\pgfsetstrokecolor{textcolor}%
\pgfsetfillcolor{textcolor}%
\pgftext[x=1.481313in,y=0.370616in,,top]{\color{textcolor}\rmfamily\fontsize{8.000000}{9.600000}\selectfont \(\displaystyle 6\)}%
\end{pgfscope}%
\begin{pgfscope}%
\pgfsetbuttcap%
\pgfsetroundjoin%
\definecolor{currentfill}{rgb}{0.000000,0.000000,0.000000}%
\pgfsetfillcolor{currentfill}%
\pgfsetlinewidth{0.803000pt}%
\definecolor{currentstroke}{rgb}{0.000000,0.000000,0.000000}%
\pgfsetstrokecolor{currentstroke}%
\pgfsetdash{}{0pt}%
\pgfsys@defobject{currentmarker}{\pgfqpoint{0.000000in}{-0.048611in}}{\pgfqpoint{0.000000in}{0.000000in}}{%
\pgfpathmoveto{\pgfqpoint{0.000000in}{0.000000in}}%
\pgfpathlineto{\pgfqpoint{0.000000in}{-0.048611in}}%
\pgfusepath{stroke,fill}%
}%
\begin{pgfscope}%
\pgfsys@transformshift{1.662024in}{0.467838in}%
\pgfsys@useobject{currentmarker}{}%
\end{pgfscope}%
\end{pgfscope}%
\begin{pgfscope}%
\definecolor{textcolor}{rgb}{0.000000,0.000000,0.000000}%
\pgfsetstrokecolor{textcolor}%
\pgfsetfillcolor{textcolor}%
\pgftext[x=1.662024in,y=0.370616in,,top]{\color{textcolor}\rmfamily\fontsize{8.000000}{9.600000}\selectfont \(\displaystyle 7\)}%
\end{pgfscope}%
\begin{pgfscope}%
\pgfsetbuttcap%
\pgfsetroundjoin%
\definecolor{currentfill}{rgb}{0.000000,0.000000,0.000000}%
\pgfsetfillcolor{currentfill}%
\pgfsetlinewidth{0.803000pt}%
\definecolor{currentstroke}{rgb}{0.000000,0.000000,0.000000}%
\pgfsetstrokecolor{currentstroke}%
\pgfsetdash{}{0pt}%
\pgfsys@defobject{currentmarker}{\pgfqpoint{0.000000in}{-0.048611in}}{\pgfqpoint{0.000000in}{0.000000in}}{%
\pgfpathmoveto{\pgfqpoint{0.000000in}{0.000000in}}%
\pgfpathlineto{\pgfqpoint{0.000000in}{-0.048611in}}%
\pgfusepath{stroke,fill}%
}%
\begin{pgfscope}%
\pgfsys@transformshift{1.842736in}{0.467838in}%
\pgfsys@useobject{currentmarker}{}%
\end{pgfscope}%
\end{pgfscope}%
\begin{pgfscope}%
\definecolor{textcolor}{rgb}{0.000000,0.000000,0.000000}%
\pgfsetstrokecolor{textcolor}%
\pgfsetfillcolor{textcolor}%
\pgftext[x=1.842736in,y=0.370616in,,top]{\color{textcolor}\rmfamily\fontsize{8.000000}{9.600000}\selectfont \(\displaystyle 8\)}%
\end{pgfscope}%
\begin{pgfscope}%
\pgfsetbuttcap%
\pgfsetroundjoin%
\definecolor{currentfill}{rgb}{0.000000,0.000000,0.000000}%
\pgfsetfillcolor{currentfill}%
\pgfsetlinewidth{0.803000pt}%
\definecolor{currentstroke}{rgb}{0.000000,0.000000,0.000000}%
\pgfsetstrokecolor{currentstroke}%
\pgfsetdash{}{0pt}%
\pgfsys@defobject{currentmarker}{\pgfqpoint{0.000000in}{-0.048611in}}{\pgfqpoint{0.000000in}{0.000000in}}{%
\pgfpathmoveto{\pgfqpoint{0.000000in}{0.000000in}}%
\pgfpathlineto{\pgfqpoint{0.000000in}{-0.048611in}}%
\pgfusepath{stroke,fill}%
}%
\begin{pgfscope}%
\pgfsys@transformshift{2.023447in}{0.467838in}%
\pgfsys@useobject{currentmarker}{}%
\end{pgfscope}%
\end{pgfscope}%
\begin{pgfscope}%
\definecolor{textcolor}{rgb}{0.000000,0.000000,0.000000}%
\pgfsetstrokecolor{textcolor}%
\pgfsetfillcolor{textcolor}%
\pgftext[x=2.023447in,y=0.370616in,,top]{\color{textcolor}\rmfamily\fontsize{8.000000}{9.600000}\selectfont \(\displaystyle 9\)}%
\end{pgfscope}%
\begin{pgfscope}%
\pgfsetbuttcap%
\pgfsetroundjoin%
\definecolor{currentfill}{rgb}{0.000000,0.000000,0.000000}%
\pgfsetfillcolor{currentfill}%
\pgfsetlinewidth{0.803000pt}%
\definecolor{currentstroke}{rgb}{0.000000,0.000000,0.000000}%
\pgfsetstrokecolor{currentstroke}%
\pgfsetdash{}{0pt}%
\pgfsys@defobject{currentmarker}{\pgfqpoint{0.000000in}{-0.048611in}}{\pgfqpoint{0.000000in}{0.000000in}}{%
\pgfpathmoveto{\pgfqpoint{0.000000in}{0.000000in}}%
\pgfpathlineto{\pgfqpoint{0.000000in}{-0.048611in}}%
\pgfusepath{stroke,fill}%
}%
\begin{pgfscope}%
\pgfsys@transformshift{2.204158in}{0.467838in}%
\pgfsys@useobject{currentmarker}{}%
\end{pgfscope}%
\end{pgfscope}%
\begin{pgfscope}%
\definecolor{textcolor}{rgb}{0.000000,0.000000,0.000000}%
\pgfsetstrokecolor{textcolor}%
\pgfsetfillcolor{textcolor}%
\pgftext[x=2.204158in,y=0.370616in,,top]{\color{textcolor}\rmfamily\fontsize{8.000000}{9.600000}\selectfont \(\displaystyle 10\)}%
\end{pgfscope}%
\begin{pgfscope}%
\pgfsetbuttcap%
\pgfsetroundjoin%
\definecolor{currentfill}{rgb}{0.000000,0.000000,0.000000}%
\pgfsetfillcolor{currentfill}%
\pgfsetlinewidth{0.803000pt}%
\definecolor{currentstroke}{rgb}{0.000000,0.000000,0.000000}%
\pgfsetstrokecolor{currentstroke}%
\pgfsetdash{}{0pt}%
\pgfsys@defobject{currentmarker}{\pgfqpoint{0.000000in}{-0.048611in}}{\pgfqpoint{0.000000in}{0.000000in}}{%
\pgfpathmoveto{\pgfqpoint{0.000000in}{0.000000in}}%
\pgfpathlineto{\pgfqpoint{0.000000in}{-0.048611in}}%
\pgfusepath{stroke,fill}%
}%
\begin{pgfscope}%
\pgfsys@transformshift{2.384870in}{0.467838in}%
\pgfsys@useobject{currentmarker}{}%
\end{pgfscope}%
\end{pgfscope}%
\begin{pgfscope}%
\definecolor{textcolor}{rgb}{0.000000,0.000000,0.000000}%
\pgfsetstrokecolor{textcolor}%
\pgfsetfillcolor{textcolor}%
\pgftext[x=2.384870in,y=0.370616in,,top]{\color{textcolor}\rmfamily\fontsize{8.000000}{9.600000}\selectfont \(\displaystyle 11\)}%
\end{pgfscope}%
\begin{pgfscope}%
\pgfsetbuttcap%
\pgfsetroundjoin%
\definecolor{currentfill}{rgb}{0.000000,0.000000,0.000000}%
\pgfsetfillcolor{currentfill}%
\pgfsetlinewidth{0.803000pt}%
\definecolor{currentstroke}{rgb}{0.000000,0.000000,0.000000}%
\pgfsetstrokecolor{currentstroke}%
\pgfsetdash{}{0pt}%
\pgfsys@defobject{currentmarker}{\pgfqpoint{0.000000in}{-0.048611in}}{\pgfqpoint{0.000000in}{0.000000in}}{%
\pgfpathmoveto{\pgfqpoint{0.000000in}{0.000000in}}%
\pgfpathlineto{\pgfqpoint{0.000000in}{-0.048611in}}%
\pgfusepath{stroke,fill}%
}%
\begin{pgfscope}%
\pgfsys@transformshift{2.565581in}{0.467838in}%
\pgfsys@useobject{currentmarker}{}%
\end{pgfscope}%
\end{pgfscope}%
\begin{pgfscope}%
\definecolor{textcolor}{rgb}{0.000000,0.000000,0.000000}%
\pgfsetstrokecolor{textcolor}%
\pgfsetfillcolor{textcolor}%
\pgftext[x=2.565581in,y=0.370616in,,top]{\color{textcolor}\rmfamily\fontsize{8.000000}{9.600000}\selectfont \(\displaystyle 12\)}%
\end{pgfscope}%
\begin{pgfscope}%
\definecolor{textcolor}{rgb}{0.000000,0.000000,0.000000}%
\pgfsetstrokecolor{textcolor}%
\pgfsetfillcolor{textcolor}%
\pgftext[x=1.571669in,y=0.207530in,,top]{\color{textcolor}\rmfamily\fontsize{8.000000}{9.600000}\selectfont Number of processors}%
\end{pgfscope}%
\begin{pgfscope}%
\pgfsetbuttcap%
\pgfsetroundjoin%
\definecolor{currentfill}{rgb}{0.000000,0.000000,0.000000}%
\pgfsetfillcolor{currentfill}%
\pgfsetlinewidth{0.803000pt}%
\definecolor{currentstroke}{rgb}{0.000000,0.000000,0.000000}%
\pgfsetstrokecolor{currentstroke}%
\pgfsetdash{}{0pt}%
\pgfsys@defobject{currentmarker}{\pgfqpoint{-0.048611in}{0.000000in}}{\pgfqpoint{0.000000in}{0.000000in}}{%
\pgfpathmoveto{\pgfqpoint{0.000000in}{0.000000in}}%
\pgfpathlineto{\pgfqpoint{-0.048611in}{0.000000in}}%
\pgfusepath{stroke,fill}%
}%
\begin{pgfscope}%
\pgfsys@transformshift{0.478365in}{0.692882in}%
\pgfsys@useobject{currentmarker}{}%
\end{pgfscope}%
\end{pgfscope}%
\begin{pgfscope}%
\definecolor{textcolor}{rgb}{0.000000,0.000000,0.000000}%
\pgfsetstrokecolor{textcolor}%
\pgfsetfillcolor{textcolor}%
\pgftext[x=0.322114in,y=0.650673in,left,base]{\color{textcolor}\rmfamily\fontsize{8.000000}{9.600000}\selectfont \(\displaystyle 2\)}%
\end{pgfscope}%
\begin{pgfscope}%
\pgfsetbuttcap%
\pgfsetroundjoin%
\definecolor{currentfill}{rgb}{0.000000,0.000000,0.000000}%
\pgfsetfillcolor{currentfill}%
\pgfsetlinewidth{0.803000pt}%
\definecolor{currentstroke}{rgb}{0.000000,0.000000,0.000000}%
\pgfsetstrokecolor{currentstroke}%
\pgfsetdash{}{0pt}%
\pgfsys@defobject{currentmarker}{\pgfqpoint{-0.048611in}{0.000000in}}{\pgfqpoint{0.000000in}{0.000000in}}{%
\pgfpathmoveto{\pgfqpoint{0.000000in}{0.000000in}}%
\pgfpathlineto{\pgfqpoint{-0.048611in}{0.000000in}}%
\pgfusepath{stroke,fill}%
}%
\begin{pgfscope}%
\pgfsys@transformshift{0.478365in}{0.994845in}%
\pgfsys@useobject{currentmarker}{}%
\end{pgfscope}%
\end{pgfscope}%
\begin{pgfscope}%
\definecolor{textcolor}{rgb}{0.000000,0.000000,0.000000}%
\pgfsetstrokecolor{textcolor}%
\pgfsetfillcolor{textcolor}%
\pgftext[x=0.322114in,y=0.952636in,left,base]{\color{textcolor}\rmfamily\fontsize{8.000000}{9.600000}\selectfont \(\displaystyle 4\)}%
\end{pgfscope}%
\begin{pgfscope}%
\pgfsetbuttcap%
\pgfsetroundjoin%
\definecolor{currentfill}{rgb}{0.000000,0.000000,0.000000}%
\pgfsetfillcolor{currentfill}%
\pgfsetlinewidth{0.803000pt}%
\definecolor{currentstroke}{rgb}{0.000000,0.000000,0.000000}%
\pgfsetstrokecolor{currentstroke}%
\pgfsetdash{}{0pt}%
\pgfsys@defobject{currentmarker}{\pgfqpoint{-0.048611in}{0.000000in}}{\pgfqpoint{0.000000in}{0.000000in}}{%
\pgfpathmoveto{\pgfqpoint{0.000000in}{0.000000in}}%
\pgfpathlineto{\pgfqpoint{-0.048611in}{0.000000in}}%
\pgfusepath{stroke,fill}%
}%
\begin{pgfscope}%
\pgfsys@transformshift{0.478365in}{1.296808in}%
\pgfsys@useobject{currentmarker}{}%
\end{pgfscope}%
\end{pgfscope}%
\begin{pgfscope}%
\definecolor{textcolor}{rgb}{0.000000,0.000000,0.000000}%
\pgfsetstrokecolor{textcolor}%
\pgfsetfillcolor{textcolor}%
\pgftext[x=0.322114in,y=1.254599in,left,base]{\color{textcolor}\rmfamily\fontsize{8.000000}{9.600000}\selectfont \(\displaystyle 6\)}%
\end{pgfscope}%
\begin{pgfscope}%
\pgfsetbuttcap%
\pgfsetroundjoin%
\definecolor{currentfill}{rgb}{0.000000,0.000000,0.000000}%
\pgfsetfillcolor{currentfill}%
\pgfsetlinewidth{0.803000pt}%
\definecolor{currentstroke}{rgb}{0.000000,0.000000,0.000000}%
\pgfsetstrokecolor{currentstroke}%
\pgfsetdash{}{0pt}%
\pgfsys@defobject{currentmarker}{\pgfqpoint{-0.048611in}{0.000000in}}{\pgfqpoint{0.000000in}{0.000000in}}{%
\pgfpathmoveto{\pgfqpoint{0.000000in}{0.000000in}}%
\pgfpathlineto{\pgfqpoint{-0.048611in}{0.000000in}}%
\pgfusepath{stroke,fill}%
}%
\begin{pgfscope}%
\pgfsys@transformshift{0.478365in}{1.598771in}%
\pgfsys@useobject{currentmarker}{}%
\end{pgfscope}%
\end{pgfscope}%
\begin{pgfscope}%
\definecolor{textcolor}{rgb}{0.000000,0.000000,0.000000}%
\pgfsetstrokecolor{textcolor}%
\pgfsetfillcolor{textcolor}%
\pgftext[x=0.322114in,y=1.556561in,left,base]{\color{textcolor}\rmfamily\fontsize{8.000000}{9.600000}\selectfont \(\displaystyle 8\)}%
\end{pgfscope}%
\begin{pgfscope}%
\pgfsetbuttcap%
\pgfsetroundjoin%
\definecolor{currentfill}{rgb}{0.000000,0.000000,0.000000}%
\pgfsetfillcolor{currentfill}%
\pgfsetlinewidth{0.803000pt}%
\definecolor{currentstroke}{rgb}{0.000000,0.000000,0.000000}%
\pgfsetstrokecolor{currentstroke}%
\pgfsetdash{}{0pt}%
\pgfsys@defobject{currentmarker}{\pgfqpoint{-0.048611in}{0.000000in}}{\pgfqpoint{0.000000in}{0.000000in}}{%
\pgfpathmoveto{\pgfqpoint{0.000000in}{0.000000in}}%
\pgfpathlineto{\pgfqpoint{-0.048611in}{0.000000in}}%
\pgfusepath{stroke,fill}%
}%
\begin{pgfscope}%
\pgfsys@transformshift{0.478365in}{1.900733in}%
\pgfsys@useobject{currentmarker}{}%
\end{pgfscope}%
\end{pgfscope}%
\begin{pgfscope}%
\definecolor{textcolor}{rgb}{0.000000,0.000000,0.000000}%
\pgfsetstrokecolor{textcolor}%
\pgfsetfillcolor{textcolor}%
\pgftext[x=0.263086in,y=1.858524in,left,base]{\color{textcolor}\rmfamily\fontsize{8.000000}{9.600000}\selectfont \(\displaystyle 10\)}%
\end{pgfscope}%
\begin{pgfscope}%
\definecolor{textcolor}{rgb}{0.000000,0.000000,0.000000}%
\pgfsetstrokecolor{textcolor}%
\pgfsetfillcolor{textcolor}%
\pgftext[x=0.207530in,y=1.282526in,,bottom,rotate=90.000000]{\color{textcolor}\rmfamily\fontsize{8.000000}{9.600000}\selectfont Speedup}%
\end{pgfscope}%
\begin{pgfscope}%
\pgfpathrectangle{\pgfqpoint{0.478365in}{0.467838in}}{\pgfqpoint{2.186607in}{1.629375in}}%
\pgfusepath{clip}%
\pgfsetrectcap%
\pgfsetroundjoin%
\pgfsetlinewidth{1.505625pt}%
\definecolor{currentstroke}{rgb}{0.121569,0.466667,0.705882}%
\pgfsetstrokecolor{currentstroke}%
\pgfsetdash{}{0pt}%
\pgfpathmoveto{\pgfqpoint{0.577757in}{0.541901in}}%
\pgfpathlineto{\pgfqpoint{0.758468in}{0.670945in}}%
\pgfpathlineto{\pgfqpoint{0.939179in}{0.798926in}}%
\pgfpathlineto{\pgfqpoint{1.119890in}{0.901246in}}%
\pgfpathlineto{\pgfqpoint{1.300602in}{0.998767in}}%
\pgfpathlineto{\pgfqpoint{1.481313in}{1.086524in}}%
\pgfpathlineto{\pgfqpoint{1.662024in}{1.118985in}}%
\pgfpathlineto{\pgfqpoint{1.842736in}{1.226709in}}%
\pgfpathlineto{\pgfqpoint{2.023447in}{1.264598in}}%
\pgfpathlineto{\pgfqpoint{2.204158in}{1.348901in}}%
\pgfpathlineto{\pgfqpoint{2.384870in}{1.431013in}}%
\pgfpathlineto{\pgfqpoint{2.565581in}{1.486671in}}%
\pgfusepath{stroke}%
\end{pgfscope}%
\begin{pgfscope}%
\pgfpathrectangle{\pgfqpoint{0.478365in}{0.467838in}}{\pgfqpoint{2.186607in}{1.629375in}}%
\pgfusepath{clip}%
\pgfsetrectcap%
\pgfsetroundjoin%
\pgfsetlinewidth{1.505625pt}%
\definecolor{currentstroke}{rgb}{1.000000,0.498039,0.054902}%
\pgfsetstrokecolor{currentstroke}%
\pgfsetdash{}{0pt}%
\pgfpathmoveto{\pgfqpoint{0.577757in}{0.541901in}}%
\pgfpathlineto{\pgfqpoint{0.758468in}{0.594602in}}%
\pgfpathlineto{\pgfqpoint{0.939179in}{0.634878in}}%
\pgfpathlineto{\pgfqpoint{1.119890in}{0.652938in}}%
\pgfpathlineto{\pgfqpoint{1.300602in}{0.658458in}}%
\pgfpathlineto{\pgfqpoint{1.481313in}{0.680334in}}%
\pgfpathlineto{\pgfqpoint{1.662024in}{0.687490in}}%
\pgfpathlineto{\pgfqpoint{1.842736in}{0.686273in}}%
\pgfpathlineto{\pgfqpoint{2.023447in}{0.663869in}}%
\pgfpathlineto{\pgfqpoint{2.204158in}{0.665257in}}%
\pgfpathlineto{\pgfqpoint{2.384870in}{0.666658in}}%
\pgfpathlineto{\pgfqpoint{2.565581in}{0.660126in}}%
\pgfusepath{stroke}%
\end{pgfscope}%
\begin{pgfscope}%
\pgfpathrectangle{\pgfqpoint{0.478365in}{0.467838in}}{\pgfqpoint{2.186607in}{1.629375in}}%
\pgfusepath{clip}%
\pgfsetbuttcap%
\pgfsetroundjoin%
\pgfsetlinewidth{0.501875pt}%
\definecolor{currentstroke}{rgb}{0.000000,0.000000,0.000000}%
\pgfsetstrokecolor{currentstroke}%
\pgfsetdash{{1.850000pt}{0.800000pt}}{0.000000pt}%
\pgfpathmoveto{\pgfqpoint{0.577757in}{0.541901in}}%
\pgfpathlineto{\pgfqpoint{0.758468in}{0.689892in}}%
\pgfpathlineto{\pgfqpoint{0.939179in}{0.834982in}}%
\pgfpathlineto{\pgfqpoint{1.119890in}{0.977255in}}%
\pgfpathlineto{\pgfqpoint{1.300602in}{1.116791in}}%
\pgfpathlineto{\pgfqpoint{1.481313in}{1.253670in}}%
\pgfpathlineto{\pgfqpoint{1.662024in}{1.387966in}}%
\pgfpathlineto{\pgfqpoint{1.842736in}{1.519752in}}%
\pgfpathlineto{\pgfqpoint{2.023447in}{1.649098in}}%
\pgfpathlineto{\pgfqpoint{2.204158in}{1.776070in}}%
\pgfpathlineto{\pgfqpoint{2.384870in}{1.900733in}}%
\pgfpathlineto{\pgfqpoint{2.565581in}{2.023151in}}%
\pgfusepath{stroke}%
\end{pgfscope}%
\begin{pgfscope}%
\pgfpathrectangle{\pgfqpoint{0.478365in}{0.467838in}}{\pgfqpoint{2.186607in}{1.629375in}}%
\pgfusepath{clip}%
\pgfsetbuttcap%
\pgfsetroundjoin%
\pgfsetlinewidth{0.501875pt}%
\definecolor{currentstroke}{rgb}{0.000000,0.000000,0.000000}%
\pgfsetstrokecolor{currentstroke}%
\pgfsetdash{{1.850000pt}{0.800000pt}}{0.000000pt}%
\pgfpathmoveto{\pgfqpoint{0.577757in}{0.541901in}}%
\pgfpathlineto{\pgfqpoint{0.758468in}{0.665431in}}%
\pgfpathlineto{\pgfqpoint{0.939179in}{0.768373in}}%
\pgfpathlineto{\pgfqpoint{1.119890in}{0.855478in}}%
\pgfpathlineto{\pgfqpoint{1.300602in}{0.930139in}}%
\pgfpathlineto{\pgfqpoint{1.481313in}{0.994845in}}%
\pgfpathlineto{\pgfqpoint{1.662024in}{1.051463in}}%
\pgfpathlineto{\pgfqpoint{1.842736in}{1.101420in}}%
\pgfpathlineto{\pgfqpoint{2.023447in}{1.145826in}}%
\pgfpathlineto{\pgfqpoint{2.204158in}{1.185558in}}%
\pgfpathlineto{\pgfqpoint{2.384870in}{1.221317in}}%
\pgfpathlineto{\pgfqpoint{2.565581in}{1.253670in}}%
\pgfusepath{stroke}%
\end{pgfscope}%
\begin{pgfscope}%
\pgfpathrectangle{\pgfqpoint{0.478365in}{0.467838in}}{\pgfqpoint{2.186607in}{1.629375in}}%
\pgfusepath{clip}%
\pgfsetbuttcap%
\pgfsetroundjoin%
\pgfsetlinewidth{0.501875pt}%
\definecolor{currentstroke}{rgb}{0.000000,0.000000,0.000000}%
\pgfsetstrokecolor{currentstroke}%
\pgfsetdash{{1.850000pt}{0.800000pt}}{0.000000pt}%
\pgfpathmoveto{\pgfqpoint{0.577757in}{0.541901in}}%
\pgfpathlineto{\pgfqpoint{0.758468in}{0.592228in}}%
\pgfpathlineto{\pgfqpoint{0.939179in}{0.617392in}}%
\pgfpathlineto{\pgfqpoint{1.119890in}{0.632490in}}%
\pgfpathlineto{\pgfqpoint{1.300602in}{0.642555in}}%
\pgfpathlineto{\pgfqpoint{1.481313in}{0.649745in}}%
\pgfpathlineto{\pgfqpoint{1.662024in}{0.655137in}}%
\pgfpathlineto{\pgfqpoint{1.842736in}{0.659331in}}%
\pgfpathlineto{\pgfqpoint{2.023447in}{0.662686in}}%
\pgfpathlineto{\pgfqpoint{2.204158in}{0.665431in}}%
\pgfpathlineto{\pgfqpoint{2.384870in}{0.667719in}}%
\pgfpathlineto{\pgfqpoint{2.565581in}{0.669654in}}%
\pgfusepath{stroke}%
\end{pgfscope}%
\begin{pgfscope}%
\pgfsetrectcap%
\pgfsetmiterjoin%
\pgfsetlinewidth{0.803000pt}%
\definecolor{currentstroke}{rgb}{0.000000,0.000000,0.000000}%
\pgfsetstrokecolor{currentstroke}%
\pgfsetdash{}{0pt}%
\pgfpathmoveto{\pgfqpoint{0.478365in}{0.467838in}}%
\pgfpathlineto{\pgfqpoint{0.478365in}{2.097213in}}%
\pgfusepath{stroke}%
\end{pgfscope}%
\begin{pgfscope}%
\pgfsetrectcap%
\pgfsetmiterjoin%
\pgfsetlinewidth{0.803000pt}%
\definecolor{currentstroke}{rgb}{0.000000,0.000000,0.000000}%
\pgfsetstrokecolor{currentstroke}%
\pgfsetdash{}{0pt}%
\pgfpathmoveto{\pgfqpoint{2.664972in}{0.467838in}}%
\pgfpathlineto{\pgfqpoint{2.664972in}{2.097213in}}%
\pgfusepath{stroke}%
\end{pgfscope}%
\begin{pgfscope}%
\pgfsetrectcap%
\pgfsetmiterjoin%
\pgfsetlinewidth{0.803000pt}%
\definecolor{currentstroke}{rgb}{0.000000,0.000000,0.000000}%
\pgfsetstrokecolor{currentstroke}%
\pgfsetdash{}{0pt}%
\pgfpathmoveto{\pgfqpoint{0.478365in}{0.467838in}}%
\pgfpathlineto{\pgfqpoint{2.664972in}{0.467838in}}%
\pgfusepath{stroke}%
\end{pgfscope}%
\begin{pgfscope}%
\pgfsetrectcap%
\pgfsetmiterjoin%
\pgfsetlinewidth{0.803000pt}%
\definecolor{currentstroke}{rgb}{0.000000,0.000000,0.000000}%
\pgfsetstrokecolor{currentstroke}%
\pgfsetdash{}{0pt}%
\pgfpathmoveto{\pgfqpoint{0.478365in}{2.097213in}}%
\pgfpathlineto{\pgfqpoint{2.664972in}{2.097213in}}%
\pgfusepath{stroke}%
\end{pgfscope}%
\begin{pgfscope}%
\definecolor{textcolor}{rgb}{0.000000,0.000000,0.000000}%
\pgfsetstrokecolor{textcolor}%
\pgfsetfillcolor{textcolor}%
\pgftext[x=1.759247in,y=1.327894in,left,base]{\color{textcolor}\rmfamily\fontsize{8.000000}{9.600000}\selectfont \(\displaystyle f=0.01\)}%
\end{pgfscope}%
\begin{pgfscope}%
\definecolor{textcolor}{rgb}{0.000000,0.000000,0.000000}%
\pgfsetstrokecolor{textcolor}%
\pgfsetfillcolor{textcolor}%
\pgftext[x=1.759247in,y=0.954241in,left,base]{\color{textcolor}\rmfamily\fontsize{8.000000}{9.600000}\selectfont \(\displaystyle f=0.1\)}%
\end{pgfscope}%
\begin{pgfscope}%
\definecolor{textcolor}{rgb}{0.000000,0.000000,0.000000}%
\pgfsetstrokecolor{textcolor}%
\pgfsetfillcolor{textcolor}%
\pgftext[x=1.759247in,y=0.530137in,left,base]{\color{textcolor}\rmfamily\fontsize{8.000000}{9.600000}\selectfont \(\displaystyle f=0.5\)}%
\end{pgfscope}%
\begin{pgfscope}%
\pgfsetrectcap%
\pgfsetroundjoin%
\pgfsetlinewidth{1.505625pt}%
\definecolor{currentstroke}{rgb}{0.121569,0.466667,0.705882}%
\pgfsetstrokecolor{currentstroke}%
\pgfsetdash{}{0pt}%
\pgfpathmoveto{\pgfqpoint{0.578365in}{1.951684in}}%
\pgfpathlineto{\pgfqpoint{0.800588in}{1.951684in}}%
\pgfusepath{stroke}%
\end{pgfscope}%
\begin{pgfscope}%
\definecolor{textcolor}{rgb}{0.000000,0.000000,0.000000}%
\pgfsetstrokecolor{textcolor}%
\pgfsetfillcolor{textcolor}%
\pgftext[x=0.889476in,y=1.912795in,left,base]{\color{textcolor}\rmfamily\fontsize{8.000000}{9.600000}\selectfont With Perple\_X}%
\end{pgfscope}%
\begin{pgfscope}%
\pgfsetrectcap%
\pgfsetroundjoin%
\pgfsetlinewidth{1.505625pt}%
\definecolor{currentstroke}{rgb}{1.000000,0.498039,0.054902}%
\pgfsetstrokecolor{currentstroke}%
\pgfsetdash{}{0pt}%
\pgfpathmoveto{\pgfqpoint{0.578365in}{1.785505in}}%
\pgfpathlineto{\pgfqpoint{0.800588in}{1.785505in}}%
\pgfusepath{stroke}%
\end{pgfscope}%
\begin{pgfscope}%
\definecolor{textcolor}{rgb}{0.000000,0.000000,0.000000}%
\pgfsetstrokecolor{textcolor}%
\pgfsetfillcolor{textcolor}%
\pgftext[x=0.889476in,y=1.746616in,left,base]{\color{textcolor}\rmfamily\fontsize{8.000000}{9.600000}\selectfont Without Perple\_X}%
\end{pgfscope}%
\end{pgfpicture}%
\makeatother%
\endgroup%

        \caption{Particle property plugin.}
        \label{fig:scaling_particle_property}
    \end{subfigure}
    \hfill
    \begin{subfigure}{0.49\textwidth}
        \centering
        %% Creator: Matplotlib, PGF backend
%%
%% To include the figure in your LaTeX document, write
%%   \input{<filename>.pgf}
%%
%% Make sure the required packages are loaded in your preamble
%%   \usepackage{pgf}
%%
%% Figures using additional raster images can only be included by \input if
%% they are in the same directory as the main LaTeX file. For loading figures
%% from other directories you can use the `import` package
%%   \usepackage{import}
%% and then include the figures with
%%   \import{<path to file>}{<filename>.pgf}
%%
%% Matplotlib used the following preamble
%%   \usepackage{fontspec}
%%   \setmainfont{DejaVuSerif.ttf}[Path=/home/connor/.local/lib/python3.8/site-packages/matplotlib/mpl-data/fonts/ttf/]
%%   \setsansfont{DejaVuSans.ttf}[Path=/home/connor/.local/lib/python3.8/site-packages/matplotlib/mpl-data/fonts/ttf/]
%%   \setmonofont{DejaVuSansMono.ttf}[Path=/home/connor/.local/lib/python3.8/site-packages/matplotlib/mpl-data/fonts/ttf/]
%%
\begingroup%
\makeatletter%
\begin{pgfpicture}%
\pgfpathrectangle{\pgfpointorigin}{\pgfqpoint{3.898197in}{3.017211in}}%
\pgfusepath{use as bounding box, clip}%
\begin{pgfscope}%
\pgfsetbuttcap%
\pgfsetmiterjoin%
\definecolor{currentfill}{rgb}{1.000000,1.000000,1.000000}%
\pgfsetfillcolor{currentfill}%
\pgfsetlinewidth{0.000000pt}%
\definecolor{currentstroke}{rgb}{1.000000,1.000000,1.000000}%
\pgfsetstrokecolor{currentstroke}%
\pgfsetdash{}{0pt}%
\pgfpathmoveto{\pgfqpoint{0.000000in}{-0.000000in}}%
\pgfpathlineto{\pgfqpoint{3.898197in}{-0.000000in}}%
\pgfpathlineto{\pgfqpoint{3.898197in}{3.017211in}}%
\pgfpathlineto{\pgfqpoint{0.000000in}{3.017211in}}%
\pgfpathclose%
\pgfusepath{fill}%
\end{pgfscope}%
\begin{pgfscope}%
\pgfsetbuttcap%
\pgfsetmiterjoin%
\definecolor{currentfill}{rgb}{1.000000,1.000000,1.000000}%
\pgfsetfillcolor{currentfill}%
\pgfsetlinewidth{0.000000pt}%
\definecolor{currentstroke}{rgb}{0.000000,0.000000,0.000000}%
\pgfsetstrokecolor{currentstroke}%
\pgfsetstrokeopacity{0.000000}%
\pgfsetdash{}{0pt}%
\pgfpathmoveto{\pgfqpoint{0.511159in}{0.467838in}}%
\pgfpathlineto{\pgfqpoint{3.798197in}{0.467838in}}%
\pgfpathlineto{\pgfqpoint{3.798197in}{2.917211in}}%
\pgfpathlineto{\pgfqpoint{0.511159in}{2.917211in}}%
\pgfpathclose%
\pgfusepath{fill}%
\end{pgfscope}%
\begin{pgfscope}%
\pgfsetbuttcap%
\pgfsetroundjoin%
\definecolor{currentfill}{rgb}{0.000000,0.000000,0.000000}%
\pgfsetfillcolor{currentfill}%
\pgfsetlinewidth{0.803000pt}%
\definecolor{currentstroke}{rgb}{0.000000,0.000000,0.000000}%
\pgfsetstrokecolor{currentstroke}%
\pgfsetdash{}{0pt}%
\pgfsys@defobject{currentmarker}{\pgfqpoint{0.000000in}{-0.048611in}}{\pgfqpoint{0.000000in}{0.000000in}}{%
\pgfpathmoveto{\pgfqpoint{0.000000in}{0.000000in}}%
\pgfpathlineto{\pgfqpoint{0.000000in}{-0.048611in}}%
\pgfusepath{stroke,fill}%
}%
\begin{pgfscope}%
\pgfsys@transformshift{0.932226in}{0.467838in}%
\pgfsys@useobject{currentmarker}{}%
\end{pgfscope}%
\end{pgfscope}%
\begin{pgfscope}%
\definecolor{textcolor}{rgb}{0.000000,0.000000,0.000000}%
\pgfsetstrokecolor{textcolor}%
\pgfsetfillcolor{textcolor}%
\pgftext[x=0.932226in,y=0.370616in,,top]{\color{textcolor}\rmfamily\fontsize{8.000000}{9.600000}\selectfont \(\displaystyle 2\)}%
\end{pgfscope}%
\begin{pgfscope}%
\pgfsetbuttcap%
\pgfsetroundjoin%
\definecolor{currentfill}{rgb}{0.000000,0.000000,0.000000}%
\pgfsetfillcolor{currentfill}%
\pgfsetlinewidth{0.803000pt}%
\definecolor{currentstroke}{rgb}{0.000000,0.000000,0.000000}%
\pgfsetstrokecolor{currentstroke}%
\pgfsetdash{}{0pt}%
\pgfsys@defobject{currentmarker}{\pgfqpoint{0.000000in}{-0.048611in}}{\pgfqpoint{0.000000in}{0.000000in}}{%
\pgfpathmoveto{\pgfqpoint{0.000000in}{0.000000in}}%
\pgfpathlineto{\pgfqpoint{0.000000in}{-0.048611in}}%
\pgfusepath{stroke,fill}%
}%
\begin{pgfscope}%
\pgfsys@transformshift{1.475538in}{0.467838in}%
\pgfsys@useobject{currentmarker}{}%
\end{pgfscope}%
\end{pgfscope}%
\begin{pgfscope}%
\definecolor{textcolor}{rgb}{0.000000,0.000000,0.000000}%
\pgfsetstrokecolor{textcolor}%
\pgfsetfillcolor{textcolor}%
\pgftext[x=1.475538in,y=0.370616in,,top]{\color{textcolor}\rmfamily\fontsize{8.000000}{9.600000}\selectfont \(\displaystyle 4\)}%
\end{pgfscope}%
\begin{pgfscope}%
\pgfsetbuttcap%
\pgfsetroundjoin%
\definecolor{currentfill}{rgb}{0.000000,0.000000,0.000000}%
\pgfsetfillcolor{currentfill}%
\pgfsetlinewidth{0.803000pt}%
\definecolor{currentstroke}{rgb}{0.000000,0.000000,0.000000}%
\pgfsetstrokecolor{currentstroke}%
\pgfsetdash{}{0pt}%
\pgfsys@defobject{currentmarker}{\pgfqpoint{0.000000in}{-0.048611in}}{\pgfqpoint{0.000000in}{0.000000in}}{%
\pgfpathmoveto{\pgfqpoint{0.000000in}{0.000000in}}%
\pgfpathlineto{\pgfqpoint{0.000000in}{-0.048611in}}%
\pgfusepath{stroke,fill}%
}%
\begin{pgfscope}%
\pgfsys@transformshift{2.018850in}{0.467838in}%
\pgfsys@useobject{currentmarker}{}%
\end{pgfscope}%
\end{pgfscope}%
\begin{pgfscope}%
\definecolor{textcolor}{rgb}{0.000000,0.000000,0.000000}%
\pgfsetstrokecolor{textcolor}%
\pgfsetfillcolor{textcolor}%
\pgftext[x=2.018850in,y=0.370616in,,top]{\color{textcolor}\rmfamily\fontsize{8.000000}{9.600000}\selectfont \(\displaystyle 6\)}%
\end{pgfscope}%
\begin{pgfscope}%
\pgfsetbuttcap%
\pgfsetroundjoin%
\definecolor{currentfill}{rgb}{0.000000,0.000000,0.000000}%
\pgfsetfillcolor{currentfill}%
\pgfsetlinewidth{0.803000pt}%
\definecolor{currentstroke}{rgb}{0.000000,0.000000,0.000000}%
\pgfsetstrokecolor{currentstroke}%
\pgfsetdash{}{0pt}%
\pgfsys@defobject{currentmarker}{\pgfqpoint{0.000000in}{-0.048611in}}{\pgfqpoint{0.000000in}{0.000000in}}{%
\pgfpathmoveto{\pgfqpoint{0.000000in}{0.000000in}}%
\pgfpathlineto{\pgfqpoint{0.000000in}{-0.048611in}}%
\pgfusepath{stroke,fill}%
}%
\begin{pgfscope}%
\pgfsys@transformshift{2.562162in}{0.467838in}%
\pgfsys@useobject{currentmarker}{}%
\end{pgfscope}%
\end{pgfscope}%
\begin{pgfscope}%
\definecolor{textcolor}{rgb}{0.000000,0.000000,0.000000}%
\pgfsetstrokecolor{textcolor}%
\pgfsetfillcolor{textcolor}%
\pgftext[x=2.562162in,y=0.370616in,,top]{\color{textcolor}\rmfamily\fontsize{8.000000}{9.600000}\selectfont \(\displaystyle 8\)}%
\end{pgfscope}%
\begin{pgfscope}%
\pgfsetbuttcap%
\pgfsetroundjoin%
\definecolor{currentfill}{rgb}{0.000000,0.000000,0.000000}%
\pgfsetfillcolor{currentfill}%
\pgfsetlinewidth{0.803000pt}%
\definecolor{currentstroke}{rgb}{0.000000,0.000000,0.000000}%
\pgfsetstrokecolor{currentstroke}%
\pgfsetdash{}{0pt}%
\pgfsys@defobject{currentmarker}{\pgfqpoint{0.000000in}{-0.048611in}}{\pgfqpoint{0.000000in}{0.000000in}}{%
\pgfpathmoveto{\pgfqpoint{0.000000in}{0.000000in}}%
\pgfpathlineto{\pgfqpoint{0.000000in}{-0.048611in}}%
\pgfusepath{stroke,fill}%
}%
\begin{pgfscope}%
\pgfsys@transformshift{3.105474in}{0.467838in}%
\pgfsys@useobject{currentmarker}{}%
\end{pgfscope}%
\end{pgfscope}%
\begin{pgfscope}%
\definecolor{textcolor}{rgb}{0.000000,0.000000,0.000000}%
\pgfsetstrokecolor{textcolor}%
\pgfsetfillcolor{textcolor}%
\pgftext[x=3.105474in,y=0.370616in,,top]{\color{textcolor}\rmfamily\fontsize{8.000000}{9.600000}\selectfont \(\displaystyle 10\)}%
\end{pgfscope}%
\begin{pgfscope}%
\pgfsetbuttcap%
\pgfsetroundjoin%
\definecolor{currentfill}{rgb}{0.000000,0.000000,0.000000}%
\pgfsetfillcolor{currentfill}%
\pgfsetlinewidth{0.803000pt}%
\definecolor{currentstroke}{rgb}{0.000000,0.000000,0.000000}%
\pgfsetstrokecolor{currentstroke}%
\pgfsetdash{}{0pt}%
\pgfsys@defobject{currentmarker}{\pgfqpoint{0.000000in}{-0.048611in}}{\pgfqpoint{0.000000in}{0.000000in}}{%
\pgfpathmoveto{\pgfqpoint{0.000000in}{0.000000in}}%
\pgfpathlineto{\pgfqpoint{0.000000in}{-0.048611in}}%
\pgfusepath{stroke,fill}%
}%
\begin{pgfscope}%
\pgfsys@transformshift{3.648786in}{0.467838in}%
\pgfsys@useobject{currentmarker}{}%
\end{pgfscope}%
\end{pgfscope}%
\begin{pgfscope}%
\definecolor{textcolor}{rgb}{0.000000,0.000000,0.000000}%
\pgfsetstrokecolor{textcolor}%
\pgfsetfillcolor{textcolor}%
\pgftext[x=3.648786in,y=0.370616in,,top]{\color{textcolor}\rmfamily\fontsize{8.000000}{9.600000}\selectfont \(\displaystyle 12\)}%
\end{pgfscope}%
\begin{pgfscope}%
\definecolor{textcolor}{rgb}{0.000000,0.000000,0.000000}%
\pgfsetstrokecolor{textcolor}%
\pgfsetfillcolor{textcolor}%
\pgftext[x=2.154678in,y=0.207530in,,top]{\color{textcolor}\rmfamily\fontsize{8.000000}{9.600000}\selectfont Number of processors}%
\end{pgfscope}%
\begin{pgfscope}%
\pgfsetbuttcap%
\pgfsetroundjoin%
\definecolor{currentfill}{rgb}{0.000000,0.000000,0.000000}%
\pgfsetfillcolor{currentfill}%
\pgfsetlinewidth{0.803000pt}%
\definecolor{currentstroke}{rgb}{0.000000,0.000000,0.000000}%
\pgfsetstrokecolor{currentstroke}%
\pgfsetdash{}{0pt}%
\pgfsys@defobject{currentmarker}{\pgfqpoint{-0.048611in}{0.000000in}}{\pgfqpoint{0.000000in}{0.000000in}}{%
\pgfpathmoveto{\pgfqpoint{0.000000in}{0.000000in}}%
\pgfpathlineto{\pgfqpoint{-0.048611in}{0.000000in}}%
\pgfusepath{stroke,fill}%
}%
\begin{pgfscope}%
\pgfsys@transformshift{0.511159in}{0.579173in}%
\pgfsys@useobject{currentmarker}{}%
\end{pgfscope}%
\end{pgfscope}%
\begin{pgfscope}%
\definecolor{textcolor}{rgb}{0.000000,0.000000,0.000000}%
\pgfsetstrokecolor{textcolor}%
\pgfsetfillcolor{textcolor}%
\pgftext[x=0.263086in,y=0.536964in,left,base]{\color{textcolor}\rmfamily\fontsize{8.000000}{9.600000}\selectfont \(\displaystyle 1.0\)}%
\end{pgfscope}%
\begin{pgfscope}%
\pgfsetbuttcap%
\pgfsetroundjoin%
\definecolor{currentfill}{rgb}{0.000000,0.000000,0.000000}%
\pgfsetfillcolor{currentfill}%
\pgfsetlinewidth{0.803000pt}%
\definecolor{currentstroke}{rgb}{0.000000,0.000000,0.000000}%
\pgfsetstrokecolor{currentstroke}%
\pgfsetdash{}{0pt}%
\pgfsys@defobject{currentmarker}{\pgfqpoint{-0.048611in}{0.000000in}}{\pgfqpoint{0.000000in}{0.000000in}}{%
\pgfpathmoveto{\pgfqpoint{0.000000in}{0.000000in}}%
\pgfpathlineto{\pgfqpoint{-0.048611in}{0.000000in}}%
\pgfusepath{stroke,fill}%
}%
\begin{pgfscope}%
\pgfsys@transformshift{0.511159in}{1.073758in}%
\pgfsys@useobject{currentmarker}{}%
\end{pgfscope}%
\end{pgfscope}%
\begin{pgfscope}%
\definecolor{textcolor}{rgb}{0.000000,0.000000,0.000000}%
\pgfsetstrokecolor{textcolor}%
\pgfsetfillcolor{textcolor}%
\pgftext[x=0.263086in,y=1.031549in,left,base]{\color{textcolor}\rmfamily\fontsize{8.000000}{9.600000}\selectfont \(\displaystyle 1.2\)}%
\end{pgfscope}%
\begin{pgfscope}%
\pgfsetbuttcap%
\pgfsetroundjoin%
\definecolor{currentfill}{rgb}{0.000000,0.000000,0.000000}%
\pgfsetfillcolor{currentfill}%
\pgfsetlinewidth{0.803000pt}%
\definecolor{currentstroke}{rgb}{0.000000,0.000000,0.000000}%
\pgfsetstrokecolor{currentstroke}%
\pgfsetdash{}{0pt}%
\pgfsys@defobject{currentmarker}{\pgfqpoint{-0.048611in}{0.000000in}}{\pgfqpoint{0.000000in}{0.000000in}}{%
\pgfpathmoveto{\pgfqpoint{0.000000in}{0.000000in}}%
\pgfpathlineto{\pgfqpoint{-0.048611in}{0.000000in}}%
\pgfusepath{stroke,fill}%
}%
\begin{pgfscope}%
\pgfsys@transformshift{0.511159in}{1.568343in}%
\pgfsys@useobject{currentmarker}{}%
\end{pgfscope}%
\end{pgfscope}%
\begin{pgfscope}%
\definecolor{textcolor}{rgb}{0.000000,0.000000,0.000000}%
\pgfsetstrokecolor{textcolor}%
\pgfsetfillcolor{textcolor}%
\pgftext[x=0.263086in,y=1.526134in,left,base]{\color{textcolor}\rmfamily\fontsize{8.000000}{9.600000}\selectfont \(\displaystyle 1.4\)}%
\end{pgfscope}%
\begin{pgfscope}%
\pgfsetbuttcap%
\pgfsetroundjoin%
\definecolor{currentfill}{rgb}{0.000000,0.000000,0.000000}%
\pgfsetfillcolor{currentfill}%
\pgfsetlinewidth{0.803000pt}%
\definecolor{currentstroke}{rgb}{0.000000,0.000000,0.000000}%
\pgfsetstrokecolor{currentstroke}%
\pgfsetdash{}{0pt}%
\pgfsys@defobject{currentmarker}{\pgfqpoint{-0.048611in}{0.000000in}}{\pgfqpoint{0.000000in}{0.000000in}}{%
\pgfpathmoveto{\pgfqpoint{0.000000in}{0.000000in}}%
\pgfpathlineto{\pgfqpoint{-0.048611in}{0.000000in}}%
\pgfusepath{stroke,fill}%
}%
\begin{pgfscope}%
\pgfsys@transformshift{0.511159in}{2.062928in}%
\pgfsys@useobject{currentmarker}{}%
\end{pgfscope}%
\end{pgfscope}%
\begin{pgfscope}%
\definecolor{textcolor}{rgb}{0.000000,0.000000,0.000000}%
\pgfsetstrokecolor{textcolor}%
\pgfsetfillcolor{textcolor}%
\pgftext[x=0.263086in,y=2.020719in,left,base]{\color{textcolor}\rmfamily\fontsize{8.000000}{9.600000}\selectfont \(\displaystyle 1.6\)}%
\end{pgfscope}%
\begin{pgfscope}%
\pgfsetbuttcap%
\pgfsetroundjoin%
\definecolor{currentfill}{rgb}{0.000000,0.000000,0.000000}%
\pgfsetfillcolor{currentfill}%
\pgfsetlinewidth{0.803000pt}%
\definecolor{currentstroke}{rgb}{0.000000,0.000000,0.000000}%
\pgfsetstrokecolor{currentstroke}%
\pgfsetdash{}{0pt}%
\pgfsys@defobject{currentmarker}{\pgfqpoint{-0.048611in}{0.000000in}}{\pgfqpoint{0.000000in}{0.000000in}}{%
\pgfpathmoveto{\pgfqpoint{0.000000in}{0.000000in}}%
\pgfpathlineto{\pgfqpoint{-0.048611in}{0.000000in}}%
\pgfusepath{stroke,fill}%
}%
\begin{pgfscope}%
\pgfsys@transformshift{0.511159in}{2.557513in}%
\pgfsys@useobject{currentmarker}{}%
\end{pgfscope}%
\end{pgfscope}%
\begin{pgfscope}%
\definecolor{textcolor}{rgb}{0.000000,0.000000,0.000000}%
\pgfsetstrokecolor{textcolor}%
\pgfsetfillcolor{textcolor}%
\pgftext[x=0.263086in,y=2.515304in,left,base]{\color{textcolor}\rmfamily\fontsize{8.000000}{9.600000}\selectfont \(\displaystyle 1.8\)}%
\end{pgfscope}%
\begin{pgfscope}%
\definecolor{textcolor}{rgb}{0.000000,0.000000,0.000000}%
\pgfsetstrokecolor{textcolor}%
\pgfsetfillcolor{textcolor}%
\pgftext[x=0.207530in,y=1.692525in,,bottom,rotate=90.000000]{\color{textcolor}\rmfamily\fontsize{8.000000}{9.600000}\selectfont Speedup}%
\end{pgfscope}%
\begin{pgfscope}%
\pgfpathrectangle{\pgfqpoint{0.511159in}{0.467838in}}{\pgfqpoint{3.287038in}{2.449373in}}%
\pgfusepath{clip}%
\pgfsetrectcap%
\pgfsetroundjoin%
\pgfsetlinewidth{1.505625pt}%
\definecolor{currentstroke}{rgb}{0.121569,0.466667,0.705882}%
\pgfsetstrokecolor{currentstroke}%
\pgfsetdash{}{0pt}%
\pgfpathmoveto{\pgfqpoint{0.660570in}{0.579173in}}%
\pgfpathlineto{\pgfqpoint{0.932226in}{0.820284in}}%
\pgfpathlineto{\pgfqpoint{1.203882in}{1.477721in}}%
\pgfpathlineto{\pgfqpoint{1.475538in}{1.955944in}}%
\pgfpathlineto{\pgfqpoint{1.747194in}{2.025433in}}%
\pgfpathlineto{\pgfqpoint{2.018850in}{2.265689in}}%
\pgfpathlineto{\pgfqpoint{2.290506in}{2.249813in}}%
\pgfpathlineto{\pgfqpoint{2.562162in}{2.501447in}}%
\pgfpathlineto{\pgfqpoint{2.833818in}{2.265689in}}%
\pgfpathlineto{\pgfqpoint{3.105474in}{2.483724in}}%
\pgfpathlineto{\pgfqpoint{3.377130in}{2.414241in}}%
\pgfpathlineto{\pgfqpoint{3.648786in}{2.805876in}}%
\pgfusepath{stroke}%
\end{pgfscope}%
\begin{pgfscope}%
\pgfsetrectcap%
\pgfsetmiterjoin%
\pgfsetlinewidth{0.803000pt}%
\definecolor{currentstroke}{rgb}{0.000000,0.000000,0.000000}%
\pgfsetstrokecolor{currentstroke}%
\pgfsetdash{}{0pt}%
\pgfpathmoveto{\pgfqpoint{0.511159in}{0.467838in}}%
\pgfpathlineto{\pgfqpoint{0.511159in}{2.917211in}}%
\pgfusepath{stroke}%
\end{pgfscope}%
\begin{pgfscope}%
\pgfsetrectcap%
\pgfsetmiterjoin%
\pgfsetlinewidth{0.803000pt}%
\definecolor{currentstroke}{rgb}{0.000000,0.000000,0.000000}%
\pgfsetstrokecolor{currentstroke}%
\pgfsetdash{}{0pt}%
\pgfpathmoveto{\pgfqpoint{3.798197in}{0.467838in}}%
\pgfpathlineto{\pgfqpoint{3.798197in}{2.917211in}}%
\pgfusepath{stroke}%
\end{pgfscope}%
\begin{pgfscope}%
\pgfsetrectcap%
\pgfsetmiterjoin%
\pgfsetlinewidth{0.803000pt}%
\definecolor{currentstroke}{rgb}{0.000000,0.000000,0.000000}%
\pgfsetstrokecolor{currentstroke}%
\pgfsetdash{}{0pt}%
\pgfpathmoveto{\pgfqpoint{0.511159in}{0.467838in}}%
\pgfpathlineto{\pgfqpoint{3.798197in}{0.467838in}}%
\pgfusepath{stroke}%
\end{pgfscope}%
\begin{pgfscope}%
\pgfsetrectcap%
\pgfsetmiterjoin%
\pgfsetlinewidth{0.803000pt}%
\definecolor{currentstroke}{rgb}{0.000000,0.000000,0.000000}%
\pgfsetstrokecolor{currentstroke}%
\pgfsetdash{}{0pt}%
\pgfpathmoveto{\pgfqpoint{0.511159in}{2.917211in}}%
\pgfpathlineto{\pgfqpoint{3.798197in}{2.917211in}}%
\pgfusepath{stroke}%
\end{pgfscope}%
\begin{pgfscope}%
\pgfsetrectcap%
\pgfsetroundjoin%
\pgfsetlinewidth{1.505625pt}%
\definecolor{currentstroke}{rgb}{0.121569,0.466667,0.705882}%
\pgfsetstrokecolor{currentstroke}%
\pgfsetdash{}{0pt}%
\pgfpathmoveto{\pgfqpoint{0.611159in}{2.771682in}}%
\pgfpathlineto{\pgfqpoint{0.833381in}{2.771682in}}%
\pgfusepath{stroke}%
\end{pgfscope}%
\begin{pgfscope}%
\definecolor{textcolor}{rgb}{0.000000,0.000000,0.000000}%
\pgfsetstrokecolor{textcolor}%
\pgfsetfillcolor{textcolor}%
\pgftext[x=0.922270in,y=2.732793in,left,base]{\color{textcolor}\rmfamily\fontsize{8.000000}{9.600000}\selectfont With Perple\_X}%
\end{pgfscope}%
\end{pgfpicture}%
\makeatother%
\endgroup%

        \caption{Material model plugin.}
        \label{fig:scaling_material_model}
    \end{subfigure}
    \caption{
        Speedup of the plugins compared with theoretical constraints. 
        Note that the problem size has remained fixed for each run.
    }
    \label{fig:scaling}
\end{figure}

Of primary importance to the performance of the plugins is how well they scale with the number of processors.
The greatest number of processors that was used in this work was \num{12}, but geodynamical simulations have processor counts that often run into the thousands of cores and good scaling performance is essential in order to take advantage of them.

A common metric for measuring scaling performance is the \textit{speedup} $S$. This is defined as:

\begin{equation*}
    S(p) = \frac{t(1)}{t(p)},
\end{equation*}

where $t(1)$ is the runtime for the problem on a single processor and $t(p)$ the runtime on $p$ processors. 
Naturally the optimal speedup achievable as $p$ increases is $p$ itself (i.e. doubling the number of cores should halve the compute time).
However, this is never achievable in reality as some parts of the code will always need to be run sequentially rather than in parallel.
There will also be additional overhead to consider that derives from increasing parallelism and this often scales with $p$.

Speedup curves may be predicted with Amdahl's Law~\parencite{amdahl_validity_1967}:

\begin{equation*}
    t(p) = f t(1) + (1 - f) \frac{t(1)}{p}.
\end{equation*}

It simply states that the fraction of the code that must be run sequentially, $f$, will experience no speedup and the fraction of the code that can be run in parallel, $1-f$, will experience the optimal speedup of $p$.
The `law' is a major simplification and does not account for any parallel-induced overhead, but it serves as a useful technique for modelling $f$.

The speedups for each plugin are shown in Figure~\ref{fig:scaling} and they are plotted against the curves predicted by Amdahl's Law for given values of $f$.
For the particle property plugin (Figure~\ref{fig:scaling_particle_property}) one sees that $f$ lies between \num{0.01} and \num{0.1}, which is very near optimal and means that significant speedup is observed as the number of processors increases.
Conversely, the control model setup shown in the same plot has $f$ roughly equal to \num{0.5}.

Turning our attention to the material model plugin (Figure~\ref{fig:scaling_material_model}), the speedup is much less pronounced and $f$ is shown to lie approximately between \num{0.5} and \num{0.6}.

\subsubsection{Load balancing}

\begin{figure}
    \centering
    %% Creator: Matplotlib, PGF backend
%%
%% To include the figure in your LaTeX document, write
%%   \input{<filename>.pgf}
%%
%% Make sure the required packages are loaded in your preamble
%%   \usepackage{pgf}
%%
%% Figures using additional raster images can only be included by \input if
%% they are in the same directory as the main LaTeX file. For loading figures
%% from other directories you can use the `import` package
%%   \usepackage{import}
%% and then include the figures with
%%   \import{<path to file>}{<filename>.pgf}
%%
%% Matplotlib used the following preamble
%%   \usepackage{fontspec}
%%   \setmainfont{DejaVuSerif.ttf}[Path=/home/connor/.local/lib/python3.8/site-packages/matplotlib/mpl-data/fonts/ttf/]
%%   \setsansfont{DejaVuSans.ttf}[Path=/home/connor/.local/lib/python3.8/site-packages/matplotlib/mpl-data/fonts/ttf/]
%%   \setmonofont{DejaVuSansMono.ttf}[Path=/home/connor/.local/lib/python3.8/site-packages/matplotlib/mpl-data/fonts/ttf/]
%%
\begingroup%
\makeatletter%
\begin{pgfpicture}%
\pgfpathrectangle{\pgfpointorigin}{\pgfqpoint{3.924431in}{3.046073in}}%
\pgfusepath{use as bounding box, clip}%
\begin{pgfscope}%
\pgfsetbuttcap%
\pgfsetmiterjoin%
\definecolor{currentfill}{rgb}{1.000000,1.000000,1.000000}%
\pgfsetfillcolor{currentfill}%
\pgfsetlinewidth{0.000000pt}%
\definecolor{currentstroke}{rgb}{1.000000,1.000000,1.000000}%
\pgfsetstrokecolor{currentstroke}%
\pgfsetdash{}{0pt}%
\pgfpathmoveto{\pgfqpoint{0.000000in}{0.000000in}}%
\pgfpathlineto{\pgfqpoint{3.924431in}{0.000000in}}%
\pgfpathlineto{\pgfqpoint{3.924431in}{3.046073in}}%
\pgfpathlineto{\pgfqpoint{0.000000in}{3.046073in}}%
\pgfpathclose%
\pgfusepath{fill}%
\end{pgfscope}%
\begin{pgfscope}%
\pgfsetbuttcap%
\pgfsetmiterjoin%
\definecolor{currentfill}{rgb}{1.000000,1.000000,1.000000}%
\pgfsetfillcolor{currentfill}%
\pgfsetlinewidth{0.000000pt}%
\definecolor{currentstroke}{rgb}{0.000000,0.000000,0.000000}%
\pgfsetstrokecolor{currentstroke}%
\pgfsetstrokeopacity{0.000000}%
\pgfsetdash{}{0pt}%
\pgfpathmoveto{\pgfqpoint{0.537394in}{0.484854in}}%
\pgfpathlineto{\pgfqpoint{3.824431in}{0.484854in}}%
\pgfpathlineto{\pgfqpoint{3.824431in}{2.934227in}}%
\pgfpathlineto{\pgfqpoint{0.537394in}{2.934227in}}%
\pgfpathclose%
\pgfusepath{fill}%
\end{pgfscope}%
\begin{pgfscope}%
\pgfpathrectangle{\pgfqpoint{0.537394in}{0.484854in}}{\pgfqpoint{3.287038in}{2.449373in}}%
\pgfusepath{clip}%
\pgfsetbuttcap%
\pgfsetroundjoin%
\definecolor{currentfill}{rgb}{0.121569,0.466667,0.705882}%
\pgfsetfillcolor{currentfill}%
\pgfsetfillopacity{0.300000}%
\pgfsetlinewidth{0.000000pt}%
\definecolor{currentstroke}{rgb}{0.000000,0.000000,0.000000}%
\pgfsetstrokecolor{currentstroke}%
\pgfsetdash{}{0pt}%
\pgfpathmoveto{\pgfqpoint{0.537394in}{1.567842in}}%
\pgfpathlineto{\pgfqpoint{0.537394in}{1.527356in}}%
\pgfpathlineto{\pgfqpoint{0.756530in}{1.527356in}}%
\pgfpathlineto{\pgfqpoint{0.975666in}{1.527356in}}%
\pgfpathlineto{\pgfqpoint{1.194801in}{1.527356in}}%
\pgfpathlineto{\pgfqpoint{1.413937in}{1.527356in}}%
\pgfpathlineto{\pgfqpoint{1.633073in}{1.486871in}}%
\pgfpathlineto{\pgfqpoint{1.852209in}{1.446385in}}%
\pgfpathlineto{\pgfqpoint{2.071345in}{1.324929in}}%
\pgfpathlineto{\pgfqpoint{2.217215in}{1.162987in}}%
\pgfpathlineto{\pgfqpoint{2.303424in}{1.162987in}}%
\pgfpathlineto{\pgfqpoint{2.365092in}{1.284443in}}%
\pgfpathlineto{\pgfqpoint{2.413246in}{1.243958in}}%
\pgfpathlineto{\pgfqpoint{2.452847in}{1.243958in}}%
\pgfpathlineto{\pgfqpoint{2.486549in}{1.122501in}}%
\pgfpathlineto{\pgfqpoint{2.515976in}{1.001045in}}%
\pgfpathlineto{\pgfqpoint{2.542171in}{0.960559in}}%
\pgfpathlineto{\pgfqpoint{2.565846in}{0.960559in}}%
\pgfpathlineto{\pgfqpoint{2.587505in}{0.960559in}}%
\pgfpathlineto{\pgfqpoint{2.607523in}{0.960559in}}%
\pgfpathlineto{\pgfqpoint{2.626183in}{1.162987in}}%
\pgfpathlineto{\pgfqpoint{2.643572in}{1.122501in}}%
\pgfpathlineto{\pgfqpoint{2.659840in}{1.162987in}}%
\pgfpathlineto{\pgfqpoint{2.675166in}{1.122501in}}%
\pgfpathlineto{\pgfqpoint{2.689692in}{1.122501in}}%
\pgfpathlineto{\pgfqpoint{2.703534in}{1.122501in}}%
\pgfpathlineto{\pgfqpoint{2.716788in}{1.122501in}}%
\pgfpathlineto{\pgfqpoint{2.729536in}{0.960559in}}%
\pgfpathlineto{\pgfqpoint{2.741843in}{0.879588in}}%
\pgfpathlineto{\pgfqpoint{2.753768in}{0.879588in}}%
\pgfpathlineto{\pgfqpoint{2.765360in}{0.960559in}}%
\pgfpathlineto{\pgfqpoint{2.776662in}{0.758131in}}%
\pgfpathlineto{\pgfqpoint{2.787712in}{1.122501in}}%
\pgfpathlineto{\pgfqpoint{2.798544in}{1.041530in}}%
\pgfpathlineto{\pgfqpoint{2.809184in}{0.960559in}}%
\pgfpathlineto{\pgfqpoint{2.819657in}{0.920073in}}%
\pgfpathlineto{\pgfqpoint{2.829990in}{0.839102in}}%
\pgfpathlineto{\pgfqpoint{2.840204in}{0.798617in}}%
\pgfpathlineto{\pgfqpoint{2.850323in}{0.798617in}}%
\pgfpathlineto{\pgfqpoint{2.860365in}{0.879588in}}%
\pgfpathlineto{\pgfqpoint{2.870349in}{1.041530in}}%
\pgfpathlineto{\pgfqpoint{2.880294in}{1.122501in}}%
\pgfpathlineto{\pgfqpoint{2.890216in}{1.203472in}}%
\pgfpathlineto{\pgfqpoint{2.900131in}{1.203472in}}%
\pgfpathlineto{\pgfqpoint{2.910056in}{1.324929in}}%
\pgfpathlineto{\pgfqpoint{2.920006in}{1.243958in}}%
\pgfpathlineto{\pgfqpoint{2.929996in}{1.041530in}}%
\pgfpathlineto{\pgfqpoint{2.940040in}{0.960559in}}%
\pgfpathlineto{\pgfqpoint{2.950155in}{1.162987in}}%
\pgfpathlineto{\pgfqpoint{2.960354in}{1.122501in}}%
\pgfpathlineto{\pgfqpoint{2.970653in}{1.082016in}}%
\pgfpathlineto{\pgfqpoint{2.981067in}{1.041530in}}%
\pgfpathlineto{\pgfqpoint{2.991611in}{1.082016in}}%
\pgfpathlineto{\pgfqpoint{3.002300in}{0.839102in}}%
\pgfpathlineto{\pgfqpoint{3.013152in}{0.839102in}}%
\pgfpathlineto{\pgfqpoint{3.024182in}{0.677160in}}%
\pgfpathlineto{\pgfqpoint{3.035406in}{0.717646in}}%
\pgfpathlineto{\pgfqpoint{3.046826in}{0.636675in}}%
\pgfpathlineto{\pgfqpoint{3.058134in}{0.636675in}}%
\pgfpathlineto{\pgfqpoint{3.069290in}{0.717646in}}%
\pgfpathlineto{\pgfqpoint{3.080344in}{0.798617in}}%
\pgfpathlineto{\pgfqpoint{3.091337in}{1.082016in}}%
\pgfpathlineto{\pgfqpoint{3.102311in}{1.041530in}}%
\pgfpathlineto{\pgfqpoint{3.113303in}{1.082016in}}%
\pgfpathlineto{\pgfqpoint{3.124347in}{1.082016in}}%
\pgfpathlineto{\pgfqpoint{3.135474in}{0.960559in}}%
\pgfpathlineto{\pgfqpoint{3.146717in}{0.920073in}}%
\pgfpathlineto{\pgfqpoint{3.158082in}{0.879588in}}%
\pgfpathlineto{\pgfqpoint{3.169596in}{0.758131in}}%
\pgfpathlineto{\pgfqpoint{3.181287in}{0.798617in}}%
\pgfpathlineto{\pgfqpoint{3.193182in}{0.798617in}}%
\pgfpathlineto{\pgfqpoint{3.205304in}{0.879588in}}%
\pgfpathlineto{\pgfqpoint{3.217678in}{0.879588in}}%
\pgfpathlineto{\pgfqpoint{3.230327in}{0.920073in}}%
\pgfpathlineto{\pgfqpoint{3.243273in}{0.879588in}}%
\pgfpathlineto{\pgfqpoint{3.256537in}{0.960559in}}%
\pgfpathlineto{\pgfqpoint{3.270140in}{0.798617in}}%
\pgfpathlineto{\pgfqpoint{3.284100in}{0.798617in}}%
\pgfpathlineto{\pgfqpoint{3.298435in}{0.879588in}}%
\pgfpathlineto{\pgfqpoint{3.313164in}{0.758131in}}%
\pgfpathlineto{\pgfqpoint{3.328301in}{0.596189in}}%
\pgfpathlineto{\pgfqpoint{3.343863in}{0.677160in}}%
\pgfpathlineto{\pgfqpoint{3.359865in}{0.879588in}}%
\pgfpathlineto{\pgfqpoint{3.376321in}{0.879588in}}%
\pgfpathlineto{\pgfqpoint{3.393243in}{1.001045in}}%
\pgfpathlineto{\pgfqpoint{3.410646in}{1.122501in}}%
\pgfpathlineto{\pgfqpoint{3.428543in}{1.001045in}}%
\pgfpathlineto{\pgfqpoint{3.446946in}{1.001045in}}%
\pgfpathlineto{\pgfqpoint{3.465870in}{1.082016in}}%
\pgfpathlineto{\pgfqpoint{3.485327in}{0.960559in}}%
\pgfpathlineto{\pgfqpoint{3.505319in}{0.879588in}}%
\pgfpathlineto{\pgfqpoint{3.525848in}{0.717646in}}%
\pgfpathlineto{\pgfqpoint{3.546928in}{0.717646in}}%
\pgfpathlineto{\pgfqpoint{3.568573in}{0.677160in}}%
\pgfpathlineto{\pgfqpoint{3.590798in}{0.717646in}}%
\pgfpathlineto{\pgfqpoint{3.613621in}{0.717646in}}%
\pgfpathlineto{\pgfqpoint{3.637059in}{0.879588in}}%
\pgfpathlineto{\pgfqpoint{3.661130in}{0.960559in}}%
\pgfpathlineto{\pgfqpoint{3.685855in}{0.920073in}}%
\pgfpathlineto{\pgfqpoint{3.711237in}{0.879588in}}%
\pgfpathlineto{\pgfqpoint{3.737282in}{0.920073in}}%
\pgfpathlineto{\pgfqpoint{3.764011in}{0.839102in}}%
\pgfpathlineto{\pgfqpoint{3.791445in}{0.717646in}}%
\pgfpathlineto{\pgfqpoint{3.819606in}{0.596189in}}%
\pgfpathlineto{\pgfqpoint{3.824431in}{0.596189in}}%
\pgfpathlineto{\pgfqpoint{3.824431in}{2.620465in}}%
\pgfpathlineto{\pgfqpoint{3.824431in}{2.620465in}}%
\pgfpathlineto{\pgfqpoint{3.819606in}{2.579979in}}%
\pgfpathlineto{\pgfqpoint{3.791445in}{2.458523in}}%
\pgfpathlineto{\pgfqpoint{3.764011in}{2.418037in}}%
\pgfpathlineto{\pgfqpoint{3.737282in}{2.377552in}}%
\pgfpathlineto{\pgfqpoint{3.711237in}{2.499008in}}%
\pgfpathlineto{\pgfqpoint{3.685855in}{2.579979in}}%
\pgfpathlineto{\pgfqpoint{3.661130in}{2.822892in}}%
\pgfpathlineto{\pgfqpoint{3.637059in}{2.822892in}}%
\pgfpathlineto{\pgfqpoint{3.613621in}{2.458523in}}%
\pgfpathlineto{\pgfqpoint{3.590798in}{2.337066in}}%
\pgfpathlineto{\pgfqpoint{3.568573in}{2.256095in}}%
\pgfpathlineto{\pgfqpoint{3.546928in}{2.013182in}}%
\pgfpathlineto{\pgfqpoint{3.525848in}{2.053668in}}%
\pgfpathlineto{\pgfqpoint{3.505319in}{2.094153in}}%
\pgfpathlineto{\pgfqpoint{3.485327in}{1.972697in}}%
\pgfpathlineto{\pgfqpoint{3.465870in}{2.013182in}}%
\pgfpathlineto{\pgfqpoint{3.446946in}{2.013182in}}%
\pgfpathlineto{\pgfqpoint{3.428543in}{2.215610in}}%
\pgfpathlineto{\pgfqpoint{3.410646in}{2.215610in}}%
\pgfpathlineto{\pgfqpoint{3.393243in}{2.134639in}}%
\pgfpathlineto{\pgfqpoint{3.376321in}{2.094153in}}%
\pgfpathlineto{\pgfqpoint{3.359865in}{2.256095in}}%
\pgfpathlineto{\pgfqpoint{3.343863in}{2.134639in}}%
\pgfpathlineto{\pgfqpoint{3.328301in}{2.215610in}}%
\pgfpathlineto{\pgfqpoint{3.313164in}{2.134639in}}%
\pgfpathlineto{\pgfqpoint{3.298435in}{2.013182in}}%
\pgfpathlineto{\pgfqpoint{3.284100in}{2.134639in}}%
\pgfpathlineto{\pgfqpoint{3.270140in}{2.053668in}}%
\pgfpathlineto{\pgfqpoint{3.256537in}{2.175124in}}%
\pgfpathlineto{\pgfqpoint{3.243273in}{2.215610in}}%
\pgfpathlineto{\pgfqpoint{3.230327in}{2.175124in}}%
\pgfpathlineto{\pgfqpoint{3.217678in}{2.296581in}}%
\pgfpathlineto{\pgfqpoint{3.205304in}{2.296581in}}%
\pgfpathlineto{\pgfqpoint{3.193182in}{2.134639in}}%
\pgfpathlineto{\pgfqpoint{3.181287in}{2.094153in}}%
\pgfpathlineto{\pgfqpoint{3.169596in}{2.094153in}}%
\pgfpathlineto{\pgfqpoint{3.158082in}{1.972697in}}%
\pgfpathlineto{\pgfqpoint{3.146717in}{1.851240in}}%
\pgfpathlineto{\pgfqpoint{3.135474in}{1.891726in}}%
\pgfpathlineto{\pgfqpoint{3.124347in}{1.851240in}}%
\pgfpathlineto{\pgfqpoint{3.113303in}{2.053668in}}%
\pgfpathlineto{\pgfqpoint{3.102311in}{2.094153in}}%
\pgfpathlineto{\pgfqpoint{3.091337in}{2.094153in}}%
\pgfpathlineto{\pgfqpoint{3.080344in}{2.094153in}}%
\pgfpathlineto{\pgfqpoint{3.069290in}{2.094153in}}%
\pgfpathlineto{\pgfqpoint{3.058134in}{2.215610in}}%
\pgfpathlineto{\pgfqpoint{3.046826in}{2.296581in}}%
\pgfpathlineto{\pgfqpoint{3.035406in}{2.134639in}}%
\pgfpathlineto{\pgfqpoint{3.024182in}{2.215610in}}%
\pgfpathlineto{\pgfqpoint{3.013152in}{1.972697in}}%
\pgfpathlineto{\pgfqpoint{3.002300in}{1.972697in}}%
\pgfpathlineto{\pgfqpoint{2.991611in}{2.134639in}}%
\pgfpathlineto{\pgfqpoint{2.981067in}{2.215610in}}%
\pgfpathlineto{\pgfqpoint{2.970653in}{2.013182in}}%
\pgfpathlineto{\pgfqpoint{2.960354in}{2.013182in}}%
\pgfpathlineto{\pgfqpoint{2.950155in}{2.013182in}}%
\pgfpathlineto{\pgfqpoint{2.940040in}{1.932211in}}%
\pgfpathlineto{\pgfqpoint{2.929996in}{1.932211in}}%
\pgfpathlineto{\pgfqpoint{2.920006in}{1.851240in}}%
\pgfpathlineto{\pgfqpoint{2.910056in}{2.013182in}}%
\pgfpathlineto{\pgfqpoint{2.900131in}{2.053668in}}%
\pgfpathlineto{\pgfqpoint{2.890216in}{2.053668in}}%
\pgfpathlineto{\pgfqpoint{2.880294in}{2.094153in}}%
\pgfpathlineto{\pgfqpoint{2.870349in}{2.175124in}}%
\pgfpathlineto{\pgfqpoint{2.860365in}{2.296581in}}%
\pgfpathlineto{\pgfqpoint{2.850323in}{2.377552in}}%
\pgfpathlineto{\pgfqpoint{2.840204in}{2.256095in}}%
\pgfpathlineto{\pgfqpoint{2.829990in}{2.256095in}}%
\pgfpathlineto{\pgfqpoint{2.819657in}{2.175124in}}%
\pgfpathlineto{\pgfqpoint{2.809184in}{2.418037in}}%
\pgfpathlineto{\pgfqpoint{2.798544in}{2.296581in}}%
\pgfpathlineto{\pgfqpoint{2.787712in}{2.175124in}}%
\pgfpathlineto{\pgfqpoint{2.776662in}{2.134639in}}%
\pgfpathlineto{\pgfqpoint{2.765360in}{2.175124in}}%
\pgfpathlineto{\pgfqpoint{2.753768in}{2.296581in}}%
\pgfpathlineto{\pgfqpoint{2.741843in}{2.215610in}}%
\pgfpathlineto{\pgfqpoint{2.729536in}{2.134639in}}%
\pgfpathlineto{\pgfqpoint{2.716788in}{2.175124in}}%
\pgfpathlineto{\pgfqpoint{2.703534in}{2.175124in}}%
\pgfpathlineto{\pgfqpoint{2.689692in}{2.215610in}}%
\pgfpathlineto{\pgfqpoint{2.675166in}{2.215610in}}%
\pgfpathlineto{\pgfqpoint{2.659840in}{2.215610in}}%
\pgfpathlineto{\pgfqpoint{2.643572in}{2.134639in}}%
\pgfpathlineto{\pgfqpoint{2.626183in}{2.134639in}}%
\pgfpathlineto{\pgfqpoint{2.607523in}{2.215610in}}%
\pgfpathlineto{\pgfqpoint{2.587505in}{2.134639in}}%
\pgfpathlineto{\pgfqpoint{2.565846in}{2.053668in}}%
\pgfpathlineto{\pgfqpoint{2.542171in}{2.013182in}}%
\pgfpathlineto{\pgfqpoint{2.515976in}{2.094153in}}%
\pgfpathlineto{\pgfqpoint{2.486549in}{2.053668in}}%
\pgfpathlineto{\pgfqpoint{2.452847in}{1.972697in}}%
\pgfpathlineto{\pgfqpoint{2.413246in}{1.770269in}}%
\pgfpathlineto{\pgfqpoint{2.365092in}{1.729784in}}%
\pgfpathlineto{\pgfqpoint{2.303424in}{1.810755in}}%
\pgfpathlineto{\pgfqpoint{2.217215in}{1.770269in}}%
\pgfpathlineto{\pgfqpoint{2.071345in}{1.729784in}}%
\pgfpathlineto{\pgfqpoint{1.852209in}{1.648813in}}%
\pgfpathlineto{\pgfqpoint{1.633073in}{1.689298in}}%
\pgfpathlineto{\pgfqpoint{1.413937in}{1.608327in}}%
\pgfpathlineto{\pgfqpoint{1.194801in}{1.608327in}}%
\pgfpathlineto{\pgfqpoint{0.975666in}{1.608327in}}%
\pgfpathlineto{\pgfqpoint{0.756530in}{1.567842in}}%
\pgfpathlineto{\pgfqpoint{0.537394in}{1.567842in}}%
\pgfpathclose%
\pgfusepath{fill}%
\end{pgfscope}%
\begin{pgfscope}%
\pgfsetbuttcap%
\pgfsetroundjoin%
\definecolor{currentfill}{rgb}{0.000000,0.000000,0.000000}%
\pgfsetfillcolor{currentfill}%
\pgfsetlinewidth{0.803000pt}%
\definecolor{currentstroke}{rgb}{0.000000,0.000000,0.000000}%
\pgfsetstrokecolor{currentstroke}%
\pgfsetdash{}{0pt}%
\pgfsys@defobject{currentmarker}{\pgfqpoint{0.000000in}{-0.048611in}}{\pgfqpoint{0.000000in}{0.000000in}}{%
\pgfpathmoveto{\pgfqpoint{0.000000in}{0.000000in}}%
\pgfpathlineto{\pgfqpoint{0.000000in}{-0.048611in}}%
\pgfusepath{stroke,fill}%
}%
\begin{pgfscope}%
\pgfsys@transformshift{0.537394in}{0.484854in}%
\pgfsys@useobject{currentmarker}{}%
\end{pgfscope}%
\end{pgfscope}%
\begin{pgfscope}%
\definecolor{textcolor}{rgb}{0.000000,0.000000,0.000000}%
\pgfsetstrokecolor{textcolor}%
\pgfsetfillcolor{textcolor}%
\pgftext[x=0.537394in,y=0.387632in,,top]{\color{textcolor}\rmfamily\fontsize{8.000000}{9.600000}\selectfont \(\displaystyle 0.0\)}%
\end{pgfscope}%
\begin{pgfscope}%
\pgfsetbuttcap%
\pgfsetroundjoin%
\definecolor{currentfill}{rgb}{0.000000,0.000000,0.000000}%
\pgfsetfillcolor{currentfill}%
\pgfsetlinewidth{0.803000pt}%
\definecolor{currentstroke}{rgb}{0.000000,0.000000,0.000000}%
\pgfsetstrokecolor{currentstroke}%
\pgfsetdash{}{0pt}%
\pgfsys@defobject{currentmarker}{\pgfqpoint{0.000000in}{-0.048611in}}{\pgfqpoint{0.000000in}{0.000000in}}{%
\pgfpathmoveto{\pgfqpoint{0.000000in}{0.000000in}}%
\pgfpathlineto{\pgfqpoint{0.000000in}{-0.048611in}}%
\pgfusepath{stroke,fill}%
}%
\begin{pgfscope}%
\pgfsys@transformshift{0.975666in}{0.484854in}%
\pgfsys@useobject{currentmarker}{}%
\end{pgfscope}%
\end{pgfscope}%
\begin{pgfscope}%
\definecolor{textcolor}{rgb}{0.000000,0.000000,0.000000}%
\pgfsetstrokecolor{textcolor}%
\pgfsetfillcolor{textcolor}%
\pgftext[x=0.975666in,y=0.387632in,,top]{\color{textcolor}\rmfamily\fontsize{8.000000}{9.600000}\selectfont \(\displaystyle 0.2\)}%
\end{pgfscope}%
\begin{pgfscope}%
\pgfsetbuttcap%
\pgfsetroundjoin%
\definecolor{currentfill}{rgb}{0.000000,0.000000,0.000000}%
\pgfsetfillcolor{currentfill}%
\pgfsetlinewidth{0.803000pt}%
\definecolor{currentstroke}{rgb}{0.000000,0.000000,0.000000}%
\pgfsetstrokecolor{currentstroke}%
\pgfsetdash{}{0pt}%
\pgfsys@defobject{currentmarker}{\pgfqpoint{0.000000in}{-0.048611in}}{\pgfqpoint{0.000000in}{0.000000in}}{%
\pgfpathmoveto{\pgfqpoint{0.000000in}{0.000000in}}%
\pgfpathlineto{\pgfqpoint{0.000000in}{-0.048611in}}%
\pgfusepath{stroke,fill}%
}%
\begin{pgfscope}%
\pgfsys@transformshift{1.413937in}{0.484854in}%
\pgfsys@useobject{currentmarker}{}%
\end{pgfscope}%
\end{pgfscope}%
\begin{pgfscope}%
\definecolor{textcolor}{rgb}{0.000000,0.000000,0.000000}%
\pgfsetstrokecolor{textcolor}%
\pgfsetfillcolor{textcolor}%
\pgftext[x=1.413937in,y=0.387632in,,top]{\color{textcolor}\rmfamily\fontsize{8.000000}{9.600000}\selectfont \(\displaystyle 0.4\)}%
\end{pgfscope}%
\begin{pgfscope}%
\pgfsetbuttcap%
\pgfsetroundjoin%
\definecolor{currentfill}{rgb}{0.000000,0.000000,0.000000}%
\pgfsetfillcolor{currentfill}%
\pgfsetlinewidth{0.803000pt}%
\definecolor{currentstroke}{rgb}{0.000000,0.000000,0.000000}%
\pgfsetstrokecolor{currentstroke}%
\pgfsetdash{}{0pt}%
\pgfsys@defobject{currentmarker}{\pgfqpoint{0.000000in}{-0.048611in}}{\pgfqpoint{0.000000in}{0.000000in}}{%
\pgfpathmoveto{\pgfqpoint{0.000000in}{0.000000in}}%
\pgfpathlineto{\pgfqpoint{0.000000in}{-0.048611in}}%
\pgfusepath{stroke,fill}%
}%
\begin{pgfscope}%
\pgfsys@transformshift{1.852209in}{0.484854in}%
\pgfsys@useobject{currentmarker}{}%
\end{pgfscope}%
\end{pgfscope}%
\begin{pgfscope}%
\definecolor{textcolor}{rgb}{0.000000,0.000000,0.000000}%
\pgfsetstrokecolor{textcolor}%
\pgfsetfillcolor{textcolor}%
\pgftext[x=1.852209in,y=0.387632in,,top]{\color{textcolor}\rmfamily\fontsize{8.000000}{9.600000}\selectfont \(\displaystyle 0.6\)}%
\end{pgfscope}%
\begin{pgfscope}%
\pgfsetbuttcap%
\pgfsetroundjoin%
\definecolor{currentfill}{rgb}{0.000000,0.000000,0.000000}%
\pgfsetfillcolor{currentfill}%
\pgfsetlinewidth{0.803000pt}%
\definecolor{currentstroke}{rgb}{0.000000,0.000000,0.000000}%
\pgfsetstrokecolor{currentstroke}%
\pgfsetdash{}{0pt}%
\pgfsys@defobject{currentmarker}{\pgfqpoint{0.000000in}{-0.048611in}}{\pgfqpoint{0.000000in}{0.000000in}}{%
\pgfpathmoveto{\pgfqpoint{0.000000in}{0.000000in}}%
\pgfpathlineto{\pgfqpoint{0.000000in}{-0.048611in}}%
\pgfusepath{stroke,fill}%
}%
\begin{pgfscope}%
\pgfsys@transformshift{2.290481in}{0.484854in}%
\pgfsys@useobject{currentmarker}{}%
\end{pgfscope}%
\end{pgfscope}%
\begin{pgfscope}%
\definecolor{textcolor}{rgb}{0.000000,0.000000,0.000000}%
\pgfsetstrokecolor{textcolor}%
\pgfsetfillcolor{textcolor}%
\pgftext[x=2.290481in,y=0.387632in,,top]{\color{textcolor}\rmfamily\fontsize{8.000000}{9.600000}\selectfont \(\displaystyle 0.8\)}%
\end{pgfscope}%
\begin{pgfscope}%
\pgfsetbuttcap%
\pgfsetroundjoin%
\definecolor{currentfill}{rgb}{0.000000,0.000000,0.000000}%
\pgfsetfillcolor{currentfill}%
\pgfsetlinewidth{0.803000pt}%
\definecolor{currentstroke}{rgb}{0.000000,0.000000,0.000000}%
\pgfsetstrokecolor{currentstroke}%
\pgfsetdash{}{0pt}%
\pgfsys@defobject{currentmarker}{\pgfqpoint{0.000000in}{-0.048611in}}{\pgfqpoint{0.000000in}{0.000000in}}{%
\pgfpathmoveto{\pgfqpoint{0.000000in}{0.000000in}}%
\pgfpathlineto{\pgfqpoint{0.000000in}{-0.048611in}}%
\pgfusepath{stroke,fill}%
}%
\begin{pgfscope}%
\pgfsys@transformshift{2.728752in}{0.484854in}%
\pgfsys@useobject{currentmarker}{}%
\end{pgfscope}%
\end{pgfscope}%
\begin{pgfscope}%
\definecolor{textcolor}{rgb}{0.000000,0.000000,0.000000}%
\pgfsetstrokecolor{textcolor}%
\pgfsetfillcolor{textcolor}%
\pgftext[x=2.728752in,y=0.387632in,,top]{\color{textcolor}\rmfamily\fontsize{8.000000}{9.600000}\selectfont \(\displaystyle 1.0\)}%
\end{pgfscope}%
\begin{pgfscope}%
\pgfsetbuttcap%
\pgfsetroundjoin%
\definecolor{currentfill}{rgb}{0.000000,0.000000,0.000000}%
\pgfsetfillcolor{currentfill}%
\pgfsetlinewidth{0.803000pt}%
\definecolor{currentstroke}{rgb}{0.000000,0.000000,0.000000}%
\pgfsetstrokecolor{currentstroke}%
\pgfsetdash{}{0pt}%
\pgfsys@defobject{currentmarker}{\pgfqpoint{0.000000in}{-0.048611in}}{\pgfqpoint{0.000000in}{0.000000in}}{%
\pgfpathmoveto{\pgfqpoint{0.000000in}{0.000000in}}%
\pgfpathlineto{\pgfqpoint{0.000000in}{-0.048611in}}%
\pgfusepath{stroke,fill}%
}%
\begin{pgfscope}%
\pgfsys@transformshift{3.167024in}{0.484854in}%
\pgfsys@useobject{currentmarker}{}%
\end{pgfscope}%
\end{pgfscope}%
\begin{pgfscope}%
\definecolor{textcolor}{rgb}{0.000000,0.000000,0.000000}%
\pgfsetstrokecolor{textcolor}%
\pgfsetfillcolor{textcolor}%
\pgftext[x=3.167024in,y=0.387632in,,top]{\color{textcolor}\rmfamily\fontsize{8.000000}{9.600000}\selectfont \(\displaystyle 1.2\)}%
\end{pgfscope}%
\begin{pgfscope}%
\pgfsetbuttcap%
\pgfsetroundjoin%
\definecolor{currentfill}{rgb}{0.000000,0.000000,0.000000}%
\pgfsetfillcolor{currentfill}%
\pgfsetlinewidth{0.803000pt}%
\definecolor{currentstroke}{rgb}{0.000000,0.000000,0.000000}%
\pgfsetstrokecolor{currentstroke}%
\pgfsetdash{}{0pt}%
\pgfsys@defobject{currentmarker}{\pgfqpoint{0.000000in}{-0.048611in}}{\pgfqpoint{0.000000in}{0.000000in}}{%
\pgfpathmoveto{\pgfqpoint{0.000000in}{0.000000in}}%
\pgfpathlineto{\pgfqpoint{0.000000in}{-0.048611in}}%
\pgfusepath{stroke,fill}%
}%
\begin{pgfscope}%
\pgfsys@transformshift{3.605296in}{0.484854in}%
\pgfsys@useobject{currentmarker}{}%
\end{pgfscope}%
\end{pgfscope}%
\begin{pgfscope}%
\definecolor{textcolor}{rgb}{0.000000,0.000000,0.000000}%
\pgfsetstrokecolor{textcolor}%
\pgfsetfillcolor{textcolor}%
\pgftext[x=3.605296in,y=0.387632in,,top]{\color{textcolor}\rmfamily\fontsize{8.000000}{9.600000}\selectfont \(\displaystyle 1.4\)}%
\end{pgfscope}%
\begin{pgfscope}%
\definecolor{textcolor}{rgb}{0.000000,0.000000,0.000000}%
\pgfsetstrokecolor{textcolor}%
\pgfsetfillcolor{textcolor}%
\pgftext[x=2.180913in,y=0.224546in,,top]{\color{textcolor}\rmfamily\fontsize{8.000000}{9.600000}\selectfont Time (\(\displaystyle \times 10^6 \, \mathrm{yr}\))}%
\end{pgfscope}%
\begin{pgfscope}%
\pgfsetbuttcap%
\pgfsetroundjoin%
\definecolor{currentfill}{rgb}{0.000000,0.000000,0.000000}%
\pgfsetfillcolor{currentfill}%
\pgfsetlinewidth{0.803000pt}%
\definecolor{currentstroke}{rgb}{0.000000,0.000000,0.000000}%
\pgfsetstrokecolor{currentstroke}%
\pgfsetdash{}{0pt}%
\pgfsys@defobject{currentmarker}{\pgfqpoint{-0.048611in}{0.000000in}}{\pgfqpoint{0.000000in}{0.000000in}}{%
\pgfpathmoveto{\pgfqpoint{0.000000in}{0.000000in}}%
\pgfpathlineto{\pgfqpoint{-0.048611in}{0.000000in}}%
\pgfusepath{stroke,fill}%
}%
\begin{pgfscope}%
\pgfsys@transformshift{0.537394in}{0.879588in}%
\pgfsys@useobject{currentmarker}{}%
\end{pgfscope}%
\end{pgfscope}%
\begin{pgfscope}%
\definecolor{textcolor}{rgb}{0.000000,0.000000,0.000000}%
\pgfsetstrokecolor{textcolor}%
\pgfsetfillcolor{textcolor}%
\pgftext[x=0.263086in,y=0.837379in,left,base]{\color{textcolor}\rmfamily\fontsize{8.000000}{9.600000}\selectfont \(\displaystyle 150\)}%
\end{pgfscope}%
\begin{pgfscope}%
\pgfsetbuttcap%
\pgfsetroundjoin%
\definecolor{currentfill}{rgb}{0.000000,0.000000,0.000000}%
\pgfsetfillcolor{currentfill}%
\pgfsetlinewidth{0.803000pt}%
\definecolor{currentstroke}{rgb}{0.000000,0.000000,0.000000}%
\pgfsetstrokecolor{currentstroke}%
\pgfsetdash{}{0pt}%
\pgfsys@defobject{currentmarker}{\pgfqpoint{-0.048611in}{0.000000in}}{\pgfqpoint{0.000000in}{0.000000in}}{%
\pgfpathmoveto{\pgfqpoint{0.000000in}{0.000000in}}%
\pgfpathlineto{\pgfqpoint{-0.048611in}{0.000000in}}%
\pgfusepath{stroke,fill}%
}%
\begin{pgfscope}%
\pgfsys@transformshift{0.537394in}{1.284443in}%
\pgfsys@useobject{currentmarker}{}%
\end{pgfscope}%
\end{pgfscope}%
\begin{pgfscope}%
\definecolor{textcolor}{rgb}{0.000000,0.000000,0.000000}%
\pgfsetstrokecolor{textcolor}%
\pgfsetfillcolor{textcolor}%
\pgftext[x=0.263086in,y=1.242234in,left,base]{\color{textcolor}\rmfamily\fontsize{8.000000}{9.600000}\selectfont \(\displaystyle 160\)}%
\end{pgfscope}%
\begin{pgfscope}%
\pgfsetbuttcap%
\pgfsetroundjoin%
\definecolor{currentfill}{rgb}{0.000000,0.000000,0.000000}%
\pgfsetfillcolor{currentfill}%
\pgfsetlinewidth{0.803000pt}%
\definecolor{currentstroke}{rgb}{0.000000,0.000000,0.000000}%
\pgfsetstrokecolor{currentstroke}%
\pgfsetdash{}{0pt}%
\pgfsys@defobject{currentmarker}{\pgfqpoint{-0.048611in}{0.000000in}}{\pgfqpoint{0.000000in}{0.000000in}}{%
\pgfpathmoveto{\pgfqpoint{0.000000in}{0.000000in}}%
\pgfpathlineto{\pgfqpoint{-0.048611in}{0.000000in}}%
\pgfusepath{stroke,fill}%
}%
\begin{pgfscope}%
\pgfsys@transformshift{0.537394in}{1.689298in}%
\pgfsys@useobject{currentmarker}{}%
\end{pgfscope}%
\end{pgfscope}%
\begin{pgfscope}%
\definecolor{textcolor}{rgb}{0.000000,0.000000,0.000000}%
\pgfsetstrokecolor{textcolor}%
\pgfsetfillcolor{textcolor}%
\pgftext[x=0.263086in,y=1.647089in,left,base]{\color{textcolor}\rmfamily\fontsize{8.000000}{9.600000}\selectfont \(\displaystyle 170\)}%
\end{pgfscope}%
\begin{pgfscope}%
\pgfsetbuttcap%
\pgfsetroundjoin%
\definecolor{currentfill}{rgb}{0.000000,0.000000,0.000000}%
\pgfsetfillcolor{currentfill}%
\pgfsetlinewidth{0.803000pt}%
\definecolor{currentstroke}{rgb}{0.000000,0.000000,0.000000}%
\pgfsetstrokecolor{currentstroke}%
\pgfsetdash{}{0pt}%
\pgfsys@defobject{currentmarker}{\pgfqpoint{-0.048611in}{0.000000in}}{\pgfqpoint{0.000000in}{0.000000in}}{%
\pgfpathmoveto{\pgfqpoint{0.000000in}{0.000000in}}%
\pgfpathlineto{\pgfqpoint{-0.048611in}{0.000000in}}%
\pgfusepath{stroke,fill}%
}%
\begin{pgfscope}%
\pgfsys@transformshift{0.537394in}{2.094153in}%
\pgfsys@useobject{currentmarker}{}%
\end{pgfscope}%
\end{pgfscope}%
\begin{pgfscope}%
\definecolor{textcolor}{rgb}{0.000000,0.000000,0.000000}%
\pgfsetstrokecolor{textcolor}%
\pgfsetfillcolor{textcolor}%
\pgftext[x=0.263086in,y=2.051944in,left,base]{\color{textcolor}\rmfamily\fontsize{8.000000}{9.600000}\selectfont \(\displaystyle 180\)}%
\end{pgfscope}%
\begin{pgfscope}%
\pgfsetbuttcap%
\pgfsetroundjoin%
\definecolor{currentfill}{rgb}{0.000000,0.000000,0.000000}%
\pgfsetfillcolor{currentfill}%
\pgfsetlinewidth{0.803000pt}%
\definecolor{currentstroke}{rgb}{0.000000,0.000000,0.000000}%
\pgfsetstrokecolor{currentstroke}%
\pgfsetdash{}{0pt}%
\pgfsys@defobject{currentmarker}{\pgfqpoint{-0.048611in}{0.000000in}}{\pgfqpoint{0.000000in}{0.000000in}}{%
\pgfpathmoveto{\pgfqpoint{0.000000in}{0.000000in}}%
\pgfpathlineto{\pgfqpoint{-0.048611in}{0.000000in}}%
\pgfusepath{stroke,fill}%
}%
\begin{pgfscope}%
\pgfsys@transformshift{0.537394in}{2.499008in}%
\pgfsys@useobject{currentmarker}{}%
\end{pgfscope}%
\end{pgfscope}%
\begin{pgfscope}%
\definecolor{textcolor}{rgb}{0.000000,0.000000,0.000000}%
\pgfsetstrokecolor{textcolor}%
\pgfsetfillcolor{textcolor}%
\pgftext[x=0.263086in,y=2.456799in,left,base]{\color{textcolor}\rmfamily\fontsize{8.000000}{9.600000}\selectfont \(\displaystyle 190\)}%
\end{pgfscope}%
\begin{pgfscope}%
\pgfsetbuttcap%
\pgfsetroundjoin%
\definecolor{currentfill}{rgb}{0.000000,0.000000,0.000000}%
\pgfsetfillcolor{currentfill}%
\pgfsetlinewidth{0.803000pt}%
\definecolor{currentstroke}{rgb}{0.000000,0.000000,0.000000}%
\pgfsetstrokecolor{currentstroke}%
\pgfsetdash{}{0pt}%
\pgfsys@defobject{currentmarker}{\pgfqpoint{-0.048611in}{0.000000in}}{\pgfqpoint{0.000000in}{0.000000in}}{%
\pgfpathmoveto{\pgfqpoint{0.000000in}{0.000000in}}%
\pgfpathlineto{\pgfqpoint{-0.048611in}{0.000000in}}%
\pgfusepath{stroke,fill}%
}%
\begin{pgfscope}%
\pgfsys@transformshift{0.537394in}{2.903863in}%
\pgfsys@useobject{currentmarker}{}%
\end{pgfscope}%
\end{pgfscope}%
\begin{pgfscope}%
\definecolor{textcolor}{rgb}{0.000000,0.000000,0.000000}%
\pgfsetstrokecolor{textcolor}%
\pgfsetfillcolor{textcolor}%
\pgftext[x=0.263086in,y=2.861654in,left,base]{\color{textcolor}\rmfamily\fontsize{8.000000}{9.600000}\selectfont \(\displaystyle 200\)}%
\end{pgfscope}%
\begin{pgfscope}%
\definecolor{textcolor}{rgb}{0.000000,0.000000,0.000000}%
\pgfsetstrokecolor{textcolor}%
\pgfsetfillcolor{textcolor}%
\pgftext[x=0.207530in,y=1.709541in,,bottom,rotate=90.000000]{\color{textcolor}\rmfamily\fontsize{8.000000}{9.600000}\selectfont Number of particles}%
\end{pgfscope}%
\begin{pgfscope}%
\pgfpathrectangle{\pgfqpoint{0.537394in}{0.484854in}}{\pgfqpoint{3.287038in}{2.449373in}}%
\pgfusepath{clip}%
\pgfsetrectcap%
\pgfsetroundjoin%
\pgfsetlinewidth{1.505625pt}%
\definecolor{currentstroke}{rgb}{0.121569,0.466667,0.705882}%
\pgfsetstrokecolor{currentstroke}%
\pgfsetdash{}{0pt}%
\pgfpathmoveto{\pgfqpoint{0.537394in}{1.554348in}}%
\pgfpathlineto{\pgfqpoint{0.756530in}{1.554348in}}%
\pgfpathlineto{\pgfqpoint{0.975666in}{1.554348in}}%
\pgfpathlineto{\pgfqpoint{1.194801in}{1.554348in}}%
\pgfpathlineto{\pgfqpoint{1.413937in}{1.554348in}}%
\pgfpathlineto{\pgfqpoint{1.633073in}{1.554348in}}%
\pgfpathlineto{\pgfqpoint{1.852209in}{1.554348in}}%
\pgfpathlineto{\pgfqpoint{2.071345in}{1.554348in}}%
\pgfpathlineto{\pgfqpoint{2.217215in}{1.554348in}}%
\pgfpathlineto{\pgfqpoint{2.303424in}{1.554348in}}%
\pgfpathlineto{\pgfqpoint{2.365092in}{1.554348in}}%
\pgfpathlineto{\pgfqpoint{2.413246in}{1.554348in}}%
\pgfpathlineto{\pgfqpoint{2.452847in}{1.554348in}}%
\pgfpathlineto{\pgfqpoint{2.486549in}{1.554348in}}%
\pgfpathlineto{\pgfqpoint{2.515976in}{1.554348in}}%
\pgfpathlineto{\pgfqpoint{2.542171in}{1.554348in}}%
\pgfpathlineto{\pgfqpoint{2.565846in}{1.554348in}}%
\pgfpathlineto{\pgfqpoint{2.587505in}{1.554348in}}%
\pgfpathlineto{\pgfqpoint{2.607523in}{1.554348in}}%
\pgfpathlineto{\pgfqpoint{2.626183in}{1.554348in}}%
\pgfpathlineto{\pgfqpoint{2.643572in}{1.554348in}}%
\pgfpathlineto{\pgfqpoint{2.659840in}{1.554348in}}%
\pgfpathlineto{\pgfqpoint{2.675166in}{1.554348in}}%
\pgfpathlineto{\pgfqpoint{2.689692in}{1.554348in}}%
\pgfpathlineto{\pgfqpoint{2.703534in}{1.554348in}}%
\pgfpathlineto{\pgfqpoint{2.716788in}{1.554348in}}%
\pgfpathlineto{\pgfqpoint{2.729536in}{1.554348in}}%
\pgfpathlineto{\pgfqpoint{2.741843in}{1.554348in}}%
\pgfpathlineto{\pgfqpoint{2.753768in}{1.554348in}}%
\pgfpathlineto{\pgfqpoint{2.765360in}{1.554348in}}%
\pgfpathlineto{\pgfqpoint{2.776662in}{1.554348in}}%
\pgfpathlineto{\pgfqpoint{2.787712in}{1.554348in}}%
\pgfpathlineto{\pgfqpoint{2.798544in}{1.554348in}}%
\pgfpathlineto{\pgfqpoint{2.809184in}{1.554348in}}%
\pgfpathlineto{\pgfqpoint{2.819657in}{1.554348in}}%
\pgfpathlineto{\pgfqpoint{2.829990in}{1.554348in}}%
\pgfpathlineto{\pgfqpoint{2.840204in}{1.554348in}}%
\pgfpathlineto{\pgfqpoint{2.850323in}{1.554348in}}%
\pgfpathlineto{\pgfqpoint{2.860365in}{1.554348in}}%
\pgfpathlineto{\pgfqpoint{2.870349in}{1.554348in}}%
\pgfpathlineto{\pgfqpoint{2.880294in}{1.554348in}}%
\pgfpathlineto{\pgfqpoint{2.890216in}{1.554348in}}%
\pgfpathlineto{\pgfqpoint{2.900131in}{1.554348in}}%
\pgfpathlineto{\pgfqpoint{2.910056in}{1.554348in}}%
\pgfpathlineto{\pgfqpoint{2.920006in}{1.554348in}}%
\pgfpathlineto{\pgfqpoint{2.929996in}{1.554348in}}%
\pgfpathlineto{\pgfqpoint{2.940040in}{1.554348in}}%
\pgfpathlineto{\pgfqpoint{2.950155in}{1.554348in}}%
\pgfpathlineto{\pgfqpoint{2.960354in}{1.554348in}}%
\pgfpathlineto{\pgfqpoint{2.970653in}{1.554348in}}%
\pgfpathlineto{\pgfqpoint{2.981067in}{1.554348in}}%
\pgfpathlineto{\pgfqpoint{2.991611in}{1.554348in}}%
\pgfpathlineto{\pgfqpoint{3.002300in}{1.554348in}}%
\pgfpathlineto{\pgfqpoint{3.013152in}{1.554348in}}%
\pgfpathlineto{\pgfqpoint{3.024182in}{1.554348in}}%
\pgfpathlineto{\pgfqpoint{3.035406in}{1.554348in}}%
\pgfpathlineto{\pgfqpoint{3.046826in}{1.554348in}}%
\pgfpathlineto{\pgfqpoint{3.058134in}{1.554348in}}%
\pgfpathlineto{\pgfqpoint{3.069290in}{1.554348in}}%
\pgfpathlineto{\pgfqpoint{3.080344in}{1.554348in}}%
\pgfpathlineto{\pgfqpoint{3.091337in}{1.554348in}}%
\pgfpathlineto{\pgfqpoint{3.102311in}{1.554348in}}%
\pgfpathlineto{\pgfqpoint{3.113303in}{1.554348in}}%
\pgfpathlineto{\pgfqpoint{3.124347in}{1.554348in}}%
\pgfpathlineto{\pgfqpoint{3.135474in}{1.554348in}}%
\pgfpathlineto{\pgfqpoint{3.146717in}{1.554348in}}%
\pgfpathlineto{\pgfqpoint{3.158082in}{1.554348in}}%
\pgfpathlineto{\pgfqpoint{3.169596in}{1.554348in}}%
\pgfpathlineto{\pgfqpoint{3.181287in}{1.554348in}}%
\pgfpathlineto{\pgfqpoint{3.193182in}{1.554348in}}%
\pgfpathlineto{\pgfqpoint{3.205304in}{1.554348in}}%
\pgfpathlineto{\pgfqpoint{3.217678in}{1.554348in}}%
\pgfpathlineto{\pgfqpoint{3.230327in}{1.554348in}}%
\pgfpathlineto{\pgfqpoint{3.243273in}{1.554348in}}%
\pgfpathlineto{\pgfqpoint{3.256537in}{1.554348in}}%
\pgfpathlineto{\pgfqpoint{3.270140in}{1.554348in}}%
\pgfpathlineto{\pgfqpoint{3.284100in}{1.554348in}}%
\pgfpathlineto{\pgfqpoint{3.298435in}{1.554348in}}%
\pgfpathlineto{\pgfqpoint{3.313164in}{1.554348in}}%
\pgfpathlineto{\pgfqpoint{3.328301in}{1.554348in}}%
\pgfpathlineto{\pgfqpoint{3.343863in}{1.554348in}}%
\pgfpathlineto{\pgfqpoint{3.359865in}{1.554348in}}%
\pgfpathlineto{\pgfqpoint{3.376321in}{1.554348in}}%
\pgfpathlineto{\pgfqpoint{3.393243in}{1.554348in}}%
\pgfpathlineto{\pgfqpoint{3.410646in}{1.554348in}}%
\pgfpathlineto{\pgfqpoint{3.428543in}{1.554348in}}%
\pgfpathlineto{\pgfqpoint{3.446946in}{1.554348in}}%
\pgfpathlineto{\pgfqpoint{3.465870in}{1.554348in}}%
\pgfpathlineto{\pgfqpoint{3.485327in}{1.554348in}}%
\pgfpathlineto{\pgfqpoint{3.505319in}{1.554348in}}%
\pgfpathlineto{\pgfqpoint{3.525848in}{1.554348in}}%
\pgfpathlineto{\pgfqpoint{3.546928in}{1.554348in}}%
\pgfpathlineto{\pgfqpoint{3.568573in}{1.554348in}}%
\pgfpathlineto{\pgfqpoint{3.590798in}{1.554348in}}%
\pgfpathlineto{\pgfqpoint{3.613621in}{1.554348in}}%
\pgfpathlineto{\pgfqpoint{3.637059in}{1.554348in}}%
\pgfpathlineto{\pgfqpoint{3.661130in}{1.554348in}}%
\pgfpathlineto{\pgfqpoint{3.685855in}{1.554348in}}%
\pgfpathlineto{\pgfqpoint{3.711237in}{1.554348in}}%
\pgfpathlineto{\pgfqpoint{3.737282in}{1.554348in}}%
\pgfpathlineto{\pgfqpoint{3.764011in}{1.554348in}}%
\pgfpathlineto{\pgfqpoint{3.791445in}{1.554348in}}%
\pgfpathlineto{\pgfqpoint{3.819606in}{1.554348in}}%
\pgfpathlineto{\pgfqpoint{3.824431in}{1.554348in}}%
\pgfusepath{stroke}%
\end{pgfscope}%
\begin{pgfscope}%
\pgfsetrectcap%
\pgfsetmiterjoin%
\pgfsetlinewidth{0.803000pt}%
\definecolor{currentstroke}{rgb}{0.000000,0.000000,0.000000}%
\pgfsetstrokecolor{currentstroke}%
\pgfsetdash{}{0pt}%
\pgfpathmoveto{\pgfqpoint{0.537394in}{0.484854in}}%
\pgfpathlineto{\pgfqpoint{0.537394in}{2.934227in}}%
\pgfusepath{stroke}%
\end{pgfscope}%
\begin{pgfscope}%
\pgfsetrectcap%
\pgfsetmiterjoin%
\pgfsetlinewidth{0.803000pt}%
\definecolor{currentstroke}{rgb}{0.000000,0.000000,0.000000}%
\pgfsetstrokecolor{currentstroke}%
\pgfsetdash{}{0pt}%
\pgfpathmoveto{\pgfqpoint{3.824431in}{0.484854in}}%
\pgfpathlineto{\pgfqpoint{3.824431in}{2.934227in}}%
\pgfusepath{stroke}%
\end{pgfscope}%
\begin{pgfscope}%
\pgfsetrectcap%
\pgfsetmiterjoin%
\pgfsetlinewidth{0.803000pt}%
\definecolor{currentstroke}{rgb}{0.000000,0.000000,0.000000}%
\pgfsetstrokecolor{currentstroke}%
\pgfsetdash{}{0pt}%
\pgfpathmoveto{\pgfqpoint{0.537394in}{0.484854in}}%
\pgfpathlineto{\pgfqpoint{3.824431in}{0.484854in}}%
\pgfusepath{stroke}%
\end{pgfscope}%
\begin{pgfscope}%
\pgfsetrectcap%
\pgfsetmiterjoin%
\pgfsetlinewidth{0.803000pt}%
\definecolor{currentstroke}{rgb}{0.000000,0.000000,0.000000}%
\pgfsetstrokecolor{currentstroke}%
\pgfsetdash{}{0pt}%
\pgfpathmoveto{\pgfqpoint{0.537394in}{2.934227in}}%
\pgfpathlineto{\pgfqpoint{3.824431in}{2.934227in}}%
\pgfusepath{stroke}%
\end{pgfscope}%
\begin{pgfscope}%
\pgfsetrectcap%
\pgfsetroundjoin%
\pgfsetlinewidth{1.505625pt}%
\definecolor{currentstroke}{rgb}{0.121569,0.466667,0.705882}%
\pgfsetstrokecolor{currentstroke}%
\pgfsetdash{}{0pt}%
\pgfpathmoveto{\pgfqpoint{0.637394in}{2.788698in}}%
\pgfpathlineto{\pgfqpoint{0.859616in}{2.788698in}}%
\pgfusepath{stroke}%
\end{pgfscope}%
\begin{pgfscope}%
\definecolor{textcolor}{rgb}{0.000000,0.000000,0.000000}%
\pgfsetstrokecolor{textcolor}%
\pgfsetfillcolor{textcolor}%
\pgftext[x=0.948505in,y=2.749809in,left,base]{\color{textcolor}\rmfamily\fontsize{8.000000}{9.600000}\selectfont Average}%
\end{pgfscope}%
\begin{pgfscope}%
\pgfsetbuttcap%
\pgfsetmiterjoin%
\definecolor{currentfill}{rgb}{0.121569,0.466667,0.705882}%
\pgfsetfillcolor{currentfill}%
\pgfsetfillopacity{0.300000}%
\pgfsetlinewidth{0.000000pt}%
\definecolor{currentstroke}{rgb}{0.000000,0.000000,0.000000}%
\pgfsetstrokecolor{currentstroke}%
\pgfsetstrokeopacity{0.300000}%
\pgfsetdash{}{0pt}%
\pgfpathmoveto{\pgfqpoint{0.637394in}{2.585150in}}%
\pgfpathlineto{\pgfqpoint{0.859616in}{2.585150in}}%
\pgfpathlineto{\pgfqpoint{0.859616in}{2.662928in}}%
\pgfpathlineto{\pgfqpoint{0.637394in}{2.662928in}}%
\pgfpathclose%
\pgfusepath{fill}%
\end{pgfscope}%
\begin{pgfscope}%
\definecolor{textcolor}{rgb}{0.000000,0.000000,0.000000}%
\pgfsetstrokecolor{textcolor}%
\pgfsetfillcolor{textcolor}%
\pgftext[x=0.948505in,y=2.585150in,left,base]{\color{textcolor}\rmfamily\fontsize{8.000000}{9.600000}\selectfont Range}%
\end{pgfscope}%
\end{pgfpicture}%
\makeatother%
\endgroup%

    \caption{
        Load balancing across processors for the particle property plugin.
        The data shown reflect the number of particles per process and the range indicates the maximum and minimum number of particles for each.
        The data were collected using 2000 particles on 12 MPI processes.
    }
    \label{fig:load_balancing_particle_property}
\end{figure}

When running a program in parallel, it is important not only that the code be divided into pieces that can run on each processor, but also that the divisions be made equally in order to avoid processors idling unnecessarily.

The load balancing information for the particle property plugin is shown in Figure~\ref{fig:load_balancing_particle_property}.
At first the particles are almost perfectly uniformly distributed about the domain and have very low velocities, leading to a very small difference between the maximum and minimum number of particles per process.
However, as the plumes begin to form at around \num{500000} years (see Figure~\ref{fig:closed_box} for screenshots), the particle velocities increase and the uniformity breaks down.
At the end of the simulation there is an imbalance of approximately \SI{30}{\percent} between the number of particles per processor.
The particle distribution will follow a normal distribution and so this increase is simply a consequence of the normal curve gradually smoothing out.
Due to the fact that the model setup does not permit material to leave the simulation, the number of particles was unchanged and so the average number of particles per process was constant.

No results were reported for the material model plugin because, as no adaptive mesh refinement was found to have occurred during the simulation, the load was perfectly balanced between the cells throughout.

\subsubsection{Cache usage}

Cache utilisation for the Perple\_X wrapper was recorded for both plugins over a range of tolerances and the results are shown in Figure~\ref{fig:cache_usage}. 
In all cases, the first time step has a significantly lower hit rate due to mandatory misses (i.e. the cache is empty so there are no results to reuse).
This is followed by a spike to a peak value for that tolerance, likely caused by the fact that the initial velocity in the entire domain is close to zero and therefore the $p$-$T$-$X$ conditions are unchanged between time steps resulting in a cache hit.
Around \num{500000} years drops in the hit rate can be observed.
This time corresponds with the point where turbulent flow becomes more obvious and the particle velocities increase leading to greater variation in the pressure and temperature causing cache misses.

Although this general trend may be observed for both plots, the hit rate for the material model plugin is extremely high at around \SI{99.5}{\percent}.
This is due to the large number of iterations as a consequence of using the operator splitting method.
At each time step, so as to effectively model the rapid reaction rate of the partially melting substance, a great many evaluations are made of the material model with very tiny variations in the $p$-$T$-$X$ conditions, leading to a very high rate of cache reuse, even for high tolerances.

A final observation from the figure may be made regarding the particle property plugin.
Right at the end of the simulation a jump can be seen in the cache hit rate.
This may be explained by observing that there is also a sudden drop in the time step duration, meaning that the particles have displaced less and their respective $p$-$T$-$X$ conditions will not have changed much.

\begin{figure}[htb]
    \centering
    \begin{subfigure}{0.49\textwidth}
        %% Creator: Matplotlib, PGF backend
%%
%% To include the figure in your LaTeX document, write
%%   \input{<filename>.pgf}
%%
%% Make sure the required packages are loaded in your preamble
%%   \usepackage{pgf}
%%
%% Figures using additional raster images can only be included by \input if
%% they are in the same directory as the main LaTeX file. For loading figures
%% from other directories you can use the `import` package
%%   \usepackage{import}
%% and then include the figures with
%%   \import{<path to file>}{<filename>.pgf}
%%
%% Matplotlib used the following preamble
%%   \usepackage{fontspec}
%%   \setmainfont{DejaVuSerif.ttf}[Path=/home/connor/.local/lib/python3.8/site-packages/matplotlib/mpl-data/fonts/ttf/]
%%   \setsansfont{DejaVuSans.ttf}[Path=/home/connor/.local/lib/python3.8/site-packages/matplotlib/mpl-data/fonts/ttf/]
%%   \setmonofont{DejaVuSansMono.ttf}[Path=/home/connor/.local/lib/python3.8/site-packages/matplotlib/mpl-data/fonts/ttf/]
%%
\begingroup%
\makeatletter%
\begin{pgfpicture}%
\pgfpathrectangle{\pgfpointorigin}{\pgfqpoint{3.898197in}{3.018785in}}%
\pgfusepath{use as bounding box, clip}%
\begin{pgfscope}%
\pgfsetbuttcap%
\pgfsetmiterjoin%
\definecolor{currentfill}{rgb}{1.000000,1.000000,1.000000}%
\pgfsetfillcolor{currentfill}%
\pgfsetlinewidth{0.000000pt}%
\definecolor{currentstroke}{rgb}{1.000000,1.000000,1.000000}%
\pgfsetstrokecolor{currentstroke}%
\pgfsetdash{}{0pt}%
\pgfpathmoveto{\pgfqpoint{0.000000in}{0.000000in}}%
\pgfpathlineto{\pgfqpoint{3.898197in}{0.000000in}}%
\pgfpathlineto{\pgfqpoint{3.898197in}{3.018785in}}%
\pgfpathlineto{\pgfqpoint{0.000000in}{3.018785in}}%
\pgfpathclose%
\pgfusepath{fill}%
\end{pgfscope}%
\begin{pgfscope}%
\pgfsetbuttcap%
\pgfsetmiterjoin%
\definecolor{currentfill}{rgb}{1.000000,1.000000,1.000000}%
\pgfsetfillcolor{currentfill}%
\pgfsetlinewidth{0.000000pt}%
\definecolor{currentstroke}{rgb}{0.000000,0.000000,0.000000}%
\pgfsetstrokecolor{currentstroke}%
\pgfsetstrokeopacity{0.000000}%
\pgfsetdash{}{0pt}%
\pgfpathmoveto{\pgfqpoint{0.511159in}{0.469412in}}%
\pgfpathlineto{\pgfqpoint{3.798197in}{0.469412in}}%
\pgfpathlineto{\pgfqpoint{3.798197in}{2.918785in}}%
\pgfpathlineto{\pgfqpoint{0.511159in}{2.918785in}}%
\pgfpathclose%
\pgfusepath{fill}%
\end{pgfscope}%
\begin{pgfscope}%
\pgfsetbuttcap%
\pgfsetroundjoin%
\definecolor{currentfill}{rgb}{0.000000,0.000000,0.000000}%
\pgfsetfillcolor{currentfill}%
\pgfsetlinewidth{0.803000pt}%
\definecolor{currentstroke}{rgb}{0.000000,0.000000,0.000000}%
\pgfsetstrokecolor{currentstroke}%
\pgfsetdash{}{0pt}%
\pgfsys@defobject{currentmarker}{\pgfqpoint{0.000000in}{-0.048611in}}{\pgfqpoint{0.000000in}{0.000000in}}{%
\pgfpathmoveto{\pgfqpoint{0.000000in}{0.000000in}}%
\pgfpathlineto{\pgfqpoint{0.000000in}{-0.048611in}}%
\pgfusepath{stroke,fill}%
}%
\begin{pgfscope}%
\pgfsys@transformshift{0.660570in}{0.469412in}%
\pgfsys@useobject{currentmarker}{}%
\end{pgfscope}%
\end{pgfscope}%
\begin{pgfscope}%
\definecolor{textcolor}{rgb}{0.000000,0.000000,0.000000}%
\pgfsetstrokecolor{textcolor}%
\pgfsetfillcolor{textcolor}%
\pgftext[x=0.660570in,y=0.372189in,,top]{\color{textcolor}\rmfamily\fontsize{8.000000}{9.600000}\selectfont \(\displaystyle 0\)}%
\end{pgfscope}%
\begin{pgfscope}%
\pgfsetbuttcap%
\pgfsetroundjoin%
\definecolor{currentfill}{rgb}{0.000000,0.000000,0.000000}%
\pgfsetfillcolor{currentfill}%
\pgfsetlinewidth{0.803000pt}%
\definecolor{currentstroke}{rgb}{0.000000,0.000000,0.000000}%
\pgfsetstrokecolor{currentstroke}%
\pgfsetdash{}{0pt}%
\pgfsys@defobject{currentmarker}{\pgfqpoint{0.000000in}{-0.048611in}}{\pgfqpoint{0.000000in}{0.000000in}}{%
\pgfpathmoveto{\pgfqpoint{0.000000in}{0.000000in}}%
\pgfpathlineto{\pgfqpoint{0.000000in}{-0.048611in}}%
\pgfusepath{stroke,fill}%
}%
\begin{pgfscope}%
\pgfsys@transformshift{1.058999in}{0.469412in}%
\pgfsys@useobject{currentmarker}{}%
\end{pgfscope}%
\end{pgfscope}%
\begin{pgfscope}%
\definecolor{textcolor}{rgb}{0.000000,0.000000,0.000000}%
\pgfsetstrokecolor{textcolor}%
\pgfsetfillcolor{textcolor}%
\pgftext[x=1.058999in,y=0.372189in,,top]{\color{textcolor}\rmfamily\fontsize{8.000000}{9.600000}\selectfont \(\displaystyle 200000\)}%
\end{pgfscope}%
\begin{pgfscope}%
\pgfsetbuttcap%
\pgfsetroundjoin%
\definecolor{currentfill}{rgb}{0.000000,0.000000,0.000000}%
\pgfsetfillcolor{currentfill}%
\pgfsetlinewidth{0.803000pt}%
\definecolor{currentstroke}{rgb}{0.000000,0.000000,0.000000}%
\pgfsetstrokecolor{currentstroke}%
\pgfsetdash{}{0pt}%
\pgfsys@defobject{currentmarker}{\pgfqpoint{0.000000in}{-0.048611in}}{\pgfqpoint{0.000000in}{0.000000in}}{%
\pgfpathmoveto{\pgfqpoint{0.000000in}{0.000000in}}%
\pgfpathlineto{\pgfqpoint{0.000000in}{-0.048611in}}%
\pgfusepath{stroke,fill}%
}%
\begin{pgfscope}%
\pgfsys@transformshift{1.457427in}{0.469412in}%
\pgfsys@useobject{currentmarker}{}%
\end{pgfscope}%
\end{pgfscope}%
\begin{pgfscope}%
\definecolor{textcolor}{rgb}{0.000000,0.000000,0.000000}%
\pgfsetstrokecolor{textcolor}%
\pgfsetfillcolor{textcolor}%
\pgftext[x=1.457427in,y=0.372189in,,top]{\color{textcolor}\rmfamily\fontsize{8.000000}{9.600000}\selectfont \(\displaystyle 400000\)}%
\end{pgfscope}%
\begin{pgfscope}%
\pgfsetbuttcap%
\pgfsetroundjoin%
\definecolor{currentfill}{rgb}{0.000000,0.000000,0.000000}%
\pgfsetfillcolor{currentfill}%
\pgfsetlinewidth{0.803000pt}%
\definecolor{currentstroke}{rgb}{0.000000,0.000000,0.000000}%
\pgfsetstrokecolor{currentstroke}%
\pgfsetdash{}{0pt}%
\pgfsys@defobject{currentmarker}{\pgfqpoint{0.000000in}{-0.048611in}}{\pgfqpoint{0.000000in}{0.000000in}}{%
\pgfpathmoveto{\pgfqpoint{0.000000in}{0.000000in}}%
\pgfpathlineto{\pgfqpoint{0.000000in}{-0.048611in}}%
\pgfusepath{stroke,fill}%
}%
\begin{pgfscope}%
\pgfsys@transformshift{1.855856in}{0.469412in}%
\pgfsys@useobject{currentmarker}{}%
\end{pgfscope}%
\end{pgfscope}%
\begin{pgfscope}%
\definecolor{textcolor}{rgb}{0.000000,0.000000,0.000000}%
\pgfsetstrokecolor{textcolor}%
\pgfsetfillcolor{textcolor}%
\pgftext[x=1.855856in,y=0.372189in,,top]{\color{textcolor}\rmfamily\fontsize{8.000000}{9.600000}\selectfont \(\displaystyle 600000\)}%
\end{pgfscope}%
\begin{pgfscope}%
\pgfsetbuttcap%
\pgfsetroundjoin%
\definecolor{currentfill}{rgb}{0.000000,0.000000,0.000000}%
\pgfsetfillcolor{currentfill}%
\pgfsetlinewidth{0.803000pt}%
\definecolor{currentstroke}{rgb}{0.000000,0.000000,0.000000}%
\pgfsetstrokecolor{currentstroke}%
\pgfsetdash{}{0pt}%
\pgfsys@defobject{currentmarker}{\pgfqpoint{0.000000in}{-0.048611in}}{\pgfqpoint{0.000000in}{0.000000in}}{%
\pgfpathmoveto{\pgfqpoint{0.000000in}{0.000000in}}%
\pgfpathlineto{\pgfqpoint{0.000000in}{-0.048611in}}%
\pgfusepath{stroke,fill}%
}%
\begin{pgfscope}%
\pgfsys@transformshift{2.254285in}{0.469412in}%
\pgfsys@useobject{currentmarker}{}%
\end{pgfscope}%
\end{pgfscope}%
\begin{pgfscope}%
\definecolor{textcolor}{rgb}{0.000000,0.000000,0.000000}%
\pgfsetstrokecolor{textcolor}%
\pgfsetfillcolor{textcolor}%
\pgftext[x=2.254285in,y=0.372189in,,top]{\color{textcolor}\rmfamily\fontsize{8.000000}{9.600000}\selectfont \(\displaystyle 800000\)}%
\end{pgfscope}%
\begin{pgfscope}%
\pgfsetbuttcap%
\pgfsetroundjoin%
\definecolor{currentfill}{rgb}{0.000000,0.000000,0.000000}%
\pgfsetfillcolor{currentfill}%
\pgfsetlinewidth{0.803000pt}%
\definecolor{currentstroke}{rgb}{0.000000,0.000000,0.000000}%
\pgfsetstrokecolor{currentstroke}%
\pgfsetdash{}{0pt}%
\pgfsys@defobject{currentmarker}{\pgfqpoint{0.000000in}{-0.048611in}}{\pgfqpoint{0.000000in}{0.000000in}}{%
\pgfpathmoveto{\pgfqpoint{0.000000in}{0.000000in}}%
\pgfpathlineto{\pgfqpoint{0.000000in}{-0.048611in}}%
\pgfusepath{stroke,fill}%
}%
\begin{pgfscope}%
\pgfsys@transformshift{2.652714in}{0.469412in}%
\pgfsys@useobject{currentmarker}{}%
\end{pgfscope}%
\end{pgfscope}%
\begin{pgfscope}%
\definecolor{textcolor}{rgb}{0.000000,0.000000,0.000000}%
\pgfsetstrokecolor{textcolor}%
\pgfsetfillcolor{textcolor}%
\pgftext[x=2.652714in,y=0.372189in,,top]{\color{textcolor}\rmfamily\fontsize{8.000000}{9.600000}\selectfont \(\displaystyle 1000000\)}%
\end{pgfscope}%
\begin{pgfscope}%
\pgfsetbuttcap%
\pgfsetroundjoin%
\definecolor{currentfill}{rgb}{0.000000,0.000000,0.000000}%
\pgfsetfillcolor{currentfill}%
\pgfsetlinewidth{0.803000pt}%
\definecolor{currentstroke}{rgb}{0.000000,0.000000,0.000000}%
\pgfsetstrokecolor{currentstroke}%
\pgfsetdash{}{0pt}%
\pgfsys@defobject{currentmarker}{\pgfqpoint{0.000000in}{-0.048611in}}{\pgfqpoint{0.000000in}{0.000000in}}{%
\pgfpathmoveto{\pgfqpoint{0.000000in}{0.000000in}}%
\pgfpathlineto{\pgfqpoint{0.000000in}{-0.048611in}}%
\pgfusepath{stroke,fill}%
}%
\begin{pgfscope}%
\pgfsys@transformshift{3.051143in}{0.469412in}%
\pgfsys@useobject{currentmarker}{}%
\end{pgfscope}%
\end{pgfscope}%
\begin{pgfscope}%
\definecolor{textcolor}{rgb}{0.000000,0.000000,0.000000}%
\pgfsetstrokecolor{textcolor}%
\pgfsetfillcolor{textcolor}%
\pgftext[x=3.051143in,y=0.372189in,,top]{\color{textcolor}\rmfamily\fontsize{8.000000}{9.600000}\selectfont \(\displaystyle 1200000\)}%
\end{pgfscope}%
\begin{pgfscope}%
\pgfsetbuttcap%
\pgfsetroundjoin%
\definecolor{currentfill}{rgb}{0.000000,0.000000,0.000000}%
\pgfsetfillcolor{currentfill}%
\pgfsetlinewidth{0.803000pt}%
\definecolor{currentstroke}{rgb}{0.000000,0.000000,0.000000}%
\pgfsetstrokecolor{currentstroke}%
\pgfsetdash{}{0pt}%
\pgfsys@defobject{currentmarker}{\pgfqpoint{0.000000in}{-0.048611in}}{\pgfqpoint{0.000000in}{0.000000in}}{%
\pgfpathmoveto{\pgfqpoint{0.000000in}{0.000000in}}%
\pgfpathlineto{\pgfqpoint{0.000000in}{-0.048611in}}%
\pgfusepath{stroke,fill}%
}%
\begin{pgfscope}%
\pgfsys@transformshift{3.449571in}{0.469412in}%
\pgfsys@useobject{currentmarker}{}%
\end{pgfscope}%
\end{pgfscope}%
\begin{pgfscope}%
\definecolor{textcolor}{rgb}{0.000000,0.000000,0.000000}%
\pgfsetstrokecolor{textcolor}%
\pgfsetfillcolor{textcolor}%
\pgftext[x=3.449571in,y=0.372189in,,top]{\color{textcolor}\rmfamily\fontsize{8.000000}{9.600000}\selectfont \(\displaystyle 1400000\)}%
\end{pgfscope}%
\begin{pgfscope}%
\definecolor{textcolor}{rgb}{0.000000,0.000000,0.000000}%
\pgfsetstrokecolor{textcolor}%
\pgfsetfillcolor{textcolor}%
\pgftext[x=2.154678in,y=0.209104in,,top]{\color{textcolor}\rmfamily\fontsize{8.000000}{9.600000}\selectfont Time (years)}%
\end{pgfscope}%
\begin{pgfscope}%
\pgfsetbuttcap%
\pgfsetroundjoin%
\definecolor{currentfill}{rgb}{0.000000,0.000000,0.000000}%
\pgfsetfillcolor{currentfill}%
\pgfsetlinewidth{0.803000pt}%
\definecolor{currentstroke}{rgb}{0.000000,0.000000,0.000000}%
\pgfsetstrokecolor{currentstroke}%
\pgfsetdash{}{0pt}%
\pgfsys@defobject{currentmarker}{\pgfqpoint{-0.048611in}{0.000000in}}{\pgfqpoint{0.000000in}{0.000000in}}{%
\pgfpathmoveto{\pgfqpoint{0.000000in}{0.000000in}}%
\pgfpathlineto{\pgfqpoint{-0.048611in}{0.000000in}}%
\pgfusepath{stroke,fill}%
}%
\begin{pgfscope}%
\pgfsys@transformshift{0.511159in}{0.578513in}%
\pgfsys@useobject{currentmarker}{}%
\end{pgfscope}%
\end{pgfscope}%
\begin{pgfscope}%
\definecolor{textcolor}{rgb}{0.000000,0.000000,0.000000}%
\pgfsetstrokecolor{textcolor}%
\pgfsetfillcolor{textcolor}%
\pgftext[x=0.263086in,y=0.536304in,left,base]{\color{textcolor}\rmfamily\fontsize{8.000000}{9.600000}\selectfont \(\displaystyle 0.0\)}%
\end{pgfscope}%
\begin{pgfscope}%
\pgfsetbuttcap%
\pgfsetroundjoin%
\definecolor{currentfill}{rgb}{0.000000,0.000000,0.000000}%
\pgfsetfillcolor{currentfill}%
\pgfsetlinewidth{0.803000pt}%
\definecolor{currentstroke}{rgb}{0.000000,0.000000,0.000000}%
\pgfsetstrokecolor{currentstroke}%
\pgfsetdash{}{0pt}%
\pgfsys@defobject{currentmarker}{\pgfqpoint{-0.048611in}{0.000000in}}{\pgfqpoint{0.000000in}{0.000000in}}{%
\pgfpathmoveto{\pgfqpoint{0.000000in}{0.000000in}}%
\pgfpathlineto{\pgfqpoint{-0.048611in}{0.000000in}}%
\pgfusepath{stroke,fill}%
}%
\begin{pgfscope}%
\pgfsys@transformshift{0.511159in}{1.025194in}%
\pgfsys@useobject{currentmarker}{}%
\end{pgfscope}%
\end{pgfscope}%
\begin{pgfscope}%
\definecolor{textcolor}{rgb}{0.000000,0.000000,0.000000}%
\pgfsetstrokecolor{textcolor}%
\pgfsetfillcolor{textcolor}%
\pgftext[x=0.263086in,y=0.982985in,left,base]{\color{textcolor}\rmfamily\fontsize{8.000000}{9.600000}\selectfont \(\displaystyle 0.2\)}%
\end{pgfscope}%
\begin{pgfscope}%
\pgfsetbuttcap%
\pgfsetroundjoin%
\definecolor{currentfill}{rgb}{0.000000,0.000000,0.000000}%
\pgfsetfillcolor{currentfill}%
\pgfsetlinewidth{0.803000pt}%
\definecolor{currentstroke}{rgb}{0.000000,0.000000,0.000000}%
\pgfsetstrokecolor{currentstroke}%
\pgfsetdash{}{0pt}%
\pgfsys@defobject{currentmarker}{\pgfqpoint{-0.048611in}{0.000000in}}{\pgfqpoint{0.000000in}{0.000000in}}{%
\pgfpathmoveto{\pgfqpoint{0.000000in}{0.000000in}}%
\pgfpathlineto{\pgfqpoint{-0.048611in}{0.000000in}}%
\pgfusepath{stroke,fill}%
}%
\begin{pgfscope}%
\pgfsys@transformshift{0.511159in}{1.471875in}%
\pgfsys@useobject{currentmarker}{}%
\end{pgfscope}%
\end{pgfscope}%
\begin{pgfscope}%
\definecolor{textcolor}{rgb}{0.000000,0.000000,0.000000}%
\pgfsetstrokecolor{textcolor}%
\pgfsetfillcolor{textcolor}%
\pgftext[x=0.263086in,y=1.429665in,left,base]{\color{textcolor}\rmfamily\fontsize{8.000000}{9.600000}\selectfont \(\displaystyle 0.4\)}%
\end{pgfscope}%
\begin{pgfscope}%
\pgfsetbuttcap%
\pgfsetroundjoin%
\definecolor{currentfill}{rgb}{0.000000,0.000000,0.000000}%
\pgfsetfillcolor{currentfill}%
\pgfsetlinewidth{0.803000pt}%
\definecolor{currentstroke}{rgb}{0.000000,0.000000,0.000000}%
\pgfsetstrokecolor{currentstroke}%
\pgfsetdash{}{0pt}%
\pgfsys@defobject{currentmarker}{\pgfqpoint{-0.048611in}{0.000000in}}{\pgfqpoint{0.000000in}{0.000000in}}{%
\pgfpathmoveto{\pgfqpoint{0.000000in}{0.000000in}}%
\pgfpathlineto{\pgfqpoint{-0.048611in}{0.000000in}}%
\pgfusepath{stroke,fill}%
}%
\begin{pgfscope}%
\pgfsys@transformshift{0.511159in}{1.918555in}%
\pgfsys@useobject{currentmarker}{}%
\end{pgfscope}%
\end{pgfscope}%
\begin{pgfscope}%
\definecolor{textcolor}{rgb}{0.000000,0.000000,0.000000}%
\pgfsetstrokecolor{textcolor}%
\pgfsetfillcolor{textcolor}%
\pgftext[x=0.263086in,y=1.876346in,left,base]{\color{textcolor}\rmfamily\fontsize{8.000000}{9.600000}\selectfont \(\displaystyle 0.6\)}%
\end{pgfscope}%
\begin{pgfscope}%
\pgfsetbuttcap%
\pgfsetroundjoin%
\definecolor{currentfill}{rgb}{0.000000,0.000000,0.000000}%
\pgfsetfillcolor{currentfill}%
\pgfsetlinewidth{0.803000pt}%
\definecolor{currentstroke}{rgb}{0.000000,0.000000,0.000000}%
\pgfsetstrokecolor{currentstroke}%
\pgfsetdash{}{0pt}%
\pgfsys@defobject{currentmarker}{\pgfqpoint{-0.048611in}{0.000000in}}{\pgfqpoint{0.000000in}{0.000000in}}{%
\pgfpathmoveto{\pgfqpoint{0.000000in}{0.000000in}}%
\pgfpathlineto{\pgfqpoint{-0.048611in}{0.000000in}}%
\pgfusepath{stroke,fill}%
}%
\begin{pgfscope}%
\pgfsys@transformshift{0.511159in}{2.365236in}%
\pgfsys@useobject{currentmarker}{}%
\end{pgfscope}%
\end{pgfscope}%
\begin{pgfscope}%
\definecolor{textcolor}{rgb}{0.000000,0.000000,0.000000}%
\pgfsetstrokecolor{textcolor}%
\pgfsetfillcolor{textcolor}%
\pgftext[x=0.263086in,y=2.323027in,left,base]{\color{textcolor}\rmfamily\fontsize{8.000000}{9.600000}\selectfont \(\displaystyle 0.8\)}%
\end{pgfscope}%
\begin{pgfscope}%
\pgfsetbuttcap%
\pgfsetroundjoin%
\definecolor{currentfill}{rgb}{0.000000,0.000000,0.000000}%
\pgfsetfillcolor{currentfill}%
\pgfsetlinewidth{0.803000pt}%
\definecolor{currentstroke}{rgb}{0.000000,0.000000,0.000000}%
\pgfsetstrokecolor{currentstroke}%
\pgfsetdash{}{0pt}%
\pgfsys@defobject{currentmarker}{\pgfqpoint{-0.048611in}{0.000000in}}{\pgfqpoint{0.000000in}{0.000000in}}{%
\pgfpathmoveto{\pgfqpoint{0.000000in}{0.000000in}}%
\pgfpathlineto{\pgfqpoint{-0.048611in}{0.000000in}}%
\pgfusepath{stroke,fill}%
}%
\begin{pgfscope}%
\pgfsys@transformshift{0.511159in}{2.811916in}%
\pgfsys@useobject{currentmarker}{}%
\end{pgfscope}%
\end{pgfscope}%
\begin{pgfscope}%
\definecolor{textcolor}{rgb}{0.000000,0.000000,0.000000}%
\pgfsetstrokecolor{textcolor}%
\pgfsetfillcolor{textcolor}%
\pgftext[x=0.263086in,y=2.769707in,left,base]{\color{textcolor}\rmfamily\fontsize{8.000000}{9.600000}\selectfont \(\displaystyle 1.0\)}%
\end{pgfscope}%
\begin{pgfscope}%
\definecolor{textcolor}{rgb}{0.000000,0.000000,0.000000}%
\pgfsetstrokecolor{textcolor}%
\pgfsetfillcolor{textcolor}%
\pgftext[x=0.207530in,y=1.694098in,,bottom,rotate=90.000000]{\color{textcolor}\rmfamily\fontsize{8.000000}{9.600000}\selectfont Hit rate}%
\end{pgfscope}%
\begin{pgfscope}%
\pgfpathrectangle{\pgfqpoint{0.511159in}{0.469412in}}{\pgfqpoint{3.287038in}{2.449373in}}%
\pgfusepath{clip}%
\pgfsetrectcap%
\pgfsetroundjoin%
\pgfsetlinewidth{1.505625pt}%
\definecolor{currentstroke}{rgb}{0.121569,0.466667,0.705882}%
\pgfsetstrokecolor{currentstroke}%
\pgfsetdash{}{0pt}%
\pgfpathmoveto{\pgfqpoint{0.660570in}{1.512076in}}%
\pgfpathlineto{\pgfqpoint{0.859784in}{2.805216in}}%
\pgfpathlineto{\pgfqpoint{1.058999in}{2.807450in}}%
\pgfpathlineto{\pgfqpoint{1.258213in}{2.789582in}}%
\pgfpathlineto{\pgfqpoint{1.457427in}{2.765015in}}%
\pgfpathlineto{\pgfqpoint{1.656642in}{2.720347in}}%
\pgfpathlineto{\pgfqpoint{1.855856in}{2.304934in}}%
\pgfpathlineto{\pgfqpoint{2.055071in}{2.182097in}}%
\pgfpathlineto{\pgfqpoint{2.187680in}{2.211131in}}%
\pgfpathlineto{\pgfqpoint{2.266052in}{2.255799in}}%
\pgfpathlineto{\pgfqpoint{2.322114in}{2.298234in}}%
\pgfpathlineto{\pgfqpoint{2.365890in}{2.318334in}}%
\pgfpathlineto{\pgfqpoint{2.401891in}{2.313868in}}%
\pgfpathlineto{\pgfqpoint{2.432529in}{2.380870in}}%
\pgfpathlineto{\pgfqpoint{2.459280in}{2.371936in}}%
\pgfpathlineto{\pgfqpoint{2.483095in}{2.376403in}}%
\pgfpathlineto{\pgfqpoint{2.504617in}{2.418838in}}%
\pgfpathlineto{\pgfqpoint{2.524307in}{2.392037in}}%
\pgfpathlineto{\pgfqpoint{2.542505in}{2.432238in}}%
\pgfpathlineto{\pgfqpoint{2.559469in}{2.418838in}}%
\pgfpathlineto{\pgfqpoint{2.575277in}{2.454572in}}%
\pgfpathlineto{\pgfqpoint{2.590066in}{2.467972in}}%
\pgfpathlineto{\pgfqpoint{2.603999in}{2.459039in}}%
\pgfpathlineto{\pgfqpoint{2.617204in}{2.505940in}}%
\pgfpathlineto{\pgfqpoint{2.629788in}{2.497007in}}%
\pgfpathlineto{\pgfqpoint{2.641838in}{2.508174in}}%
\pgfpathlineto{\pgfqpoint{2.653426in}{2.503707in}}%
\pgfpathlineto{\pgfqpoint{2.664614in}{2.510407in}}%
\pgfpathlineto{\pgfqpoint{2.675455in}{2.528274in}}%
\pgfpathlineto{\pgfqpoint{2.685993in}{2.519341in}}%
\pgfpathlineto{\pgfqpoint{2.696268in}{2.519341in}}%
\pgfpathlineto{\pgfqpoint{2.706314in}{2.530508in}}%
\pgfpathlineto{\pgfqpoint{2.716161in}{2.512640in}}%
\pgfpathlineto{\pgfqpoint{2.725834in}{2.550608in}}%
\pgfpathlineto{\pgfqpoint{2.735355in}{2.532741in}}%
\pgfpathlineto{\pgfqpoint{2.744748in}{2.561775in}}%
\pgfpathlineto{\pgfqpoint{2.754034in}{2.532741in}}%
\pgfpathlineto{\pgfqpoint{2.763232in}{2.552842in}}%
\pgfpathlineto{\pgfqpoint{2.772361in}{2.548375in}}%
\pgfpathlineto{\pgfqpoint{2.781438in}{2.559542in}}%
\pgfpathlineto{\pgfqpoint{2.790479in}{2.548375in}}%
\pgfpathlineto{\pgfqpoint{2.799499in}{2.537208in}}%
\pgfpathlineto{\pgfqpoint{2.808513in}{2.581876in}}%
\pgfpathlineto{\pgfqpoint{2.817536in}{2.550608in}}%
\pgfpathlineto{\pgfqpoint{2.826581in}{2.546141in}}%
\pgfpathlineto{\pgfqpoint{2.835662in}{2.561775in}}%
\pgfpathlineto{\pgfqpoint{2.844794in}{2.550608in}}%
\pgfpathlineto{\pgfqpoint{2.853989in}{2.568476in}}%
\pgfpathlineto{\pgfqpoint{2.863261in}{2.559542in}}%
\pgfpathlineto{\pgfqpoint{2.872623in}{2.550608in}}%
\pgfpathlineto{\pgfqpoint{2.882090in}{2.534974in}}%
\pgfpathlineto{\pgfqpoint{2.891676in}{2.534974in}}%
\pgfpathlineto{\pgfqpoint{2.901394in}{2.543908in}}%
\pgfpathlineto{\pgfqpoint{2.911259in}{2.530508in}}%
\pgfpathlineto{\pgfqpoint{2.921286in}{2.517107in}}%
\pgfpathlineto{\pgfqpoint{2.931490in}{2.492540in}}%
\pgfpathlineto{\pgfqpoint{2.941872in}{2.497007in}}%
\pgfpathlineto{\pgfqpoint{2.952152in}{2.503707in}}%
\pgfpathlineto{\pgfqpoint{2.962294in}{2.490306in}}%
\pgfpathlineto{\pgfqpoint{2.972342in}{2.505940in}}%
\pgfpathlineto{\pgfqpoint{2.982336in}{2.499240in}}%
\pgfpathlineto{\pgfqpoint{2.992313in}{2.508174in}}%
\pgfpathlineto{\pgfqpoint{3.002305in}{2.503707in}}%
\pgfpathlineto{\pgfqpoint{3.012345in}{2.490306in}}%
\pgfpathlineto{\pgfqpoint{3.022461in}{2.481373in}}%
\pgfpathlineto{\pgfqpoint{3.032682in}{2.476906in}}%
\pgfpathlineto{\pgfqpoint{3.043013in}{2.467972in}}%
\pgfpathlineto{\pgfqpoint{3.053481in}{2.492540in}}%
\pgfpathlineto{\pgfqpoint{3.064109in}{2.514874in}}%
\pgfpathlineto{\pgfqpoint{3.074923in}{2.497007in}}%
\pgfpathlineto{\pgfqpoint{3.085943in}{2.510407in}}%
\pgfpathlineto{\pgfqpoint{3.097192in}{2.476906in}}%
\pgfpathlineto{\pgfqpoint{3.108691in}{2.483606in}}%
\pgfpathlineto{\pgfqpoint{3.120460in}{2.472439in}}%
\pgfpathlineto{\pgfqpoint{3.132518in}{2.481373in}}%
\pgfpathlineto{\pgfqpoint{3.144884in}{2.452339in}}%
\pgfpathlineto{\pgfqpoint{3.157575in}{2.456805in}}%
\pgfpathlineto{\pgfqpoint{3.170607in}{2.472439in}}%
\pgfpathlineto{\pgfqpoint{3.183997in}{2.459039in}}%
\pgfpathlineto{\pgfqpoint{3.197758in}{2.434471in}}%
\pgfpathlineto{\pgfqpoint{3.211906in}{2.436705in}}%
\pgfpathlineto{\pgfqpoint{3.226453in}{2.438938in}}%
\pgfpathlineto{\pgfqpoint{3.241412in}{2.438938in}}%
\pgfpathlineto{\pgfqpoint{3.256797in}{2.423304in}}%
\pgfpathlineto{\pgfqpoint{3.272618in}{2.430005in}}%
\pgfpathlineto{\pgfqpoint{3.288887in}{2.423304in}}%
\pgfpathlineto{\pgfqpoint{3.305618in}{2.409904in}}%
\pgfpathlineto{\pgfqpoint{3.322821in}{2.398737in}}%
\pgfpathlineto{\pgfqpoint{3.340509in}{2.392037in}}%
\pgfpathlineto{\pgfqpoint{3.358683in}{2.409904in}}%
\pgfpathlineto{\pgfqpoint{3.377346in}{2.371936in}}%
\pgfpathlineto{\pgfqpoint{3.396510in}{2.414371in}}%
\pgfpathlineto{\pgfqpoint{3.416187in}{2.389803in}}%
\pgfpathlineto{\pgfqpoint{3.436392in}{2.416604in}}%
\pgfpathlineto{\pgfqpoint{3.457140in}{2.389803in}}%
\pgfpathlineto{\pgfqpoint{3.478447in}{2.394270in}}%
\pgfpathlineto{\pgfqpoint{3.500330in}{2.385336in}}%
\pgfpathlineto{\pgfqpoint{3.522807in}{2.400970in}}%
\pgfpathlineto{\pgfqpoint{3.545882in}{2.387570in}}%
\pgfpathlineto{\pgfqpoint{3.569559in}{2.396503in}}%
\pgfpathlineto{\pgfqpoint{3.593858in}{2.398737in}}%
\pgfpathlineto{\pgfqpoint{3.618798in}{2.389803in}}%
\pgfpathlineto{\pgfqpoint{3.644399in}{2.427771in}}%
\pgfpathlineto{\pgfqpoint{3.648786in}{2.735981in}}%
\pgfusepath{stroke}%
\end{pgfscope}%
\begin{pgfscope}%
\pgfpathrectangle{\pgfqpoint{0.511159in}{0.469412in}}{\pgfqpoint{3.287038in}{2.449373in}}%
\pgfusepath{clip}%
\pgfsetrectcap%
\pgfsetroundjoin%
\pgfsetlinewidth{1.505625pt}%
\definecolor{currentstroke}{rgb}{1.000000,0.498039,0.054902}%
\pgfsetstrokecolor{currentstroke}%
\pgfsetdash{}{0pt}%
\pgfpathmoveto{\pgfqpoint{0.660570in}{0.714751in}}%
\pgfpathlineto{\pgfqpoint{0.859784in}{2.767248in}}%
\pgfpathlineto{\pgfqpoint{1.058999in}{2.668979in}}%
\pgfpathlineto{\pgfqpoint{1.258213in}{1.670647in}}%
\pgfpathlineto{\pgfqpoint{1.457427in}{1.215033in}}%
\pgfpathlineto{\pgfqpoint{1.656642in}{1.018494in}}%
\pgfpathlineto{\pgfqpoint{1.855856in}{1.047528in}}%
\pgfpathlineto{\pgfqpoint{2.055071in}{1.027427in}}%
\pgfpathlineto{\pgfqpoint{2.187680in}{1.098896in}}%
\pgfpathlineto{\pgfqpoint{2.266052in}{1.125697in}}%
\pgfpathlineto{\pgfqpoint{2.322114in}{1.219500in}}%
\pgfpathlineto{\pgfqpoint{2.365890in}{1.223967in}}%
\pgfpathlineto{\pgfqpoint{2.401891in}{1.241834in}}%
\pgfpathlineto{\pgfqpoint{2.432529in}{1.286502in}}%
\pgfpathlineto{\pgfqpoint{2.459280in}{1.306603in}}%
\pgfpathlineto{\pgfqpoint{2.483095in}{1.364671in}}%
\pgfpathlineto{\pgfqpoint{2.504617in}{1.306603in}}%
\pgfpathlineto{\pgfqpoint{2.524307in}{1.393706in}}%
\pgfpathlineto{\pgfqpoint{2.542505in}{1.375838in}}%
\pgfpathlineto{\pgfqpoint{2.559469in}{1.442840in}}%
\pgfpathlineto{\pgfqpoint{2.575277in}{1.391472in}}%
\pgfpathlineto{\pgfqpoint{2.590066in}{1.424973in}}%
\pgfpathlineto{\pgfqpoint{2.603999in}{1.465174in}}%
\pgfpathlineto{\pgfqpoint{2.617204in}{1.440607in}}%
\pgfpathlineto{\pgfqpoint{2.629788in}{1.402639in}}%
\pgfpathlineto{\pgfqpoint{2.641838in}{1.418273in}}%
\pgfpathlineto{\pgfqpoint{2.653426in}{1.436140in}}%
\pgfpathlineto{\pgfqpoint{2.664614in}{1.469641in}}%
\pgfpathlineto{\pgfqpoint{2.675455in}{1.420506in}}%
\pgfpathlineto{\pgfqpoint{2.685993in}{1.451774in}}%
\pgfpathlineto{\pgfqpoint{2.696268in}{1.420506in}}%
\pgfpathlineto{\pgfqpoint{2.706314in}{1.400406in}}%
\pgfpathlineto{\pgfqpoint{2.716161in}{1.375838in}}%
\pgfpathlineto{\pgfqpoint{2.725834in}{1.429440in}}%
\pgfpathlineto{\pgfqpoint{2.735355in}{1.424973in}}%
\pgfpathlineto{\pgfqpoint{2.744748in}{1.445074in}}%
\pgfpathlineto{\pgfqpoint{2.754034in}{1.427207in}}%
\pgfpathlineto{\pgfqpoint{2.763232in}{1.433907in}}%
\pgfpathlineto{\pgfqpoint{2.772361in}{1.418273in}}%
\pgfpathlineto{\pgfqpoint{2.781438in}{1.445074in}}%
\pgfpathlineto{\pgfqpoint{2.790479in}{1.418273in}}%
\pgfpathlineto{\pgfqpoint{2.799499in}{1.395939in}}%
\pgfpathlineto{\pgfqpoint{2.808513in}{1.422740in}}%
\pgfpathlineto{\pgfqpoint{2.817536in}{1.402639in}}%
\pgfpathlineto{\pgfqpoint{2.826581in}{1.395939in}}%
\pgfpathlineto{\pgfqpoint{2.835662in}{1.395939in}}%
\pgfpathlineto{\pgfqpoint{2.844794in}{1.409339in}}%
\pgfpathlineto{\pgfqpoint{2.853989in}{1.413806in}}%
\pgfpathlineto{\pgfqpoint{2.863261in}{1.427207in}}%
\pgfpathlineto{\pgfqpoint{2.872623in}{1.433907in}}%
\pgfpathlineto{\pgfqpoint{2.882090in}{1.420506in}}%
\pgfpathlineto{\pgfqpoint{2.891676in}{1.424973in}}%
\pgfpathlineto{\pgfqpoint{2.901394in}{1.378072in}}%
\pgfpathlineto{\pgfqpoint{2.911259in}{1.433907in}}%
\pgfpathlineto{\pgfqpoint{2.921286in}{1.429440in}}%
\pgfpathlineto{\pgfqpoint{2.931490in}{1.476341in}}%
\pgfpathlineto{\pgfqpoint{2.941872in}{1.411573in}}%
\pgfpathlineto{\pgfqpoint{2.952152in}{1.391472in}}%
\pgfpathlineto{\pgfqpoint{2.962294in}{1.422740in}}%
\pgfpathlineto{\pgfqpoint{2.972342in}{1.447307in}}%
\pgfpathlineto{\pgfqpoint{2.982336in}{1.389239in}}%
\pgfpathlineto{\pgfqpoint{2.992313in}{1.445074in}}%
\pgfpathlineto{\pgfqpoint{3.002305in}{1.440607in}}%
\pgfpathlineto{\pgfqpoint{3.012345in}{1.445074in}}%
\pgfpathlineto{\pgfqpoint{3.022461in}{1.467408in}}%
\pgfpathlineto{\pgfqpoint{3.032682in}{1.451774in}}%
\pgfpathlineto{\pgfqpoint{3.043013in}{1.456241in}}%
\pgfpathlineto{\pgfqpoint{3.053481in}{1.476341in}}%
\pgfpathlineto{\pgfqpoint{3.064109in}{1.494209in}}%
\pgfpathlineto{\pgfqpoint{3.074923in}{1.467408in}}%
\pgfpathlineto{\pgfqpoint{3.085943in}{1.427207in}}%
\pgfpathlineto{\pgfqpoint{3.097192in}{1.445074in}}%
\pgfpathlineto{\pgfqpoint{3.108691in}{1.456241in}}%
\pgfpathlineto{\pgfqpoint{3.120460in}{1.451774in}}%
\pgfpathlineto{\pgfqpoint{3.132518in}{1.480808in}}%
\pgfpathlineto{\pgfqpoint{3.144884in}{1.467408in}}%
\pgfpathlineto{\pgfqpoint{3.157575in}{1.462941in}}%
\pgfpathlineto{\pgfqpoint{3.170607in}{1.489742in}}%
\pgfpathlineto{\pgfqpoint{3.183997in}{1.471875in}}%
\pgfpathlineto{\pgfqpoint{3.197758in}{1.509842in}}%
\pgfpathlineto{\pgfqpoint{3.211906in}{1.507609in}}%
\pgfpathlineto{\pgfqpoint{3.226453in}{1.476341in}}%
\pgfpathlineto{\pgfqpoint{3.241412in}{1.500909in}}%
\pgfpathlineto{\pgfqpoint{3.256797in}{1.465174in}}%
\pgfpathlineto{\pgfqpoint{3.272618in}{1.507609in}}%
\pgfpathlineto{\pgfqpoint{3.288887in}{1.505376in}}%
\pgfpathlineto{\pgfqpoint{3.305618in}{1.494209in}}%
\pgfpathlineto{\pgfqpoint{3.322821in}{1.474108in}}%
\pgfpathlineto{\pgfqpoint{3.340509in}{1.494209in}}%
\pgfpathlineto{\pgfqpoint{3.358683in}{1.478575in}}%
\pgfpathlineto{\pgfqpoint{3.377346in}{1.498675in}}%
\pgfpathlineto{\pgfqpoint{3.396510in}{1.500909in}}%
\pgfpathlineto{\pgfqpoint{3.416187in}{1.491975in}}%
\pgfpathlineto{\pgfqpoint{3.436392in}{1.491975in}}%
\pgfpathlineto{\pgfqpoint{3.457140in}{1.498675in}}%
\pgfpathlineto{\pgfqpoint{3.478447in}{1.483042in}}%
\pgfpathlineto{\pgfqpoint{3.500330in}{1.500909in}}%
\pgfpathlineto{\pgfqpoint{3.522807in}{1.487508in}}%
\pgfpathlineto{\pgfqpoint{3.545882in}{1.478575in}}%
\pgfpathlineto{\pgfqpoint{3.569559in}{1.516543in}}%
\pgfpathlineto{\pgfqpoint{3.593858in}{1.496442in}}%
\pgfpathlineto{\pgfqpoint{3.618798in}{1.478575in}}%
\pgfpathlineto{\pgfqpoint{3.644399in}{1.465174in}}%
\pgfpathlineto{\pgfqpoint{3.648786in}{1.710849in}}%
\pgfusepath{stroke}%
\end{pgfscope}%
\begin{pgfscope}%
\pgfpathrectangle{\pgfqpoint{0.511159in}{0.469412in}}{\pgfqpoint{3.287038in}{2.449373in}}%
\pgfusepath{clip}%
\pgfsetrectcap%
\pgfsetroundjoin%
\pgfsetlinewidth{1.505625pt}%
\definecolor{currentstroke}{rgb}{0.172549,0.627451,0.172549}%
\pgfsetstrokecolor{currentstroke}%
\pgfsetdash{}{0pt}%
\pgfpathmoveto{\pgfqpoint{0.660570in}{0.596381in}}%
\pgfpathlineto{\pgfqpoint{0.859784in}{1.387005in}}%
\pgfpathlineto{\pgfqpoint{1.058999in}{0.830888in}}%
\pgfpathlineto{\pgfqpoint{1.258213in}{0.723685in}}%
\pgfpathlineto{\pgfqpoint{1.457427in}{0.656683in}}%
\pgfpathlineto{\pgfqpoint{1.656642in}{0.641049in}}%
\pgfpathlineto{\pgfqpoint{1.855856in}{0.634348in}}%
\pgfpathlineto{\pgfqpoint{2.055071in}{0.632115in}}%
\pgfpathlineto{\pgfqpoint{2.187680in}{0.663383in}}%
\pgfpathlineto{\pgfqpoint{2.266052in}{0.663383in}}%
\pgfpathlineto{\pgfqpoint{2.322114in}{0.663383in}}%
\pgfpathlineto{\pgfqpoint{2.365890in}{0.683483in}}%
\pgfpathlineto{\pgfqpoint{2.401891in}{0.699117in}}%
\pgfpathlineto{\pgfqpoint{2.432529in}{0.694650in}}%
\pgfpathlineto{\pgfqpoint{2.459280in}{0.721451in}}%
\pgfpathlineto{\pgfqpoint{2.483095in}{0.701351in}}%
\pgfpathlineto{\pgfqpoint{2.504617in}{0.714751in}}%
\pgfpathlineto{\pgfqpoint{2.524307in}{0.692417in}}%
\pgfpathlineto{\pgfqpoint{2.542505in}{0.710284in}}%
\pgfpathlineto{\pgfqpoint{2.559469in}{0.701351in}}%
\pgfpathlineto{\pgfqpoint{2.575277in}{0.676783in}}%
\pgfpathlineto{\pgfqpoint{2.590066in}{0.676783in}}%
\pgfpathlineto{\pgfqpoint{2.603999in}{0.692417in}}%
\pgfpathlineto{\pgfqpoint{2.617204in}{0.703584in}}%
\pgfpathlineto{\pgfqpoint{2.629788in}{0.692417in}}%
\pgfpathlineto{\pgfqpoint{2.641838in}{0.696884in}}%
\pgfpathlineto{\pgfqpoint{2.653426in}{0.683483in}}%
\pgfpathlineto{\pgfqpoint{2.664614in}{0.701351in}}%
\pgfpathlineto{\pgfqpoint{2.675455in}{0.712518in}}%
\pgfpathlineto{\pgfqpoint{2.685993in}{0.699117in}}%
\pgfpathlineto{\pgfqpoint{2.696268in}{0.701351in}}%
\pgfpathlineto{\pgfqpoint{2.706314in}{0.687950in}}%
\pgfpathlineto{\pgfqpoint{2.716161in}{0.699117in}}%
\pgfpathlineto{\pgfqpoint{2.725834in}{0.683483in}}%
\pgfpathlineto{\pgfqpoint{2.735355in}{0.696884in}}%
\pgfpathlineto{\pgfqpoint{2.744748in}{0.705817in}}%
\pgfpathlineto{\pgfqpoint{2.754034in}{0.712518in}}%
\pgfpathlineto{\pgfqpoint{2.763232in}{0.714751in}}%
\pgfpathlineto{\pgfqpoint{2.772361in}{0.687950in}}%
\pgfpathlineto{\pgfqpoint{2.781438in}{0.701351in}}%
\pgfpathlineto{\pgfqpoint{2.790479in}{0.694650in}}%
\pgfpathlineto{\pgfqpoint{2.799499in}{0.690184in}}%
\pgfpathlineto{\pgfqpoint{2.808513in}{0.701351in}}%
\pgfpathlineto{\pgfqpoint{2.817536in}{0.703584in}}%
\pgfpathlineto{\pgfqpoint{2.826581in}{0.705817in}}%
\pgfpathlineto{\pgfqpoint{2.835662in}{0.687950in}}%
\pgfpathlineto{\pgfqpoint{2.844794in}{0.721451in}}%
\pgfpathlineto{\pgfqpoint{2.853989in}{0.694650in}}%
\pgfpathlineto{\pgfqpoint{2.863261in}{0.710284in}}%
\pgfpathlineto{\pgfqpoint{2.872623in}{0.750485in}}%
\pgfpathlineto{\pgfqpoint{2.882090in}{0.714751in}}%
\pgfpathlineto{\pgfqpoint{2.891676in}{0.725918in}}%
\pgfpathlineto{\pgfqpoint{2.901394in}{0.708051in}}%
\pgfpathlineto{\pgfqpoint{2.911259in}{0.725918in}}%
\pgfpathlineto{\pgfqpoint{2.921286in}{0.754952in}}%
\pgfpathlineto{\pgfqpoint{2.931490in}{0.730385in}}%
\pgfpathlineto{\pgfqpoint{2.941872in}{0.746019in}}%
\pgfpathlineto{\pgfqpoint{2.952152in}{0.748252in}}%
\pgfpathlineto{\pgfqpoint{2.962294in}{0.777286in}}%
\pgfpathlineto{\pgfqpoint{2.972342in}{0.746019in}}%
\pgfpathlineto{\pgfqpoint{2.982336in}{0.741552in}}%
\pgfpathlineto{\pgfqpoint{2.992313in}{0.743785in}}%
\pgfpathlineto{\pgfqpoint{3.002305in}{0.754952in}}%
\pgfpathlineto{\pgfqpoint{3.012345in}{0.777286in}}%
\pgfpathlineto{\pgfqpoint{3.022461in}{0.772819in}}%
\pgfpathlineto{\pgfqpoint{3.032682in}{0.792920in}}%
\pgfpathlineto{\pgfqpoint{3.043013in}{0.781753in}}%
\pgfpathlineto{\pgfqpoint{3.053481in}{0.772819in}}%
\pgfpathlineto{\pgfqpoint{3.064109in}{0.830888in}}%
\pgfpathlineto{\pgfqpoint{3.074923in}{0.801854in}}%
\pgfpathlineto{\pgfqpoint{3.085943in}{0.770586in}}%
\pgfpathlineto{\pgfqpoint{3.097192in}{0.790687in}}%
\pgfpathlineto{\pgfqpoint{3.108691in}{0.790687in}}%
\pgfpathlineto{\pgfqpoint{3.120460in}{0.759419in}}%
\pgfpathlineto{\pgfqpoint{3.132518in}{0.788453in}}%
\pgfpathlineto{\pgfqpoint{3.144884in}{0.781753in}}%
\pgfpathlineto{\pgfqpoint{3.157575in}{0.770586in}}%
\pgfpathlineto{\pgfqpoint{3.170607in}{0.790687in}}%
\pgfpathlineto{\pgfqpoint{3.183997in}{0.795154in}}%
\pgfpathlineto{\pgfqpoint{3.197758in}{0.824188in}}%
\pgfpathlineto{\pgfqpoint{3.211906in}{0.772819in}}%
\pgfpathlineto{\pgfqpoint{3.226453in}{0.824188in}}%
\pgfpathlineto{\pgfqpoint{3.241412in}{0.792920in}}%
\pgfpathlineto{\pgfqpoint{3.256797in}{0.817488in}}%
\pgfpathlineto{\pgfqpoint{3.272618in}{0.819721in}}%
\pgfpathlineto{\pgfqpoint{3.288887in}{0.806321in}}%
\pgfpathlineto{\pgfqpoint{3.305618in}{0.810787in}}%
\pgfpathlineto{\pgfqpoint{3.322821in}{0.766119in}}%
\pgfpathlineto{\pgfqpoint{3.340509in}{0.797387in}}%
\pgfpathlineto{\pgfqpoint{3.358683in}{0.801854in}}%
\pgfpathlineto{\pgfqpoint{3.377346in}{0.833121in}}%
\pgfpathlineto{\pgfqpoint{3.396510in}{0.817488in}}%
\pgfpathlineto{\pgfqpoint{3.416187in}{0.804087in}}%
\pgfpathlineto{\pgfqpoint{3.436392in}{0.819721in}}%
\pgfpathlineto{\pgfqpoint{3.457140in}{0.815254in}}%
\pgfpathlineto{\pgfqpoint{3.478447in}{0.817488in}}%
\pgfpathlineto{\pgfqpoint{3.500330in}{0.846522in}}%
\pgfpathlineto{\pgfqpoint{3.522807in}{0.786220in}}%
\pgfpathlineto{\pgfqpoint{3.545882in}{0.808554in}}%
\pgfpathlineto{\pgfqpoint{3.569559in}{0.839822in}}%
\pgfpathlineto{\pgfqpoint{3.593858in}{0.783986in}}%
\pgfpathlineto{\pgfqpoint{3.618798in}{0.815254in}}%
\pgfpathlineto{\pgfqpoint{3.644399in}{0.817488in}}%
\pgfpathlineto{\pgfqpoint{3.648786in}{0.893423in}}%
\pgfusepath{stroke}%
\end{pgfscope}%
\begin{pgfscope}%
\pgfpathrectangle{\pgfqpoint{0.511159in}{0.469412in}}{\pgfqpoint{3.287038in}{2.449373in}}%
\pgfusepath{clip}%
\pgfsetrectcap%
\pgfsetroundjoin%
\pgfsetlinewidth{1.505625pt}%
\definecolor{currentstroke}{rgb}{0.839216,0.152941,0.156863}%
\pgfsetstrokecolor{currentstroke}%
\pgfsetdash{}{0pt}%
\pgfpathmoveto{\pgfqpoint{0.660570in}{0.580747in}}%
\pgfpathlineto{\pgfqpoint{0.859784in}{0.667850in}}%
\pgfpathlineto{\pgfqpoint{1.058999in}{0.616481in}}%
\pgfpathlineto{\pgfqpoint{1.258213in}{0.600847in}}%
\pgfpathlineto{\pgfqpoint{1.457427in}{0.600847in}}%
\pgfpathlineto{\pgfqpoint{1.656642in}{0.580747in}}%
\pgfpathlineto{\pgfqpoint{1.855856in}{0.585214in}}%
\pgfpathlineto{\pgfqpoint{2.055071in}{0.585214in}}%
\pgfpathlineto{\pgfqpoint{2.187680in}{0.589680in}}%
\pgfpathlineto{\pgfqpoint{2.266052in}{0.587447in}}%
\pgfpathlineto{\pgfqpoint{2.322114in}{0.589680in}}%
\pgfpathlineto{\pgfqpoint{2.365890in}{0.589680in}}%
\pgfpathlineto{\pgfqpoint{2.401891in}{0.589680in}}%
\pgfpathlineto{\pgfqpoint{2.432529in}{0.589680in}}%
\pgfpathlineto{\pgfqpoint{2.459280in}{0.600847in}}%
\pgfpathlineto{\pgfqpoint{2.483095in}{0.587447in}}%
\pgfpathlineto{\pgfqpoint{2.504617in}{0.589680in}}%
\pgfpathlineto{\pgfqpoint{2.524307in}{0.589680in}}%
\pgfpathlineto{\pgfqpoint{2.542505in}{0.591914in}}%
\pgfpathlineto{\pgfqpoint{2.559469in}{0.603081in}}%
\pgfpathlineto{\pgfqpoint{2.575277in}{0.582980in}}%
\pgfpathlineto{\pgfqpoint{2.590066in}{0.589680in}}%
\pgfpathlineto{\pgfqpoint{2.603999in}{0.589680in}}%
\pgfpathlineto{\pgfqpoint{2.617204in}{0.594147in}}%
\pgfpathlineto{\pgfqpoint{2.629788in}{0.582980in}}%
\pgfpathlineto{\pgfqpoint{2.641838in}{0.589680in}}%
\pgfpathlineto{\pgfqpoint{2.653426in}{0.585214in}}%
\pgfpathlineto{\pgfqpoint{2.664614in}{0.591914in}}%
\pgfpathlineto{\pgfqpoint{2.675455in}{0.582980in}}%
\pgfpathlineto{\pgfqpoint{2.685993in}{0.589680in}}%
\pgfpathlineto{\pgfqpoint{2.696268in}{0.589680in}}%
\pgfpathlineto{\pgfqpoint{2.706314in}{0.585214in}}%
\pgfpathlineto{\pgfqpoint{2.716161in}{0.589680in}}%
\pgfpathlineto{\pgfqpoint{2.725834in}{0.582980in}}%
\pgfpathlineto{\pgfqpoint{2.735355in}{0.591914in}}%
\pgfpathlineto{\pgfqpoint{2.744748in}{0.585214in}}%
\pgfpathlineto{\pgfqpoint{2.754034in}{0.589680in}}%
\pgfpathlineto{\pgfqpoint{2.763232in}{0.585214in}}%
\pgfpathlineto{\pgfqpoint{2.772361in}{0.596381in}}%
\pgfpathlineto{\pgfqpoint{2.781438in}{0.594147in}}%
\pgfpathlineto{\pgfqpoint{2.790479in}{0.603081in}}%
\pgfpathlineto{\pgfqpoint{2.799499in}{0.596381in}}%
\pgfpathlineto{\pgfqpoint{2.808513in}{0.589680in}}%
\pgfpathlineto{\pgfqpoint{2.817536in}{0.594147in}}%
\pgfpathlineto{\pgfqpoint{2.826581in}{0.589680in}}%
\pgfpathlineto{\pgfqpoint{2.835662in}{0.580747in}}%
\pgfpathlineto{\pgfqpoint{2.844794in}{0.594147in}}%
\pgfpathlineto{\pgfqpoint{2.853989in}{0.585214in}}%
\pgfpathlineto{\pgfqpoint{2.863261in}{0.587447in}}%
\pgfpathlineto{\pgfqpoint{2.872623in}{0.596381in}}%
\pgfpathlineto{\pgfqpoint{2.882090in}{0.603081in}}%
\pgfpathlineto{\pgfqpoint{2.891676in}{0.596381in}}%
\pgfpathlineto{\pgfqpoint{2.901394in}{0.591914in}}%
\pgfpathlineto{\pgfqpoint{2.911259in}{0.591914in}}%
\pgfpathlineto{\pgfqpoint{2.921286in}{0.607548in}}%
\pgfpathlineto{\pgfqpoint{2.931490in}{0.594147in}}%
\pgfpathlineto{\pgfqpoint{2.941872in}{0.591914in}}%
\pgfpathlineto{\pgfqpoint{2.952152in}{0.591914in}}%
\pgfpathlineto{\pgfqpoint{2.962294in}{0.609781in}}%
\pgfpathlineto{\pgfqpoint{2.972342in}{0.594147in}}%
\pgfpathlineto{\pgfqpoint{2.982336in}{0.600847in}}%
\pgfpathlineto{\pgfqpoint{2.992313in}{0.607548in}}%
\pgfpathlineto{\pgfqpoint{3.002305in}{0.612014in}}%
\pgfpathlineto{\pgfqpoint{3.012345in}{0.594147in}}%
\pgfpathlineto{\pgfqpoint{3.022461in}{0.596381in}}%
\pgfpathlineto{\pgfqpoint{3.032682in}{0.609781in}}%
\pgfpathlineto{\pgfqpoint{3.043013in}{0.607548in}}%
\pgfpathlineto{\pgfqpoint{3.053481in}{0.603081in}}%
\pgfpathlineto{\pgfqpoint{3.064109in}{0.609781in}}%
\pgfpathlineto{\pgfqpoint{3.074923in}{0.596381in}}%
\pgfpathlineto{\pgfqpoint{3.085943in}{0.600847in}}%
\pgfpathlineto{\pgfqpoint{3.097192in}{0.609781in}}%
\pgfpathlineto{\pgfqpoint{3.108691in}{0.612014in}}%
\pgfpathlineto{\pgfqpoint{3.120460in}{0.609781in}}%
\pgfpathlineto{\pgfqpoint{3.132518in}{0.594147in}}%
\pgfpathlineto{\pgfqpoint{3.144884in}{0.603081in}}%
\pgfpathlineto{\pgfqpoint{3.157575in}{0.609781in}}%
\pgfpathlineto{\pgfqpoint{3.170607in}{0.609781in}}%
\pgfpathlineto{\pgfqpoint{3.183997in}{0.596381in}}%
\pgfpathlineto{\pgfqpoint{3.197758in}{0.632115in}}%
\pgfpathlineto{\pgfqpoint{3.211906in}{0.605314in}}%
\pgfpathlineto{\pgfqpoint{3.226453in}{0.612014in}}%
\pgfpathlineto{\pgfqpoint{3.241412in}{0.616481in}}%
\pgfpathlineto{\pgfqpoint{3.256797in}{0.607548in}}%
\pgfpathlineto{\pgfqpoint{3.272618in}{0.596381in}}%
\pgfpathlineto{\pgfqpoint{3.288887in}{0.612014in}}%
\pgfpathlineto{\pgfqpoint{3.305618in}{0.614248in}}%
\pgfpathlineto{\pgfqpoint{3.322821in}{0.591914in}}%
\pgfpathlineto{\pgfqpoint{3.340509in}{0.612014in}}%
\pgfpathlineto{\pgfqpoint{3.358683in}{0.612014in}}%
\pgfpathlineto{\pgfqpoint{3.377346in}{0.614248in}}%
\pgfpathlineto{\pgfqpoint{3.396510in}{0.603081in}}%
\pgfpathlineto{\pgfqpoint{3.416187in}{0.605314in}}%
\pgfpathlineto{\pgfqpoint{3.436392in}{0.612014in}}%
\pgfpathlineto{\pgfqpoint{3.457140in}{0.607548in}}%
\pgfpathlineto{\pgfqpoint{3.478447in}{0.623181in}}%
\pgfpathlineto{\pgfqpoint{3.500330in}{0.616481in}}%
\pgfpathlineto{\pgfqpoint{3.522807in}{0.607548in}}%
\pgfpathlineto{\pgfqpoint{3.545882in}{0.609781in}}%
\pgfpathlineto{\pgfqpoint{3.569559in}{0.607548in}}%
\pgfpathlineto{\pgfqpoint{3.593858in}{0.607548in}}%
\pgfpathlineto{\pgfqpoint{3.618798in}{0.605314in}}%
\pgfpathlineto{\pgfqpoint{3.644399in}{0.618715in}}%
\pgfpathlineto{\pgfqpoint{3.648786in}{0.643282in}}%
\pgfusepath{stroke}%
\end{pgfscope}%
\begin{pgfscope}%
\pgfsetrectcap%
\pgfsetmiterjoin%
\pgfsetlinewidth{0.803000pt}%
\definecolor{currentstroke}{rgb}{0.000000,0.000000,0.000000}%
\pgfsetstrokecolor{currentstroke}%
\pgfsetdash{}{0pt}%
\pgfpathmoveto{\pgfqpoint{0.511159in}{0.469412in}}%
\pgfpathlineto{\pgfqpoint{0.511159in}{2.918785in}}%
\pgfusepath{stroke}%
\end{pgfscope}%
\begin{pgfscope}%
\pgfsetrectcap%
\pgfsetmiterjoin%
\pgfsetlinewidth{0.803000pt}%
\definecolor{currentstroke}{rgb}{0.000000,0.000000,0.000000}%
\pgfsetstrokecolor{currentstroke}%
\pgfsetdash{}{0pt}%
\pgfpathmoveto{\pgfqpoint{3.798197in}{0.469412in}}%
\pgfpathlineto{\pgfqpoint{3.798197in}{2.918785in}}%
\pgfusepath{stroke}%
\end{pgfscope}%
\begin{pgfscope}%
\pgfsetrectcap%
\pgfsetmiterjoin%
\pgfsetlinewidth{0.803000pt}%
\definecolor{currentstroke}{rgb}{0.000000,0.000000,0.000000}%
\pgfsetstrokecolor{currentstroke}%
\pgfsetdash{}{0pt}%
\pgfpathmoveto{\pgfqpoint{0.511159in}{0.469412in}}%
\pgfpathlineto{\pgfqpoint{3.798197in}{0.469412in}}%
\pgfusepath{stroke}%
\end{pgfscope}%
\begin{pgfscope}%
\pgfsetrectcap%
\pgfsetmiterjoin%
\pgfsetlinewidth{0.803000pt}%
\definecolor{currentstroke}{rgb}{0.000000,0.000000,0.000000}%
\pgfsetstrokecolor{currentstroke}%
\pgfsetdash{}{0pt}%
\pgfpathmoveto{\pgfqpoint{0.511159in}{2.918785in}}%
\pgfpathlineto{\pgfqpoint{3.798197in}{2.918785in}}%
\pgfusepath{stroke}%
\end{pgfscope}%
\begin{pgfscope}%
\pgfsetrectcap%
\pgfsetroundjoin%
\pgfsetlinewidth{1.505625pt}%
\definecolor{currentstroke}{rgb}{0.121569,0.466667,0.705882}%
\pgfsetstrokecolor{currentstroke}%
\pgfsetdash{}{0pt}%
\pgfpathmoveto{\pgfqpoint{1.876781in}{1.946963in}}%
\pgfpathlineto{\pgfqpoint{2.099003in}{1.946963in}}%
\pgfusepath{stroke}%
\end{pgfscope}%
\begin{pgfscope}%
\definecolor{textcolor}{rgb}{0.000000,0.000000,0.000000}%
\pgfsetstrokecolor{textcolor}%
\pgfsetfillcolor{textcolor}%
\pgftext[x=2.187892in,y=1.908074in,left,base]{\color{textcolor}\rmfamily\fontsize{8.000000}{9.600000}\selectfont 1e-2}%
\end{pgfscope}%
\begin{pgfscope}%
\pgfsetrectcap%
\pgfsetroundjoin%
\pgfsetlinewidth{1.505625pt}%
\definecolor{currentstroke}{rgb}{1.000000,0.498039,0.054902}%
\pgfsetstrokecolor{currentstroke}%
\pgfsetdash{}{0pt}%
\pgfpathmoveto{\pgfqpoint{1.876781in}{1.783877in}}%
\pgfpathlineto{\pgfqpoint{2.099003in}{1.783877in}}%
\pgfusepath{stroke}%
\end{pgfscope}%
\begin{pgfscope}%
\definecolor{textcolor}{rgb}{0.000000,0.000000,0.000000}%
\pgfsetstrokecolor{textcolor}%
\pgfsetfillcolor{textcolor}%
\pgftext[x=2.187892in,y=1.744988in,left,base]{\color{textcolor}\rmfamily\fontsize{8.000000}{9.600000}\selectfont 1e-3}%
\end{pgfscope}%
\begin{pgfscope}%
\pgfsetrectcap%
\pgfsetroundjoin%
\pgfsetlinewidth{1.505625pt}%
\definecolor{currentstroke}{rgb}{0.172549,0.627451,0.172549}%
\pgfsetstrokecolor{currentstroke}%
\pgfsetdash{}{0pt}%
\pgfpathmoveto{\pgfqpoint{1.876781in}{1.620791in}}%
\pgfpathlineto{\pgfqpoint{2.099003in}{1.620791in}}%
\pgfusepath{stroke}%
\end{pgfscope}%
\begin{pgfscope}%
\definecolor{textcolor}{rgb}{0.000000,0.000000,0.000000}%
\pgfsetstrokecolor{textcolor}%
\pgfsetfillcolor{textcolor}%
\pgftext[x=2.187892in,y=1.581902in,left,base]{\color{textcolor}\rmfamily\fontsize{8.000000}{9.600000}\selectfont 1e-4}%
\end{pgfscope}%
\begin{pgfscope}%
\pgfsetrectcap%
\pgfsetroundjoin%
\pgfsetlinewidth{1.505625pt}%
\definecolor{currentstroke}{rgb}{0.839216,0.152941,0.156863}%
\pgfsetstrokecolor{currentstroke}%
\pgfsetdash{}{0pt}%
\pgfpathmoveto{\pgfqpoint{1.876781in}{1.457705in}}%
\pgfpathlineto{\pgfqpoint{2.099003in}{1.457705in}}%
\pgfusepath{stroke}%
\end{pgfscope}%
\begin{pgfscope}%
\definecolor{textcolor}{rgb}{0.000000,0.000000,0.000000}%
\pgfsetstrokecolor{textcolor}%
\pgfsetfillcolor{textcolor}%
\pgftext[x=2.187892in,y=1.418816in,left,base]{\color{textcolor}\rmfamily\fontsize{8.000000}{9.600000}\selectfont 1e-5}%
\end{pgfscope}%
\end{pgfpicture}%
\makeatother%
\endgroup%

        \caption{Particle property plugin.}
        \label{fig:cache_usage_particle_property}
        \end{subfigure}
    \hfill
    \begin{subfigure}{0.49\textwidth}
        \centering
        %% Creator: Matplotlib, PGF backend
%%
%% To include the figure in your LaTeX document, write
%%   \input{<filename>.pgf}
%%
%% Make sure the required packages are loaded in your preamble
%%   \usepackage{pgf}
%%
%% Figures using additional raster images can only be included by \input if
%% they are in the same directory as the main LaTeX file. For loading figures
%% from other directories you can use the `import` package
%%   \usepackage{import}
%% and then include the figures with
%%   \import{<path to file>}{<filename>.pgf}
%%
%% Matplotlib used the following preamble
%%   \usepackage{fontspec}
%%   \setmainfont{DejaVuSerif.ttf}[Path=/home/connor/.local/lib/python3.8/site-packages/matplotlib/mpl-data/fonts/ttf/]
%%   \setsansfont{DejaVuSans.ttf}[Path=/home/connor/.local/lib/python3.8/site-packages/matplotlib/mpl-data/fonts/ttf/]
%%   \setmonofont{DejaVuSansMono.ttf}[Path=/home/connor/.local/lib/python3.8/site-packages/matplotlib/mpl-data/fonts/ttf/]
%%
\begingroup%
\makeatletter%
\begin{pgfpicture}%
\pgfpathrectangle{\pgfpointorigin}{\pgfqpoint{2.986359in}{3.942354in}}%
\pgfusepath{use as bounding box, clip}%
\begin{pgfscope}%
\pgfsetbuttcap%
\pgfsetmiterjoin%
\definecolor{currentfill}{rgb}{1.000000,1.000000,1.000000}%
\pgfsetfillcolor{currentfill}%
\pgfsetlinewidth{0.000000pt}%
\definecolor{currentstroke}{rgb}{1.000000,1.000000,1.000000}%
\pgfsetstrokecolor{currentstroke}%
\pgfsetdash{}{0pt}%
\pgfpathmoveto{\pgfqpoint{0.000000in}{0.000000in}}%
\pgfpathlineto{\pgfqpoint{2.986359in}{0.000000in}}%
\pgfpathlineto{\pgfqpoint{2.986359in}{3.942354in}}%
\pgfpathlineto{\pgfqpoint{0.000000in}{3.942354in}}%
\pgfpathclose%
\pgfusepath{fill}%
\end{pgfscope}%
\begin{pgfscope}%
\pgfsetbuttcap%
\pgfsetmiterjoin%
\definecolor{currentfill}{rgb}{1.000000,1.000000,1.000000}%
\pgfsetfillcolor{currentfill}%
\pgfsetlinewidth{0.000000pt}%
\definecolor{currentstroke}{rgb}{0.000000,0.000000,0.000000}%
\pgfsetstrokecolor{currentstroke}%
\pgfsetstrokeopacity{0.000000}%
\pgfsetdash{}{0pt}%
\pgfpathmoveto{\pgfqpoint{0.629216in}{1.472354in}}%
\pgfpathlineto{\pgfqpoint{2.886359in}{1.472354in}}%
\pgfpathlineto{\pgfqpoint{2.886359in}{3.842354in}}%
\pgfpathlineto{\pgfqpoint{0.629216in}{3.842354in}}%
\pgfpathclose%
\pgfusepath{fill}%
\end{pgfscope}%
\begin{pgfscope}%
\pgfsetbuttcap%
\pgfsetroundjoin%
\definecolor{currentfill}{rgb}{0.000000,0.000000,0.000000}%
\pgfsetfillcolor{currentfill}%
\pgfsetlinewidth{0.803000pt}%
\definecolor{currentstroke}{rgb}{0.000000,0.000000,0.000000}%
\pgfsetstrokecolor{currentstroke}%
\pgfsetdash{}{0pt}%
\pgfsys@defobject{currentmarker}{\pgfqpoint{0.000000in}{-0.048611in}}{\pgfqpoint{0.000000in}{0.000000in}}{%
\pgfpathmoveto{\pgfqpoint{0.000000in}{0.000000in}}%
\pgfpathlineto{\pgfqpoint{0.000000in}{-0.048611in}}%
\pgfusepath{stroke,fill}%
}%
\begin{pgfscope}%
\pgfsys@transformshift{0.731814in}{1.472354in}%
\pgfsys@useobject{currentmarker}{}%
\end{pgfscope}%
\end{pgfscope}%
\begin{pgfscope}%
\definecolor{textcolor}{rgb}{0.000000,0.000000,0.000000}%
\pgfsetstrokecolor{textcolor}%
\pgfsetfillcolor{textcolor}%
\pgftext[x=0.731814in,y=1.375132in,,top]{\color{textcolor}\rmfamily\fontsize{8.000000}{9.600000}\selectfont \(\displaystyle 0.0\)}%
\end{pgfscope}%
\begin{pgfscope}%
\pgfsetbuttcap%
\pgfsetroundjoin%
\definecolor{currentfill}{rgb}{0.000000,0.000000,0.000000}%
\pgfsetfillcolor{currentfill}%
\pgfsetlinewidth{0.803000pt}%
\definecolor{currentstroke}{rgb}{0.000000,0.000000,0.000000}%
\pgfsetstrokecolor{currentstroke}%
\pgfsetdash{}{0pt}%
\pgfsys@defobject{currentmarker}{\pgfqpoint{0.000000in}{-0.048611in}}{\pgfqpoint{0.000000in}{0.000000in}}{%
\pgfpathmoveto{\pgfqpoint{0.000000in}{0.000000in}}%
\pgfpathlineto{\pgfqpoint{0.000000in}{-0.048611in}}%
\pgfusepath{stroke,fill}%
}%
\begin{pgfscope}%
\pgfsys@transformshift{1.415796in}{1.472354in}%
\pgfsys@useobject{currentmarker}{}%
\end{pgfscope}%
\end{pgfscope}%
\begin{pgfscope}%
\definecolor{textcolor}{rgb}{0.000000,0.000000,0.000000}%
\pgfsetstrokecolor{textcolor}%
\pgfsetfillcolor{textcolor}%
\pgftext[x=1.415796in,y=1.375132in,,top]{\color{textcolor}\rmfamily\fontsize{8.000000}{9.600000}\selectfont \(\displaystyle 0.5\)}%
\end{pgfscope}%
\begin{pgfscope}%
\pgfsetbuttcap%
\pgfsetroundjoin%
\definecolor{currentfill}{rgb}{0.000000,0.000000,0.000000}%
\pgfsetfillcolor{currentfill}%
\pgfsetlinewidth{0.803000pt}%
\definecolor{currentstroke}{rgb}{0.000000,0.000000,0.000000}%
\pgfsetstrokecolor{currentstroke}%
\pgfsetdash{}{0pt}%
\pgfsys@defobject{currentmarker}{\pgfqpoint{0.000000in}{-0.048611in}}{\pgfqpoint{0.000000in}{0.000000in}}{%
\pgfpathmoveto{\pgfqpoint{0.000000in}{0.000000in}}%
\pgfpathlineto{\pgfqpoint{0.000000in}{-0.048611in}}%
\pgfusepath{stroke,fill}%
}%
\begin{pgfscope}%
\pgfsys@transformshift{2.099779in}{1.472354in}%
\pgfsys@useobject{currentmarker}{}%
\end{pgfscope}%
\end{pgfscope}%
\begin{pgfscope}%
\definecolor{textcolor}{rgb}{0.000000,0.000000,0.000000}%
\pgfsetstrokecolor{textcolor}%
\pgfsetfillcolor{textcolor}%
\pgftext[x=2.099779in,y=1.375132in,,top]{\color{textcolor}\rmfamily\fontsize{8.000000}{9.600000}\selectfont \(\displaystyle 1.0\)}%
\end{pgfscope}%
\begin{pgfscope}%
\pgfsetbuttcap%
\pgfsetroundjoin%
\definecolor{currentfill}{rgb}{0.000000,0.000000,0.000000}%
\pgfsetfillcolor{currentfill}%
\pgfsetlinewidth{0.803000pt}%
\definecolor{currentstroke}{rgb}{0.000000,0.000000,0.000000}%
\pgfsetstrokecolor{currentstroke}%
\pgfsetdash{}{0pt}%
\pgfsys@defobject{currentmarker}{\pgfqpoint{0.000000in}{-0.048611in}}{\pgfqpoint{0.000000in}{0.000000in}}{%
\pgfpathmoveto{\pgfqpoint{0.000000in}{0.000000in}}%
\pgfpathlineto{\pgfqpoint{0.000000in}{-0.048611in}}%
\pgfusepath{stroke,fill}%
}%
\begin{pgfscope}%
\pgfsys@transformshift{2.783761in}{1.472354in}%
\pgfsys@useobject{currentmarker}{}%
\end{pgfscope}%
\end{pgfscope}%
\begin{pgfscope}%
\definecolor{textcolor}{rgb}{0.000000,0.000000,0.000000}%
\pgfsetstrokecolor{textcolor}%
\pgfsetfillcolor{textcolor}%
\pgftext[x=2.783761in,y=1.375132in,,top]{\color{textcolor}\rmfamily\fontsize{8.000000}{9.600000}\selectfont \(\displaystyle 1.5\)}%
\end{pgfscope}%
\begin{pgfscope}%
\pgfsetbuttcap%
\pgfsetroundjoin%
\definecolor{currentfill}{rgb}{0.000000,0.000000,0.000000}%
\pgfsetfillcolor{currentfill}%
\pgfsetlinewidth{0.803000pt}%
\definecolor{currentstroke}{rgb}{0.000000,0.000000,0.000000}%
\pgfsetstrokecolor{currentstroke}%
\pgfsetdash{}{0pt}%
\pgfsys@defobject{currentmarker}{\pgfqpoint{-0.048611in}{0.000000in}}{\pgfqpoint{0.000000in}{0.000000in}}{%
\pgfpathmoveto{\pgfqpoint{0.000000in}{0.000000in}}%
\pgfpathlineto{\pgfqpoint{-0.048611in}{0.000000in}}%
\pgfusepath{stroke,fill}%
}%
\begin{pgfscope}%
\pgfsys@transformshift{0.629216in}{1.743211in}%
\pgfsys@useobject{currentmarker}{}%
\end{pgfscope}%
\end{pgfscope}%
\begin{pgfscope}%
\definecolor{textcolor}{rgb}{0.000000,0.000000,0.000000}%
\pgfsetstrokecolor{textcolor}%
\pgfsetfillcolor{textcolor}%
\pgftext[x=0.322114in,y=1.701002in,left,base]{\color{textcolor}\rmfamily\fontsize{8.000000}{9.600000}\selectfont \(\displaystyle 98.5\)}%
\end{pgfscope}%
\begin{pgfscope}%
\pgfsetbuttcap%
\pgfsetroundjoin%
\definecolor{currentfill}{rgb}{0.000000,0.000000,0.000000}%
\pgfsetfillcolor{currentfill}%
\pgfsetlinewidth{0.803000pt}%
\definecolor{currentstroke}{rgb}{0.000000,0.000000,0.000000}%
\pgfsetstrokecolor{currentstroke}%
\pgfsetdash{}{0pt}%
\pgfsys@defobject{currentmarker}{\pgfqpoint{-0.048611in}{0.000000in}}{\pgfqpoint{0.000000in}{0.000000in}}{%
\pgfpathmoveto{\pgfqpoint{0.000000in}{0.000000in}}%
\pgfpathlineto{\pgfqpoint{-0.048611in}{0.000000in}}%
\pgfusepath{stroke,fill}%
}%
\begin{pgfscope}%
\pgfsys@transformshift{0.629216in}{2.420354in}%
\pgfsys@useobject{currentmarker}{}%
\end{pgfscope}%
\end{pgfscope}%
\begin{pgfscope}%
\definecolor{textcolor}{rgb}{0.000000,0.000000,0.000000}%
\pgfsetstrokecolor{textcolor}%
\pgfsetfillcolor{textcolor}%
\pgftext[x=0.322114in,y=2.378145in,left,base]{\color{textcolor}\rmfamily\fontsize{8.000000}{9.600000}\selectfont \(\displaystyle 99.0\)}%
\end{pgfscope}%
\begin{pgfscope}%
\pgfsetbuttcap%
\pgfsetroundjoin%
\definecolor{currentfill}{rgb}{0.000000,0.000000,0.000000}%
\pgfsetfillcolor{currentfill}%
\pgfsetlinewidth{0.803000pt}%
\definecolor{currentstroke}{rgb}{0.000000,0.000000,0.000000}%
\pgfsetstrokecolor{currentstroke}%
\pgfsetdash{}{0pt}%
\pgfsys@defobject{currentmarker}{\pgfqpoint{-0.048611in}{0.000000in}}{\pgfqpoint{0.000000in}{0.000000in}}{%
\pgfpathmoveto{\pgfqpoint{0.000000in}{0.000000in}}%
\pgfpathlineto{\pgfqpoint{-0.048611in}{0.000000in}}%
\pgfusepath{stroke,fill}%
}%
\begin{pgfscope}%
\pgfsys@transformshift{0.629216in}{3.097497in}%
\pgfsys@useobject{currentmarker}{}%
\end{pgfscope}%
\end{pgfscope}%
\begin{pgfscope}%
\definecolor{textcolor}{rgb}{0.000000,0.000000,0.000000}%
\pgfsetstrokecolor{textcolor}%
\pgfsetfillcolor{textcolor}%
\pgftext[x=0.322114in,y=3.055288in,left,base]{\color{textcolor}\rmfamily\fontsize{8.000000}{9.600000}\selectfont \(\displaystyle 99.5\)}%
\end{pgfscope}%
\begin{pgfscope}%
\pgfsetbuttcap%
\pgfsetroundjoin%
\definecolor{currentfill}{rgb}{0.000000,0.000000,0.000000}%
\pgfsetfillcolor{currentfill}%
\pgfsetlinewidth{0.803000pt}%
\definecolor{currentstroke}{rgb}{0.000000,0.000000,0.000000}%
\pgfsetstrokecolor{currentstroke}%
\pgfsetdash{}{0pt}%
\pgfsys@defobject{currentmarker}{\pgfqpoint{-0.048611in}{0.000000in}}{\pgfqpoint{0.000000in}{0.000000in}}{%
\pgfpathmoveto{\pgfqpoint{0.000000in}{0.000000in}}%
\pgfpathlineto{\pgfqpoint{-0.048611in}{0.000000in}}%
\pgfusepath{stroke,fill}%
}%
\begin{pgfscope}%
\pgfsys@transformshift{0.629216in}{3.774640in}%
\pgfsys@useobject{currentmarker}{}%
\end{pgfscope}%
\end{pgfscope}%
\begin{pgfscope}%
\definecolor{textcolor}{rgb}{0.000000,0.000000,0.000000}%
\pgfsetstrokecolor{textcolor}%
\pgfsetfillcolor{textcolor}%
\pgftext[x=0.263086in,y=3.732430in,left,base]{\color{textcolor}\rmfamily\fontsize{8.000000}{9.600000}\selectfont \(\displaystyle 100.0\)}%
\end{pgfscope}%
\begin{pgfscope}%
\definecolor{textcolor}{rgb}{0.000000,0.000000,0.000000}%
\pgfsetstrokecolor{textcolor}%
\pgfsetfillcolor{textcolor}%
\pgftext[x=0.207530in,y=2.657354in,,bottom,rotate=90.000000]{\color{textcolor}\rmfamily\fontsize{8.000000}{9.600000}\selectfont Hit rate (\%)}%
\end{pgfscope}%
\begin{pgfscope}%
\pgfpathrectangle{\pgfqpoint{0.629216in}{1.472354in}}{\pgfqpoint{2.257143in}{2.370000in}}%
\pgfusepath{clip}%
\pgfsetrectcap%
\pgfsetroundjoin%
\pgfsetlinewidth{1.505625pt}%
\definecolor{currentstroke}{rgb}{0.121569,0.466667,0.705882}%
\pgfsetstrokecolor{currentstroke}%
\pgfsetdash{}{0pt}%
\pgfpathmoveto{\pgfqpoint{0.731814in}{2.420354in}}%
\pgfpathlineto{\pgfqpoint{0.868610in}{3.774640in}}%
\pgfpathlineto{\pgfqpoint{1.005407in}{3.774640in}}%
\pgfpathlineto{\pgfqpoint{1.142203in}{3.774640in}}%
\pgfpathlineto{\pgfqpoint{1.279000in}{3.774640in}}%
\pgfpathlineto{\pgfqpoint{1.348261in}{3.774640in}}%
\pgfpathlineto{\pgfqpoint{1.390141in}{3.774640in}}%
\pgfpathlineto{\pgfqpoint{1.421955in}{3.774640in}}%
\pgfpathlineto{\pgfqpoint{1.448179in}{3.639211in}}%
\pgfpathlineto{\pgfqpoint{1.470824in}{3.639211in}}%
\pgfpathlineto{\pgfqpoint{1.491009in}{3.639211in}}%
\pgfpathlineto{\pgfqpoint{1.509432in}{3.639211in}}%
\pgfpathlineto{\pgfqpoint{1.526393in}{3.639211in}}%
\pgfpathlineto{\pgfqpoint{1.542042in}{3.639211in}}%
\pgfpathlineto{\pgfqpoint{1.556723in}{3.639211in}}%
\pgfpathlineto{\pgfqpoint{1.570691in}{3.639211in}}%
\pgfpathlineto{\pgfqpoint{1.584141in}{3.639211in}}%
\pgfpathlineto{\pgfqpoint{1.597229in}{3.639211in}}%
\pgfpathlineto{\pgfqpoint{1.610082in}{3.639211in}}%
\pgfpathlineto{\pgfqpoint{1.622811in}{3.639211in}}%
\pgfpathlineto{\pgfqpoint{1.635514in}{3.639211in}}%
\pgfpathlineto{\pgfqpoint{1.648280in}{3.639211in}}%
\pgfpathlineto{\pgfqpoint{1.661192in}{3.639211in}}%
\pgfpathlineto{\pgfqpoint{1.673978in}{3.639211in}}%
\pgfpathlineto{\pgfqpoint{1.686423in}{3.774640in}}%
\pgfpathlineto{\pgfqpoint{1.698685in}{3.774640in}}%
\pgfpathlineto{\pgfqpoint{1.710859in}{3.639211in}}%
\pgfpathlineto{\pgfqpoint{1.723059in}{3.774640in}}%
\pgfpathlineto{\pgfqpoint{1.735394in}{3.774640in}}%
\pgfpathlineto{\pgfqpoint{1.747957in}{3.774640in}}%
\pgfpathlineto{\pgfqpoint{1.760837in}{3.639211in}}%
\pgfpathlineto{\pgfqpoint{1.774167in}{3.639211in}}%
\pgfpathlineto{\pgfqpoint{1.788108in}{3.639211in}}%
\pgfpathlineto{\pgfqpoint{1.802741in}{3.639211in}}%
\pgfpathlineto{\pgfqpoint{1.818026in}{3.639211in}}%
\pgfpathlineto{\pgfqpoint{1.832248in}{3.639211in}}%
\pgfpathlineto{\pgfqpoint{1.848979in}{3.639211in}}%
\pgfpathlineto{\pgfqpoint{1.866427in}{3.639211in}}%
\pgfpathlineto{\pgfqpoint{1.884060in}{3.639211in}}%
\pgfpathlineto{\pgfqpoint{1.901454in}{3.639211in}}%
\pgfpathlineto{\pgfqpoint{1.919929in}{3.639211in}}%
\pgfpathlineto{\pgfqpoint{1.933485in}{3.639211in}}%
\pgfpathlineto{\pgfqpoint{1.950853in}{3.639211in}}%
\pgfpathlineto{\pgfqpoint{1.973789in}{3.639211in}}%
\pgfpathlineto{\pgfqpoint{1.999776in}{3.639211in}}%
\pgfpathlineto{\pgfqpoint{2.027650in}{3.639211in}}%
\pgfpathlineto{\pgfqpoint{2.057076in}{3.774640in}}%
\pgfpathlineto{\pgfqpoint{2.088231in}{3.774640in}}%
\pgfpathlineto{\pgfqpoint{2.121099in}{3.774640in}}%
\pgfpathlineto{\pgfqpoint{2.155838in}{3.774640in}}%
\pgfpathlineto{\pgfqpoint{2.192705in}{3.774640in}}%
\pgfpathlineto{\pgfqpoint{2.231946in}{3.774640in}}%
\pgfpathlineto{\pgfqpoint{2.273831in}{3.774640in}}%
\pgfpathlineto{\pgfqpoint{2.318472in}{3.774640in}}%
\pgfpathlineto{\pgfqpoint{2.366069in}{3.774640in}}%
\pgfpathlineto{\pgfqpoint{2.416730in}{3.774640in}}%
\pgfpathlineto{\pgfqpoint{2.470753in}{3.774640in}}%
\pgfpathlineto{\pgfqpoint{2.528321in}{3.774640in}}%
\pgfpathlineto{\pgfqpoint{2.589345in}{3.774640in}}%
\pgfpathlineto{\pgfqpoint{2.653624in}{3.774640in}}%
\pgfpathlineto{\pgfqpoint{2.721219in}{3.774640in}}%
\pgfpathlineto{\pgfqpoint{2.783761in}{3.774640in}}%
\pgfusepath{stroke}%
\end{pgfscope}%
\begin{pgfscope}%
\pgfpathrectangle{\pgfqpoint{0.629216in}{1.472354in}}{\pgfqpoint{2.257143in}{2.370000in}}%
\pgfusepath{clip}%
\pgfsetrectcap%
\pgfsetroundjoin%
\pgfsetlinewidth{1.505625pt}%
\definecolor{currentstroke}{rgb}{1.000000,0.498039,0.054902}%
\pgfsetstrokecolor{currentstroke}%
\pgfsetdash{}{0pt}%
\pgfpathmoveto{\pgfqpoint{0.802022in}{1.467354in}}%
\pgfpathlineto{\pgfqpoint{0.868610in}{3.774640in}}%
\pgfpathlineto{\pgfqpoint{1.005407in}{3.774640in}}%
\pgfpathlineto{\pgfqpoint{1.142203in}{3.774640in}}%
\pgfpathlineto{\pgfqpoint{1.279000in}{3.774640in}}%
\pgfpathlineto{\pgfqpoint{1.348261in}{3.639211in}}%
\pgfpathlineto{\pgfqpoint{1.390141in}{3.639211in}}%
\pgfpathlineto{\pgfqpoint{1.421955in}{3.639211in}}%
\pgfpathlineto{\pgfqpoint{1.448179in}{3.503783in}}%
\pgfpathlineto{\pgfqpoint{1.470824in}{3.503783in}}%
\pgfpathlineto{\pgfqpoint{1.491009in}{3.503783in}}%
\pgfpathlineto{\pgfqpoint{1.509432in}{3.503783in}}%
\pgfpathlineto{\pgfqpoint{1.526393in}{3.503783in}}%
\pgfpathlineto{\pgfqpoint{1.542042in}{3.503783in}}%
\pgfpathlineto{\pgfqpoint{1.556723in}{3.368354in}}%
\pgfpathlineto{\pgfqpoint{1.570691in}{3.368354in}}%
\pgfpathlineto{\pgfqpoint{1.584141in}{3.368354in}}%
\pgfpathlineto{\pgfqpoint{1.597229in}{3.368354in}}%
\pgfpathlineto{\pgfqpoint{1.610082in}{3.368354in}}%
\pgfpathlineto{\pgfqpoint{1.622811in}{3.368354in}}%
\pgfpathlineto{\pgfqpoint{1.635514in}{3.368354in}}%
\pgfpathlineto{\pgfqpoint{1.648280in}{3.368354in}}%
\pgfpathlineto{\pgfqpoint{1.661192in}{3.368354in}}%
\pgfpathlineto{\pgfqpoint{1.673978in}{3.368354in}}%
\pgfpathlineto{\pgfqpoint{1.686423in}{3.368354in}}%
\pgfpathlineto{\pgfqpoint{1.698685in}{3.368354in}}%
\pgfpathlineto{\pgfqpoint{1.710859in}{3.368354in}}%
\pgfpathlineto{\pgfqpoint{1.723059in}{3.368354in}}%
\pgfpathlineto{\pgfqpoint{1.735394in}{3.368354in}}%
\pgfpathlineto{\pgfqpoint{1.747957in}{3.368354in}}%
\pgfpathlineto{\pgfqpoint{1.760837in}{3.368354in}}%
\pgfpathlineto{\pgfqpoint{1.774167in}{3.232925in}}%
\pgfpathlineto{\pgfqpoint{1.788103in}{3.232925in}}%
\pgfpathlineto{\pgfqpoint{1.802681in}{3.368354in}}%
\pgfpathlineto{\pgfqpoint{1.817954in}{3.368354in}}%
\pgfpathlineto{\pgfqpoint{1.833966in}{3.368354in}}%
\pgfpathlineto{\pgfqpoint{1.850017in}{3.368354in}}%
\pgfpathlineto{\pgfqpoint{1.867586in}{3.503783in}}%
\pgfpathlineto{\pgfqpoint{1.885596in}{3.368354in}}%
\pgfpathlineto{\pgfqpoint{1.903446in}{3.368354in}}%
\pgfpathlineto{\pgfqpoint{1.922076in}{3.368354in}}%
\pgfpathlineto{\pgfqpoint{1.941648in}{3.368354in}}%
\pgfpathlineto{\pgfqpoint{1.952398in}{3.232925in}}%
\pgfpathlineto{\pgfqpoint{1.974978in}{3.503783in}}%
\pgfpathlineto{\pgfqpoint{2.000776in}{3.503783in}}%
\pgfpathlineto{\pgfqpoint{2.028730in}{3.503783in}}%
\pgfpathlineto{\pgfqpoint{2.058429in}{3.503783in}}%
\pgfpathlineto{\pgfqpoint{2.089840in}{3.503783in}}%
\pgfpathlineto{\pgfqpoint{2.122980in}{3.503783in}}%
\pgfpathlineto{\pgfqpoint{2.158111in}{3.503783in}}%
\pgfpathlineto{\pgfqpoint{2.195289in}{3.503783in}}%
\pgfpathlineto{\pgfqpoint{2.234900in}{3.639211in}}%
\pgfpathlineto{\pgfqpoint{2.276921in}{3.639211in}}%
\pgfpathlineto{\pgfqpoint{2.321615in}{3.639211in}}%
\pgfpathlineto{\pgfqpoint{2.369253in}{3.639211in}}%
\pgfpathlineto{\pgfqpoint{2.419903in}{3.639211in}}%
\pgfpathlineto{\pgfqpoint{2.473908in}{3.639211in}}%
\pgfpathlineto{\pgfqpoint{2.531457in}{3.639211in}}%
\pgfpathlineto{\pgfqpoint{2.592446in}{3.639211in}}%
\pgfpathlineto{\pgfqpoint{2.656672in}{3.639211in}}%
\pgfpathlineto{\pgfqpoint{2.724203in}{3.639211in}}%
\pgfpathlineto{\pgfqpoint{2.783761in}{3.639211in}}%
\pgfusepath{stroke}%
\end{pgfscope}%
\begin{pgfscope}%
\pgfpathrectangle{\pgfqpoint{0.629216in}{1.472354in}}{\pgfqpoint{2.257143in}{2.370000in}}%
\pgfusepath{clip}%
\pgfsetrectcap%
\pgfsetroundjoin%
\pgfsetlinewidth{1.505625pt}%
\definecolor{currentstroke}{rgb}{0.172549,0.627451,0.172549}%
\pgfsetstrokecolor{currentstroke}%
\pgfsetdash{}{0pt}%
\pgfpathmoveto{\pgfqpoint{0.803872in}{1.467354in}}%
\pgfpathlineto{\pgfqpoint{0.868610in}{3.774640in}}%
\pgfpathlineto{\pgfqpoint{1.005407in}{3.774640in}}%
\pgfpathlineto{\pgfqpoint{1.142203in}{3.774640in}}%
\pgfpathlineto{\pgfqpoint{1.279000in}{3.774640in}}%
\pgfpathlineto{\pgfqpoint{1.348261in}{3.639211in}}%
\pgfpathlineto{\pgfqpoint{1.390141in}{3.503783in}}%
\pgfpathlineto{\pgfqpoint{1.421955in}{3.503783in}}%
\pgfpathlineto{\pgfqpoint{1.448179in}{3.368354in}}%
\pgfpathlineto{\pgfqpoint{1.470824in}{3.368354in}}%
\pgfpathlineto{\pgfqpoint{1.491009in}{3.368354in}}%
\pgfpathlineto{\pgfqpoint{1.509432in}{3.232925in}}%
\pgfpathlineto{\pgfqpoint{1.526393in}{3.232925in}}%
\pgfpathlineto{\pgfqpoint{1.542042in}{3.232925in}}%
\pgfpathlineto{\pgfqpoint{1.556723in}{3.232925in}}%
\pgfpathlineto{\pgfqpoint{1.570691in}{3.232925in}}%
\pgfpathlineto{\pgfqpoint{1.584141in}{3.232925in}}%
\pgfpathlineto{\pgfqpoint{1.597229in}{3.232925in}}%
\pgfpathlineto{\pgfqpoint{1.610082in}{3.097497in}}%
\pgfpathlineto{\pgfqpoint{1.622811in}{3.097497in}}%
\pgfpathlineto{\pgfqpoint{1.635514in}{3.097497in}}%
\pgfpathlineto{\pgfqpoint{1.648280in}{3.097497in}}%
\pgfpathlineto{\pgfqpoint{1.661192in}{3.232925in}}%
\pgfpathlineto{\pgfqpoint{1.673978in}{3.097497in}}%
\pgfpathlineto{\pgfqpoint{1.686423in}{3.097497in}}%
\pgfpathlineto{\pgfqpoint{1.698685in}{3.097497in}}%
\pgfpathlineto{\pgfqpoint{1.710859in}{3.097497in}}%
\pgfpathlineto{\pgfqpoint{1.723059in}{3.097497in}}%
\pgfpathlineto{\pgfqpoint{1.735394in}{3.097497in}}%
\pgfpathlineto{\pgfqpoint{1.747957in}{3.097497in}}%
\pgfpathlineto{\pgfqpoint{1.760837in}{3.097497in}}%
\pgfpathlineto{\pgfqpoint{1.774165in}{2.962068in}}%
\pgfpathlineto{\pgfqpoint{1.788101in}{3.097497in}}%
\pgfpathlineto{\pgfqpoint{1.802681in}{3.232925in}}%
\pgfpathlineto{\pgfqpoint{1.817964in}{3.097497in}}%
\pgfpathlineto{\pgfqpoint{1.833968in}{3.232925in}}%
\pgfpathlineto{\pgfqpoint{1.850807in}{3.232925in}}%
\pgfpathlineto{\pgfqpoint{1.868425in}{3.232925in}}%
\pgfpathlineto{\pgfqpoint{1.886393in}{3.232925in}}%
\pgfpathlineto{\pgfqpoint{1.904298in}{3.232925in}}%
\pgfpathlineto{\pgfqpoint{1.923008in}{3.232925in}}%
\pgfpathlineto{\pgfqpoint{1.942332in}{3.232925in}}%
\pgfpathlineto{\pgfqpoint{1.956483in}{3.097497in}}%
\pgfpathlineto{\pgfqpoint{1.979601in}{3.368354in}}%
\pgfpathlineto{\pgfqpoint{2.006138in}{3.368354in}}%
\pgfpathlineto{\pgfqpoint{2.034542in}{3.368354in}}%
\pgfpathlineto{\pgfqpoint{2.064625in}{3.368354in}}%
\pgfpathlineto{\pgfqpoint{2.096351in}{3.368354in}}%
\pgfpathlineto{\pgfqpoint{2.129899in}{3.368354in}}%
\pgfpathlineto{\pgfqpoint{2.165292in}{3.368354in}}%
\pgfpathlineto{\pgfqpoint{2.202851in}{3.503783in}}%
\pgfpathlineto{\pgfqpoint{2.242901in}{3.503783in}}%
\pgfpathlineto{\pgfqpoint{2.285492in}{3.503783in}}%
\pgfpathlineto{\pgfqpoint{2.330879in}{3.503783in}}%
\pgfpathlineto{\pgfqpoint{2.379270in}{3.503783in}}%
\pgfpathlineto{\pgfqpoint{2.430831in}{3.503783in}}%
\pgfpathlineto{\pgfqpoint{2.485859in}{3.503783in}}%
\pgfpathlineto{\pgfqpoint{2.544521in}{3.503783in}}%
\pgfpathlineto{\pgfqpoint{2.606558in}{3.503783in}}%
\pgfpathlineto{\pgfqpoint{2.671754in}{3.639211in}}%
\pgfpathlineto{\pgfqpoint{2.740194in}{3.639211in}}%
\pgfpathlineto{\pgfqpoint{2.783761in}{3.503783in}}%
\pgfusepath{stroke}%
\end{pgfscope}%
\begin{pgfscope}%
\pgfpathrectangle{\pgfqpoint{0.629216in}{1.472354in}}{\pgfqpoint{2.257143in}{2.370000in}}%
\pgfusepath{clip}%
\pgfsetrectcap%
\pgfsetroundjoin%
\pgfsetlinewidth{1.505625pt}%
\definecolor{currentstroke}{rgb}{0.839216,0.152941,0.156863}%
\pgfsetstrokecolor{currentstroke}%
\pgfsetdash{}{0pt}%
\pgfpathmoveto{\pgfqpoint{0.803872in}{1.467354in}}%
\pgfpathlineto{\pgfqpoint{0.868610in}{3.774640in}}%
\pgfpathlineto{\pgfqpoint{1.005407in}{3.774640in}}%
\pgfpathlineto{\pgfqpoint{1.142203in}{3.639211in}}%
\pgfpathlineto{\pgfqpoint{1.279000in}{3.639211in}}%
\pgfpathlineto{\pgfqpoint{1.348261in}{3.503783in}}%
\pgfpathlineto{\pgfqpoint{1.390141in}{3.368354in}}%
\pgfpathlineto{\pgfqpoint{1.421955in}{3.368354in}}%
\pgfpathlineto{\pgfqpoint{1.448179in}{3.232925in}}%
\pgfpathlineto{\pgfqpoint{1.470824in}{3.232925in}}%
\pgfpathlineto{\pgfqpoint{1.491009in}{3.097497in}}%
\pgfpathlineto{\pgfqpoint{1.509432in}{3.097497in}}%
\pgfpathlineto{\pgfqpoint{1.526393in}{3.097497in}}%
\pgfpathlineto{\pgfqpoint{1.542042in}{2.962068in}}%
\pgfpathlineto{\pgfqpoint{1.556723in}{2.962068in}}%
\pgfpathlineto{\pgfqpoint{1.570691in}{2.962068in}}%
\pgfpathlineto{\pgfqpoint{1.584141in}{2.962068in}}%
\pgfpathlineto{\pgfqpoint{1.597229in}{2.962068in}}%
\pgfpathlineto{\pgfqpoint{1.610082in}{2.962068in}}%
\pgfpathlineto{\pgfqpoint{1.622811in}{2.962068in}}%
\pgfpathlineto{\pgfqpoint{1.635514in}{2.962068in}}%
\pgfpathlineto{\pgfqpoint{1.648280in}{2.962068in}}%
\pgfpathlineto{\pgfqpoint{1.661192in}{2.962068in}}%
\pgfpathlineto{\pgfqpoint{1.673978in}{2.962068in}}%
\pgfpathlineto{\pgfqpoint{1.686423in}{2.962068in}}%
\pgfpathlineto{\pgfqpoint{1.698685in}{2.826640in}}%
\pgfpathlineto{\pgfqpoint{1.710859in}{2.962068in}}%
\pgfpathlineto{\pgfqpoint{1.723059in}{2.962068in}}%
\pgfpathlineto{\pgfqpoint{1.735394in}{2.962068in}}%
\pgfpathlineto{\pgfqpoint{1.747957in}{2.962068in}}%
\pgfpathlineto{\pgfqpoint{1.760837in}{2.962068in}}%
\pgfpathlineto{\pgfqpoint{1.774165in}{2.826640in}}%
\pgfpathlineto{\pgfqpoint{1.788101in}{2.826640in}}%
\pgfpathlineto{\pgfqpoint{1.802681in}{2.962068in}}%
\pgfpathlineto{\pgfqpoint{1.817964in}{2.826640in}}%
\pgfpathlineto{\pgfqpoint{1.833968in}{2.962068in}}%
\pgfpathlineto{\pgfqpoint{1.850807in}{2.962068in}}%
\pgfpathlineto{\pgfqpoint{1.868425in}{2.962068in}}%
\pgfpathlineto{\pgfqpoint{1.886393in}{2.962068in}}%
\pgfpathlineto{\pgfqpoint{1.904298in}{2.962068in}}%
\pgfpathlineto{\pgfqpoint{1.923008in}{3.097497in}}%
\pgfpathlineto{\pgfqpoint{1.942331in}{3.097497in}}%
\pgfpathlineto{\pgfqpoint{1.956480in}{2.826640in}}%
\pgfpathlineto{\pgfqpoint{1.979601in}{3.097497in}}%
\pgfpathlineto{\pgfqpoint{2.006137in}{3.232925in}}%
\pgfpathlineto{\pgfqpoint{2.034541in}{3.232925in}}%
\pgfpathlineto{\pgfqpoint{2.064625in}{3.232925in}}%
\pgfpathlineto{\pgfqpoint{2.096351in}{3.232925in}}%
\pgfpathlineto{\pgfqpoint{2.129897in}{3.232925in}}%
\pgfpathlineto{\pgfqpoint{2.165289in}{3.368354in}}%
\pgfpathlineto{\pgfqpoint{2.202843in}{3.368354in}}%
\pgfpathlineto{\pgfqpoint{2.242893in}{3.368354in}}%
\pgfpathlineto{\pgfqpoint{2.285479in}{3.368354in}}%
\pgfpathlineto{\pgfqpoint{2.330852in}{3.368354in}}%
\pgfpathlineto{\pgfqpoint{2.379242in}{3.368354in}}%
\pgfpathlineto{\pgfqpoint{2.430810in}{3.368354in}}%
\pgfpathlineto{\pgfqpoint{2.485822in}{3.503783in}}%
\pgfpathlineto{\pgfqpoint{2.544537in}{3.503783in}}%
\pgfpathlineto{\pgfqpoint{2.606519in}{3.503783in}}%
\pgfpathlineto{\pgfqpoint{2.671640in}{3.503783in}}%
\pgfpathlineto{\pgfqpoint{2.739973in}{3.503783in}}%
\pgfpathlineto{\pgfqpoint{2.783761in}{3.368354in}}%
\pgfusepath{stroke}%
\end{pgfscope}%
\begin{pgfscope}%
\pgfsetrectcap%
\pgfsetmiterjoin%
\pgfsetlinewidth{0.803000pt}%
\definecolor{currentstroke}{rgb}{0.000000,0.000000,0.000000}%
\pgfsetstrokecolor{currentstroke}%
\pgfsetdash{}{0pt}%
\pgfpathmoveto{\pgfqpoint{0.629216in}{1.472354in}}%
\pgfpathlineto{\pgfqpoint{0.629216in}{3.842354in}}%
\pgfusepath{stroke}%
\end{pgfscope}%
\begin{pgfscope}%
\pgfsetrectcap%
\pgfsetmiterjoin%
\pgfsetlinewidth{0.803000pt}%
\definecolor{currentstroke}{rgb}{0.000000,0.000000,0.000000}%
\pgfsetstrokecolor{currentstroke}%
\pgfsetdash{}{0pt}%
\pgfpathmoveto{\pgfqpoint{2.886359in}{1.472354in}}%
\pgfpathlineto{\pgfqpoint{2.886359in}{3.842354in}}%
\pgfusepath{stroke}%
\end{pgfscope}%
\begin{pgfscope}%
\pgfsetrectcap%
\pgfsetmiterjoin%
\pgfsetlinewidth{0.803000pt}%
\definecolor{currentstroke}{rgb}{0.000000,0.000000,0.000000}%
\pgfsetstrokecolor{currentstroke}%
\pgfsetdash{}{0pt}%
\pgfpathmoveto{\pgfqpoint{0.629216in}{1.472354in}}%
\pgfpathlineto{\pgfqpoint{2.886359in}{1.472354in}}%
\pgfusepath{stroke}%
\end{pgfscope}%
\begin{pgfscope}%
\pgfsetrectcap%
\pgfsetmiterjoin%
\pgfsetlinewidth{0.803000pt}%
\definecolor{currentstroke}{rgb}{0.000000,0.000000,0.000000}%
\pgfsetstrokecolor{currentstroke}%
\pgfsetdash{}{0pt}%
\pgfpathmoveto{\pgfqpoint{0.629216in}{3.842354in}}%
\pgfpathlineto{\pgfqpoint{2.886359in}{3.842354in}}%
\pgfusepath{stroke}%
\end{pgfscope}%
\begin{pgfscope}%
\pgfsetrectcap%
\pgfsetroundjoin%
\pgfsetlinewidth{1.505625pt}%
\definecolor{currentstroke}{rgb}{0.121569,0.466667,0.705882}%
\pgfsetstrokecolor{currentstroke}%
\pgfsetdash{}{0pt}%
\pgfpathmoveto{\pgfqpoint{2.155929in}{2.042205in}}%
\pgfpathlineto{\pgfqpoint{2.350374in}{2.042205in}}%
\pgfusepath{stroke}%
\end{pgfscope}%
\begin{pgfscope}%
\definecolor{textcolor}{rgb}{0.000000,0.000000,0.000000}%
\pgfsetstrokecolor{textcolor}%
\pgfsetfillcolor{textcolor}%
\pgftext[x=2.428152in,y=2.008177in,left,base]{\color{textcolor}\rmfamily\fontsize{7.000000}{8.400000}\selectfont 1.0\%}%
\end{pgfscope}%
\begin{pgfscope}%
\pgfsetrectcap%
\pgfsetroundjoin%
\pgfsetlinewidth{1.505625pt}%
\definecolor{currentstroke}{rgb}{1.000000,0.498039,0.054902}%
\pgfsetstrokecolor{currentstroke}%
\pgfsetdash{}{0pt}%
\pgfpathmoveto{\pgfqpoint{2.155929in}{1.899505in}}%
\pgfpathlineto{\pgfqpoint{2.350374in}{1.899505in}}%
\pgfusepath{stroke}%
\end{pgfscope}%
\begin{pgfscope}%
\definecolor{textcolor}{rgb}{0.000000,0.000000,0.000000}%
\pgfsetstrokecolor{textcolor}%
\pgfsetfillcolor{textcolor}%
\pgftext[x=2.428152in,y=1.865477in,left,base]{\color{textcolor}\rmfamily\fontsize{7.000000}{8.400000}\selectfont 0.1\%}%
\end{pgfscope}%
\begin{pgfscope}%
\pgfsetrectcap%
\pgfsetroundjoin%
\pgfsetlinewidth{1.505625pt}%
\definecolor{currentstroke}{rgb}{0.172549,0.627451,0.172549}%
\pgfsetstrokecolor{currentstroke}%
\pgfsetdash{}{0pt}%
\pgfpathmoveto{\pgfqpoint{2.155929in}{1.756805in}}%
\pgfpathlineto{\pgfqpoint{2.350374in}{1.756805in}}%
\pgfusepath{stroke}%
\end{pgfscope}%
\begin{pgfscope}%
\definecolor{textcolor}{rgb}{0.000000,0.000000,0.000000}%
\pgfsetstrokecolor{textcolor}%
\pgfsetfillcolor{textcolor}%
\pgftext[x=2.428152in,y=1.722777in,left,base]{\color{textcolor}\rmfamily\fontsize{7.000000}{8.400000}\selectfont 0.01\%}%
\end{pgfscope}%
\begin{pgfscope}%
\pgfsetrectcap%
\pgfsetroundjoin%
\pgfsetlinewidth{1.505625pt}%
\definecolor{currentstroke}{rgb}{0.839216,0.152941,0.156863}%
\pgfsetstrokecolor{currentstroke}%
\pgfsetdash{}{0pt}%
\pgfpathmoveto{\pgfqpoint{2.155929in}{1.614105in}}%
\pgfpathlineto{\pgfqpoint{2.350374in}{1.614105in}}%
\pgfusepath{stroke}%
\end{pgfscope}%
\begin{pgfscope}%
\definecolor{textcolor}{rgb}{0.000000,0.000000,0.000000}%
\pgfsetstrokecolor{textcolor}%
\pgfsetfillcolor{textcolor}%
\pgftext[x=2.428152in,y=1.580077in,left,base]{\color{textcolor}\rmfamily\fontsize{7.000000}{8.400000}\selectfont 0.001\%}%
\end{pgfscope}%
\begin{pgfscope}%
\pgfsetbuttcap%
\pgfsetmiterjoin%
\definecolor{currentfill}{rgb}{1.000000,1.000000,1.000000}%
\pgfsetfillcolor{currentfill}%
\pgfsetlinewidth{0.000000pt}%
\definecolor{currentstroke}{rgb}{0.000000,0.000000,0.000000}%
\pgfsetstrokecolor{currentstroke}%
\pgfsetstrokeopacity{0.000000}%
\pgfsetdash{}{0pt}%
\pgfpathmoveto{\pgfqpoint{0.629216in}{0.484854in}}%
\pgfpathlineto{\pgfqpoint{2.886359in}{0.484854in}}%
\pgfpathlineto{\pgfqpoint{2.886359in}{1.472354in}}%
\pgfpathlineto{\pgfqpoint{0.629216in}{1.472354in}}%
\pgfpathclose%
\pgfusepath{fill}%
\end{pgfscope}%
\begin{pgfscope}%
\pgfsetbuttcap%
\pgfsetroundjoin%
\definecolor{currentfill}{rgb}{0.000000,0.000000,0.000000}%
\pgfsetfillcolor{currentfill}%
\pgfsetlinewidth{0.803000pt}%
\definecolor{currentstroke}{rgb}{0.000000,0.000000,0.000000}%
\pgfsetstrokecolor{currentstroke}%
\pgfsetdash{}{0pt}%
\pgfsys@defobject{currentmarker}{\pgfqpoint{0.000000in}{-0.048611in}}{\pgfqpoint{0.000000in}{0.000000in}}{%
\pgfpathmoveto{\pgfqpoint{0.000000in}{0.000000in}}%
\pgfpathlineto{\pgfqpoint{0.000000in}{-0.048611in}}%
\pgfusepath{stroke,fill}%
}%
\begin{pgfscope}%
\pgfsys@transformshift{0.731814in}{0.484854in}%
\pgfsys@useobject{currentmarker}{}%
\end{pgfscope}%
\end{pgfscope}%
\begin{pgfscope}%
\definecolor{textcolor}{rgb}{0.000000,0.000000,0.000000}%
\pgfsetstrokecolor{textcolor}%
\pgfsetfillcolor{textcolor}%
\pgftext[x=0.731814in,y=0.387632in,,top]{\color{textcolor}\rmfamily\fontsize{8.000000}{9.600000}\selectfont \(\displaystyle 0.0\)}%
\end{pgfscope}%
\begin{pgfscope}%
\pgfsetbuttcap%
\pgfsetroundjoin%
\definecolor{currentfill}{rgb}{0.000000,0.000000,0.000000}%
\pgfsetfillcolor{currentfill}%
\pgfsetlinewidth{0.803000pt}%
\definecolor{currentstroke}{rgb}{0.000000,0.000000,0.000000}%
\pgfsetstrokecolor{currentstroke}%
\pgfsetdash{}{0pt}%
\pgfsys@defobject{currentmarker}{\pgfqpoint{0.000000in}{-0.048611in}}{\pgfqpoint{0.000000in}{0.000000in}}{%
\pgfpathmoveto{\pgfqpoint{0.000000in}{0.000000in}}%
\pgfpathlineto{\pgfqpoint{0.000000in}{-0.048611in}}%
\pgfusepath{stroke,fill}%
}%
\begin{pgfscope}%
\pgfsys@transformshift{1.415796in}{0.484854in}%
\pgfsys@useobject{currentmarker}{}%
\end{pgfscope}%
\end{pgfscope}%
\begin{pgfscope}%
\definecolor{textcolor}{rgb}{0.000000,0.000000,0.000000}%
\pgfsetstrokecolor{textcolor}%
\pgfsetfillcolor{textcolor}%
\pgftext[x=1.415796in,y=0.387632in,,top]{\color{textcolor}\rmfamily\fontsize{8.000000}{9.600000}\selectfont \(\displaystyle 0.5\)}%
\end{pgfscope}%
\begin{pgfscope}%
\pgfsetbuttcap%
\pgfsetroundjoin%
\definecolor{currentfill}{rgb}{0.000000,0.000000,0.000000}%
\pgfsetfillcolor{currentfill}%
\pgfsetlinewidth{0.803000pt}%
\definecolor{currentstroke}{rgb}{0.000000,0.000000,0.000000}%
\pgfsetstrokecolor{currentstroke}%
\pgfsetdash{}{0pt}%
\pgfsys@defobject{currentmarker}{\pgfqpoint{0.000000in}{-0.048611in}}{\pgfqpoint{0.000000in}{0.000000in}}{%
\pgfpathmoveto{\pgfqpoint{0.000000in}{0.000000in}}%
\pgfpathlineto{\pgfqpoint{0.000000in}{-0.048611in}}%
\pgfusepath{stroke,fill}%
}%
\begin{pgfscope}%
\pgfsys@transformshift{2.099779in}{0.484854in}%
\pgfsys@useobject{currentmarker}{}%
\end{pgfscope}%
\end{pgfscope}%
\begin{pgfscope}%
\definecolor{textcolor}{rgb}{0.000000,0.000000,0.000000}%
\pgfsetstrokecolor{textcolor}%
\pgfsetfillcolor{textcolor}%
\pgftext[x=2.099779in,y=0.387632in,,top]{\color{textcolor}\rmfamily\fontsize{8.000000}{9.600000}\selectfont \(\displaystyle 1.0\)}%
\end{pgfscope}%
\begin{pgfscope}%
\pgfsetbuttcap%
\pgfsetroundjoin%
\definecolor{currentfill}{rgb}{0.000000,0.000000,0.000000}%
\pgfsetfillcolor{currentfill}%
\pgfsetlinewidth{0.803000pt}%
\definecolor{currentstroke}{rgb}{0.000000,0.000000,0.000000}%
\pgfsetstrokecolor{currentstroke}%
\pgfsetdash{}{0pt}%
\pgfsys@defobject{currentmarker}{\pgfqpoint{0.000000in}{-0.048611in}}{\pgfqpoint{0.000000in}{0.000000in}}{%
\pgfpathmoveto{\pgfqpoint{0.000000in}{0.000000in}}%
\pgfpathlineto{\pgfqpoint{0.000000in}{-0.048611in}}%
\pgfusepath{stroke,fill}%
}%
\begin{pgfscope}%
\pgfsys@transformshift{2.783761in}{0.484854in}%
\pgfsys@useobject{currentmarker}{}%
\end{pgfscope}%
\end{pgfscope}%
\begin{pgfscope}%
\definecolor{textcolor}{rgb}{0.000000,0.000000,0.000000}%
\pgfsetstrokecolor{textcolor}%
\pgfsetfillcolor{textcolor}%
\pgftext[x=2.783761in,y=0.387632in,,top]{\color{textcolor}\rmfamily\fontsize{8.000000}{9.600000}\selectfont \(\displaystyle 1.5\)}%
\end{pgfscope}%
\begin{pgfscope}%
\definecolor{textcolor}{rgb}{0.000000,0.000000,0.000000}%
\pgfsetstrokecolor{textcolor}%
\pgfsetfillcolor{textcolor}%
\pgftext[x=1.757788in,y=0.224546in,,top]{\color{textcolor}\rmfamily\fontsize{8.000000}{9.600000}\selectfont Time (\(\displaystyle \times 10^6 \, \mathrm{yr}\))}%
\end{pgfscope}%
\begin{pgfscope}%
\pgfsetbuttcap%
\pgfsetroundjoin%
\definecolor{currentfill}{rgb}{0.000000,0.000000,0.000000}%
\pgfsetfillcolor{currentfill}%
\pgfsetlinewidth{0.803000pt}%
\definecolor{currentstroke}{rgb}{0.000000,0.000000,0.000000}%
\pgfsetstrokecolor{currentstroke}%
\pgfsetdash{}{0pt}%
\pgfsys@defobject{currentmarker}{\pgfqpoint{-0.048611in}{0.000000in}}{\pgfqpoint{0.000000in}{0.000000in}}{%
\pgfpathmoveto{\pgfqpoint{0.000000in}{0.000000in}}%
\pgfpathlineto{\pgfqpoint{-0.048611in}{0.000000in}}%
\pgfusepath{stroke,fill}%
}%
\begin{pgfscope}%
\pgfsys@transformshift{0.629216in}{0.934759in}%
\pgfsys@useobject{currentmarker}{}%
\end{pgfscope}%
\end{pgfscope}%
\begin{pgfscope}%
\definecolor{textcolor}{rgb}{0.000000,0.000000,0.000000}%
\pgfsetstrokecolor{textcolor}%
\pgfsetfillcolor{textcolor}%
\pgftext[x=0.413937in,y=0.892549in,left,base]{\color{textcolor}\rmfamily\fontsize{8.000000}{9.600000}\selectfont \(\displaystyle 50\)}%
\end{pgfscope}%
\begin{pgfscope}%
\pgfsetbuttcap%
\pgfsetroundjoin%
\definecolor{currentfill}{rgb}{0.000000,0.000000,0.000000}%
\pgfsetfillcolor{currentfill}%
\pgfsetlinewidth{0.803000pt}%
\definecolor{currentstroke}{rgb}{0.000000,0.000000,0.000000}%
\pgfsetstrokecolor{currentstroke}%
\pgfsetdash{}{0pt}%
\pgfsys@defobject{currentmarker}{\pgfqpoint{-0.048611in}{0.000000in}}{\pgfqpoint{0.000000in}{0.000000in}}{%
\pgfpathmoveto{\pgfqpoint{0.000000in}{0.000000in}}%
\pgfpathlineto{\pgfqpoint{-0.048611in}{0.000000in}}%
\pgfusepath{stroke,fill}%
}%
\begin{pgfscope}%
\pgfsys@transformshift{0.629216in}{1.427468in}%
\pgfsys@useobject{currentmarker}{}%
\end{pgfscope}%
\end{pgfscope}%
\begin{pgfscope}%
\definecolor{textcolor}{rgb}{0.000000,0.000000,0.000000}%
\pgfsetstrokecolor{textcolor}%
\pgfsetfillcolor{textcolor}%
\pgftext[x=0.354908in,y=1.385259in,left,base]{\color{textcolor}\rmfamily\fontsize{8.000000}{9.600000}\selectfont \(\displaystyle 100\)}%
\end{pgfscope}%
\begin{pgfscope}%
\definecolor{textcolor}{rgb}{0.000000,0.000000,0.000000}%
\pgfsetstrokecolor{textcolor}%
\pgfsetfillcolor{textcolor}%
\pgftext[x=0.299353in,y=0.978604in,,bottom,rotate=90.000000]{\color{textcolor}\rmfamily\fontsize{8.000000}{9.600000}\selectfont Timestep (\(\displaystyle \times 10^3 \, \mathrm{yr}\))}%
\end{pgfscope}%
\begin{pgfscope}%
\pgfpathrectangle{\pgfqpoint{0.629216in}{0.484854in}}{\pgfqpoint{2.257143in}{0.987500in}}%
\pgfusepath{clip}%
\pgfsetrectcap%
\pgfsetroundjoin%
\pgfsetlinewidth{1.505625pt}%
\definecolor{currentstroke}{rgb}{0.121569,0.466667,0.705882}%
\pgfsetstrokecolor{currentstroke}%
\pgfsetdash{}{0pt}%
\pgfpathmoveto{\pgfqpoint{0.731814in}{1.427468in}}%
\pgfpathlineto{\pgfqpoint{0.868610in}{1.427468in}}%
\pgfpathlineto{\pgfqpoint{1.005407in}{1.427468in}}%
\pgfpathlineto{\pgfqpoint{1.142203in}{1.427468in}}%
\pgfpathlineto{\pgfqpoint{1.279000in}{0.940979in}}%
\pgfpathlineto{\pgfqpoint{1.348261in}{0.743727in}}%
\pgfpathlineto{\pgfqpoint{1.390141in}{0.671226in}}%
\pgfpathlineto{\pgfqpoint{1.421955in}{0.630953in}}%
\pgfpathlineto{\pgfqpoint{1.448179in}{0.605176in}}%
\pgfpathlineto{\pgfqpoint{1.470824in}{0.587449in}}%
\pgfpathlineto{\pgfqpoint{1.491009in}{0.574760in}}%
\pgfpathlineto{\pgfqpoint{1.509432in}{0.564234in}}%
\pgfpathlineto{\pgfqpoint{1.526393in}{0.554773in}}%
\pgfpathlineto{\pgfqpoint{1.542042in}{0.547804in}}%
\pgfpathlineto{\pgfqpoint{1.556723in}{0.542668in}}%
\pgfpathlineto{\pgfqpoint{1.570691in}{0.538938in}}%
\pgfpathlineto{\pgfqpoint{1.584141in}{0.536327in}}%
\pgfpathlineto{\pgfqpoint{1.597229in}{0.534640in}}%
\pgfpathlineto{\pgfqpoint{1.610082in}{0.533745in}}%
\pgfpathlineto{\pgfqpoint{1.622811in}{0.533555in}}%
\pgfpathlineto{\pgfqpoint{1.635514in}{0.534009in}}%
\pgfpathlineto{\pgfqpoint{1.648280in}{0.535062in}}%
\pgfpathlineto{\pgfqpoint{1.661192in}{0.534151in}}%
\pgfpathlineto{\pgfqpoint{1.673978in}{0.531698in}}%
\pgfpathlineto{\pgfqpoint{1.686423in}{0.530382in}}%
\pgfpathlineto{\pgfqpoint{1.698685in}{0.529741in}}%
\pgfpathlineto{\pgfqpoint{1.710859in}{0.529939in}}%
\pgfpathlineto{\pgfqpoint{1.723059in}{0.530900in}}%
\pgfpathlineto{\pgfqpoint{1.735394in}{0.532549in}}%
\pgfpathlineto{\pgfqpoint{1.747957in}{0.534831in}}%
\pgfpathlineto{\pgfqpoint{1.760837in}{0.538071in}}%
\pgfpathlineto{\pgfqpoint{1.774167in}{0.542473in}}%
\pgfpathlineto{\pgfqpoint{1.788108in}{0.547462in}}%
\pgfpathlineto{\pgfqpoint{1.802741in}{0.552156in}}%
\pgfpathlineto{\pgfqpoint{1.818026in}{0.544496in}}%
\pgfpathlineto{\pgfqpoint{1.832248in}{0.562571in}}%
\pgfpathlineto{\pgfqpoint{1.848979in}{0.567735in}}%
\pgfpathlineto{\pgfqpoint{1.866427in}{0.569074in}}%
\pgfpathlineto{\pgfqpoint{1.884060in}{0.567348in}}%
\pgfpathlineto{\pgfqpoint{1.901454in}{0.575130in}}%
\pgfpathlineto{\pgfqpoint{1.919929in}{0.539702in}}%
\pgfpathlineto{\pgfqpoint{1.933485in}{0.567161in}}%
\pgfpathlineto{\pgfqpoint{1.950853in}{0.607271in}}%
\pgfpathlineto{\pgfqpoint{1.973789in}{0.629249in}}%
\pgfpathlineto{\pgfqpoint{1.999776in}{0.642838in}}%
\pgfpathlineto{\pgfqpoint{2.027650in}{0.654021in}}%
\pgfpathlineto{\pgfqpoint{2.057076in}{0.666478in}}%
\pgfpathlineto{\pgfqpoint{2.088231in}{0.678813in}}%
\pgfpathlineto{\pgfqpoint{2.121099in}{0.692292in}}%
\pgfpathlineto{\pgfqpoint{2.155838in}{0.707623in}}%
\pgfpathlineto{\pgfqpoint{2.192705in}{0.724725in}}%
\pgfpathlineto{\pgfqpoint{2.231946in}{0.743768in}}%
\pgfpathlineto{\pgfqpoint{2.273831in}{0.763620in}}%
\pgfpathlineto{\pgfqpoint{2.318472in}{0.784917in}}%
\pgfpathlineto{\pgfqpoint{2.366069in}{0.806991in}}%
\pgfpathlineto{\pgfqpoint{2.416730in}{0.831207in}}%
\pgfpathlineto{\pgfqpoint{2.470753in}{0.856738in}}%
\pgfpathlineto{\pgfqpoint{2.528321in}{0.881642in}}%
\pgfpathlineto{\pgfqpoint{2.589345in}{0.905084in}}%
\pgfpathlineto{\pgfqpoint{2.653624in}{0.928969in}}%
\pgfpathlineto{\pgfqpoint{2.721219in}{0.892578in}}%
\pgfusepath{stroke}%
\end{pgfscope}%
\begin{pgfscope}%
\pgfsetrectcap%
\pgfsetmiterjoin%
\pgfsetlinewidth{0.803000pt}%
\definecolor{currentstroke}{rgb}{0.000000,0.000000,0.000000}%
\pgfsetstrokecolor{currentstroke}%
\pgfsetdash{}{0pt}%
\pgfpathmoveto{\pgfqpoint{0.629216in}{0.484854in}}%
\pgfpathlineto{\pgfqpoint{0.629216in}{1.472354in}}%
\pgfusepath{stroke}%
\end{pgfscope}%
\begin{pgfscope}%
\pgfsetrectcap%
\pgfsetmiterjoin%
\pgfsetlinewidth{0.803000pt}%
\definecolor{currentstroke}{rgb}{0.000000,0.000000,0.000000}%
\pgfsetstrokecolor{currentstroke}%
\pgfsetdash{}{0pt}%
\pgfpathmoveto{\pgfqpoint{2.886359in}{0.484854in}}%
\pgfpathlineto{\pgfqpoint{2.886359in}{1.472354in}}%
\pgfusepath{stroke}%
\end{pgfscope}%
\begin{pgfscope}%
\pgfsetrectcap%
\pgfsetmiterjoin%
\pgfsetlinewidth{0.803000pt}%
\definecolor{currentstroke}{rgb}{0.000000,0.000000,0.000000}%
\pgfsetstrokecolor{currentstroke}%
\pgfsetdash{}{0pt}%
\pgfpathmoveto{\pgfqpoint{0.629216in}{0.484854in}}%
\pgfpathlineto{\pgfqpoint{2.886359in}{0.484854in}}%
\pgfusepath{stroke}%
\end{pgfscope}%
\begin{pgfscope}%
\pgfsetrectcap%
\pgfsetmiterjoin%
\pgfsetlinewidth{0.803000pt}%
\definecolor{currentstroke}{rgb}{0.000000,0.000000,0.000000}%
\pgfsetstrokecolor{currentstroke}%
\pgfsetdash{}{0pt}%
\pgfpathmoveto{\pgfqpoint{0.629216in}{1.472354in}}%
\pgfpathlineto{\pgfqpoint{2.886359in}{1.472354in}}%
\pgfusepath{stroke}%
\end{pgfscope}%
\end{pgfpicture}%
\makeatother%
\endgroup%

        \caption{Material model plugin.}
        \label{fig:cache_usage_material_model}
    \end{subfigure}
    \caption{Cache utilisation and time step size against time.}
    \label{fig:cache_usage}
\end{figure}
