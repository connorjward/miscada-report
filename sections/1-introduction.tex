Within the mantle and lithosphere, partial melting can occur whenever there is a significant change in either the pressure ($p$), temperature ($T$), or chemical composition ($X$), that is, the amount of various compounds, usually oxides, that the mantle material contains.
Some examples of this in action would be a rising mantle plume, bringing hot material to the surface, or a subduction zone or mid-ocean ridge, where volatile compounds such as \ce{H2O} and \ce{CO2} are added to the composition, reducing the solidus temperature.
Correct modelling of this melt is important because it plays a significant role in determining the dynamics of such systems.

\vspace{5mm}

The modelling of the convection of mantle material is done by \textit{geodynamical codes}.
These calculate the velocity, pressure and temperature fields by solving the compressible Stokes equation as well as an additional temperature equation using the finite-element method.

This work will be using the modern geodynamical code ASPECT~\parencite{kronbichlerHighAccuracyMantle2012}.
It is built upon the powerful finite-element library deal.II~\parencite{bangerth_dealiigeneral-purpose_2007} and improves upon previous codes (e.g. ConMan:~\cite{king_conman_1990}, and CITCOM:~\cite{moresi_accuracy_1996}) by leveraging modern numerical methods, the most important to this work being adaptive mesh refinement and parallel performance that scales to thousands of processors.

\vspace{5mm}

Calculating melt volume within geodynamical codes is typically done by assuming a parametrised melting function~\parencite[e.g.][]{dannbergCompressibleMagmaMantle2016, katz_new_2003}.
However, significant simplifications are involved and no account is taken of the chemical composition.
In order to get a more realistic calculation of the melt volume, so-called \textit{thermodynamical codes} must be used.
These are codes that accept $p$-$T$-$X$ input conditions and calculate the stable phases of the system (e.g. Olivine, Quartz, melt) via a minimisation of the Gibbs energy.
The Gibbs energy is one of several potentials widely used in thermodynamics and it is used here because it is minimised when the system is stable.
Inside the code, it is represented by a series of nonlinear equations that are defined in thermodynamic databases \parencite[e.g.][]{hollandMeltingPeridotitesGranites2018}.
These codes also output a variety of other thermodynamical information, such as the entropy and density, that would be useful to more general geodynamic simulations not involving partial melting.
See~\textcite{connollyPrimerGibbsEnergy2017} for a more detailed description of this process.

At present there exist many different thermodynamical codes, all specialising in some particular \mbox{process} or region of the mantle and lithosphere.
Examples include: MELTS \parencite{ghiorsoChemicalMassTransfer1994}, 
THERMOCALC \parencite{powell_calculating_1998}, 
HeFESTo \parencite{stixrudeThermodynamicsMantleMinerals2011}, 
BurnMan \parencite{cottaarBurnManLowerMantle2014}, and
RCrust \parencite{mayneRcrustToolCalculating2016}.
In this work we will be using 
Perple\_X~
\parencite{connollyComputationPhaseEquilibria2005,connollyGeodynamicEquationState2009}.
Perple\_X is different to many of the other solvers because it converts the nonlinear problem into a linear one by discretising the nonlinear phase functions into a series of \textit{pseudocompounds}.
This is described in detail in \textcite{connollyComputationPhaseEquilibria2005}.
Unlike the nonlinear solvers, this linear formulation of the problem may be solved efficiently using the Simplex algorithm and has guaranteed convergence, at the expense of introducing artefacts into the final phase diagram. 

Previous efforts to integrate Perple\_X into geodynamical codes have taken one of two approaches.
They have either precomputed the thermodynamical properties of interest and stored them in a lookup table~\parencite[e.g.][]{magniDeepWaterRecycling2014,bouilholNumericalApproachMelting2015,freeburnNumericalModelsMagmatic2017} or, more recently, they have made direct calls to Perple\_X, performing the minimisations during the simulation~\parencite[e.g.][]{kaislaniemiLithosphereDestabilizationMelt2018}.
The former method is fast to run during the simulation but preparation of the lookup table can take a long time and it is not completely exact because interpolation is required at each reading.
It also will only work for a fixed composition so is unsuitable for modelling melt transport where the composition will necessarily change.
By contrast, the latter approach produces very accurate results and it will work with a varying composition, but until recently this has been prohibitively expensive to calculate.

Although the examples above used the older code CITCOM instead of ASPECT, ASPECT already includes some integration with Perple\_X \parencite{myhill_perplex_2018}.
However, the integration code focuses on calculating material properties such as the density and compressibilities, rather than our specific investigation of partial melting.
It is additionally both slow to run and difficult to read, both areas that this project aims to improve upon.

\vspace{5mm}

So far we have only discussed measuring the amount and composition of the partially molten rock, disregarding the actual melt migration dynamics essential to a realistic model.
The principal difficulty in modelling melt migration is in handling the length and time scales that differ massively between the solid and liquid materials.
Melt transport occurs very quickly in comparison to the slow motion of the solid mantle material and this poses challenges for the finite element software, in particular if adaptive mesh refinement is not supported.

The simplest method to model partial melting inside of a geodynamical model is called \textit{batch melting} and involves disregarding the separate melt dynamics altogether.
The melt components advect alongside the solid residue so the overall bulk composition does not change.
This is the most straightforward to implement but also the least realistic.

Another possible approach is \textit{fractional melting}.
Here, the melt advects with the residue until a melt percolation threshold is reached, at which point the melt is removed from the bulk composition of the simulation.
This is designed to model the situation where the small pockets of melt become connected and it can flow to the surface.
Since the process happens much more quickly than the advection of the rest of the mantle, it is modelled as being simultaneous.
Fractional melting allows for a more realistic simulation of the remaining depleted residue, but the melt dynamics are still ignored.
The method has the additional disadvantage that the overall amount of material will decrease over time, making long-running simulations unfeasible.
See~\textcite{kaislaniemiLithosphereDestabilizationMelt2018} for an example comparing batch and fractional melting with Perple\_X and CITCOM.

Finally, the most accurate solution possible is to consider the melt as a distinct material that advects separately to the solid residue, unlike the previous methods mentioned that only track the residue.
\mbox{Solving} this requires complex changes to the underlying equations of the geodynamical codes.
So-called \textit{two-phase flow} is naturally more computationally intensive than the previous approximations but it has already been implemented in ASPECT~\parencite{dannbergCompressibleMagmaMantle2016}.

\vspace{5mm}

In this work we present new software integrating Perple\_X into ASPECT, allowing for analysis of melt amounts and composition.
Implementations of all three types of melt model are provided.
We begin with a detailed discussion of the implementation and design choices (Section~\ref{sec:solution}) before applying the code to some model setups (Section~\ref{sec:results}) and finally discussing issues and potential improvements (Section~\ref{sec:discussion}).
